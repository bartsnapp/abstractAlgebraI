\documentclass{ximera}

\usepackage[T1]{fontenc}
\usepackage{stix2}
\usepackage{gillius}
\usepackage{resizegather}
%\usepackage{rsfso} fancy cal
\DeclareMathAlphabet{\mathcal}{OMS}{cmsy}{m}{n} %less fancy cal


\usepackage{multicol}


\usepackage{tikz-cd}
\usepackage{tkz-euclide} %% compass
\usetkzobj{all}  %% tkzCompass
\tikzset{>=stealth}
\tikzcdset{arrow style=tikz}
\usetikzlibrary{math} %% for assigning variables
%\usetikzlibrary{fadings}

\usepackage{colortbl,boldline,makecell} %% group tables


\usepackage[sans]{dsfont}

\usepackage{stmaryrd,pifont}

\graphicspath{
  {./}
  {fields/}
  }     



\let\oldbibliography\thebibliography%% to compact bib
\renewcommand{\thebibliography}[1]{%
  \oldbibliography{#1}%
  \setlength{\itemsep}{0pt}%
}
\renewcommand\refname{} %% no name needed!


\DefineVerbatimEnvironment{macaulay2}{Verbatim}{numbers=left,frame=lines,label=Macaulay2,labelposition=topline}

\DefineVerbatimEnvironment{gap}{Verbatim}{numbers=left,frame=lines,label=GAP,labelposition=topline}

%%% This next bit of code defines all our theorem environments
\makeatletter
\let\c@theorem\relax
\let\c@corollary\relax
%\let\c@example\relax
\makeatother

\let\definition\relax
\let\enddefinition\relax

\let\theorem\relax
\let\endtheorem\relax

\let\proposition\relax
\let\endproposition\relax

\let\exercise\relax
\let\endexercise\relax

\let\question\relax
\let\endquestion\relax

\let\remark\relax
\let\endremark\relax

\let\corollary\relax
\let\endcorollary\relax


\let\example\relax
\let\endexample\relax

\let\warning\relax
\let\endwarning\relax

\let\lemma\relax
\let\endlemma\relax


\let\algorithm\relax
\let\endalgorithm\relax
\usepackage{algpseudocode}

\newtheoremstyle{SlantTheorem}{\topsep}{\topsep}%%% space between body and thm
		{\slshape}                      %%% Thm body font
		{}                              %%% Indent amount (empty = no indent)
		{\bfseries\sffamily}            %%% Thm head font
		{}                              %%% Punctuation after thm head
		{3ex}                           %%% Space after thm head
		{\thmname{#1}\thmnumber{ #2}\thmnote{ \bfseries(#3)}}%%% Thm head spec
\theoremstyle{SlantTheorem}
\newtheorem{theorem}{Theorem}
%\newtheorem{definition}[theorem]{Definition}
%\newtheorem{proposition}[theorem]{Proposition}
%% \newtheorem*{dfnn}{Definition}
%% \newtheorem{ques}{Question}[theorem]
%% \newtheorem*{war}{WARNING}
%% \newtheorem*{cor}{Corollary}
%% \newtheorem*{eg}{Example}
\newtheorem*{remark}{Remark}
\newtheorem*{touchstone}{Touchstone}
\newtheorem{corollary}{Corollary}[theorem]
\newtheorem*{warning}{WARNING}
\newtheorem{example}[corollary]{Example}
\newtheorem{lemma}[theorem]{Lemma}




\newtheoremstyle{Definition}{\topsep}{\topsep}%%% space between body and thm
		{}                              %%% Thm body font
		{}                              %%% Indent amount (empty = no indent)
		{\bfseries\sffamily}            %%% Thm head font
		{}                              %%% Punctuation after thm head
		{3ex}                           %%% Space after thm head
		{\thmname{#1}\thmnumber{ #2}\thmnote{ \bfseries(#3)}}%%% Thm head spec
\theoremstyle{Definition}
\newtheorem{definition}[theorem]{Definition}



\let\algorithm\relax
\let\endalgorithm\relax
\newtheoremstyle{Alg}{\topsep}{\topsep}%%% space between body and thm
		{}                              %%% Thm body font
		{}                              %%% Indent amount (empty = no indent)
		{\bfseries\sffamily}            %%% Thm head font
		{}                              %%% Punctuation after thm head
		{3ex}                           %%% Space after thm head
		{\thmname{#1}\thmnumber{ #2}\thmnote{ \bfseries(#3)}}%%% Thm head spec
\theoremstyle{Alg}
\newtheorem*{algorithm}{Algorithm}
\newtheorem*{construction}{Construction}




\newtheoremstyle{Exercise}{\topsep}{\topsep} %%% space between body and thm
		{}                           %%% Thm body font
		{}                           %%% Indent amount (empty = no indent)
		{\bfseries\sffamily}         %%% Thm head font
		{)}                          %%% Punctuation after thm head
		{ }                          %%% Space after thm head
		{\thmnumber{#2}\thmnote{ \bfseries(#3)}}%%% Thm head spec
\theoremstyle{Exercise}
\newtheorem{exercise}[corollary]{}%[theorem]

%% \newtheoremstyle{Question}{\topsep}{\topsep} %%% space between body and thm
%% 		{\bfseries}                  %%% Thm body font
%% 		{3ex}                        %%% Indent amount (empty = no indent)
%% 		{}                           %%% Thm head font
%% 		{}                           %%% Punctuation after thm head
%% 		{}                           %%% Space after thm head
%% 		{\thmnumber{#2}\thmnote{ \bfseries(#3)}}%%% Thm head spec
\newtheoremstyle{Question}{3em}{3em} %%% space between body and thm
		{\large\bfseries}                           %%% Thm body font
		{}                           %%% Indent amount (empty = no indent)
		{}                         %%% Thm head font
		{}                          %%% Punctuation after thm head
		{0em}                          %%% Space after thm head
		{}%%% Thm head spec
\theoremstyle{Question}
\newtheorem*{question}{}






\renewcommand{\tilde}{\widetilde}
\renewcommand{\bar}{\overline}
\renewcommand{\hat}{\widehat}
\newcommand{\N}{\mathbb N}
\newcommand{\Z}{\mathbb Z}
\newcommand{\R}{\mathbb R}
\newcommand{\Q}{\mathbb Q}
\newcommand{\C}{\mathbb C}
\newcommand{\V}{\mathbb V}
\newcommand{\I}{\mathbb I}
\newcommand{\A}{\mathbb A}
\renewcommand{\o}{\mathbf o}
\newcommand{\iso}{\simeq}
\newcommand{\ph}{\varphi}
\newcommand{\Cf}{\mathcal{C}}
\newcommand{\IZ}{\mathrm{Int}(\Z)}
\newcommand{\dsum}{\oplus}
\newcommand{\directsum}{\bigoplus}
\newcommand{\union}{\bigcup}
\newcommand{\subgp}{\leq}
\newcommand{\normal}{\trianglelefteq}
\renewcommand{\i}{\mathfrak}
\renewcommand{\a}{\mathfrak{a}}
\renewcommand{\b}{\mathfrak{b}}
\newcommand{\m}{\mathfrak{m}}
\newcommand{\p}{\mathfrak{p}}
\newcommand{\q}{\mathfrak{q}}
\newcommand{\dfn}[1]{\textbf{#1}\index{#1}}
\let\hom\relax
\DeclareMathOperator{\mat}{Mat}
\DeclareMathOperator{\ann}{Ann}
\DeclareMathOperator{\h}{ht}
\DeclareMathOperator{\tr}{tr}
\DeclareMathOperator{\hom}{Hom}
\DeclareMathOperator{\Span}{Span}
\DeclareMathOperator{\spec}{Spec}
\DeclareMathOperator{\maxspec}{MaxSpec}
\DeclareMathOperator{\aut}{Aut}
\DeclareMathOperator{\ass}{Ass}
\DeclareMathOperator{\lcm}{lcm}
\DeclareMathOperator{\ff}{Frac}
\DeclareMathOperator{\im}{Im}
\DeclareMathOperator{\syz}{Syz}
\DeclareMathOperator{\gr}{Gr}
\DeclareMathOperator{\multideg}{multideg}
\renewcommand{\ker}{\mathop{\mathrm{Ker}}\nolimits}
\newcommand{\coker}{\mathop{\mathrm{Coker}}\nolimits}
\newcommand{\lps}{[\hspace{-0.25ex}[}
\newcommand{\rps}{]\hspace{-0.25ex}]}
\newcommand{\into}{\hookrightarrow}
\newcommand{\onto}{\twoheadrightarrow}
\newcommand{\tensor}{\otimes}
\newcommand{\x}{\mathbf{x}}
\newcommand{\X}{\mathbf X}
\newcommand{\Y}{\mathbf Y}
\renewcommand{\k}{\boldsymbol{\kappa}}
\renewcommand{\emptyset}{\varnothing}
\renewcommand{\qedsymbol}{$\blacksquare$}
\renewcommand{\l}{\ell}
\newcommand{\1}{\mathds{1}}
\newcommand{\lto}{\mathop{\longrightarrow\,}\limits}
\newcommand{\rad}{\sqrt}
\newcommand{\hf}{H}
\newcommand{\hs}{H\!S}
\newcommand{\hp}{H\!P}
\renewcommand{\vec}{\mathbf}
\let\temp\phi
\let\phi\varphi
\let\eulerphi\temp


\renewcommand{\epsilon}{\varepsilon}
\renewcommand{\subset}{\subseteq}
\renewcommand{\supset}{\supseteq}
\newcommand{\macaulay}{\normalfont\textsl{Macaulay2}}
\newcommand{\GAP}{\normalfont\textsf{GAP}}
\newcommand{\invlim}{\varprojlim}
\renewcommand{\le}{\leqslant}
\renewcommand{\ge}{\geqslant}
\newcommand{\valpha}{{\boldsymbol\alpha}}
\newcommand{\vbeta}{{\boldsymbol\beta}}
\newcommand{\vgamma}{{\boldsymbol\gamma}}
\newcommand{\dotp}{\bullet}
\newcommand{\lc}{\normalfont\textsc{lc}}
\newcommand{\lt}{\normalfont\textsc{lt}}
\newcommand{\lm}{\normalfont\textsc{lm}}
\newcommand{\from}{\leftarrow}
\newcommand{\transpose}{\intercal}
\newcommand{\grad}{\boldsymbol\nabla}
\newcommand{\curl}{\boldsymbol{\nabla\times}}
\renewcommand{\d}{\, d}
\newcommand{\<}{\langle}
\renewcommand{\>}{\rangle}

%\renewcommand{\proofname}{Sketch of Proof}


\renewenvironment{proof}[1][Proof]
  {\begin{trivlist}\item[\hskip \labelsep \itshape \bfseries #1{}\hspace{2ex}]\upshape}
{\qed\end{trivlist}}

\newenvironment{sketch}[1][Sketch of Proof]
  {\begin{trivlist}\item[\hskip \labelsep \itshape \bfseries #1{}\hspace{2ex}]\upshape}
{\qed\end{trivlist}}



\makeatletter
\renewcommand\section{\@startsection{paragraph}{10}{\z@}%
                                     {-3.25ex\@plus -1ex \@minus -.2ex}%
                                     {1.5ex \@plus .2ex}%
                                     {\normalfont\large\sffamily\bfseries}}
\renewcommand\subsection{\@startsection{subparagraph}{10}{\z@}%
                                    {3.25ex \@plus1ex \@minus.2ex}%
                                    {-1em}%
                                    {\normalfont\normalsize\sffamily\bfseries}}
\makeatother

%% Fix weird index/bib issue.
\makeatletter
\gdef\ttl@savemark{\sectionmark{}}
\makeatother


\makeatletter
%% no number for refs
\newcommand\frontstyle{%
  \def\activitystyle{activity-chapter}
  \def\maketitle{%
    \addtocounter{titlenumber}{1}%
                    {\flushleft\small\sffamily\bfseries\@pretitle\par\vspace{-1.5em}}%
                    {\flushleft\LARGE\sffamily\bfseries\@title \par }%
                    {\vskip .6em\noindent\textit\theabstract\setcounter{problem}{0}\setcounter{sectiontitlenumber}{0}}%
                    \par\vspace{2em}
                    \phantomsection\addcontentsline{toc}{section}{\textbf{\@title}}%
                  }}
\makeatother



\NewEnviron{annotate}{\vspace{-.3cm}\small \itshape \BODY \vspace{.3cm}}


%%%% TIKZ STUFF

%% N-GON code
\tikzset{
    pics/tikzngon/.style={
        code={
        \tikzmath{\xx = #1;\rr=1.7;}
        \draw[ultra thick,rounded corners=.05mm] ({\rr*sin(0*360/\xx)},{\rr*cos(0*360/\xx)})
        \foreach \x in {-1,0,...,\xx}
        {
        -- ({\rr*sin(\x*360/\xx)},{\rr*cos(\x*360/\xx)})
        }
           -- cycle;
  }}}

%% N-GON code (even)
\tikzset{
    pics/tikzegon/.style={
        code={
        \tikzmath{\xx = #1;\rr=1.7;}
        \draw[ultra thick,rounded corners=.05mm] ({\rr*sin(0*360/\xx+180/\xx)},{\rr*cos(0*360/\xx+180/\xx)})
        \foreach \x in {-1,0,...,\xx}
           {
           -- ({\rr*sin(\x*360/\xx+180/\xx)},{\rr*cos(\x*360/\xx+180/\xx)}) 
           }
           -- cycle;
  }}}




%% N-CLOCK code
\tikzset{
    pics/tikznclock/.style={
        code={
        \tikzmath{\xx = #1;\rr=1.7;\dd=.4;}
        \foreach \x in {1,...,\xx}
        \pgfmathtruncatemacro{\xy}{\x-1}
           {
             \node[circle,fill=black,inner sep=0pt, minimum size=13pt,text=white]
             at ({(\rr-\dd)*sin((\x-1)*360/(\xx)},{(\rr-\dd)*cos((\x-1)*360/\xx}) {\normalfont\bfseries\sffamily\small {\xy}};
           }
  \draw[thick] (0,0) circle (\rr);
  }}}



%% barcode from
%% https://tex.stackexchange.com/questions/6895/is-there-a-good-latex-package-for-generating-barcodes
%% NOT CURRENTLY USED!


\def\barcode#1#2#3#4#5#6#7{\begingroup%
  \dimen0=0.1em
  \def\stack##1##2{\oalign{##1\cr\hidewidth##2\hidewidth}}%
  \def\0##1{\kern##1\dimen0}%
  \def\1##1{\vrule height10ex width##1\dimen0}%
  \def\L##1{\ifcase##1\bc3211##1\or\bc2221##1\or\bc2122##1\or\bc1411##1%
    \or\bc1132##1\or\bc1231##1\or\bc1114##1\or\bc1312##1\or\bc1213##1%
    \or\bc3112##1\fi}%
  \def\R##1{\bgroup\let\next\1\let\1\0\let\0\next\L##1\egroup}%
  \def\G##1{\bgroup\let\bc\bcg\L##1\egroup}% reverse
  \def\bc##1##2##3##4##5{\stack{\0##1\1##2\0##3\1##4}##5}%
  \def\bcg##1##2##3##4##5{\stack{\0##4\1##3\0##2\1##1}##5}%
  \def\bcR##1##2##3##4##5##6{\R##1\R##2\R##3\R##4\R##5\R##6\11\01\11\09%
    \endgroup}%
  \stack{\09}#1\11\01\11\L#2%
  \ifcase#1\L#3\L#4\L#5\L#6\L#7\or\L#3\G#4\L#5\G#6\G#7%
    \or\L#3\G#4\G#5\L#6\G#7\or\L#3\G#4\G#5\G#6\L#7%
    \or\G#3\L#4\L#5\G#6\G#7\or\G#3\G#4\L#5\L#6\G#7%
    \or\G#3\G#4\G#5\L#6\L#7\or\G#3\L#4\G#5\L#6\G#7%
    \or\G#3\L#4\G#5\G#6\L#7\or\G#3\G#4\L#5\G#6\L#7%
  \fi\01\11\01\11\01\bcR}


\author{Bart Snapp}

\title{Noether's isomorphism theorem}

\begin{document}
\begin{abstract}
  We introduce Noether's isomorphism theorem, also known as the first
  isomorphism theorem.
\end{abstract}
\maketitle

\index{first isomorphism theorem}
Sometimes called the \textit{first isomorphism theorem}, Noether's
isomorphism theorem first appeared in her paper \link[\textit{Abstrakter
  Aufbau der Idealtheorie in algebraischen Zahl- und
  Funktionenk\"orpern}]{https://link.springer.com/article/10.1007/BF01209152}.
For those who aren't in-the-know, \link[Emmy Noether]{https://en.wikipedia.org/wiki/Emmy\_Noether} was one of the
greatest mathematicians of the 20th century. Let's see her isomorphism theorem.



\begin{theorem}[Noether's isomorphism theorem]\label{T:NI}\index{Noether's isomorphism theorem}\index{first isomorphism theorem}
  If $\phi:G\to H$ is a homomorphism of groups, then
  \[
  \frac{G}{\ker(\phi)} \iso \im(\phi).
  \]
  \begin{proof}
    To prove this theorem, we will construct the isomorphism
    above. We'll do this in a number of steps.


    \textbf{First we define a map from
      $\boldsymbol{\frac{G}{\ker(\phi)}}$ to
      $\boldsymbol{\im(\phi)}$.}  Let $a\in G$. Set
    \begin{align*}
      \eta: \frac{G}{\ker(\phi)} &\to \im(\phi)\\
      a\ker(\phi) &\mapsto \phi(a).
    \end{align*}
    \textbf{We must check this map is
      well-defined.}\index{well-defined} Suppose for $a,b\in G$
    \[
    a \ker(\phi) = b\ker(\phi).
    \]
    We must show that $\phi(a) = \phi(b)$. First note that if $a
    \ker(\phi) = b\ker(\phi)$, this means that $a\in
    b\ker(\phi)$. Hence $a = bk$ for some $k\in\ker(\phi)$. Now write
    with me
    \begin{align*}
      \phi(a) &=\phi(bk)\\
      &=\phi(b)\phi(k)\\
      &=\phi(b) e_H\\
      &=\phi(b).
    \end{align*}
    Since $\phi(a) = \phi(b)$, we see that $\eta$ is well-defined.


    
    \textbf{Now we must show that $\boldsymbol\eta$ is a
      homomorphism.} First recall that $\ker(\phi)$ is a normal
    subgroup of $G$, Lemma~\ref{L:kerN}, and so
    \[
    (a\ker(\phi)) (b\ker(\phi))  =  ab\ker(\phi)
    \]
    by Lemma~\ref{L:cpwd}.  Now write with me,
    \begin{align*}
    \eta\left((a\ker(\phi))( b\ker(\phi))\right) &= \eta(ab\ker(\phi))\\
    &= \phi(ab)\\
    &= \phi(a)\phi(b)\\
    &= \eta(a\ker(\phi)) \eta(b\ker(\phi)).
    \end{align*}



    \textbf{We must show that $\boldsymbol\eta$ is bijective.} Note
    $\eta$ is surjective by the definition of $\im(\phi)$.  To see that
    $\eta$ is injective, note that if $\eta(a\ker(\phi)) = e_H$, then
    \[
    \phi(a) = e_H
    \]
    and so $a\in \ker(\phi)$, meaning
    \[
    a\equiv e_G \pmod{\ker(\phi)}.
    \]
    Hence by Lemma~\ref{L:kerinj}, $\eta$ is injective.

    We have now shown $\eta$ is a bijective homomorphism, and hence an
    isomorphism. We conclude $\frac{G}{\ker(\phi)} \iso \im(\phi)$.
  \end{proof}
\end{theorem}

Noether's isomorphism theorem above is saying something powerful about
group homomorphisms. It's saying that the \textbf{image of any
  homomorphism} from a group $G$ to any group $H$ is \textbf{always
  isomorphic to a quotient group $\boldsymbol{G/N}$}.


\begin{corollary}[Canonical isomorphism]
  If $\pi:G\onto H$, is a surjective homomorphism of groups, then
  \[
  \frac{G}{\ker(\pi)} \iso H.
  \]
\end{corollary}

\begin{example}[$\boldsymbol{\pmb\Z\pmb\to\pmb\Z_3}$]
  Consider the homomorphism of groups
  \begin{align*}
    \pi:\Z &\to \Z_3\\
    1 &\mapsto 1.
  \end{align*}
  Here $\ker(\pi)= 3\Z$, hence we see by Noether's isomorphism
  theorem that
  \[
  \frac{\Z}{3\Z}\iso \Z_3.
  \]
\end{example}



\begin{example}[Single variable calculus]
  Let $\R[x]$ denote the group of all polynomials in one variable with
  real coefficients under addition. Meaning
  \[
  f = c_nx^{n} + c_{n-1}x^{n-1} + \cdots + c_1x + c_0
  \]
  where $n\in\N\cup\{0\}$ and each $c_i\in\R$.  For each $f$ in
  $\R[x]$, let $f'$ denote the derivative of $f$. The map
  \begin{align*}
    \frac{d}{dx}:\R[x]&\to\R[x]\\
    f  &\mapsto f'\\
    c_i x^i &\mapsto i\cdot c_i x^{i-1}
  \end{align*}
  is a homomorphism of additive groups with
  \[
  \ker\left(\frac{d}{dx}\right) = \{c:c\in \R\}
  \]
  and
  \[
  \im\left(\frac{d}{dx}\right) =\R[x].
  \]
  Moreover, by Noether's isomorphism theorem,
  \[
  \frac{\R[x]}{\ker\left(\frac{d}{dx}\right)} \iso \im\left(\frac{d}{dx}\right).
  \]
  The isomorphisms are given by
  \begin{align*}
    \frac{d}{dx}:\frac{\R[x]}{\ker\left(\frac{d}{dx}\right)} &\to \im\left(\frac{d}{dx}\right)\\
    f+\ker\left(\frac{d}{dx}\right) &\mapsto f'
  \end{align*}
  and
  \begin{align*}
    \iota:   \im\left(\frac{d}{dx}\right) &\to \frac{\R[x]}{\ker\left(\frac{d}{dx}\right)} \\
    f &\mapsto \int f \d x\\
    c_ix^i &\mapsto \frac{c_ix^{i+1}}{i+1} + \ker\left(\frac{d}{dx}\right).
  \end{align*}
\end{example}



So let's pause and reflect a little bit now. A standard calculus
problem goes something like this:

\begin{quote}
  \textit{Find the signed area between the $x$-axis and the function
    $f$ on the interval $[a,b]$.}
\end{quote}

From the example above we should see there is a connection between
solving a question like this one and homomorphisms. Let's make that
explicit now.

We can introduce two new homomorphims. First, the ``evaluation'' map,
  \begin{align*}
    [-]_a^b: \frac{\R[x]}{\ker\left(\frac{d}{dx}\right)}&\to\R\\
    f+ \ker\left(\frac{d}{dx}\right) &\mapsto f(b)-f(a).
  \end{align*}
I'll leave it to you, my friend, to show that this a well-defined
homomorphism from $(\R[x]/\ker\left(d/dx\right),+)$ to $(\R,+)$.


Second, we need the signed ``area'' map,
  \begin{align*}
    \alpha_a^b :\R[x]&\to \R\\
    f &\mapsto \left(\begin{minipage}{2in}signed area between the $x$-axis and the function $f$ on
    the interval $[a,b]$.\end{minipage}\right)
  \end{align*}
Again, it's your task to show that this a homomorphism from
$(\R[x],+)$ to $(\R,+)$.

Finally, we can represent the solution to the calculus problem above
with the following diagram:
\[
\begin{tikzcd}
  \R[x]=\im\left(\frac{d}{dx}\right) \ar[d,"{\alpha_a^b}",swap] \ar[r,"\iota"]  &  \frac{\R[x]}{\ker\left(\frac{d}{dx}\right)} \ar[d,"{[-]_a^b}"]\\
  \R \ar[r,equal,"\text{F.T.C.}"]  & \R
\end{tikzcd}
\]
The \index{Fundamental Theorem of Calculus}Fundamental Theorem of
Calculus says that this diagram commutes, meaning
\[
  [-]_a^b\circ\iota = \alpha_a^b.
\]
Let's do a really easy example, we'll compute
\[
\int_1^5 x\d x.
\]
In the diagram we see:
\[
\begin{tikzpicture}
\matrix[matrix of math nodes,row sep=2cm,column sep=2cm] (m) {%
\R[x]=\im\left(\frac{d}{dx}\right) & \R[x]/\ker(d/dx) \\
\R & \R \\};
\path   (m-1-1) node[above left=2ex and 5ex] (a) {$\scriptstyle f'(x)= x$}
        (m-1-2) node[above right=2ex and 5ex] (b) {$\scriptstyle \int f' \d x=\frac{x^2}{2}+C$}
        (m-2-1) node[below left=1.5ex and 5ex] (c) {$\scriptstyle \int_1^5 x \d x = ?$}
        (m-2-2) node[below right=1.5ex and 5ex] (d) {$\scriptstyle  12=\left[\frac{x^2}{2}+C\right]_1^5$};

\draw[->] (m-1-1) -- node[above] {$\iota$} ++ (m-1-2);
\draw[->] (m-1-1) -- node[left] {$\alpha_1^5$} ++(m-2-1);
\draw[->] (m-1-2) -- node[right] {$[-]_1^5$} ++(m-2-2);
\draw[double] (m-2-1) -- node[above] {\text{F.T.C.}} ++(m-2-2);

\draw[toarrow] (a) to[bend left] (b);
\draw[toarrow] (b) to[bend left] (d);
\draw[toarrow] (a) to[bend right] (c);
\draw[double] (c) to[bend right] (d);
\end{tikzpicture}
\]
So by the Fundamental Theorem of Calculus, we see the answer is $12$.
I like this diagram, and this way of thinking about calculus, because
it clarifies exactly what is happening at each step, disambiguation
the symbolic manipulation of functions from the statement of the
Fundamental Theorem of Calculus.





\end{document}
 
