\documentclass{ximera}

\usepackage[T1]{fontenc}
\usepackage{stix2}
\usepackage{gillius}
\usepackage{resizegather}
%\usepackage{rsfso} fancy cal
\DeclareMathAlphabet{\mathcal}{OMS}{cmsy}{m}{n} %less fancy cal


\usepackage{multicol}


\usepackage{tikz-cd}
\usepackage{tkz-euclide} %% compass
\usetkzobj{all}  %% tkzCompass
\tikzset{>=stealth}
\tikzcdset{arrow style=tikz}
\usetikzlibrary{math} %% for assigning variables
%\usetikzlibrary{fadings}

\usepackage{colortbl,boldline,makecell} %% group tables


\usepackage[sans]{dsfont}

\usepackage{stmaryrd,pifont}

\graphicspath{
  {./}
  {fields/}
  }     



\let\oldbibliography\thebibliography%% to compact bib
\renewcommand{\thebibliography}[1]{%
  \oldbibliography{#1}%
  \setlength{\itemsep}{0pt}%
}
\renewcommand\refname{} %% no name needed!


\DefineVerbatimEnvironment{macaulay2}{Verbatim}{numbers=left,frame=lines,label=Macaulay2,labelposition=topline}

\DefineVerbatimEnvironment{gap}{Verbatim}{numbers=left,frame=lines,label=GAP,labelposition=topline}

%%% This next bit of code defines all our theorem environments
\makeatletter
\let\c@theorem\relax
\let\c@corollary\relax
%\let\c@example\relax
\makeatother

\let\definition\relax
\let\enddefinition\relax

\let\theorem\relax
\let\endtheorem\relax

\let\proposition\relax
\let\endproposition\relax

\let\exercise\relax
\let\endexercise\relax

\let\question\relax
\let\endquestion\relax

\let\remark\relax
\let\endremark\relax

\let\corollary\relax
\let\endcorollary\relax


\let\example\relax
\let\endexample\relax

\let\warning\relax
\let\endwarning\relax

\let\lemma\relax
\let\endlemma\relax


\let\algorithm\relax
\let\endalgorithm\relax
\usepackage{algpseudocode}

\newtheoremstyle{SlantTheorem}{\topsep}{\topsep}%%% space between body and thm
		{\slshape}                      %%% Thm body font
		{}                              %%% Indent amount (empty = no indent)
		{\bfseries\sffamily}            %%% Thm head font
		{}                              %%% Punctuation after thm head
		{3ex}                           %%% Space after thm head
		{\thmname{#1}\thmnumber{ #2}\thmnote{ \bfseries(#3)}}%%% Thm head spec
\theoremstyle{SlantTheorem}
\newtheorem{theorem}{Theorem}
%\newtheorem{definition}[theorem]{Definition}
%\newtheorem{proposition}[theorem]{Proposition}
%% \newtheorem*{dfnn}{Definition}
%% \newtheorem{ques}{Question}[theorem]
%% \newtheorem*{war}{WARNING}
%% \newtheorem*{cor}{Corollary}
%% \newtheorem*{eg}{Example}
\newtheorem*{remark}{Remark}
\newtheorem*{touchstone}{Touchstone}
\newtheorem{corollary}{Corollary}[theorem]
\newtheorem*{warning}{WARNING}
\newtheorem{example}[corollary]{Example}
\newtheorem{lemma}[theorem]{Lemma}




\newtheoremstyle{Definition}{\topsep}{\topsep}%%% space between body and thm
		{}                              %%% Thm body font
		{}                              %%% Indent amount (empty = no indent)
		{\bfseries\sffamily}            %%% Thm head font
		{}                              %%% Punctuation after thm head
		{3ex}                           %%% Space after thm head
		{\thmname{#1}\thmnumber{ #2}\thmnote{ \bfseries(#3)}}%%% Thm head spec
\theoremstyle{Definition}
\newtheorem{definition}[theorem]{Definition}



\let\algorithm\relax
\let\endalgorithm\relax
\newtheoremstyle{Alg}{\topsep}{\topsep}%%% space between body and thm
		{}                              %%% Thm body font
		{}                              %%% Indent amount (empty = no indent)
		{\bfseries\sffamily}            %%% Thm head font
		{}                              %%% Punctuation after thm head
		{3ex}                           %%% Space after thm head
		{\thmname{#1}\thmnumber{ #2}\thmnote{ \bfseries(#3)}}%%% Thm head spec
\theoremstyle{Alg}
\newtheorem*{algorithm}{Algorithm}
\newtheorem*{construction}{Construction}




\newtheoremstyle{Exercise}{\topsep}{\topsep} %%% space between body and thm
		{}                           %%% Thm body font
		{}                           %%% Indent amount (empty = no indent)
		{\bfseries\sffamily}         %%% Thm head font
		{)}                          %%% Punctuation after thm head
		{ }                          %%% Space after thm head
		{\thmnumber{#2}\thmnote{ \bfseries(#3)}}%%% Thm head spec
\theoremstyle{Exercise}
\newtheorem{exercise}[corollary]{}%[theorem]

%% \newtheoremstyle{Question}{\topsep}{\topsep} %%% space between body and thm
%% 		{\bfseries}                  %%% Thm body font
%% 		{3ex}                        %%% Indent amount (empty = no indent)
%% 		{}                           %%% Thm head font
%% 		{}                           %%% Punctuation after thm head
%% 		{}                           %%% Space after thm head
%% 		{\thmnumber{#2}\thmnote{ \bfseries(#3)}}%%% Thm head spec
\newtheoremstyle{Question}{3em}{3em} %%% space between body and thm
		{\large\bfseries}                           %%% Thm body font
		{}                           %%% Indent amount (empty = no indent)
		{}                         %%% Thm head font
		{}                          %%% Punctuation after thm head
		{0em}                          %%% Space after thm head
		{}%%% Thm head spec
\theoremstyle{Question}
\newtheorem*{question}{}






\renewcommand{\tilde}{\widetilde}
\renewcommand{\bar}{\overline}
\renewcommand{\hat}{\widehat}
\newcommand{\N}{\mathbb N}
\newcommand{\Z}{\mathbb Z}
\newcommand{\R}{\mathbb R}
\newcommand{\Q}{\mathbb Q}
\newcommand{\C}{\mathbb C}
\newcommand{\V}{\mathbb V}
\newcommand{\I}{\mathbb I}
\newcommand{\A}{\mathbb A}
\renewcommand{\o}{\mathbf o}
\newcommand{\iso}{\simeq}
\newcommand{\ph}{\varphi}
\newcommand{\Cf}{\mathcal{C}}
\newcommand{\IZ}{\mathrm{Int}(\Z)}
\newcommand{\dsum}{\oplus}
\newcommand{\directsum}{\bigoplus}
\newcommand{\union}{\bigcup}
\newcommand{\subgp}{\leq}
\newcommand{\normal}{\trianglelefteq}
\renewcommand{\i}{\mathfrak}
\renewcommand{\a}{\mathfrak{a}}
\renewcommand{\b}{\mathfrak{b}}
\newcommand{\m}{\mathfrak{m}}
\newcommand{\p}{\mathfrak{p}}
\newcommand{\q}{\mathfrak{q}}
\newcommand{\dfn}[1]{\textbf{#1}\index{#1}}
\let\hom\relax
\DeclareMathOperator{\mat}{Mat}
\DeclareMathOperator{\ann}{Ann}
\DeclareMathOperator{\h}{ht}
\DeclareMathOperator{\tr}{tr}
\DeclareMathOperator{\hom}{Hom}
\DeclareMathOperator{\Span}{Span}
\DeclareMathOperator{\spec}{Spec}
\DeclareMathOperator{\maxspec}{MaxSpec}
\DeclareMathOperator{\aut}{Aut}
\DeclareMathOperator{\ass}{Ass}
\DeclareMathOperator{\lcm}{lcm}
\DeclareMathOperator{\ff}{Frac}
\DeclareMathOperator{\im}{Im}
\DeclareMathOperator{\syz}{Syz}
\DeclareMathOperator{\gr}{Gr}
\DeclareMathOperator{\multideg}{multideg}
\renewcommand{\ker}{\mathop{\mathrm{Ker}}\nolimits}
\newcommand{\coker}{\mathop{\mathrm{Coker}}\nolimits}
\newcommand{\lps}{[\hspace{-0.25ex}[}
\newcommand{\rps}{]\hspace{-0.25ex}]}
\newcommand{\into}{\hookrightarrow}
\newcommand{\onto}{\twoheadrightarrow}
\newcommand{\tensor}{\otimes}
\newcommand{\x}{\mathbf{x}}
\newcommand{\X}{\mathbf X}
\newcommand{\Y}{\mathbf Y}
\renewcommand{\k}{\boldsymbol{\kappa}}
\renewcommand{\emptyset}{\varnothing}
\renewcommand{\qedsymbol}{$\blacksquare$}
\renewcommand{\l}{\ell}
\newcommand{\1}{\mathds{1}}
\newcommand{\lto}{\mathop{\longrightarrow\,}\limits}
\newcommand{\rad}{\sqrt}
\newcommand{\hf}{H}
\newcommand{\hs}{H\!S}
\newcommand{\hp}{H\!P}
\renewcommand{\vec}{\mathbf}
\let\temp\phi
\let\phi\varphi
\let\eulerphi\temp


\renewcommand{\epsilon}{\varepsilon}
\renewcommand{\subset}{\subseteq}
\renewcommand{\supset}{\supseteq}
\newcommand{\macaulay}{\normalfont\textsl{Macaulay2}}
\newcommand{\GAP}{\normalfont\textsf{GAP}}
\newcommand{\invlim}{\varprojlim}
\renewcommand{\le}{\leqslant}
\renewcommand{\ge}{\geqslant}
\newcommand{\valpha}{{\boldsymbol\alpha}}
\newcommand{\vbeta}{{\boldsymbol\beta}}
\newcommand{\vgamma}{{\boldsymbol\gamma}}
\newcommand{\dotp}{\bullet}
\newcommand{\lc}{\normalfont\textsc{lc}}
\newcommand{\lt}{\normalfont\textsc{lt}}
\newcommand{\lm}{\normalfont\textsc{lm}}
\newcommand{\from}{\leftarrow}
\newcommand{\transpose}{\intercal}
\newcommand{\grad}{\boldsymbol\nabla}
\newcommand{\curl}{\boldsymbol{\nabla\times}}
\renewcommand{\d}{\, d}
\newcommand{\<}{\langle}
\renewcommand{\>}{\rangle}

%\renewcommand{\proofname}{Sketch of Proof}


\renewenvironment{proof}[1][Proof]
  {\begin{trivlist}\item[\hskip \labelsep \itshape \bfseries #1{}\hspace{2ex}]\upshape}
{\qed\end{trivlist}}

\newenvironment{sketch}[1][Sketch of Proof]
  {\begin{trivlist}\item[\hskip \labelsep \itshape \bfseries #1{}\hspace{2ex}]\upshape}
{\qed\end{trivlist}}



\makeatletter
\renewcommand\section{\@startsection{paragraph}{10}{\z@}%
                                     {-3.25ex\@plus -1ex \@minus -.2ex}%
                                     {1.5ex \@plus .2ex}%
                                     {\normalfont\large\sffamily\bfseries}}
\renewcommand\subsection{\@startsection{subparagraph}{10}{\z@}%
                                    {3.25ex \@plus1ex \@minus.2ex}%
                                    {-1em}%
                                    {\normalfont\normalsize\sffamily\bfseries}}
\makeatother

%% Fix weird index/bib issue.
\makeatletter
\gdef\ttl@savemark{\sectionmark{}}
\makeatother


\makeatletter
%% no number for refs
\newcommand\frontstyle{%
  \def\activitystyle{activity-chapter}
  \def\maketitle{%
    \addtocounter{titlenumber}{1}%
                    {\flushleft\small\sffamily\bfseries\@pretitle\par\vspace{-1.5em}}%
                    {\flushleft\LARGE\sffamily\bfseries\@title \par }%
                    {\vskip .6em\noindent\textit\theabstract\setcounter{problem}{0}\setcounter{sectiontitlenumber}{0}}%
                    \par\vspace{2em}
                    \phantomsection\addcontentsline{toc}{section}{\textbf{\@title}}%
                  }}
\makeatother



\NewEnviron{annotate}{\vspace{-.3cm}\small \itshape \BODY \vspace{.3cm}}


%%%% TIKZ STUFF

%% N-GON code
\tikzset{
    pics/tikzngon/.style={
        code={
        \tikzmath{\xx = #1;\rr=1.7;}
        \draw[ultra thick,rounded corners=.05mm] ({\rr*sin(0*360/\xx)},{\rr*cos(0*360/\xx)})
        \foreach \x in {-1,0,...,\xx}
        {
        -- ({\rr*sin(\x*360/\xx)},{\rr*cos(\x*360/\xx)})
        }
           -- cycle;
  }}}

%% N-GON code (even)
\tikzset{
    pics/tikzegon/.style={
        code={
        \tikzmath{\xx = #1;\rr=1.7;}
        \draw[ultra thick,rounded corners=.05mm] ({\rr*sin(0*360/\xx+180/\xx)},{\rr*cos(0*360/\xx+180/\xx)})
        \foreach \x in {-1,0,...,\xx}
           {
           -- ({\rr*sin(\x*360/\xx+180/\xx)},{\rr*cos(\x*360/\xx+180/\xx)}) 
           }
           -- cycle;
  }}}




%% N-CLOCK code
\tikzset{
    pics/tikznclock/.style={
        code={
        \tikzmath{\xx = #1;\rr=1.7;\dd=.4;}
        \foreach \x in {1,...,\xx}
        \pgfmathtruncatemacro{\xy}{\x-1}
           {
             \node[circle,fill=black,inner sep=0pt, minimum size=13pt,text=white]
             at ({(\rr-\dd)*sin((\x-1)*360/(\xx)},{(\rr-\dd)*cos((\x-1)*360/\xx}) {\normalfont\bfseries\sffamily\small {\xy}};
           }
  \draw[thick] (0,0) circle (\rr);
  }}}



%% barcode from
%% https://tex.stackexchange.com/questions/6895/is-there-a-good-latex-package-for-generating-barcodes
%% NOT CURRENTLY USED!


\def\barcode#1#2#3#4#5#6#7{\begingroup%
  \dimen0=0.1em
  \def\stack##1##2{\oalign{##1\cr\hidewidth##2\hidewidth}}%
  \def\0##1{\kern##1\dimen0}%
  \def\1##1{\vrule height10ex width##1\dimen0}%
  \def\L##1{\ifcase##1\bc3211##1\or\bc2221##1\or\bc2122##1\or\bc1411##1%
    \or\bc1132##1\or\bc1231##1\or\bc1114##1\or\bc1312##1\or\bc1213##1%
    \or\bc3112##1\fi}%
  \def\R##1{\bgroup\let\next\1\let\1\0\let\0\next\L##1\egroup}%
  \def\G##1{\bgroup\let\bc\bcg\L##1\egroup}% reverse
  \def\bc##1##2##3##4##5{\stack{\0##1\1##2\0##3\1##4}##5}%
  \def\bcg##1##2##3##4##5{\stack{\0##4\1##3\0##2\1##1}##5}%
  \def\bcR##1##2##3##4##5##6{\R##1\R##2\R##3\R##4\R##5\R##6\11\01\11\09%
    \endgroup}%
  \stack{\09}#1\11\01\11\L#2%
  \ifcase#1\L#3\L#4\L#5\L#6\L#7\or\L#3\G#4\L#5\G#6\G#7%
    \or\L#3\G#4\G#5\L#6\G#7\or\L#3\G#4\G#5\G#6\L#7%
    \or\G#3\L#4\L#5\G#6\G#7\or\G#3\G#4\L#5\L#6\G#7%
    \or\G#3\G#4\G#5\L#6\L#7\or\G#3\L#4\G#5\L#6\G#7%
    \or\G#3\L#4\G#5\G#6\L#7\or\G#3\G#4\L#5\G#6\L#7%
  \fi\01\11\01\11\01\bcR}


\author{Bart Snapp}

\title{Homomorphisms}

\begin{document}
\begin{abstract}
  We introduce homomorphisms and isomorphisms as functions between
  groups.
\end{abstract}
\maketitle

We've remarked before that if you want to understand an object, then
you should try to understand the functions on an object. So now we
turn our study to the functions on groups.


\section{Homomorphisms}


A \textit{homomorphism} is a function between groups that somehow
``preserves'' the structure of the group.


\begin{definition}\index{homomorphism!of groups}\index{group homomorphism}
  Let $(G,\star)$ and $(H,\diamond)$ be groups. A function
  \[
  \phi:G\to H
  \]
  is a \textbf{homomorphism} of groups if
  \[
  \phi(a\star b) = \phi(a)\diamond\phi(b).
  \]
\end{definition}


\begin{example}[$\boldsymbol{\pmb\Z_3\pmb{\to} D_3}$]
  We can define a homomorphism
  \[
  \iota:\Z_3\to D_3
  \]
  by setting
  \[
  \iota(1) = r. 
  \]
  With this single fact, and the property of $\iota$ being a
  homomorphism, we can deduce
  \[
  \iota(2) = r^2\quad\text{and}\quad \iota(0) = e.
  \]
  Moreover, we can literally witness this homomorphism in the Cayley
  tables for the group, as the Cayley table for $\Z_3$ looks like this:
  \[
  \renewcommand{\arraystretch}{1.6}
  \begin{array}{c!{\vline width 2pt}cccccc}
    (\Z_3,+)& 0     & 1\cellcolor{blue!12!white}     & 2\cellcolor{blue!24!white}   \\  \Xhline{2pt}
    0          & 0     & 1\cellcolor{blue!12!white}    & 2\cellcolor{blue!24!white}    \\  
    1\cellcolor{blue!12!white}         & 1\cellcolor{blue!12!white}    & 2\cellcolor{blue!24!white}   & 0   \\  
    2\cellcolor{blue!24!white}        & 2\cellcolor{blue!24!white}   & 0     & 1\cellcolor{blue!12!white}   \\  
  \end{array}
  \]
  And we can see this group table in the upper left-hand corner of this table:
  \[
    \renewcommand{\arraystretch}{1.6}
    \begin{array}{c!{\vline width 2pt}cccccc}
      (D_3,\circ)& e     & r\cellcolor{blue!12!white}     & r^2\cellcolor{blue!24!white}   & f \cellcolor{red!12!white}    & r f \cellcolor{blue!33!red!24!white}  & r^2 f\cellcolor{blue!66!red!24!white} \\  \Xhline{2pt}
      e          & e     & r\cellcolor{blue!12!white}    & r^2\cellcolor{blue!24!white}   & f \cellcolor{red!12!white}    & r f \cellcolor{blue!33!red!24!white}   & r^2f \cellcolor{blue!66!red!24!white} \\  
      r\cellcolor{blue!12!white}         & r\cellcolor{blue!12!white}    & r^2\cellcolor{blue!24!white}   & e     & rf\cellcolor{blue!33!red!24!white} & r^2f \cellcolor{blue!66!red!24!white}    & f \cellcolor{red!12!white}   \\  
      r^2\cellcolor{blue!24!white}        & r^2\cellcolor{blue!24!white}   & e     & r\cellcolor{blue!12!white}    & r^2 f \cellcolor{blue!66!red!24!white}   & f \cellcolor{red!12!white} & rf \cellcolor{blue!33!red!24!white}    \\  
      f \cellcolor{red!12!white}         & f \cellcolor{red!12!white}    & r^2f \cellcolor{blue!66!red!24!white}   & r f \cellcolor{blue!33!red!24!white} & e     & r^2\cellcolor{blue!24!white}    & r\cellcolor{blue!12!white}   \\  
      r f \cellcolor{blue!33!red!24!white}        & rf \cellcolor{blue!33!red!24!white}   & f \cellcolor{red!12!white} & r^2f \cellcolor{blue!66!red!24!white}    & r\cellcolor{blue!12!white}   & e     & r^2\cellcolor{blue!24!white}    \\  
      r^2 f\cellcolor{blue!66!red!24!white}      & r^2 f\cellcolor{blue!66!red!24!white} & rf \cellcolor{blue!33!red!24!white}    & f \cellcolor{red!12!white}   & r^2\cellcolor{blue!24!white}    & r\cellcolor{blue!12!white}   & e     \\  
    \end{array}
    \]
    Hence the homomorphism preserves the structure of the group.
\end{example}


\begin{example}[$\boldsymbol{D_3\pmb{\to} \pmb\Z_3}$]\label{EG:D3-Z3}
  We can define a homomorphism $\pi:D_3\to \Z_3$ by setting
  \[
  \pi(r) = 1\quad\text{and}\quad\pi(f) = 0. 
  \]
  With these two facts, and the property of $\pi$ being a
  homomorphism, we can deduce
  \begin{align*}
    \pi(r^2) &= 2 &  \pi(r^3) &= 0 \\
    \pi(rf)  &=1 & \pi(r^2f) &= 2.
  \end{align*}
 Again, we can see \textit{some} of the structure of the group preserved.
\end{example}


\begin{example}[$\boldsymbol{\pmb{\mathcal{T}}_{12}\pmb{\to} A_4}$]\label{EG:T12-A4}
  The elements of $\mathcal{T}_{12}$ are functions from the
  tetrahedron to the tetrahedron. It is generated by
  \[
\begin{tikzpicture}
  \draw[ultra thick,rounded corners=.05mm] (0,0) -- (3,0) -- (1.5,2.8) -- cycle;
  \draw[ultra thick,rounded corners=.05mm] (3,1) -- (3,0) -- (1.5,2.8) -- cycle;
  \draw[ultra thick,dashed] (0,0) -- (3,1);
  \draw[fill] (3,1) circle (1mm) node[above right] {$1$};
  \draw[fill] (0,0) circle (1mm) node[left] {$2$};
  \draw[fill] (3,0) circle (1mm) node[right] {$3$};
  \draw[fill] (1.5,2.8) circle (1mm) node[above] {$4$};

  \draw[|->] (3,2) -- (5,2);
  \node[above] at (4,2) {$r$};
  
  \draw[ultra thick,rounded corners=.05mm] (5,0) -- (8,0) -- (6.5,2.8) -- cycle;
  \draw[ultra thick,rounded corners=.05mm] (8,1) -- (8,0) -- (6.5,2.8) -- cycle;
  \draw[ultra thick,dashed] (5,0) -- (8,1);
  \draw[fill] (8,1) circle (1mm) node[above right] {$3$};
  \draw[fill] (5,0) circle (1mm) node[left] {$1$};
  \draw[fill] (8,0) circle (1mm) node[right] {$2$};
  \draw[fill] (6.5,2.8) circle (1mm) node[above] {$4$};
\end{tikzpicture}
\]
and
\[
\begin{tikzpicture}
  \draw[ultra thick,rounded corners=.05mm] (0,0) -- (3,0) -- (1.5,2.8) -- cycle;
  \draw[ultra thick,rounded corners=.05mm] (3,1) -- (3,0) -- (1.5,2.8) -- cycle;
  \draw[ultra thick,dashed] (0,0) -- (3,1);
  \draw[fill] (3,1) circle (1mm) node[above right] {$1$};
  \draw[fill] (0,0) circle (1mm) node[left] {$2$};
  \draw[fill] (3,0) circle (1mm) node[right] {$3$};
  \draw[fill] (1.5,2.8) circle (1mm) node[above] {$4$};

  \draw[|->] (3,2) -- (5,2);
  \node[above] at (4,2) {$p$};
  
  \draw[ultra thick,rounded corners=.05mm] (5,0) -- (8,0) -- (6.5,2.8) -- cycle;
  \draw[ultra thick,rounded corners=.05mm] (8,1) -- (8,0) -- (6.5,2.8) -- cycle;
  \draw[ultra thick,dashed] (5,0) -- (8,1);
  \draw[fill] (8,1) circle (1mm) node[above right] {$1$};
  \draw[fill] (5,0) circle (1mm) node[left] {$4$};
  \draw[fill] (8,0) circle (1mm) node[right] {$2$};
  \draw[fill] (6.5,2.8) circle (1mm) node[above] {$3$};
\end{tikzpicture}
\]
However, both $r$ and $p$ correspond canonically to elements of $A_4$, namely
\[
r \leftrightarrow
\left(\begin{smallmatrix}
  1 & 2 & 3 & 4\\
  2 & 3 & 1 & 4
\end{smallmatrix}\right) = (2 \ 3) (1 \ 3)
\]
and
\[
p \leftrightarrow
\left(\begin{smallmatrix}
  1 & 2 & 3 & 4\\
  1 & 3 & 4 & 2
\end{smallmatrix}\right) = (3 \ 4) (2 \ 4).
\]
Thus
\begin{align*}
  \phi:\mathcal{T}_{12}&\to A_4\\
  r &\mapsto (2 \ 3) (1 \ 3)\\
  p &\mapsto (3 \ 4) (2 \ 4)
\end{align*}
is a homomorphism. Compare this with Example~\ref{EG:T12} and
Example~\ref{EG:A4}.
\end{example}




Believe it or not, you already know some homomorphisms! They appear
all over mathematics.

\begin{example}[Exponential functions]\label{EG:exp}\index{exponential function}
  Consider the following
  \begin{align*}
    \exp:(\R,+) &\to (\R_{> 0},\cdot) \\
    x &\mapsto e^x.
  \end{align*}
  The function $\exp$ is a homomorphism from the group $(\R,+)$ to
  $(\R_{> 0},\cdot)$ as
  \begin{align*}
    \exp(x+y) &= e^{x+y} \\
    &= e^x \cdot e^y\\
    &= \exp(x)\cdot \exp(y).
  \end{align*}
\end{example}



\begin{example}[Logarithmic functions]\label{EG:log}\index{logarithm}
  Consider the following
  \begin{align*}
    \log:(\R_{> 0},\cdot) &\to (\R,+)\\
    x &\mapsto \log(x).
  \end{align*}
  The function $\log$ is a homomorphism from the group $(\R_{> 0},\cdot)$ to $(\R,+)$ as
  \[
  \log(x\cdot y) = \log(x) + \log(y).
  \]
\end{example}


\begin{example}[Derivatives]\label{EG:der}\index{derivative}
  Recall that $C^\infty(\R,\R)$ is the additive group of functions
  from $\R$ to $\R$ whose derivatives are all continuous. Consider the
  following
  \begin{align*}
    \frac{d}{dx}:C^\infty(\R,\R) &\to C^\infty(\R,\R) \\
      f &\mapsto f'.
  \end{align*}
  The function $\frac{d}{dx}$ is a homomorphism as
  \begin{align*}
    \frac{d}{dx}(f+ g) &= f' + g' \\
    &= \frac{d}{dx}f + \frac{d}{dx}g.
  \end{align*}
\end{example}


\begin{example}[The gradient]\label{EG:grad}\index{gradient}
  Now recall that $C^\infty(\R,\R^n)$ is the set of functions from
  $\R$ to $\R^n$ whose derivatives are all continuous. Consider the
  gradient
  \begin{align*}
    \grad:C^\infty(\R,\R^n) &\to C^\infty(\R^n,\R^n)\\
    F &\mapsto \grad F.
  \end{align*}
  The function $\grad$ is a homomorphism as
  \begin{align*}
    \grad(F + G) = \grad F + \grad G.
  \end{align*}
\end{example}


\begin{example}[The homothety map]\label{EG:he}\index{homothety map}
  Let $(G,+)$ be an Abelian group and $a\in\Z_{\ge 0}$ and consider the
  following
  \begin{align*}
    \eta: G &\to G\\
    x &\mapsto \underbrace{x+ x+ \dots+ x}_{\text{$a$ copies of $x$}}.
  \end{align*}
  The function $\eta$ is a homomorphism as
  \begin{align*}
    \eta(x +  y) &= \underbrace{(x+  y) + \cdots + (x+  y)}_{\text{$a$ copies of $(x+ y)$}} \\
    &= \eta(x) +  \eta(y).
  \end{align*}
  This can be rigorously proved through induction.
\end{example}



\begin{example}[Linear transformations]\index{linear transformation}
  Let $L:\R^n\to \R^n$ be a linear transformation. In this case $L$ is
  a homomorphism of additive groups, $(\R^n,+)$.
\end{example}


\begin{exercise}\index{complex numbers}
  The set of complex numbers are defined as follows
  \[
  \C = \{ a + b i: \text{$a,b\in\R$ and $i^2 = -1$}\}.
  \]
  Given a complex number $z\in\C$, the \dfn{complex conjugate}
  $\bar{z}\in\C$ is defined as
  \begin{align*}
    z &= a+bi\\
    \bar{z} &= a- bi.
  \end{align*}
  Prove that
  \begin{align*}
    \phi:\C &\to\C\\
    z &\mapsto \bar{z}
  \end{align*}
  is a homomorphism of groups.  Additionally, prove that complex
  conjugation is \textbf{not} a linear transformation.
\end{exercise}




\begin{example}[Determinants]\label{EG:det}\index{determinant}
  Consider $GL(n)$, the set of invertible $n\times n$ matrices with
  entries in $\R$. Now consider the function
  \begin{align*}
    \det: GL(n) &\to \R^\times \\
             M  &\mapsto  \det(M).
  \end{align*}
  The function $\det$ is a homomorphism as
  \[
  \det(M \cdot N) = \det(M) \cdot \det(N).
  \]
\end{example}


\begin{exercise}\index{trace}
  Consider the set $n\times n$ matrices with entries in $\R$ under the
  operation of componentwise addition, we'll denote this group $\mat(n)$. Now consider the function
  \begin{align*}
    \tr:\mat(n) &\to \R\\
    M &\mapsto \tr(M),
  \end{align*}
  where $\tr(M)$ is defined to be the sum of the diagonal entries of
  $M$. Use induction to prove that $\tr$ is a homomorphism of groups.
\end{exercise}


\begin{lemma}[Homomorphisms preserve identity]\label{L:HPId}
  If $\phi:G\to H$ is a homomorphism of groups, then
  \[
  \phi(e_G) = e_H
  \]
  where $e_G$ is the identity of $G$ and $e_H$ is the identity of
  $H$.
  \begin{sketch}
    Use the fact that identities are unique.
  \end{sketch}
\end{lemma}

\begin{example}[Homomorphisms preserve identity]
  We will demonstrate the lemma above for homomorphisms in the
  examples above:
  \begin{description}
    \begin{multicols}{2}
    \item[$(\ref{EG:D3-Z3})$] $\pi(e) = 0$
    \item[$(\ref{EG:T12-A4})$] $\phi(e_{\mathcal{T}_{12}}) = e_{A_4}$
    \item[$(\ref{EG:exp})$] $\exp(0) = 1$
    \item[$(\ref{EG:log})$] $\log(1) = 0$
    \item[$(\ref{EG:der})$] $\frac{d}{dx}0 = 0$
    \item[$(\ref{EG:grad})$] $\grad 0 = \vec{0}$
    \item[$(\ref{EG:he})$] $\eta(0) = 0$
    \item[$(\ref{EG:det})$] $\det(I) = 1$
    \end{multicols}
  \end{description}
\end{example}




\begin{lemma}[Homomorphisms preserve inverses]\label{L:HPI}
  If $\phi:G\to H$ is a group homomorphism, then $\phi(g^{-1}) =
  \phi(g)^{-1}$.
  \begin{sketch}
    Use the definition of an inverse and the definition of a
    homomorphism.
  \end{sketch}
\end{lemma}

\begin{example}[Homomorphisms preserve inverses]
  We will demonstrate the lemma above for homomorphisms in the
  examples above:
  \begin{description}
    \begin{multicols}{2}
    \item[$(\ref{EG:D3-Z3})$] $\pi(r^{-1}) = 2$
    \item[$(\ref{EG:T12-A4})$] $\phi(p^{-1}) = (4\ 3)(2 \ 3)$
    \item[$(\ref{EG:exp})$] $\exp(-5) = 1/\exp(5)$
    \item[$(\ref{EG:log})$] $\log(1/5) = -\log(5)$
    \item[$(\ref{EG:der})$] $\frac{d}{dx}(-f) = -\frac{d}{dx}f$
    \item[$(\ref{EG:grad})$] $\grad( -F) = -\grad F$
    \item[$(\ref{EG:he})$] $\eta(-5) = -\eta( 5)$
    \item[$(\ref{EG:det})$] $\det(M^{-1}) = 1/\det(M)$
    \end{multicols}
  \end{description}
\end{example}





\begin{exercise}
  Prove that a group $G$ is Abelian if and only if the mapping
  \begin{align*}
    \phi: G &\to G\\
    g &\mapsto g^2
  \end{align*}
  is a homomorphism.
\end{exercise}


\begin{exercise}
  Prove that a group $G$ is Abelian if and only if the mapping
  \begin{align*}
    \phi: G &\to G\\
    g &\mapsto g^{-1}
  \end{align*}
  is a homomorphism.
\end{exercise}

\begin{exercise}
  Prove that a group $G$ is Abelian if and only if the mapping
  \begin{align*}
    \phi: G\times G &\to G\\
    (a,b) &\mapsto ab
  \end{align*}
  is a homomorphism.
\end{exercise}


\begin{exercise}
  Determine the number of distinct homomorphisms $\phi:\Z_m\to \Z_n$ in
  terms of $m$ and $n$. Be sure to prove that your answer is correct.
\end{exercise}

\begin{exercise}\index{Frobenius map}
  Let $p$ be a prime and consider the group of polynomials in $x$ with
  coefficients in $\Z_p$ under addition. This group is denoted by
  $\Z_p[x]$. Let $\ph:\Z_p[x]\to\Z_p[x]$ be the map given by
  $\phi(f)=f^p$. Prove that $\phi$ is a homomorphism. This map is
  called the \textbf{Frobenius map}.
\end{exercise}




\section{Subgroups, kernels and images}


Now we will investigate how homomorphisms act on subgroups. There will
also be two subgroups of particular interest, the kernel of a
homomorphism and the image of a homomorphism.



\subsection{Subgroups}

Homomorphisms map subgroups to subgroups.


\begin{lemma}[Homomorphisms preserve subgroups]\label{L:hps}
  Consider a homomorphism of groups $\phi:G\to H$, and let $S\subgp
  G$. Define
  \[
  \phi(S) := \{\phi(g): g\in S\}.
  \]
  In this case, $\phi(S) \subgp H$.
  \begin{sketch}
    Use the subgroup criterion, Lemma~\ref{T:sc}.
  \end{sketch}
\end{lemma}


\begin{example}[Homomorphisms preserve subgroups]
  Consider $\exp:(\R,+)\to(\R_{>0},\cdot)$. In this case,
  $\Z\subgp\R$ and
  \[
  \exp(\Z) = \{\dots, e^{-3},e^{-2},e^{-1},e^0,e^1,e^2,e^3,\dots\}.
  \]
  We leave it to the reader to show $\exp(\Z)\subgp \R_{>0}$.

  

  Also consider $\det:GL(n)\to \R^\times$. In this case, $O(n) \subgp
  GL(n)$ and
  \[
  \det(O(n)) = \{-1,1\}\subgp\R^\times.
  \]
\end{example}




\begin{lemma}[Preimages preserve subgroups]\label{L:pps}
  Consider a homomorphism of groups $\phi:G\to H$, and let $T\subgp
  H$. Define
  \[
  \phi^{-1}(T) := \{g\in G: \phi(g)\in T\}.
  \]
  In this case, $\phi^{-1}(T) \subgp G$.
  \begin{sketch}
    Use the subgroup criterion, Lemma~\ref{T:sc}.
  \end{sketch}
\end{lemma}

\begin{example}[Preimages preserve subgroups]
  Consider $\exp:(\R,+)\to(\R_{>0},\cdot)$. In this case,
  $S=\{e^n:n\in\Z\}\subgp\R$ and
  \[
  \exp^{-1}(S) = \Z\subgp\R.
  \]

  

  Also consider $\det:GL(n)\to \R^\times$. In this case,
  $S=\{-1,1\}\subgp \R^\times$ and
  \[
  {\det}^{-1}(S) = O(n).
  \]
\end{example}



\subsection{Kernels and images}


\begin{definition}\index{kerphi@$\ker(\phi)$}\index{kernel!group homomorphism}
  Let $\phi:G\to H$ be a group homomorphism. The \index{kernel} of
  $\phi$,
  \[
  \ker(\phi) := \{g\in G: \phi(g) = e_H\}\subset G.
  \]
  is the subset of $G$ that $\phi$ maps to the identity $e_H\in H$.
\end{definition}

\begin{example}[Kernel of the homothety map]\label{E:K1}
  Consider the homomorphism
  \begin{align*}
    \eta:\Z_{12} &\to \Z_{12}\\
    x &\mapsto 3x.
  \end{align*}
  In this case $\ker(\eta) = \{0,4,8\}\subset \Z_{12}$.
\end{example}

\begin{example}[Kernel of a surjection]\label{E:K2} 
  Consider the homomorphism
  \begin{align*}
    \pi:\Z_{12} &\to \Z_{3}\\
    n &\mapsto m & &\text{if $n\equiv m \pmod{3}$}.
  \end{align*}
  In this case $\ker(\pi) = \{0,4,8\}\subset \Z_{12}$.
\end{example}

\begin{exercise}
  Why are the kernels in Example~\ref{E:K1} and Example~\ref{E:K2} the
  same?
\end{exercise}



\begin{lemma}[Kernels are normal subgroups]\label{L:kerN}
  If $\phi:G\to H$ is a group homomorphism, then $\ker(\phi)\normal G$.
  \begin{sketch}
    Use the normal subgroup criterion, Lemma~\ref{L:nsc}.
  \end{sketch}
\end{lemma}



\begin{lemma}[Kernels and injectivity]\label{L:kerinj}
  A homomorphism of groups $\phi:G\to H$ is injective if and only if
  $\ker(\phi) = \{e_G\}$.
  \begin{proof}
    $(\Rightarrow)$ By Lemma~\ref{L:HPId}, $\phi(e_G) = e_H$. If
    $\phi$ is injective, only $e_G$ can map to $e_H$. Hence
    $\ker(\phi) = \{e_G\}$.


    $(\Leftarrow)$ Suppose that $\ker(\phi) = \{e_G\}$. We must show
    that $\phi$ is injective. Seeking a contradiction, suppose that
    there exist $a,b\in G$ with $a\ne b$ such that
    \[
    \phi(a) = \phi(b).
    \]
    Write with me
    \begin{align*}
      \phi(ab^{-1}) &=\phi(a)\phi(b^{-1})\\
      &= \phi(a) \phi(b)^{-1} & \text{(By Lemma~\ref{L:HPI})}\\
      &= e_H.
    \end{align*}
    Hence $ab^{-1}\in\ker(\phi)$ but $ab^{-1}\ne e_G$, a
    contradiction.
  \end{proof}
\end{lemma}


\begin{definition}\index{imphi@$\im(\phi)$}\index{image!group homomorphism}
  Let $\phi:G\to H$ be a group homomorphism. The \textbf{image} of
  $\phi$,
  \[
  \im(\phi) := \{\phi(g): g\in G\}\subset H.
  \]
  is the subset of $H$ containing all the values of $\phi$.
\end{definition}

\begin{example}[Image of the homothety map]
  Consider the homomorphism
  \begin{align*}
    \eta:\Z_{12} &\to \Z_{12}\\
    x &\mapsto 3x.
  \end{align*}
  In this case $\im(\eta) = \{0,3,6,9\}\subset \Z_{12}$.
\end{example}

\begin{example}[Image of an injection]\label{E:I1} 
  Consider the homomorphism
  \begin{align*}
    \iota:\Z_{3} &\to \Z_{12}\\
    x &\mapsto 4x.
  \end{align*}
  In this case $\im(\iota) = \{0,4,8\}\subset \Z_{12}$.
\end{example}


\begin{lemma}[Images and surjectivity]
    A homomorphism of groups $\phi:G\to H$ is surjective if and only
    if $\im(\phi) = H$.
    \begin{proof}
      $(\Rightarrow)$ Follows from the definition.

      $(\Leftarrow)$ Follows from the definition.
    \end{proof}
\end{lemma}


\begin{exercise}
  Consider the group $(\R^\times,\cdot)$. Prove that
  the map
  \begin{align*}
    \phi:\R^\times &\to \R^\times\\
    x &\mapsto |x|
  \end{align*}
  is a homomorphism. Explicitly identify $\ker(\phi)$ and
  $\im(\phi)$.
\end{exercise}


\begin{exercise}
  Consider the group $(\Z,+)$. Prove that
  the map
  \begin{align*}
    \phi:\Z &\to \Z_n\\
    x &\mapsto x\pmod{n}
  \end{align*}
  is a homomorphism. Explicitly identify $\ker(\phi)$ and
  $\im(\phi)$.
\end{exercise}


\begin{exercise}
  Consider the group $(\Z_{12},+)$. Prove that the map
  \begin{align*}
    \phi:\Z_{12} &\to \Z_{12}\\
    x &\mapsto 3\cdot x
  \end{align*}
  is a homomorphism. Explicitly identify $\ker(\phi)$ and
  $\im(\phi)$.
\end{exercise}



\begin{exercise}
  Consider the group $(\R^\times,\cdot)$ and let $n\in \N$. Prove that
  the map
  \begin{align*}
    \phi:\R^\times &\to \R^\times\\
    x &\mapsto x^n
  \end{align*}
  is a homomorphism. Explicitly identify $\ker(\phi)$ and
  $\im(\phi)$.
\end{exercise}

\begin{exercise}
  Consider the group $GL(2)$. Prove that the map
  \begin{align*}
    \phi:GL(2) &\to \R^\times\\
    M &\mapsto \det(M)
  \end{align*}
  is a homomorphism. Explicitly identify $\ker(\phi)$ and
  $\im(\phi)$.
\end{exercise}



\begin{exercise}
  Let $\R[x]$ denote the group of all polynomials in one variable with
  real coefficients under addition. Meaning
  \[
  f = a_n x^{n} + a_{n-1}x^{n-1} + \cdots + a_1x + a_0
  \]
  where $n\in\N\cup\{0\}$ and each $a_i\in\R$.  For each $f$ in $\R[x]$,
  let $f'$ denote the derivative of $f$. Show that the map
  \begin{align*}
    \frac{d}{dx}:\R[x]&\to\R[x]\\
    f  &\mapsto f'
  \end{align*}
  is a homomorphism of additive groups. Explicitly identify
  $\ker\left(\frac{d}{dx}\right)$ and $\im\left(\frac{d}{dx}\right)$.
\end{exercise}



\begin{exercise}
  Determine all homomorphisms from $\Z_{12}$ to $\Z_{30}$.
  \begin{hint}
    $\Z_{12}$ is cyclic, hence all homomorphisms are determined by
    where they map the generator.
  \end{hint}
\end{exercise}



\begin{theorem}[Canonical surjection]
  Let $G$ be a group and $N$ be a normal subgroup of $G$. Prove that
  \begin{align*}
  \pi:G &\to G/N\\
  g &\mapsto gN
  \end{align*}
  is a surjective homomorphism. This surjection is called the
  \dfn{canonical surjection} or \dfn{natural surjection} from $G$ to
  $G/N$.
  \begin{sketch}
    Just check that this is a surjective homomorphism.
  \end{sketch}
\end{theorem}



\begin{example}[Kernel and image of a natural surjection]
    Consider the natural surjection
  \begin{align*}
    \eta:\Z &\to \Z/3\Z\\
    x &\mapsto x+3\Z.
  \end{align*}
  In this case $\ker(\eta) = 3\Z \subset \Z$ and $\im(\eta) = \Z/3\Z$.
\end{example}


\begin{corollary}[Normal subgroups are kernels]
  Given a group $G$, every normal subgroup of $G$ is the kernel of
  some canonical surjection $\pi:G\to G/N$.
\end{corollary}




\section{Isomorphisms}

At this point we are ready to formalize our definition of an
\textit{isomorphism}.

\begin{definition}
  Let $\phi:G\to H$ be a homomorphism of groups. The map $\phi$ is an
  \dfn{isomorphism} if $\phi$ is bijective.


  Two groups $G$ and $H$ are said to be \textbf{isomorphic groups}\index{isomorphic!groups} if
  there exists an isomorphism mapping $G$ to $H$, or $H$ to $G$.  In
  this case, we write $G\iso H$.\index{G=H@$G\iso H$}
\end{definition}


This new definition says that if two finite groups are isomorphic,
then there Cayley tables for the groups that ``look the same.''

\begin{exercise}
  Prove that $\exp$ is an isomorphism from $(\R,+)$ to $(\R_{>
    0},\cdot)$.
\end{exercise}


\begin{exercise}
  Let $G = (-1,1)\subset \R$ be a group with the operation
  \[
  x\cdot y := \frac{x+y}{1+xy}.
  \]
  Prove that $G$ is isomorphic to $(\R,+)$.

  \begin{hint}
    Check out the function $\tanh: \R\to G$.
  \end{hint}
\end{exercise}

\begin{exercise}
  Prove that $\Z_n$ is isomorphic to $\Z/n\Z$.
\end{exercise}


\begin{exercise}
  Is $\Z_3\times \Z_9$ isomorphic to $\Z_{27}$? Prove that your answer
  is correct.
\end{exercise}

\begin{exercise}
  Is $\Z_3\times \Z_5$ isomorphic to $\Z_{15}$? Prove that your answer
  is correct.
\end{exercise}

\begin{exercise}
  Let $\phi:\Z_{16}\to\Z_{17}^\times$ be given by
  \[
  \phi(k)= 3^k \pmod{17}.
  \]
  Show that $\phi$ is an isomorphism. Note, one way of proving that
  $\phi$ is bijective would be to find the inverse of $\phi$. This is
  quite difficult in general and is known as the \dfn{discrete log
    problem}.
\end{exercise}



%% \section{Isomorphisms revisited}

%% \begin{exercise}
%%   Prove that addition is an associative operation  over $\Z$.
%% \end{exercise}

%% \begin{exercise}
%%   Prove that multiplication is an associative operation  over $\Z$.
%% \end{exercise}





\end{document}
