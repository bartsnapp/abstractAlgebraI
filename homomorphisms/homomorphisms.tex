\documentclass{ximera}

\usepackage[T1]{fontenc}
\usepackage{stix2}
\usepackage{gillius}
\usepackage{resizegather}
%\usepackage{rsfso} fancy cal
\DeclareMathAlphabet{\mathcal}{OMS}{cmsy}{m}{n} %less fancy cal


\usepackage{multicol}


\usepackage{tikz-cd}
\tikzset{>=stealth}
\tikzcdset{arrow style=tikz}
\usetikzlibrary{math} %% for assigning variables
%\usetikzlibrary{fadings}

\usepackage{colortbl,boldline,makecell} %% group tables


\usepackage[sans]{dsfont}

\usepackage{stmaryrd,pifont}


\let\oldbibliography\thebibliography%% to compact bib
\renewcommand{\thebibliography}[1]{%
  \oldbibliography{#1}%
  \setlength{\itemsep}{0pt}%
}
\renewcommand\refname{} %% no name needed!


\DefineVerbatimEnvironment{macaulay2}{Verbatim}{numbers=left,frame=lines,label=Macaulay2,labelposition=topline}

\DefineVerbatimEnvironment{gap}{Verbatim}{numbers=left,frame=lines,label=GAP,labelposition=topline}

%%% This next bit of code defines all our theorem environments
\makeatletter
\let\c@theorem\relax
\let\c@corollary\relax
\let\c@example\relax
\makeatother

\let\definition\relax
\let\enddefinition\relax

\let\theorem\relax
\let\endtheorem\relax

\let\proposition\relax
\let\endproposition\relax

\let\exercise\relax
\let\endexercise\relax

\let\question\relax
\let\endquestion\relax

\let\remark\relax
\let\endremark\relax

\let\corollary\relax
\let\endcorollary\relax


\let\example\relax
\let\endexample\relax

\let\warning\relax
\let\endwarning\relax

\let\lemma\relax
\let\endlemma\relax


\let\algorithm\relax
\let\endalgorithm\relax
\usepackage{algpseudocode}

\newtheoremstyle{SlantTheorem}{\topsep}{\topsep}%%% space between body and thm
		{\slshape}                      %%% Thm body font
		{}                              %%% Indent amount (empty = no indent)
		{\bfseries\sffamily}            %%% Thm head font
		{}                              %%% Punctuation after thm head
		{3ex}                           %%% Space after thm head
		{\thmname{#1}\thmnumber{ #2}\thmnote{ \bfseries(#3)}}%%% Thm head spec
\theoremstyle{SlantTheorem}
\newtheorem{theorem}{Theorem}
%\newtheorem{definition}[theorem]{Definition}
%\newtheorem{proposition}[theorem]{Proposition}
%% \newtheorem*{dfnn}{Definition}
%% \newtheorem{ques}{Question}[theorem]
%% \newtheorem*{war}{WARNING}
%% \newtheorem*{cor}{Corollary}
%% \newtheorem*{eg}{Example}
\newtheorem*{remark}{Remark}
\newtheorem*{touchstone}{Touchstone}
\newtheorem{corollary}{Corollary}[theorem]
\newtheorem*{warning}{WARNING}
\newtheorem{example}{Example}[theorem]
\newtheorem{lemma}[theorem]{Lemma}




\newtheoremstyle{Definition}{\topsep}{\topsep}%%% space between body and thm
		{}                              %%% Thm body font
		{}                              %%% Indent amount (empty = no indent)
		{\bfseries\sffamily}            %%% Thm head font
		{}                              %%% Punctuation after thm head
		{3ex}                           %%% Space after thm head
		{\thmname{#1}\thmnumber{ #2}\thmnote{ \bfseries(#3)}}%%% Thm head spec
\theoremstyle{Definition}
\newtheorem{definition}[theorem]{Definition}



\let\algorithm\relax
\let\endalgorithm\relax
\newtheoremstyle{Alg}{\topsep}{\topsep}%%% space between body and thm
		{}                              %%% Thm body font
		{}                              %%% Indent amount (empty = no indent)
		{\bfseries\sffamily}            %%% Thm head font
		{}                              %%% Punctuation after thm head
		{3ex}                           %%% Space after thm head
		{\thmname{#1}\thmnumber{ #2}\thmnote{ \bfseries(#3)}}%%% Thm head spec
\theoremstyle{Alg}
\newtheorem*{algorithm}{Algorithm}




\newtheoremstyle{Exercise}{\topsep}{\topsep} %%% space between body and thm
		{}                           %%% Thm body font
		{}                           %%% Indent amount (empty = no indent)
		{\bfseries\sffamily}         %%% Thm head font
		{)}                          %%% Punctuation after thm head
		{ }                          %%% Space after thm head
		{\thmnumber{#2}\thmnote{ \bfseries(#3)}}%%% Thm head spec
\theoremstyle{Exercise}
\newtheorem{exercise}{}[theorem]

%% \newtheoremstyle{Question}{\topsep}{\topsep} %%% space between body and thm
%% 		{\bfseries}                  %%% Thm body font
%% 		{3ex}                        %%% Indent amount (empty = no indent)
%% 		{}                           %%% Thm head font
%% 		{}                           %%% Punctuation after thm head
%% 		{}                           %%% Space after thm head
%% 		{\thmnumber{#2}\thmnote{ \bfseries(#3)}}%%% Thm head spec
\newtheoremstyle{Question}{3em}{3em} %%% space between body and thm
		{\large\bfseries}                           %%% Thm body font
		{}                           %%% Indent amount (empty = no indent)
		{}                         %%% Thm head font
		{}                          %%% Punctuation after thm head
		{0em}                          %%% Space after thm head
		{}%%% Thm head spec
\theoremstyle{Question}
\newtheorem*{question}{}






\renewcommand{\tilde}{\widetilde}
\renewcommand{\bar}{\overline}
\renewcommand{\hat}{\widehat}
\newcommand{\N}{\mathbb N}
\newcommand{\Z}{\mathbb Z}
\newcommand{\R}{\mathbb R}
\newcommand{\Q}{\mathbb Q}
\newcommand{\C}{\mathbb C}
\newcommand{\V}{\mathbb V}
\newcommand{\I}{\mathbb I}
\newcommand{\A}{\mathbb A}
\renewcommand{\o}{\mathbf o}
\newcommand{\iso}{\simeq}
\newcommand{\ph}{\varphi}
\newcommand{\Cf}{\mathcal{C}}
\newcommand{\IZ}{\mathrm{Int}(\Z)}
\newcommand{\dsum}{\oplus}
\newcommand{\directsum}{\bigoplus}
\newcommand{\union}{\bigcup}
\newcommand{\subgp}{\leq}
\newcommand{\normal}{\trianglelefteq}
\renewcommand{\i}{\mathfrak}
\renewcommand{\a}{\mathfrak{a}}
\renewcommand{\b}{\mathfrak{b}}
\newcommand{\m}{\mathfrak{m}}
\newcommand{\p}{\mathfrak{p}}
\newcommand{\q}{\mathfrak{q}}
\newcommand{\dfn}[1]{\textbf{#1}\index{#1}}
\let\hom\relax
\DeclareMathOperator{\ann}{Ann}
\DeclareMathOperator{\h}{ht}
\DeclareMathOperator{\hom}{Hom}
\DeclareMathOperator{\Span}{Span}
\DeclareMathOperator{\spec}{Spec}
\DeclareMathOperator{\maxspec}{MaxSpec}
\DeclareMathOperator{\aut}{Aut}
\DeclareMathOperator{\ass}{Ass}
\DeclareMathOperator{\lcm}{lcm}
\DeclareMathOperator{\ff}{Frac}
\DeclareMathOperator{\im}{Im}
\DeclareMathOperator{\syz}{Syz}
\DeclareMathOperator{\gr}{Gr}
\DeclareMathOperator{\multideg}{multideg}
\renewcommand{\ker}{\mathop{\mathrm{Ker}}\nolimits}
\newcommand{\coker}{\mathop{\mathrm{Coker}}\nolimits}
\newcommand{\lps}{[\hspace{-0.25ex}[}
\newcommand{\rps}{]\hspace{-0.25ex}]}
\newcommand{\into}{\hookrightarrow}
\newcommand{\onto}{\twoheadrightarrow}
\newcommand{\tensor}{\otimes}
\newcommand{\x}{\mathbf{x}}
\newcommand{\X}{\mathbf X}
\newcommand{\Y}{\mathbf Y}
\renewcommand{\k}{\boldsymbol{\kappa}}
\renewcommand{\emptyset}{\varnothing}
\renewcommand{\qedsymbol}{$\blacksquare$}
\renewcommand{\l}{\ell}
\newcommand{\1}{\mathds{1}}
\newcommand{\lto}{\mathop{\longrightarrow\,}\limits}
\newcommand{\rad}{\sqrt}
\newcommand{\hf}{H}
\newcommand{\hs}{H\!S}
\newcommand{\hp}{H\!P}
\renewcommand{\vec}{\mathbf}
\let\temp\phi
\let\phi\varphi
\let\eulerphi\temp


\renewcommand{\epsilon}{\varepsilon}
\renewcommand{\subset}{\subseteq}
\renewcommand{\supset}{\supseteq}
\newcommand{\macaulay}{\normalfont\textsl{Macaulay2}}
\newcommand{\GAP}{\normalfont\textsf{GAP}}
\newcommand{\invlim}{\varprojlim}
\renewcommand{\le}{\leqslant}
\renewcommand{\ge}{\geqslant}
\newcommand{\valpha}{{\boldsymbol\alpha}}
\newcommand{\vbeta}{{\boldsymbol\beta}}
\newcommand{\vgamma}{{\boldsymbol\gamma}}
\newcommand{\dotp}{\bullet}
\newcommand{\lc}{\normalfont\textsc{lc}}
\newcommand{\lt}{\normalfont\textsc{lt}}
\newcommand{\lm}{\normalfont\textsc{lm}}
\newcommand{\from}{\leftarrow}
\newcommand{\transpose}{\intercal}
\newcommand{\grad}{\boldsymbol\nabla}
\newcommand{\curl}{\boldsymbol{\nabla\times}}
\renewcommand{\d}{\, d}
\newcommand{\<}{\langle}
\renewcommand{\>}{\rangle}

%\renewcommand{\proofname}{Sketch of Proof}


\renewenvironment{proof}[1][Proof]
  {\begin{trivlist}\item[\hskip \labelsep \itshape \bfseries #1{}\hspace{2ex}]\upshape}
{\qed\end{trivlist}}

\newenvironment{sketch}[1][Sketch of Proof]
  {\begin{trivlist}\item[\hskip \labelsep \itshape \bfseries #1{}\hspace{2ex}]\upshape}
{\qed\end{trivlist}}



\makeatletter
\renewcommand\section{\@startsection{paragraph}{10}{\z@}%
                                     {-3.25ex\@plus -1ex \@minus -.2ex}%
                                     {1.5ex \@plus .2ex}%
                                     {\normalfont\large\sffamily\bfseries}}
\renewcommand\subsection{\@startsection{subparagraph}{10}{\z@}%
                                    {3.25ex \@plus1ex \@minus.2ex}%
                                    {-1em}%
                                    {\normalfont\normalsize\sffamily\bfseries}}
\makeatother

%% Fix weird index/bib issue.
\makeatletter
\gdef\ttl@savemark{\sectionmark{}}
\makeatother


\makeatletter
%% no number for refs
\newcommand\frontstyle{%
  \def\activitystyle{activity-chapter}
  \def\maketitle{%
    \addtocounter{titlenumber}{1}%
                    {\flushleft\small\sffamily\bfseries\@pretitle\par\vspace{-1.5em}}%
                    {\flushleft\LARGE\sffamily\bfseries\@title \par }%
                    {\vskip .6em\noindent\textit\theabstract\setcounter{problem}{0}\setcounter{sectiontitlenumber}{0}}%
                    \par\vspace{2em}
                    \phantomsection\addcontentsline{toc}{section}{\textbf{\@title}}%
                  }}
\makeatother



\NewEnviron{annotate}{\vspace{-.3cm}\small \itshape \BODY \vspace{.3cm}}


%%%% TIKZ STUFF

%% N-GON code
\tikzset{
    pics/tikzngon/.style={
        code={
        \tikzmath{\xx = #1;\rr=1.7;}
        \draw[ultra thick,rounded corners=.05mm] ({\rr*sin(0*360/\xx)},{\rr*cos(0*360/\xx)})
        \foreach \x in {0,1,...,\xx+1}
           {
           -- ({\rr*sin(\x*360/\xx)},{\rr*cos(\x*360/\xx)}) 
           }
           -- cycle;
  }}}

%% N-GON code (even)
\tikzset{
    pics/tikzegon/.style={
        code={
        \tikzmath{\xx = #1;\rr=1.7;}
        \draw[ultra thick,rounded corners=.05mm] ({\rr*sin(0*360/\xx+180/\xx)},{\rr*cos(0*360/\xx+180/\xx)})
        \foreach \x in {0,1,...,\xx+1}
           {
           -- ({\rr*sin(\x*360/\xx+180/\xx)},{\rr*cos(\x*360/\xx+180/\xx)}) 
           }
           -- cycle;
  }}}




%% N-CLOCK code
\tikzset{
    pics/tikznclock/.style={
        code={
        \tikzmath{\xx = #1;\rr=1.7;\dd=.4;}
        \foreach \x in {0,1,...,\xx-1}
           {
             \node[circle,fill=black,inner sep=0pt, minimum size=13pt,text=white]
             at ({(\rr-\dd)*sin(\x*360/\xx)},{(\rr-\dd)*cos(\x*360/\xx}) {\normalfont\bfseries\sffamily\small \x};
           }
  \draw[thick] (0,0) circle (\rr);
  }}}



%% barcode from
%% https://tex.stackexchange.com/questions/6895/is-there-a-good-latex-package-for-generating-barcodes
%% NOT CURRENTLY USED!


\def\barcode#1#2#3#4#5#6#7{\begingroup%
  \dimen0=0.1em
  \def\stack##1##2{\oalign{##1\cr\hidewidth##2\hidewidth}}%
  \def\0##1{\kern##1\dimen0}%
  \def\1##1{\vrule height10ex width##1\dimen0}%
  \def\L##1{\ifcase##1\bc3211##1\or\bc2221##1\or\bc2122##1\or\bc1411##1%
    \or\bc1132##1\or\bc1231##1\or\bc1114##1\or\bc1312##1\or\bc1213##1%
    \or\bc3112##1\fi}%
  \def\R##1{\bgroup\let\next\1\let\1\0\let\0\next\L##1\egroup}%
  \def\G##1{\bgroup\let\bc\bcg\L##1\egroup}% reverse
  \def\bc##1##2##3##4##5{\stack{\0##1\1##2\0##3\1##4}##5}%
  \def\bcg##1##2##3##4##5{\stack{\0##4\1##3\0##2\1##1}##5}%
  \def\bcR##1##2##3##4##5##6{\R##1\R##2\R##3\R##4\R##5\R##6\11\01\11\09%
    \endgroup}%
  \stack{\09}#1\11\01\11\L#2%
  \ifcase#1\L#3\L#4\L#5\L#6\L#7\or\L#3\G#4\L#5\G#6\G#7%
    \or\L#3\G#4\G#5\L#6\G#7\or\L#3\G#4\G#5\G#6\L#7%
    \or\G#3\L#4\L#5\G#6\G#7\or\G#3\G#4\L#5\L#6\G#7%
    \or\G#3\G#4\G#5\L#6\L#7\or\G#3\L#4\G#5\L#6\G#7%
    \or\G#3\L#4\G#5\G#6\L#7\or\G#3\G#4\L#5\G#6\L#7%
  \fi\01\11\01\11\01\bcR}


\author{Bart Snapp}

\title{Homomorphisms}

\begin{document}
\begin{abstract}
  We introduce homomorphisms and isomorphisms. 
\end{abstract}
\maketitle

We've remarked before that if you want to understand an object, then
you should try to understand the functions on an object. So now we
turn our study to the functions on groups.


\section{Homomorphisms}

\begin{definition}\index{homomorphism!of groups}\index{group homomorphism}
  Let $(G,\star)$ and $(H,\diamond)$ be groups. A function
  \[
  \phi:G\to H
  \]
  is a \dfn{homomorphism} of groups if
  \[
  \phi(a\star b) = \phi(a)\diamond\phi(b).
  \]
\end{definition}

Believe it or not, you already know some homomorphisms!

\begin{example}[Exponential functions]
  Consider the following:
  \begin{align*}
    \exp:(\R,+) &\to (\R_{> 0},\cdot) \\
    x &\mapsto e^x.
  \end{align*}
  The function $\exp$ is a homorphism from the group $(\R,+)$ to
  $(\R_{> 0},\cdot)$ as
  \begin{align*}
    \exp(x+y) &= e^{x+y} \\
    &= e^x \cdot e^y\\
    &= \exp(x)\cdot \exp(y).
  \end{align*}
\end{example}



\begin{example}[Logarithmic functions]
  Consider the following:
  \begin{align*}
    \log:(\R_{> 0},\cdot) &\to (\R,+)\\
    x &\mapsto \log(x).
  \end{align*}
  The function $\log$ is a homorphism from the group $(\R_{> 0},\cdot)$ to $(\R,+)$ as
  \[
  \log(x\cdot y) = \log(x) + \log(y).
  \]
\end{example}


\begin{example}[Derivatives]
  Recall that $C^\infty(\R,\R)$ is the additive group of functions
  from $\R$ to $\R$ whose derivatives are all continuous. Consider the
  following:
  \begin{align*}
    \frac{d}{dx}:C^\infty(\R,\R) &\to C^\infty(\R,\R) \\
      f &\mapsto f'.
  \end{align*}
  The function $\frac{d}{dx}$ is a homorphism as
  \begin{align*}
    \frac{d}{dx}(f+ g) &= f' + g' \\
    &= \frac{d}{dx}f + \frac{d}{dx}g.
  \end{align*}
\end{example}


\begin{example}[The gradient]
  \begin{align*}
    \grad:C^\infty(\R,\R^n) &\to C^\infty(\R^n,\R^n)\\
    F &\mapsto \grad F.
  \end{align*}
\end{example}



\begin{example}[Linear transformations]
\end{example}


\begin{exercise}
  Complex conjugation
\end{exercise}




\begin{example}[Determinants]
  Consider the set of invertible linear transformations from $\R^n\to \R^n$:
  \begin{align*}
    \det: GL(n) &\to \R^\times \\
             M  &\mapsto  \det(M)
  \end{align*}
  The function $\det$ is a homomorphism as
  \[
  \det(M \cdot N) = \det(M) \cdot \det(N).
  \]
\end{example}




\begin{lemma}[Homomorphisms preserve identity]\label{L:HPId}
  Let $\phi:G\to H$ be a homomorphism of groups. Prove that
  \[
  \phi(e_G) = e_H,
  \]
  where $e_G$ is the identity of $G$ and $e_H$ is the identity of
  $H$.
\end{lemma}



\begin{lemma}[Homomorphisms preserve inverses]\label{L:HPI}
  If $\phi:G\to H$ is a group homomorphism, then $\phi(g^{-1}) =
  \phi(g)^{-1}$.
\end{lemma}




\subsection{Isomorphisms}

\begin{definition}
  Let $\phi:G\to H$ be a homomorphism of groups. The map $\phi$ is an
  \dfn{isomorphism} if $\phi$ is bijective.
\end{definition}


\section{Kernels and images}



\begin{definition}
  Let $\phi:G\to H$ be a group homomorphism. The \dfn{kernel} of
  $\phi$,
  \[
  \ker(\phi) := \{g\in G: \phi(g) = e_H\}.
  \]
  is the subset of $G$ that $\phi$ maps to the identity $e_H\in H$.
\end{definition}


\begin{lemma}[Kernels are normal subgroups]\label{L:kerN}
  If $\phi:G\to H$ is a group homomorphism, then $\ker(\phi)\normal G$.
\end{lemma}

\begin{lemma}[Homomorphisms preserve subgroups]\label{L:hps}
  Consider a homomorphism of groups $\phi:G\to H$, and let $S\subgp
  G$. Define
  \[
  \phi(S) := \{\phi(g): g\in S\}.
  \]
  In this case, $\phi(S) \subgp H$.
\end{lemma}



\begin{lemma}[Preimages preserve subgroups]\label{L:pps}
  Consider a homomorphism of groups $\phi:G\to H$, and let $T\subgp
  G$. Define
  \[
  \phi^{-1}(T) := \{g\in G: \phi(g)\in T\}.
  \]
  In this case, $\phi^{-1}(T) \subgp G$.
\end{lemma}



\begin{lemma}[Kernels and injectivity]
  A homomorphism of groups $\phi:G\to H$ is injective if and only if
  $\ker(\phi) = \{e_G\}$.
  \begin{proof}
    $(\Rightarrow)$ By Lemma~\ref{L:HPId}, $\phi(e_G) = e_H$. If
    $\phi$ is injective, only $e_G$ can map to $e_H$. Hence
    $\ker(\phi) = \{e_G\}$.


    $(\Leftarrow)$ Suppose that $\ker(\phi) = \{e_G\}$. We must show
    that $\phi$ is injective. Seeking a contradiction, suppose that
    there exist $a,b\in G$ with $a\ne b$ such that
    \[
    \phi(a) = \phi(b).
    \]
    Write with me
    \begin{align*}
      \phi(ab^{-1}) &=\phi(a)\phi(b^{-1})\\
      &= \phi(a) \phi(b)^{-1} & \text{(By Lemma~\ref{L:HPI})}\\
      &= e_H.
    \end{align*}
    Hence $ab^{-1}\in\ker(\phi)$ but $ab^{-1}\ne e_G$, a
    contradiction.
  \end{proof}
\end{lemma}


\begin{lemma}[Images and surjectivity]
    A homomorphism of groups $\phi:G\to H$ is surjective if and only
    if $\im(\phi) = H$.
    \begin{proof}
      $(\Rightarrow)$ Follows from the definition.

      $(\Leftarrow)$ Follows from the definition.
    \end{proof}
\end{lemma}







\begin{exercise}[Canonical surjection]
  Let $G$ be a group and $N$ be a normal subgroup of $G$. Prove that
  \begin{align*}
  \pi:G &\to G/N\\
  g &\mapsto gN
  \end{align*}
  is a surjective homomorphism. This surjection is called the
  \dfn{canonical surjection} or \dfn{natural surjection} from $G$ to
  $G/N$.
\end{exercise}





\section{Isomorphisms revisited}












\begin{exercise}
  Prove that addition is an associative operation  over $\Z$.
\end{exercise}

\begin{exercise}
  Prove that multiplication is an associative operation  over $\Z$.
\end{exercise}





\end{document}
