\documentclass{ximera}

\usepackage[T1]{fontenc}
\usepackage{stix2}
\usepackage{gillius}
\usepackage{resizegather}
%\usepackage{rsfso} fancy cal
\DeclareMathAlphabet{\mathcal}{OMS}{cmsy}{m}{n} %less fancy cal


\usepackage{multicol}


\usepackage{tikz-cd}
\usepackage{tkz-euclide} %% compass
\usetkzobj{all}  %% tkzCompass
\tikzset{>=stealth}
\tikzcdset{arrow style=tikz}
\usetikzlibrary{math} %% for assigning variables
%\usetikzlibrary{fadings}

\usepackage{colortbl,boldline,makecell} %% group tables


\usepackage[sans]{dsfont}

\usepackage{stmaryrd,pifont}

\graphicspath{
  {./}
  {fields/}
  }     



\let\oldbibliography\thebibliography%% to compact bib
\renewcommand{\thebibliography}[1]{%
  \oldbibliography{#1}%
  \setlength{\itemsep}{0pt}%
}
\renewcommand\refname{} %% no name needed!


\DefineVerbatimEnvironment{macaulay2}{Verbatim}{numbers=left,frame=lines,label=Macaulay2,labelposition=topline}

\DefineVerbatimEnvironment{gap}{Verbatim}{numbers=left,frame=lines,label=GAP,labelposition=topline}

%%% This next bit of code defines all our theorem environments
\makeatletter
\let\c@theorem\relax
\let\c@corollary\relax
%\let\c@example\relax
\makeatother

\let\definition\relax
\let\enddefinition\relax

\let\theorem\relax
\let\endtheorem\relax

\let\proposition\relax
\let\endproposition\relax

\let\exercise\relax
\let\endexercise\relax

\let\question\relax
\let\endquestion\relax

\let\remark\relax
\let\endremark\relax

\let\corollary\relax
\let\endcorollary\relax


\let\example\relax
\let\endexample\relax

\let\warning\relax
\let\endwarning\relax

\let\lemma\relax
\let\endlemma\relax


\let\algorithm\relax
\let\endalgorithm\relax
\usepackage{algpseudocode}

\newtheoremstyle{SlantTheorem}{\topsep}{\topsep}%%% space between body and thm
		{\slshape}                      %%% Thm body font
		{}                              %%% Indent amount (empty = no indent)
		{\bfseries\sffamily}            %%% Thm head font
		{}                              %%% Punctuation after thm head
		{3ex}                           %%% Space after thm head
		{\thmname{#1}\thmnumber{ #2}\thmnote{ \bfseries(#3)}}%%% Thm head spec
\theoremstyle{SlantTheorem}
\newtheorem{theorem}{Theorem}
%\newtheorem{definition}[theorem]{Definition}
%\newtheorem{proposition}[theorem]{Proposition}
%% \newtheorem*{dfnn}{Definition}
%% \newtheorem{ques}{Question}[theorem]
%% \newtheorem*{war}{WARNING}
%% \newtheorem*{cor}{Corollary}
%% \newtheorem*{eg}{Example}
\newtheorem*{remark}{Remark}
\newtheorem*{touchstone}{Touchstone}
\newtheorem{corollary}{Corollary}[theorem]
\newtheorem*{warning}{WARNING}
\newtheorem{example}[corollary]{Example}
\newtheorem{lemma}[theorem]{Lemma}




\newtheoremstyle{Definition}{\topsep}{\topsep}%%% space between body and thm
		{}                              %%% Thm body font
		{}                              %%% Indent amount (empty = no indent)
		{\bfseries\sffamily}            %%% Thm head font
		{}                              %%% Punctuation after thm head
		{3ex}                           %%% Space after thm head
		{\thmname{#1}\thmnumber{ #2}\thmnote{ \bfseries(#3)}}%%% Thm head spec
\theoremstyle{Definition}
\newtheorem{definition}[theorem]{Definition}



\let\algorithm\relax
\let\endalgorithm\relax
\newtheoremstyle{Alg}{\topsep}{\topsep}%%% space between body and thm
		{}                              %%% Thm body font
		{}                              %%% Indent amount (empty = no indent)
		{\bfseries\sffamily}            %%% Thm head font
		{}                              %%% Punctuation after thm head
		{3ex}                           %%% Space after thm head
		{\thmname{#1}\thmnumber{ #2}\thmnote{ \bfseries(#3)}}%%% Thm head spec
\theoremstyle{Alg}
\newtheorem*{algorithm}{Algorithm}
\newtheorem*{construction}{Construction}




\newtheoremstyle{Exercise}{\topsep}{\topsep} %%% space between body and thm
		{}                           %%% Thm body font
		{}                           %%% Indent amount (empty = no indent)
		{\bfseries\sffamily}         %%% Thm head font
		{)}                          %%% Punctuation after thm head
		{ }                          %%% Space after thm head
		{\thmnumber{#2}\thmnote{ \bfseries(#3)}}%%% Thm head spec
\theoremstyle{Exercise}
\newtheorem{exercise}[corollary]{}%[theorem]

%% \newtheoremstyle{Question}{\topsep}{\topsep} %%% space between body and thm
%% 		{\bfseries}                  %%% Thm body font
%% 		{3ex}                        %%% Indent amount (empty = no indent)
%% 		{}                           %%% Thm head font
%% 		{}                           %%% Punctuation after thm head
%% 		{}                           %%% Space after thm head
%% 		{\thmnumber{#2}\thmnote{ \bfseries(#3)}}%%% Thm head spec
\newtheoremstyle{Question}{3em}{3em} %%% space between body and thm
		{\large\bfseries}                           %%% Thm body font
		{}                           %%% Indent amount (empty = no indent)
		{}                         %%% Thm head font
		{}                          %%% Punctuation after thm head
		{0em}                          %%% Space after thm head
		{}%%% Thm head spec
\theoremstyle{Question}
\newtheorem*{question}{}






\renewcommand{\tilde}{\widetilde}
\renewcommand{\bar}{\overline}
\renewcommand{\hat}{\widehat}
\newcommand{\N}{\mathbb N}
\newcommand{\Z}{\mathbb Z}
\newcommand{\R}{\mathbb R}
\newcommand{\Q}{\mathbb Q}
\newcommand{\C}{\mathbb C}
\newcommand{\V}{\mathbb V}
\newcommand{\I}{\mathbb I}
\newcommand{\A}{\mathbb A}
\renewcommand{\o}{\mathbf o}
\newcommand{\iso}{\simeq}
\newcommand{\ph}{\varphi}
\newcommand{\Cf}{\mathcal{C}}
\newcommand{\IZ}{\mathrm{Int}(\Z)}
\newcommand{\dsum}{\oplus}
\newcommand{\directsum}{\bigoplus}
\newcommand{\union}{\bigcup}
\newcommand{\subgp}{\leq}
\newcommand{\normal}{\trianglelefteq}
\renewcommand{\i}{\mathfrak}
\renewcommand{\a}{\mathfrak{a}}
\renewcommand{\b}{\mathfrak{b}}
\newcommand{\m}{\mathfrak{m}}
\newcommand{\p}{\mathfrak{p}}
\newcommand{\q}{\mathfrak{q}}
\newcommand{\dfn}[1]{\textbf{#1}\index{#1}}
\let\hom\relax
\DeclareMathOperator{\mat}{Mat}
\DeclareMathOperator{\ann}{Ann}
\DeclareMathOperator{\h}{ht}
\DeclareMathOperator{\tr}{tr}
\DeclareMathOperator{\hom}{Hom}
\DeclareMathOperator{\Span}{Span}
\DeclareMathOperator{\spec}{Spec}
\DeclareMathOperator{\maxspec}{MaxSpec}
\DeclareMathOperator{\aut}{Aut}
\DeclareMathOperator{\ass}{Ass}
\DeclareMathOperator{\lcm}{lcm}
\DeclareMathOperator{\ff}{Frac}
\DeclareMathOperator{\im}{Im}
\DeclareMathOperator{\syz}{Syz}
\DeclareMathOperator{\gr}{Gr}
\DeclareMathOperator{\multideg}{multideg}
\renewcommand{\ker}{\mathop{\mathrm{Ker}}\nolimits}
\newcommand{\coker}{\mathop{\mathrm{Coker}}\nolimits}
\newcommand{\lps}{[\hspace{-0.25ex}[}
\newcommand{\rps}{]\hspace{-0.25ex}]}
\newcommand{\into}{\hookrightarrow}
\newcommand{\onto}{\twoheadrightarrow}
\newcommand{\tensor}{\otimes}
\newcommand{\x}{\mathbf{x}}
\newcommand{\X}{\mathbf X}
\newcommand{\Y}{\mathbf Y}
\renewcommand{\k}{\boldsymbol{\kappa}}
\renewcommand{\emptyset}{\varnothing}
\renewcommand{\qedsymbol}{$\blacksquare$}
\renewcommand{\l}{\ell}
\newcommand{\1}{\mathds{1}}
\newcommand{\lto}{\mathop{\longrightarrow\,}\limits}
\newcommand{\rad}{\sqrt}
\newcommand{\hf}{H}
\newcommand{\hs}{H\!S}
\newcommand{\hp}{H\!P}
\renewcommand{\vec}{\mathbf}
\let\temp\phi
\let\phi\varphi
\let\eulerphi\temp


\renewcommand{\epsilon}{\varepsilon}
\renewcommand{\subset}{\subseteq}
\renewcommand{\supset}{\supseteq}
\newcommand{\macaulay}{\normalfont\textsl{Macaulay2}}
\newcommand{\GAP}{\normalfont\textsf{GAP}}
\newcommand{\invlim}{\varprojlim}
\renewcommand{\le}{\leqslant}
\renewcommand{\ge}{\geqslant}
\newcommand{\valpha}{{\boldsymbol\alpha}}
\newcommand{\vbeta}{{\boldsymbol\beta}}
\newcommand{\vgamma}{{\boldsymbol\gamma}}
\newcommand{\dotp}{\bullet}
\newcommand{\lc}{\normalfont\textsc{lc}}
\newcommand{\lt}{\normalfont\textsc{lt}}
\newcommand{\lm}{\normalfont\textsc{lm}}
\newcommand{\from}{\leftarrow}
\newcommand{\transpose}{\intercal}
\newcommand{\grad}{\boldsymbol\nabla}
\newcommand{\curl}{\boldsymbol{\nabla\times}}
\renewcommand{\d}{\, d}
\newcommand{\<}{\langle}
\renewcommand{\>}{\rangle}

%\renewcommand{\proofname}{Sketch of Proof}


\renewenvironment{proof}[1][Proof]
  {\begin{trivlist}\item[\hskip \labelsep \itshape \bfseries #1{}\hspace{2ex}]\upshape}
{\qed\end{trivlist}}

\newenvironment{sketch}[1][Sketch of Proof]
  {\begin{trivlist}\item[\hskip \labelsep \itshape \bfseries #1{}\hspace{2ex}]\upshape}
{\qed\end{trivlist}}



\makeatletter
\renewcommand\section{\@startsection{paragraph}{10}{\z@}%
                                     {-3.25ex\@plus -1ex \@minus -.2ex}%
                                     {1.5ex \@plus .2ex}%
                                     {\normalfont\large\sffamily\bfseries}}
\renewcommand\subsection{\@startsection{subparagraph}{10}{\z@}%
                                    {3.25ex \@plus1ex \@minus.2ex}%
                                    {-1em}%
                                    {\normalfont\normalsize\sffamily\bfseries}}
\makeatother

%% Fix weird index/bib issue.
\makeatletter
\gdef\ttl@savemark{\sectionmark{}}
\makeatother


\makeatletter
%% no number for refs
\newcommand\frontstyle{%
  \def\activitystyle{activity-chapter}
  \def\maketitle{%
    \addtocounter{titlenumber}{1}%
                    {\flushleft\small\sffamily\bfseries\@pretitle\par\vspace{-1.5em}}%
                    {\flushleft\LARGE\sffamily\bfseries\@title \par }%
                    {\vskip .6em\noindent\textit\theabstract\setcounter{problem}{0}\setcounter{sectiontitlenumber}{0}}%
                    \par\vspace{2em}
                    \phantomsection\addcontentsline{toc}{section}{\textbf{\@title}}%
                  }}
\makeatother



\NewEnviron{annotate}{\vspace{-.3cm}\small \itshape \BODY \vspace{.3cm}}


%%%% TIKZ STUFF

%% N-GON code
\tikzset{
    pics/tikzngon/.style={
        code={
        \tikzmath{\xx = #1;\rr=1.7;}
        \draw[ultra thick,rounded corners=.05mm] ({\rr*sin(0*360/\xx)},{\rr*cos(0*360/\xx)})
        \foreach \x in {-1,0,...,\xx}
        {
        -- ({\rr*sin(\x*360/\xx)},{\rr*cos(\x*360/\xx)})
        }
           -- cycle;
  }}}

%% N-GON code (even)
\tikzset{
    pics/tikzegon/.style={
        code={
        \tikzmath{\xx = #1;\rr=1.7;}
        \draw[ultra thick,rounded corners=.05mm] ({\rr*sin(0*360/\xx+180/\xx)},{\rr*cos(0*360/\xx+180/\xx)})
        \foreach \x in {-1,0,...,\xx}
           {
           -- ({\rr*sin(\x*360/\xx+180/\xx)},{\rr*cos(\x*360/\xx+180/\xx)}) 
           }
           -- cycle;
  }}}




%% N-CLOCK code
\tikzset{
    pics/tikznclock/.style={
        code={
        \tikzmath{\xx = #1;\rr=1.7;\dd=.4;}
        \foreach \x in {1,...,\xx}
        \pgfmathtruncatemacro{\xy}{\x-1}
           {
             \node[circle,fill=black,inner sep=0pt, minimum size=13pt,text=white]
             at ({(\rr-\dd)*sin((\x-1)*360/(\xx)},{(\rr-\dd)*cos((\x-1)*360/\xx}) {\normalfont\bfseries\sffamily\small {\xy}};
           }
  \draw[thick] (0,0) circle (\rr);
  }}}



%% barcode from
%% https://tex.stackexchange.com/questions/6895/is-there-a-good-latex-package-for-generating-barcodes
%% NOT CURRENTLY USED!


\def\barcode#1#2#3#4#5#6#7{\begingroup%
  \dimen0=0.1em
  \def\stack##1##2{\oalign{##1\cr\hidewidth##2\hidewidth}}%
  \def\0##1{\kern##1\dimen0}%
  \def\1##1{\vrule height10ex width##1\dimen0}%
  \def\L##1{\ifcase##1\bc3211##1\or\bc2221##1\or\bc2122##1\or\bc1411##1%
    \or\bc1132##1\or\bc1231##1\or\bc1114##1\or\bc1312##1\or\bc1213##1%
    \or\bc3112##1\fi}%
  \def\R##1{\bgroup\let\next\1\let\1\0\let\0\next\L##1\egroup}%
  \def\G##1{\bgroup\let\bc\bcg\L##1\egroup}% reverse
  \def\bc##1##2##3##4##5{\stack{\0##1\1##2\0##3\1##4}##5}%
  \def\bcg##1##2##3##4##5{\stack{\0##4\1##3\0##2\1##1}##5}%
  \def\bcR##1##2##3##4##5##6{\R##1\R##2\R##3\R##4\R##5\R##6\11\01\11\09%
    \endgroup}%
  \stack{\09}#1\11\01\11\L#2%
  \ifcase#1\L#3\L#4\L#5\L#6\L#7\or\L#3\G#4\L#5\G#6\G#7%
    \or\L#3\G#4\G#5\L#6\G#7\or\L#3\G#4\G#5\G#6\L#7%
    \or\G#3\L#4\L#5\G#6\G#7\or\G#3\G#4\L#5\L#6\G#7%
    \or\G#3\G#4\G#5\L#6\L#7\or\G#3\L#4\G#5\L#6\G#7%
    \or\G#3\L#4\G#5\G#6\L#7\or\G#3\G#4\L#5\G#6\L#7%
  \fi\01\11\01\11\01\bcR}


\author{Bart Snapp}

\title{Cosets}

\begin{document}
\begin{abstract}
We introduce cosets of subgroups, and show that they partition the
group.
\end{abstract}
\maketitle


\begin{definition}
  Let $S$ be a subgroup of $(G,\star)$. A \textbf{left }\dfn{coset} of $S$ is a set of
  the form
  \[
  g\star S := \{g\star s : s\in S\}
  \]
  where $g\in G$. A \textbf{right }\dfn{coset} of $S$ is a set of the form
  \[
  S\star g := \{s\star g : s\in S\}.
  \]
\end{definition}

\begin{remark}
  Below we will work exclusively with left cosets. However, all of the
  same results are true for right cosets, and in fact the proofs are
  essentially the same.
\end{remark}



\begin{example}[Cosets of $\boldsymbol{2\pmb\Z\subgp \pmb\Z}$]
  The set $2\Z = \{2\cdot x: x\in \Z\}$ is a subgroup of $\Z$. The
  left cosets of $2\Z$ are
  \begin{align*}
    0+2\Z &= \{\dots, -8,-6,-4,-2,0,2,4,6,8,\dots\},\\
    1+2\Z &= \{\dots, -7,-5,-3,-1,1,3,5,7,9,\dots\}.
  \end{align*}
\end{example}


\begin{example}[Cosets of $\boldsymbol{5\pmb\Z\subgp \pmb\Z}$]
  The set $5\Z = \{5\cdot x: x\in \Z\}$ is a subgroup of $\Z$. The
  left cosets of $5\Z$ are
  \begin{align*}
    0+5\Z &= \{\dots, -20,-15,-10,-5,0,5,10,15,20,\dots\},\\
    1+5\Z &= \{\dots, -19,-14,-9 ,-4,1,6,11,16,21,\dots\},\\
    2+5\Z &= \{\dots, -18,-13,-8 ,-3,2,7,12,17,22,\dots\},\\
    3+5\Z &= \{\dots, -17,-12,-7 ,-2,3,8,13,18,23,\dots\},\\
    4+5\Z &= \{\dots, -16,-11,-6 ,-1,4,9,14,19,24,\dots\}.
  \end{align*}
\end{example}




\begin{example}[Cosets of $\boldsymbol{\<r\>\subgp D_3}$]\index{symmetries!of the equilateral triangle}
  If we arrange a Cayley table for $D_3$ appropriately, we can ``see''
  a subgroup and its cosets. Here we see the subgroup $\<r\>$ in the
  upper left-hand corner.
    \[
    \renewcommand{\arraystretch}{1.6}
    \begin{array}{c!{\vline width 2pt}cccccc}
      (D_3,\circ)& e                              & r                              & r^2                            & f                             & r f                                   & r^2 f  \\  \Xhline{2pt}
      e          & e\cellcolor{blue!12!white}     & r\cellcolor{blue!12!white}    & r^2\cellcolor{blue!12!white}   & f \cellcolor{red!12!white}    & r f \cellcolor{blue!33!red!24!white}   & r^2f \cellcolor{blue!66!red!24!white} \\  
      r                                  & r\cellcolor{blue!12!white}    & r^2\cellcolor{blue!12!white}   & e \cellcolor{blue!12!white}    & rf\cellcolor{red!12!white} & r^2f \cellcolor{blue!33!red!24!white}    & f \cellcolor{blue!66!red!24!white}   \\  
      r^2                                 & r^2\cellcolor{blue!12!white}   & e\cellcolor{blue!12!white}     & r\cellcolor{blue!12!white}    & r^2 f \cellcolor{red!12!white}   & f \cellcolor{blue!33!red!24!white} & rf \cellcolor{blue!66!red!24!white}   \\  
      f        & f \cellcolor{red!12!white}    & r^2f \cellcolor{red!12!white}   & r f \cellcolor{red!12!white} &     &   &   \\  
      r f       & rf \cellcolor{blue!33!red!24!white}   & f \cellcolor{blue!33!red!24!white} & r^2f \cellcolor{blue!33!red!24!white}   &    &      &     \\  
      r^2 f       & r^2 f\cellcolor{blue!66!red!24!white} & rf \cellcolor{blue!66!red!24!white}    & f \cellcolor{blue!66!red!24!white}  &     &     &      \\  
    \end{array}
    \]
    The left cosets of $\<r\>$ are below $\<r\>$ and the right cosets
    of $\<r\>$ are to the right of $\<r\>$. Explicitly, the cosets are
    \begin{align*}
      \<r\> &= \{e, r, r^2\},\\
      f\<r\> &= \{f, rf, r^2f\}.
    \end{align*}
    In particular, for $x\in D_3$ we see that $x\<r\> = \<r\> x$.
\end{example}



\begin{example}[Cosets of $\boldsymbol{\<f\>\subgp D_3}$]\index{symmetries!of the equilateral triangle}
  If we arrange a Cayley table for $D_3$ appropriately, we can ``see''
  a subgroup and its cosets. Here we see the subgroup $\<f\>$ in the
  upper left-hand corner.
    \[
    \renewcommand{\arraystretch}{1.6}
    \begin{array}{c!{\vline width 2pt}cccccc}
      (D_3,\circ)& e                         & f                              & rf                            & r^2f                             & r                                    & r^2  \\  \Xhline{2pt}
      e          & e\cellcolor{blue!12!white}     & f\cellcolor{blue!12!white}    & rf\cellcolor{red!12!white}   & r^2f \cellcolor{blue!33!red!24!white}     & r  \cellcolor{blue!66!red!24!white}   & r^2 \cellcolor{blue!36!white} \\  
      f                & f\cellcolor{blue!12!white}    & e\cellcolor{blue!12!white}   & r^2 \cellcolor{red!12!white}    & f\cellcolor{blue!33!red!24!white} & r^2f \cellcolor{blue!66!red!24!white}    & rf \cellcolor{blue!36!white}   \\  
      rf                & rf\cellcolor{red!12!white}   & r\cellcolor{red!12!white}     &    &    &  &     \\  
      r^2f       & r^2f \cellcolor{blue!33!red!24!white}   & r^2 \cellcolor{blue!33!red!24!white}   &   &      &   &   \\  
      r      & r \cellcolor{blue!66!red!24!white}   & rf \cellcolor{blue!66!red!24!white} &    &   &      &     \\  
      r^2       & r^2 \cellcolor{blue!36!white} & r^2f \cellcolor{blue!36!white}    &   &    &     &      \\  
    \end{array}
    \]
    The left cosets of $\<f\>$ are below $\<f\>$ and the right cosets
    of $\<f\>$ are to the right of $\<f\>$. Explicitly, the left
    cosets are
    \begin{align*}
      \<f\> &= \{e, f\},\\
      r\<f\> &= \{r, rf\},\\
      r^2\<f\> &= \{r^2, r^2f\},
    \end{align*}
    and the right cosets are
    \begin{align*}
      \<f\> &= \{e, f\},\\
      \<f\>r &= \{r, r^2f\},\\
      \<f\> r^2&= \{r^2, rf\}.
    \end{align*}
    In particular, for $x\in D_3$ with $x\ne e$ and $x\ne f$, we see
    that $x\<f\> \ne \<f\> x$.
\end{example}



\begin{exercise}
  Let $S\subgp G$ and $a\in S$. Prove that the left coset $aS = S$.
  \begin{hint}
    Recall Lemma~\ref{L:oepr}.
  \end{hint}
\end{exercise}





\begin{lemma}[Cosets are the order of the subgroup]\label{P:CSG}
  Let $S$ be a subgroup of $G$. For any $g\in G$,
  \[
  |S| = |gS|.
  \]
  \begin{proof}
    Define a function
    \begin{align*}
      \eta: S &\to g S\\
      s &\mapsto g\cdot s.
    \end{align*}
    We claim that $\eta$ is bijective. First note that by definition
    of cosets $\eta$ is surjective.

    Now we must show that $\eta$ is injective. Suppose that $\eta(s) =
    \eta(t)$. Write with me:
    \begin{align*}
      \eta(s) &= \eta(t) \\
      gs &= gt\\
      g^{-1} g s &= g^{-1} g t\\
      s &= t.
    \end{align*}
    Thus if two elements are equal in the image of $\eta$, the must
    have come from equal elements in the preimage of $\eta$. So we see
    that $\eta$ is injective.

    Since $\eta$ is bijective, $|S|= |gS|$.
  \end{proof}
\end{lemma}

\begin{corollary}[Cosets have the same order]
  Let $S$ be a subgroup of $G$. All left cosets of $S$ have the same
  order.
\end{corollary}

\begin{remark}
  The same result holds replacing ``left'' cosets with ``right''
  cosets.
\end{remark}



\begin{lemma}[Cosets are equal or disjoint]\label{L:cosetED}
  Let $S$ be a subgroup of $G$. Left cosets of $S$ are either equal or
  disjoint, meaning for all $a,b\in G$, we have that either
  \[
  aS = bS \quad\text{or}\quad aS \cap bS = \emptyset.
  \]
  \begin{proof}
    We will prove that if left cosets are not disjoint, then they are
    equal. Suppose that there exists $c$ such that $c\in aS \cap
    bS$. Write with me
    \[
    as_1 = c = bs_2.
    \]
    So, $as_1 = bs_2$, this means
    \[
    as_1 S = bs_2 S.
    \]
    But $as_1S = aS$ and $bs_2 S = bS$, so $aS = bS$.
  \end{proof}
\end{lemma}



\begin{theorem}[Cosets partition the group]\label{T:CPG}
  Let $S$ be a subgroup of $G$, 
  \[
  G = \union_{g\in G} gS
  \]
  and for all $a,b\in G$, if $aS \ne bS$,
  \[
  aS\cap bS = \emptyset.
  \]
  \begin{proof}
    All that remains to be shown is that $G = \union_{g\in G}
    gS$. Since $e\in S$, this is true. The rest follows from
    Lemma~\ref{L:cosetED}.
  \end{proof}
\end{theorem}




%% \begin{lemma}[Criterion for coset equality]
%%   Let $S$ be a subgroup of $G$. If $a,b\in G$,
%%   \[
%%   aS = bS \Leftrightarrow ab^{-1}\in S.
%%   \]
%%   \begin{proof}
%%     $(\Rightarrow)$ If $aS = bS$, then there exists $s\in S$ with
%%     \begin{align*}
%%       a &= b\cdot s,\\
%%       ab^{-1} &= s,
%%     \end{align*}
%%     so $ab^{-1}\in S$.


%%     $(\Leftarrow)$ If $ab^{-1}\in S$, then for some $s\in S$, $a = bs$
%%     and so $aS =bS$.
%%   \end{proof}
%% \end{lemma}






\end{document}
