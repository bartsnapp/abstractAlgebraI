\documentclass{ximera}

\usepackage[T1]{fontenc}
\usepackage{stix2}
\usepackage{gillius}
\usepackage{resizegather}
%\usepackage{rsfso} fancy cal
\DeclareMathAlphabet{\mathcal}{OMS}{cmsy}{m}{n} %less fancy cal


\usepackage{multicol}


\usepackage{tikz-cd}
\usepackage{tkz-euclide} %% compass
\usetkzobj{all}  %% tkzCompass
\tikzset{>=stealth}
\tikzcdset{arrow style=tikz}
\usetikzlibrary{math} %% for assigning variables
%\usetikzlibrary{fadings}

\usepackage{colortbl,boldline,makecell} %% group tables


\usepackage[sans]{dsfont}

\usepackage{stmaryrd,pifont}

\graphicspath{
  {./}
  {fields/}
  }     



\let\oldbibliography\thebibliography%% to compact bib
\renewcommand{\thebibliography}[1]{%
  \oldbibliography{#1}%
  \setlength{\itemsep}{0pt}%
}
\renewcommand\refname{} %% no name needed!


\DefineVerbatimEnvironment{macaulay2}{Verbatim}{numbers=left,frame=lines,label=Macaulay2,labelposition=topline}

\DefineVerbatimEnvironment{gap}{Verbatim}{numbers=left,frame=lines,label=GAP,labelposition=topline}

%%% This next bit of code defines all our theorem environments
\makeatletter
\let\c@theorem\relax
\let\c@corollary\relax
%\let\c@example\relax
\makeatother

\let\definition\relax
\let\enddefinition\relax

\let\theorem\relax
\let\endtheorem\relax

\let\proposition\relax
\let\endproposition\relax

\let\exercise\relax
\let\endexercise\relax

\let\question\relax
\let\endquestion\relax

\let\remark\relax
\let\endremark\relax

\let\corollary\relax
\let\endcorollary\relax


\let\example\relax
\let\endexample\relax

\let\warning\relax
\let\endwarning\relax

\let\lemma\relax
\let\endlemma\relax


\let\algorithm\relax
\let\endalgorithm\relax
\usepackage{algpseudocode}

\newtheoremstyle{SlantTheorem}{\topsep}{\topsep}%%% space between body and thm
		{\slshape}                      %%% Thm body font
		{}                              %%% Indent amount (empty = no indent)
		{\bfseries\sffamily}            %%% Thm head font
		{}                              %%% Punctuation after thm head
		{3ex}                           %%% Space after thm head
		{\thmname{#1}\thmnumber{ #2}\thmnote{ \bfseries(#3)}}%%% Thm head spec
\theoremstyle{SlantTheorem}
\newtheorem{theorem}{Theorem}
%\newtheorem{definition}[theorem]{Definition}
%\newtheorem{proposition}[theorem]{Proposition}
%% \newtheorem*{dfnn}{Definition}
%% \newtheorem{ques}{Question}[theorem]
%% \newtheorem*{war}{WARNING}
%% \newtheorem*{cor}{Corollary}
%% \newtheorem*{eg}{Example}
\newtheorem*{remark}{Remark}
\newtheorem*{touchstone}{Touchstone}
\newtheorem{corollary}{Corollary}[theorem]
\newtheorem*{warning}{WARNING}
\newtheorem{example}[corollary]{Example}
\newtheorem{lemma}[theorem]{Lemma}




\newtheoremstyle{Definition}{\topsep}{\topsep}%%% space between body and thm
		{}                              %%% Thm body font
		{}                              %%% Indent amount (empty = no indent)
		{\bfseries\sffamily}            %%% Thm head font
		{}                              %%% Punctuation after thm head
		{3ex}                           %%% Space after thm head
		{\thmname{#1}\thmnumber{ #2}\thmnote{ \bfseries(#3)}}%%% Thm head spec
\theoremstyle{Definition}
\newtheorem{definition}[theorem]{Definition}



\let\algorithm\relax
\let\endalgorithm\relax
\newtheoremstyle{Alg}{\topsep}{\topsep}%%% space between body and thm
		{}                              %%% Thm body font
		{}                              %%% Indent amount (empty = no indent)
		{\bfseries\sffamily}            %%% Thm head font
		{}                              %%% Punctuation after thm head
		{3ex}                           %%% Space after thm head
		{\thmname{#1}\thmnumber{ #2}\thmnote{ \bfseries(#3)}}%%% Thm head spec
\theoremstyle{Alg}
\newtheorem*{algorithm}{Algorithm}
\newtheorem*{construction}{Construction}




\newtheoremstyle{Exercise}{\topsep}{\topsep} %%% space between body and thm
		{}                           %%% Thm body font
		{}                           %%% Indent amount (empty = no indent)
		{\bfseries\sffamily}         %%% Thm head font
		{)}                          %%% Punctuation after thm head
		{ }                          %%% Space after thm head
		{\thmnumber{#2}\thmnote{ \bfseries(#3)}}%%% Thm head spec
\theoremstyle{Exercise}
\newtheorem{exercise}[corollary]{}%[theorem]

%% \newtheoremstyle{Question}{\topsep}{\topsep} %%% space between body and thm
%% 		{\bfseries}                  %%% Thm body font
%% 		{3ex}                        %%% Indent amount (empty = no indent)
%% 		{}                           %%% Thm head font
%% 		{}                           %%% Punctuation after thm head
%% 		{}                           %%% Space after thm head
%% 		{\thmnumber{#2}\thmnote{ \bfseries(#3)}}%%% Thm head spec
\newtheoremstyle{Question}{3em}{3em} %%% space between body and thm
		{\large\bfseries}                           %%% Thm body font
		{}                           %%% Indent amount (empty = no indent)
		{}                         %%% Thm head font
		{}                          %%% Punctuation after thm head
		{0em}                          %%% Space after thm head
		{}%%% Thm head spec
\theoremstyle{Question}
\newtheorem*{question}{}






\renewcommand{\tilde}{\widetilde}
\renewcommand{\bar}{\overline}
\renewcommand{\hat}{\widehat}
\newcommand{\N}{\mathbb N}
\newcommand{\Z}{\mathbb Z}
\newcommand{\R}{\mathbb R}
\newcommand{\Q}{\mathbb Q}
\newcommand{\C}{\mathbb C}
\newcommand{\V}{\mathbb V}
\newcommand{\I}{\mathbb I}
\newcommand{\A}{\mathbb A}
\renewcommand{\o}{\mathbf o}
\newcommand{\iso}{\simeq}
\newcommand{\ph}{\varphi}
\newcommand{\Cf}{\mathcal{C}}
\newcommand{\IZ}{\mathrm{Int}(\Z)}
\newcommand{\dsum}{\oplus}
\newcommand{\directsum}{\bigoplus}
\newcommand{\union}{\bigcup}
\newcommand{\subgp}{\leq}
\newcommand{\normal}{\trianglelefteq}
\renewcommand{\i}{\mathfrak}
\renewcommand{\a}{\mathfrak{a}}
\renewcommand{\b}{\mathfrak{b}}
\newcommand{\m}{\mathfrak{m}}
\newcommand{\p}{\mathfrak{p}}
\newcommand{\q}{\mathfrak{q}}
\newcommand{\dfn}[1]{\textbf{#1}\index{#1}}
\let\hom\relax
\DeclareMathOperator{\mat}{Mat}
\DeclareMathOperator{\ann}{Ann}
\DeclareMathOperator{\h}{ht}
\DeclareMathOperator{\tr}{tr}
\DeclareMathOperator{\hom}{Hom}
\DeclareMathOperator{\Span}{Span}
\DeclareMathOperator{\spec}{Spec}
\DeclareMathOperator{\maxspec}{MaxSpec}
\DeclareMathOperator{\aut}{Aut}
\DeclareMathOperator{\ass}{Ass}
\DeclareMathOperator{\lcm}{lcm}
\DeclareMathOperator{\ff}{Frac}
\DeclareMathOperator{\im}{Im}
\DeclareMathOperator{\syz}{Syz}
\DeclareMathOperator{\gr}{Gr}
\DeclareMathOperator{\multideg}{multideg}
\renewcommand{\ker}{\mathop{\mathrm{Ker}}\nolimits}
\newcommand{\coker}{\mathop{\mathrm{Coker}}\nolimits}
\newcommand{\lps}{[\hspace{-0.25ex}[}
\newcommand{\rps}{]\hspace{-0.25ex}]}
\newcommand{\into}{\hookrightarrow}
\newcommand{\onto}{\twoheadrightarrow}
\newcommand{\tensor}{\otimes}
\newcommand{\x}{\mathbf{x}}
\newcommand{\X}{\mathbf X}
\newcommand{\Y}{\mathbf Y}
\renewcommand{\k}{\boldsymbol{\kappa}}
\renewcommand{\emptyset}{\varnothing}
\renewcommand{\qedsymbol}{$\blacksquare$}
\renewcommand{\l}{\ell}
\newcommand{\1}{\mathds{1}}
\newcommand{\lto}{\mathop{\longrightarrow\,}\limits}
\newcommand{\rad}{\sqrt}
\newcommand{\hf}{H}
\newcommand{\hs}{H\!S}
\newcommand{\hp}{H\!P}
\renewcommand{\vec}{\mathbf}
\let\temp\phi
\let\phi\varphi
\let\eulerphi\temp


\renewcommand{\epsilon}{\varepsilon}
\renewcommand{\subset}{\subseteq}
\renewcommand{\supset}{\supseteq}
\newcommand{\macaulay}{\normalfont\textsl{Macaulay2}}
\newcommand{\GAP}{\normalfont\textsf{GAP}}
\newcommand{\invlim}{\varprojlim}
\renewcommand{\le}{\leqslant}
\renewcommand{\ge}{\geqslant}
\newcommand{\valpha}{{\boldsymbol\alpha}}
\newcommand{\vbeta}{{\boldsymbol\beta}}
\newcommand{\vgamma}{{\boldsymbol\gamma}}
\newcommand{\dotp}{\bullet}
\newcommand{\lc}{\normalfont\textsc{lc}}
\newcommand{\lt}{\normalfont\textsc{lt}}
\newcommand{\lm}{\normalfont\textsc{lm}}
\newcommand{\from}{\leftarrow}
\newcommand{\transpose}{\intercal}
\newcommand{\grad}{\boldsymbol\nabla}
\newcommand{\curl}{\boldsymbol{\nabla\times}}
\renewcommand{\d}{\, d}
\newcommand{\<}{\langle}
\renewcommand{\>}{\rangle}

%\renewcommand{\proofname}{Sketch of Proof}


\renewenvironment{proof}[1][Proof]
  {\begin{trivlist}\item[\hskip \labelsep \itshape \bfseries #1{}\hspace{2ex}]\upshape}
{\qed\end{trivlist}}

\newenvironment{sketch}[1][Sketch of Proof]
  {\begin{trivlist}\item[\hskip \labelsep \itshape \bfseries #1{}\hspace{2ex}]\upshape}
{\qed\end{trivlist}}



\makeatletter
\renewcommand\section{\@startsection{paragraph}{10}{\z@}%
                                     {-3.25ex\@plus -1ex \@minus -.2ex}%
                                     {1.5ex \@plus .2ex}%
                                     {\normalfont\large\sffamily\bfseries}}
\renewcommand\subsection{\@startsection{subparagraph}{10}{\z@}%
                                    {3.25ex \@plus1ex \@minus.2ex}%
                                    {-1em}%
                                    {\normalfont\normalsize\sffamily\bfseries}}
\makeatother

%% Fix weird index/bib issue.
\makeatletter
\gdef\ttl@savemark{\sectionmark{}}
\makeatother


\makeatletter
%% no number for refs
\newcommand\frontstyle{%
  \def\activitystyle{activity-chapter}
  \def\maketitle{%
    \addtocounter{titlenumber}{1}%
                    {\flushleft\small\sffamily\bfseries\@pretitle\par\vspace{-1.5em}}%
                    {\flushleft\LARGE\sffamily\bfseries\@title \par }%
                    {\vskip .6em\noindent\textit\theabstract\setcounter{problem}{0}\setcounter{sectiontitlenumber}{0}}%
                    \par\vspace{2em}
                    \phantomsection\addcontentsline{toc}{section}{\textbf{\@title}}%
                  }}
\makeatother



\NewEnviron{annotate}{\vspace{-.3cm}\small \itshape \BODY \vspace{.3cm}}


%%%% TIKZ STUFF

%% N-GON code
\tikzset{
    pics/tikzngon/.style={
        code={
        \tikzmath{\xx = #1;\rr=1.7;}
        \draw[ultra thick,rounded corners=.05mm] ({\rr*sin(0*360/\xx)},{\rr*cos(0*360/\xx)})
        \foreach \x in {-1,0,...,\xx}
        {
        -- ({\rr*sin(\x*360/\xx)},{\rr*cos(\x*360/\xx)})
        }
           -- cycle;
  }}}

%% N-GON code (even)
\tikzset{
    pics/tikzegon/.style={
        code={
        \tikzmath{\xx = #1;\rr=1.7;}
        \draw[ultra thick,rounded corners=.05mm] ({\rr*sin(0*360/\xx+180/\xx)},{\rr*cos(0*360/\xx+180/\xx)})
        \foreach \x in {-1,0,...,\xx}
           {
           -- ({\rr*sin(\x*360/\xx+180/\xx)},{\rr*cos(\x*360/\xx+180/\xx)}) 
           }
           -- cycle;
  }}}




%% N-CLOCK code
\tikzset{
    pics/tikznclock/.style={
        code={
        \tikzmath{\xx = #1;\rr=1.7;\dd=.4;}
        \foreach \x in {1,...,\xx}
        \pgfmathtruncatemacro{\xy}{\x-1}
           {
             \node[circle,fill=black,inner sep=0pt, minimum size=13pt,text=white]
             at ({(\rr-\dd)*sin((\x-1)*360/(\xx)},{(\rr-\dd)*cos((\x-1)*360/\xx}) {\normalfont\bfseries\sffamily\small {\xy}};
           }
  \draw[thick] (0,0) circle (\rr);
  }}}



%% barcode from
%% https://tex.stackexchange.com/questions/6895/is-there-a-good-latex-package-for-generating-barcodes
%% NOT CURRENTLY USED!


\def\barcode#1#2#3#4#5#6#7{\begingroup%
  \dimen0=0.1em
  \def\stack##1##2{\oalign{##1\cr\hidewidth##2\hidewidth}}%
  \def\0##1{\kern##1\dimen0}%
  \def\1##1{\vrule height10ex width##1\dimen0}%
  \def\L##1{\ifcase##1\bc3211##1\or\bc2221##1\or\bc2122##1\or\bc1411##1%
    \or\bc1132##1\or\bc1231##1\or\bc1114##1\or\bc1312##1\or\bc1213##1%
    \or\bc3112##1\fi}%
  \def\R##1{\bgroup\let\next\1\let\1\0\let\0\next\L##1\egroup}%
  \def\G##1{\bgroup\let\bc\bcg\L##1\egroup}% reverse
  \def\bc##1##2##3##4##5{\stack{\0##1\1##2\0##3\1##4}##5}%
  \def\bcg##1##2##3##4##5{\stack{\0##4\1##3\0##2\1##1}##5}%
  \def\bcR##1##2##3##4##5##6{\R##1\R##2\R##3\R##4\R##5\R##6\11\01\11\09%
    \endgroup}%
  \stack{\09}#1\11\01\11\L#2%
  \ifcase#1\L#3\L#4\L#5\L#6\L#7\or\L#3\G#4\L#5\G#6\G#7%
    \or\L#3\G#4\G#5\L#6\G#7\or\L#3\G#4\G#5\G#6\L#7%
    \or\G#3\L#4\L#5\G#6\G#7\or\G#3\G#4\L#5\L#6\G#7%
    \or\G#3\G#4\G#5\L#6\L#7\or\G#3\L#4\G#5\L#6\G#7%
    \or\G#3\L#4\G#5\G#6\L#7\or\G#3\G#4\L#5\G#6\L#7%
  \fi\01\11\01\11\01\bcR}


\author{Bart Snapp}

\title{Lagrange's theorem}

\begin{document}
\begin{abstract}
  We prove Lagrange's theorem and explore its consequences.
\end{abstract}
\maketitle

\begin{definition}\index{index@$[G:S]$}
  Let $S$ be a subgroup of a finite group $G$, the number of left
  cosets of $S$ is called the \dfn{index} of $S$ in $G$, denoted
  \[
  [G:S] = \text{(number of left cosets)}.
  \]
\end{definition}



\begin{example}[Index of $\boldsymbol{2\pmb\Z$ in $\pmb\Z}$]
  The left cosets of $2\Z\subgp \Z$ are
  \begin{align*}
    0+2\Z &= \{\dots, -8,-6,-4,-2,0,2,4,6,8,\dots\},\\
    1+2\Z &= \{\dots, -7,-5,-3,-1,1,3,5,7,9,\dots\},
  \end{align*}
  hence $[\Z:2\Z] = 2$.
\end{example}

\begin{example}[Index of $\boldsymbol{5\pmb\Z$ in $\pmb\Z}$]
  The left cosets of $5\Z\subgp \Z$ are
  \begin{align*}
    0+5\Z &= \{\dots, -20,-15,-10,-5,0,5,10,15,20,\dots\},\\
    1+5\Z &= \{\dots, -19,-14,-9 ,-4,1,6,11,16,21,\dots\},\\
    2+5\Z &= \{\dots, -18,-13,-8 ,-3,2,7,12,17,22,\dots\},\\
    3+5\Z &= \{\dots, -17,-12,-7 ,-2,3,8,13,18,23,\dots\},\\
    4+5\Z &= \{\dots, -16,-11,-6 ,-1,4,9,14,19,24,\dots\},
  \end{align*}
  hence $[\Z:5\Z] = 5$.
\end{example}

\begin{exercise}
  Prove that $[\Z:n\Z]= n$.
\end{exercise}



\begin{example}[Index of  $\boldsymbol{\<r\>$ in $D_3}$]
  Consider again our partial Cayley table for $D_3$ with $\< r\>$ in
  the upper left-hand corner:
  \[
    \renewcommand{\arraystretch}{1.6}
    \begin{array}{c!{\vline width 2pt}cccccc}
      (D_3,\circ)& e                              & r                              & r^2                            & f                             & r f                                   & r^2 f  \\  \Xhline{2pt}
      e          & e\cellcolor{blue!12!white}     & r\cellcolor{blue!12!white}    & r^2\cellcolor{blue!12!white}   & f \cellcolor{red!12!white}    & r f \cellcolor{blue!33!red!24!white}   & r^2f \cellcolor{blue!66!red!24!white} \\  
      r                                  & r\cellcolor{blue!12!white}    & r^2\cellcolor{blue!12!white}   & e \cellcolor{blue!12!white}    & rf\cellcolor{red!12!white} & r^2f \cellcolor{blue!33!red!24!white}    & f \cellcolor{blue!66!red!24!white}   \\  
      r^2                                 & r^2\cellcolor{blue!12!white}   & e\cellcolor{blue!12!white}     & r\cellcolor{blue!12!white}    & r^2 f \cellcolor{red!12!white}   & f \cellcolor{blue!33!red!24!white} & rf \cellcolor{blue!66!red!24!white}   \\  
      f        & f \cellcolor{red!12!white}    & r^2f \cellcolor{red!12!white}   & r f \cellcolor{red!12!white} &     &   &   \\  
      r f       & rf \cellcolor{blue!33!red!24!white}   & f \cellcolor{blue!33!red!24!white} & r^2f \cellcolor{blue!33!red!24!white}   &    &      &     \\  
      r^2 f       & r^2 f\cellcolor{blue!66!red!24!white} & rf \cellcolor{blue!66!red!24!white}    & f \cellcolor{blue!66!red!24!white}  &     &     &      \\  
    \end{array}
    \]
    Recall the left cosets are below the subgroup. Note there are only
    two distinct left cosets. Here they are
    \begin{align*}
      \<r\> &= \{e, r, r^2\},\\
      f\<r\> &= \{f, rf, r^2f\}.
    \end{align*}
    From this we see that $[D_3: \<r\>] = 2$.
\end{example}


\begin{example}[Index of  $\boldsymbol{\<f\>$ in $D_3}$]
  Consider again our partial Cayley table for $D_3$ with $\< r\>$ in
  the upper left-hand corner:
      \[
    \renewcommand{\arraystretch}{1.6}
    \begin{array}{c!{\vline width 2pt}cccccc}
      (D_3,\circ)& e                         & f                              & rf                            & r^2f                             & r                                    & r^2  \\  \Xhline{2pt}
      e          & e\cellcolor{blue!12!white}     & f\cellcolor{blue!12!white}    & rf\cellcolor{red!12!white}   & r^2f \cellcolor{blue!33!red!24!white}     & r  \cellcolor{blue!66!red!24!white}   & r^2 \cellcolor{blue!36!white} \\  
      f                & f\cellcolor{blue!12!white}    & e\cellcolor{blue!12!white}   & r^2 \cellcolor{red!12!white}    & f\cellcolor{blue!33!red!24!white} & r^2f \cellcolor{blue!66!red!24!white}    & rf \cellcolor{blue!36!white}   \\  
      rf                & rf\cellcolor{red!12!white}   & r\cellcolor{red!12!white}     &    &    &  &     \\  
      r^2f       & r^2f \cellcolor{blue!33!red!24!white}   & r^2 \cellcolor{blue!33!red!24!white}   &   &      &   &   \\  
      r      & r \cellcolor{blue!66!red!24!white}   & rf \cellcolor{blue!66!red!24!white} &    &   &      &     \\  
      r^2       & r^2 \cellcolor{blue!36!white} & r^2f \cellcolor{blue!36!white}    &   &    &     &      \\  
    \end{array}
    \]
    Recall the left cosets are below the subgroup. Note, there are
    only $3$ distinct left cosets.  Here they are
    \begin{align*}
      \<f\> &= \{e, f\},\\
      r\<f\> &= \{r, rf\},\\
      r^2\<f\> &= \{r^2, r^2f\}.
    \end{align*}
    From this we see that $[D_3: \<f\>] = 3$.
\end{example}


\begin{exercise}
  Prove that if $G$ is a group and $S\subgp G$, then $[S:\<e\>] = |S|$.
\end{exercise}


Now we can decorate our lattice diagrams with the index of the
subgroup.

\begin{example}[Index of subgroups for $\boldsymbol{\pmb\Z_6}$]
   Here is the subgroup lattice diagram for $\Z_6$ along with the
   index of each subgroup:   
  \[
  \begin{tikzcd}
    & \Z_6 \ar[ld,-,"{2=[\Z_6:\<2\>]}",swap]  \ar[rdd,-,"{3=[\Z_6:\<3\>]}"] &       \\
    \<2\> \ar[rdd,-,"{3=[\<2\>:\<0\>]}",swap] &  &\\
   & &  \< 3\> \ar[ld,-,"{2=[\<3\>:\<0\>]}"]        \\   
    & \<0\> &
  \end{tikzcd}
  \]
\end{example}



\begin{example}[Index of subgroups for $\boldsymbol{\pmb\Z_8}$]
   Here is the subgroup lattice diagram for $\Z_8$ along with the
   index of each subgroup:   
   \[
   \begin{tikzcd}
     \Z_8  \ar[d,-,"{2 = [\Z_8:\<2\>]}"] \\
     \<2\> \ar[d,-,"{2 = [\<2\>:\<4\>]}"] \\
     \<4\> \ar[d,-,"{2 = [\<4\>:\<0\>]}"] \\   
     \<0\> 
\end{tikzcd}
\]
\end{example}


\begin{example}[Index of subgroups for $\boldsymbol{D_3}$]\index{symmetries!of the equilateral triangle}
  Here is the subgroup lattice diagram for $D_3$ along with the
   index of each subgroup:   
  \[
  \begin{tikzcd}
    & D_3 \ar[rdd,-,"3"]\ar[rrdd,-,"3"]\ar[rrrdd,-,"3"]  \ar[ld,-,"2",swap] & & &      \\
    \<r\> \ar[rdd,-,"3",swap]       &       &  & &  \\
    &       &  \< f\> \ar[ld,-,"2",swap]   &  \< rf\> \ar[lld,-,"2",swap]       &  \< r^2f\> \ar[llld,-,"2"]        \\   
    & \<e\> & & &
  \end{tikzcd}
  \]
\end{example}


\begin{example}[Index of subgroups for $\boldsymbol{D_4}$]
  Here is the subgroup lattice diagram for $D_4$ along with the index
  of each subgroup:
  \[
\begin{tikzcd}
   &    & D_4 \ar[rd,-,"2"] \ar[d,-,"2"] \ar[ld,-,"2",swap] &       \\
&\<r^2,f\>  \ar[rd,-,"2"] \ar[d,-,"2"] \ar[ld,-,"2",swap]& \<r \>  \ar[d,-,"2"]     & \<r^2,rf\> \ar[ld,-,"2",swap] \ar[d,-,"2"] \ar[rd,-,"2"]\\
\<f\>\ar[rrd,-,"2",swap]& \<r^2f\> \ar[rd,-,"2"]&\< r^2\> \ar[d,-,"2"]  &   \<rf\> \ar[ld,-,"2",swap]& \<r^3f\>\ar[lld,-,"2"]     \\   
  &     & \<e\> &  &
\end{tikzcd}
\]
\end{example}




\begin{example}[Index of subgroups for $\boldsymbol{Q_8}$]\index{quaternion group}
  Here is the subgroup lattice diagram for $Q_8$ along with the index
  of each subgroup:
  \[
  \begin{tikzcd}
    & Q_8 \ar[rd,-,"2"] \ar[d,-,"2"] \ar[ld,-,"2",swap] &       \\
    \<i\>  \ar[rd,-,"2",swap] & \<j\>  \ar[d,-,"2"]     & \<k\> \ar[ld,-,"2"]\\
    &\< -1\> \ar[d,-,"2"]&        \\   
    & \<1\> &
  \end{tikzcd}
  \]
\end{example}




\begin{example}[Index of subgroups for $\boldsymbol{A_4}$]\index{alternating group}
  Setting $\sigma = (1 \ 2 \ 3)$ and $\tau = (2 \ 3 \ 4)$, here is the
  subgroup lattice diagram for $A_4$ along with the index
  of each subgroup:
  \[
  \begin{tikzpicture}
  \node[scale=.6] at (0,0) {\begin{tikzcd}
                   &   & & A_4   \ar[lld,-,"3",swap] \ar[rdd,-,"4",swap] \ar[rrdd,-,"4",swap]  \ar[rrrdd,-,"4",swap]  \ar[rrrrdd,-,"4"]   & &  & &        \\
                   & \<\sigma\tau, \sigma^2\tau^2\> \ar[ldd,-,"2",swap] \ar[dd,-,"2"] \ar[rdd,-,"2"] &  &              & &  & &  \\
                   &                                  &            &   & \<\sigma\> \ar[ldd,-,"3",swap]& \<\tau\> \ar[lldd,-,"3",swap] & \<\ \tau\sigma^2\> \ar[llldd,-,"3",swap]& \< \sigma \tau\sigma^2 \>\ar[lllldd,-,"3"]       \\
   \<\sigma\tau\>\ar[drrr,-,"2",swap]  & \<\sigma^2\tau^2\> \ar[drr,-,"2"] & \<\sigma^2\tau\sigma^2\> \ar[dr,-,"2"]& & &  & &    \\
                   &                               &   & \< e\>       & & & &  
  \end{tikzcd}};
  \end{tikzpicture}
  \]
\end{example}




\begin{exercise}
  Draw lattice diagrams for $\Z_n$ for $n=2,3\dots, 12$, include the
  index of each subgroup containment.
\end{exercise}



And now we are ready to state and prove Lagrange's theorem.



\begin{theorem}[Lagrange's theorem]\index{Lagrange's theorem}\label{T:lagrange}
  If $G$ is a finite group with subgroup $S$, then 
  \[
  |G| = [G:S]\cdot |S|.
  \]
  \begin{proof}
    First note that by Lemma~\ref{L:CSG}, $|S| = |aS|$ for all
    $a\in G$. By Theorem~\ref{T:CPG}, $G$ is partitioned by the left
    cosets of $S$. This means that every element of $G$ is in exactly
    one left coset. Hence $|G| = [G:S]\cdot |S|$.
  \end{proof}
\end{theorem}

\textbf{We can witness the importance of Lagrange's theorem by its
corollaries.}

\begin{corollary}[Subgroup's order divides the group's order]
  If $G$ is a finite group with subgroup $S$, then the order of $S$
  divides the order of $G$.
\end{corollary}

\begin{corollary}[Element's order divides the group's order]\label{C:Leodgo}
  Let $G$ be a finite group with $a\in G$. In this case, the order of
  $a$ divides the order of $G$.
\end{corollary}


\begin{corollary}[Prime order group implies trivial subgroups]
  Let $G$ be a group where $|G|= p$, a prime number. Then the only
  subgroups of $G$ are itself and $\{e\}$.
\end{corollary}

\begin{corollary}[Prime order group implies cyclic]
  Let $G$ be a group where $|G|= p$, a prime number. Then $G$ is
  cyclic and any element $g\in G$, $g\ne e$, can act as a generator.
\end{corollary}


\begin{corollary}[Index is multiplicative]
  Let $S \subgp T \subgp G$ be subgroups of a finite group
  $G$:
  \[
  \begin{tikzcd}[column sep=0pt]
    G\ar[d,-,"{[G:T]}",swap] & & G\ar[dd,-,"{[G:S] = [G:T]\cdot [T:S]}"]\\
    T \ar[d,-,"{[T:S]}",swap] & \Rightarrow& \\
    S &  & S
  \end{tikzcd}
  \]
  Then
  \[
  [G:S] = [G:T]\cdot [T:S].
  \]
\end{corollary}


\begin{corollary}[Fermat's little theorem]\index{Fermat's little theorem}
  Let $p\in \N$ be a prime number. Given $a\in\Z_p$ where $a\ne 0$,
  \[
  a^{p-1}\equiv 1 \pmod{p}.
  \]
\end{corollary}


\begin{corollary}[Euler's little theorem]\index{Euler's little theorem}
  Let $n\in \N$. Given $a\in\Z_n$ where $a\ne 0$,
  \[
  a^{\eulerphi(n)}\equiv 1 \pmod{n}.
  \]
\end{corollary}



\begin{exercise}
  Let $G$ be a group with subgroups $H$ and $K$. If $|H|=7$ and $|K| =
  9$, what are the possible values for $|H\cap K|$?
\end{exercise}


\begin{exercise}
  Let $G$ be a group with subgroups $H$ and $K$. If $|H|=15$ and $|K|
  = 24$, what are the possible values for $|H\cap K|$?
\end{exercise}


\begin{exercise}
  Let $G$ be a group with subgroups $H$ and $K$. If $|H|=12$ and $|K|
  = 24$, what are the possible values for $|H\cap K|$?
\end{exercise}

\begin{exercise}
  Suppose that $K$ is a proper subgroup of $H$, and $H$ is a proper
  subgroup of $G$. If $|K| =12$ and $|G| = 108$, what are the possible
  orders of $H$?
\end{exercise}


\section{Converse to Lagrange?}


The converse to Lagrange's theorem would be:
\begin{quote}
  If $G$ is a group and $n$ divides $|G|$, then $G$ has a subgroup of
  order $n$.
\end{quote}

You may wonder, is the converse of Lagrange's theorem true? The answer
is ``no.''

\begin{example}[Counterexample to converse of Lagrange]\index{tetrahedral group}\index{alternating group}\index{symmetries!of the regular tetrahedron}
  Consider a group $G\iso A_4\iso \mathcal{T}_{12}$. The order of $G$
  is $12$, yet $G$ has no subgroup of order $6$.
  \begin{proof}
    Seeking a contradiction, suppose that $A_4$ has a subgroup
    $H\subgp A_4$ of order $6$. In this case, by Lagrange's theorem, we
    see that
    \[
    [A_4:H] = 2.
    \]
    We claim that for all $a\in A_4$, $a^2\in H$. If $a\in H$, this is
    clear, as $H$ is closed under the group operation. Assume that
    $a\notin H$ and $a^2\notin H$. Then since there are only $2$ left
    cosets of $H$ in $A_4$ and the cosets partition the group by
    Theorem~\ref{T:CPG}, we see that
    \[
    a^2\in aH.
    \]
    But this means
    \begin{align*}
      a^2 &= ah\\
      a   &= h,
    \end{align*}
    meaning $a\in H$, a contradiction. Thus for all $a\in A_4$,
    $a^2\in H$. But this cannot be, as each $3$-cycle in $A_4$ is the
    square of some element and there are eight distinct $3$-cycles in
    $A_4$:
    \[
    (1\ 2\ 3), (1\ 3\ 2), (1\ 2\ 4), (1\ 4\ 2),
    (1\ 3\ 4), (1\ 4\ 3), (2\ 3\ 4), (2\ 4\ 3),
    \]
    Since $8>6$, we have arrived at a contradiction.
  \end{proof}
\end{example}

Now that we know there is no converse to Lagrange's theorem, we will
state several partial converses.


\begin{exercise}
  Let $G$ be a cyclic of order $n$, then if $d|n$, $G$ has a subgroup
  of order $d$.
\end{exercise}




The theorem below provides a partial converse. We will not prove
Sylow's theorem in this course, but it is good to know that such a
theorem exists.

\begin{theorem}[Sylow's theorem]
  Let $p$ be a prime number. If $p^n$ divides $|G|$, then for $k =
  1,\dots, n$ there exists subgroups $H_p, H_{p^2},\dots,H_{p^n}$ such
  that
  \[
  |H_{p^k}| = p^k.
  \]
  The subgroup $H_{p^n}$ is called a \textbf{Sylow
    $\boldsymbol{p}$-subgroup}\index{Sylow p-subgroup@Sylow $p$-subgroup}
\end{theorem}

\begin{corollary}[Prime order subgroups]
  Let $G$ be a group and $p$ be a prime number that divides
  $|G|$. Then $G$ has a subgroup of order $p$.
\end{corollary}

\begin{exercise}
  Find a Sylow $p$-subgroup of $S_3$, $S_4$, and $S_5$.
\end{exercise}





\section{Lagrange's question}

The formal definition of a group came well after the time of
Lagrange. This being the case, it is somewhat interesting that a
fundamental theorem in group theory bears his name. How did this
happen? Let's give a little history lesson. Given a quadratic equation
\[
ax^2 + bx + c = 0
\]
we've (as humans) have know the \textit{quadratic
  formula}\index{quadratic formula} for nearly $4000$ years. Here it
is
\[
x = \frac{-b \pm \sqrt{b^2-4ac}}{2a}
\]
Later Ferro and Tartaglia solved the cubic formula around $1500$ CE. Given,
\[
ax^3 + bx^2 + cx + d = 0
\]
we can find a solution with
\begin{gather}
\begin{align*}
  x &= \sqrt[3]{\left(\frac{-b^3}{27a^3}+\frac{bc}{6a^2}-\frac{d}{2a}\right)+\sqrt{\left(\frac{-b^3}{27a^3}+\frac{bc}{6a^2}-\frac{d}{2a}\right)^2 + \left(\frac{c}{3a}-\frac{b^2}{9a^2}\right)^3}} \\
    &+ \sqrt[3]{\left(\frac{-b^3}{27a^3}+\frac{bc}{6a^2}-\frac{d}{2a}\right)+\sqrt{\left(\frac{-b^3}{27a^3}+\frac{bc}{6a^2}-\frac{d}{2a}\right)^2 - \left(\frac{c}{3a}-\frac{b^2}{9a^2}\right)^3}} -\frac{b}{3a}.
\end{align*}
\end{gather}
This formula was popularized by Cardano in the mid $1500$s.  Around
this time, Cardano's student, Ferrari, joined the quest. He was able
to defeat Tartaglia in a ``mathematical duel'' and also discovered the
quartic formula.  Given
\[
ax^4 + bx^3 + cx^2 + dx + e = 0
\]
we can find solutions with
\begin{gather}
  \begin{align*} %% Copied from https://github.com/curtisbright/quartic/
    x &= \frac{1}{12a}\Big(-3b\pm\big(\sqrt{3\big(3b^2-8ac+2a\root3\of{4\big(2c^3-9bcd+27ad^2+27b^2e-72ace+\sqrt{(2c^3-9bcd+27ad^2+27b^2e-72ace)^2-4(c^2-3bd+12ae)^3}\big)}+2a\root3\of{4\big(2c^3-9bcd+27ad^2+27b^2e-72ace-\sqrt{(2c^3-9bcd+27ad^2+27b^2e-72ace)^2-4(c^2-3bd+12ae)^3}\big)}\big)}\\
    &\pm\sqrt{3\big(3b^2-8ac+2a\big({-1+\sqrt{-3}\over2}\big)\root3\of{4\big(2c^3-9bcd+27ad^2+27b^2e-72ace+\sqrt{(2c^3-9bcd+27ad^2+27b^2e-72ace)^2-4(c^2-3bd+12ae)^3}\big)}+2a\big({-1-\sqrt{-3}\over2}\big)\root3\of{4\big(2c^3-9bcd+27ad^2+27b^2e-72ace-\sqrt{(2c^3-9bcd+27ad^2+27b^2e-72ace)^2-4(c^2-3bd+12ae)^3}\big)}\big)}\big)\\
    &\pm\sgn\big(\big(\sgn\big(-b^3+4abc-8a^2d\big)-{1\over2}\big)\big(\sgn\big(\max((2c^3-9bcd+27ad^2+27b^2e-72ace)^2-4(c^2-3bd+12ae)^3,\min(3b^2-8ac,3b^4+16a^2c^2+16a^2bd-16ab^2c-64a^3e))\big)-{1\over2}\big)\big)\sqrt{3\big(3b^2-8ac+2a\big({-1-\sqrt{-3}\over2}\big)\root3\of{4\big(2c^3-9bcd+27ad^2+27b^2e-72ace+\sqrt{(2c^3-9bcd+27ad^2+27b^2e-72ace)^2-4(c^2-3bd+12ae)^3}\big)}+2a\big({-1+\sqrt{-3}\over2}\big)\root3\of{4\big(2c^3-9bcd+27ad^2+27b^2e-72ace-\sqrt{(2c^3-9bcd+27ad^2+27b^2e-72ace)^2-4(c^2-3bd+12ae)^3}\big)}\big)}\Big).
  \end{align*}
\end{gather}
If you, dear reader, cannot read the formula above, do not despair. I,
an author, cannot read it either. We will need a bigger sheet of
paper.  I leave this task to the happy children of the future.  Let us
also send our greetings to them, ``Hello children of the future.''



Two hundred years pass, and Lagrange enters the scene. Lagrange wanted
further formulas that would solve equations of degree $5$, $6$, and so
on. As part of this pursuit, Lagrange sought the answer to the
following question:


\begin{quote}
  Consider a polynomial of $n$ variables, if one permutes the
  variables, will the number of different polynomials found always
  divide $n!$?
\end{quote}

So, for example, if $F(x_1,x_2,x_3) = x+ y+z$, then every permutation
of the variables will be equal to $F(x_1,x_2,x_3)$, and $1|3$.

On the other hand, if $F(x_1,x_2,x_3) = x_1$, then we find $3$
different polynomials, and $3|3$.

Of course, $F$ need not be linear, and could be quite
complicated. Check out our next lemma.



\begin{lemma}[Permutations that fix $\boldsymbol F$ form a subgroup]
  Let $F(x_1,\dots,x_n)$ be a polynomial of $n$ variables.
  \[
  H = \{\epsilon\in S_n: F(x_{\epsilon(1)},\dots, x_{\epsilon(n)}) = F(x_1,\dots,x_n)\},
  \]
  then $H\subgp S_n$
  \begin{proof}
    First note that $e\in H$ and that $H\subset S_n$. Now note that if
    $\sigma\in H$, $\sigma^{-1}\in H$. Also, $\epsilon\circ \sigma^{-1}\in
    H$. Thus by the subgroup criterion, Theorem~\ref{T:sc}, $H\subgp
    S_n$.
  \end{proof}
\end{lemma}


\begin{lemma}[Cosets and polynomials]\label{L:cap}
  Let $F(x_1,\dots,x_n)$ be a polynomial of $n$ variables and consider
  the subgroup of $S_n$
  \[
  H = \{\epsilon\in S_n: F(x_{\epsilon(1)},\dots, x_{\epsilon(n)}) = F(x_1,\dots,x_n)\}.
  \]
  We have that 
  \[
  F(x_{\sigma(1)},\dots, x_{\sigma(n)}) = F(x_{\tau(1)},\dots, x_{\tau(n)})
  \]
  if and only if there exists $\pi \in S_n$ such that $\sigma,\tau\in\pi H$.
  \begin{proof}
    $(\Rightarrow)$ Suppose that
    \[
    F(x_{\sigma(1)},\dots, x_{\sigma(n)}) = F(x_{\tau(1)},\dots, x_{\tau(n)})
    \]
    Now, note that if $\epsilon \in H$,
    \[
    F(x_{\sigma\circ\epsilon(1)},\dots, x_{\sigma\circ\epsilon(n)}) =
    F(x_{\tau\circ\epsilon(1)},\dots,
    x_{\tau\circ\epsilon(n)}),
    \]
    meaning that $\sigma H \cap \tau H \ne \emptyset$. By
    Theorem~\ref{T:CPG}, $S_n$ is partitioned by the left cosets of
    $H$. This means that every permutation of $S_n$ is found in some
    left coset, and that cosets are either equal or disjoint. This
    means that both $\sigma$ and $\tau$ are in the same left coset of
    $H$. Hence there exists $\pi \sigma,\tau\in\pi H\in S_n$ such that
    $\sigma,\tau\in\pi H$.

    $(\Leftarrow)$ Suppose that $\sigma,\tau\in \pi H$. Then
    there exist $\epsilon_1,\epsilon_2\in H$ such that
    \begin{align*}
      F(x_{\sigma(1)},\dots, x_{\sigma(n)}) &=
    F(x_{\pi\circ\epsilon_1(1)},\dots,
    x_{\pi\circ\epsilon_1(n)}), \\
      F(x_{\tau(1)},\dots, x_{\tau(n)}) &=
    F(x_{\pi\circ\epsilon_2(1)},\dots,
    x_{\pi\circ\epsilon_2(n)}).
    \end{align*}
    Since the right-hand sides of both of these equations are equal,
    we have completed the proof of the lemma.
  \end{proof}
\end{lemma}

\begin{theorem}[A resolution to the question]
  Let $F(x_1,\dots,x_n)$ be a polynomial of $n$ variables and consider
  \[
  H = \{\sigma\in S_n: F(x_{\sigma(1)},\dots, x_{\sigma(n)}) = F(x_1,\dots,x_n)\}.
  \]
  The number of distinct polynomials found by permuting the
  variables of $F$ will divide $n!$.
    \begin{proof}
      From Lemma~\ref{L:cap}, we see that $[S_n:H]$ is the number of
      distinct polynomials found by permuting the variables. By
      Lagrange's theorem, Theorem~\ref{T:lagrange}, the index of a
      subgroup divides the order of the group. Since the order of
      $S_n$ is $n!$, we have proved the theorem.
      \end{proof}
\end{theorem}


However, mathematics is hard. Even with this answer Lagrange was
unable to find a formula for a degree $5$ equation, making (perhaps in
his eyes) his work a \textbf{failure}. But Lagrange, despite failing
in this one regard, \textbf{was not a failure.} He laid the
foundations for mathematics that he would not live to see.




For some interesting extra reading check out:
\begin{itemize}
\item \link[\textit{A characterization of the cyclic groups by subgroup indices}, G.\ Oman and T.\ Slattum, The College Math Journal, January (2016), 29--33]
  {https://www.jstor.org/stable/10.4169/college.math.j.47.1.29}.
\item \link[\textit{A history of Lagrange's theorem on groups},
  R.\ Roth, Mathematics Magazine, April (2001), 99--198]{https://www.jstor.org/stable/2690624}.
\end{itemize}

\end{document}
