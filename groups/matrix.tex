\documentclass{ximera}

\usepackage[T1]{fontenc}
\usepackage{stix2}
\usepackage{gillius}
\usepackage{resizegather}
%\usepackage{rsfso} fancy cal
\DeclareMathAlphabet{\mathcal}{OMS}{cmsy}{m}{n} %less fancy cal


\usepackage{multicol}


\usepackage{tikz-cd}
\usepackage{tkz-euclide} %% compass
\usetkzobj{all}  %% tkzCompass
\tikzset{>=stealth}
\tikzcdset{arrow style=tikz}
\usetikzlibrary{math} %% for assigning variables
%\usetikzlibrary{fadings}

\usepackage{colortbl,boldline,makecell} %% group tables


\usepackage[sans]{dsfont}

\usepackage{stmaryrd,pifont}

\graphicspath{
  {./}
  {fields/}
  }     



\let\oldbibliography\thebibliography%% to compact bib
\renewcommand{\thebibliography}[1]{%
  \oldbibliography{#1}%
  \setlength{\itemsep}{0pt}%
}
\renewcommand\refname{} %% no name needed!


\DefineVerbatimEnvironment{macaulay2}{Verbatim}{numbers=left,frame=lines,label=Macaulay2,labelposition=topline}

\DefineVerbatimEnvironment{gap}{Verbatim}{numbers=left,frame=lines,label=GAP,labelposition=topline}

%%% This next bit of code defines all our theorem environments
\makeatletter
\let\c@theorem\relax
\let\c@corollary\relax
%\let\c@example\relax
\makeatother

\let\definition\relax
\let\enddefinition\relax

\let\theorem\relax
\let\endtheorem\relax

\let\proposition\relax
\let\endproposition\relax

\let\exercise\relax
\let\endexercise\relax

\let\question\relax
\let\endquestion\relax

\let\remark\relax
\let\endremark\relax

\let\corollary\relax
\let\endcorollary\relax


\let\example\relax
\let\endexample\relax

\let\warning\relax
\let\endwarning\relax

\let\lemma\relax
\let\endlemma\relax


\let\algorithm\relax
\let\endalgorithm\relax
\usepackage{algpseudocode}

\newtheoremstyle{SlantTheorem}{\topsep}{\topsep}%%% space between body and thm
		{\slshape}                      %%% Thm body font
		{}                              %%% Indent amount (empty = no indent)
		{\bfseries\sffamily}            %%% Thm head font
		{}                              %%% Punctuation after thm head
		{3ex}                           %%% Space after thm head
		{\thmname{#1}\thmnumber{ #2}\thmnote{ \bfseries(#3)}}%%% Thm head spec
\theoremstyle{SlantTheorem}
\newtheorem{theorem}{Theorem}
%\newtheorem{definition}[theorem]{Definition}
%\newtheorem{proposition}[theorem]{Proposition}
%% \newtheorem*{dfnn}{Definition}
%% \newtheorem{ques}{Question}[theorem]
%% \newtheorem*{war}{WARNING}
%% \newtheorem*{cor}{Corollary}
%% \newtheorem*{eg}{Example}
\newtheorem*{remark}{Remark}
\newtheorem*{touchstone}{Touchstone}
\newtheorem{corollary}{Corollary}[theorem]
\newtheorem*{warning}{WARNING}
\newtheorem{example}[corollary]{Example}
\newtheorem{lemma}[theorem]{Lemma}




\newtheoremstyle{Definition}{\topsep}{\topsep}%%% space between body and thm
		{}                              %%% Thm body font
		{}                              %%% Indent amount (empty = no indent)
		{\bfseries\sffamily}            %%% Thm head font
		{}                              %%% Punctuation after thm head
		{3ex}                           %%% Space after thm head
		{\thmname{#1}\thmnumber{ #2}\thmnote{ \bfseries(#3)}}%%% Thm head spec
\theoremstyle{Definition}
\newtheorem{definition}[theorem]{Definition}



\let\algorithm\relax
\let\endalgorithm\relax
\newtheoremstyle{Alg}{\topsep}{\topsep}%%% space between body and thm
		{}                              %%% Thm body font
		{}                              %%% Indent amount (empty = no indent)
		{\bfseries\sffamily}            %%% Thm head font
		{}                              %%% Punctuation after thm head
		{3ex}                           %%% Space after thm head
		{\thmname{#1}\thmnumber{ #2}\thmnote{ \bfseries(#3)}}%%% Thm head spec
\theoremstyle{Alg}
\newtheorem*{algorithm}{Algorithm}
\newtheorem*{construction}{Construction}




\newtheoremstyle{Exercise}{\topsep}{\topsep} %%% space between body and thm
		{}                           %%% Thm body font
		{}                           %%% Indent amount (empty = no indent)
		{\bfseries\sffamily}         %%% Thm head font
		{)}                          %%% Punctuation after thm head
		{ }                          %%% Space after thm head
		{\thmnumber{#2}\thmnote{ \bfseries(#3)}}%%% Thm head spec
\theoremstyle{Exercise}
\newtheorem{exercise}[corollary]{}%[theorem]

%% \newtheoremstyle{Question}{\topsep}{\topsep} %%% space between body and thm
%% 		{\bfseries}                  %%% Thm body font
%% 		{3ex}                        %%% Indent amount (empty = no indent)
%% 		{}                           %%% Thm head font
%% 		{}                           %%% Punctuation after thm head
%% 		{}                           %%% Space after thm head
%% 		{\thmnumber{#2}\thmnote{ \bfseries(#3)}}%%% Thm head spec
\newtheoremstyle{Question}{3em}{3em} %%% space between body and thm
		{\large\bfseries}                           %%% Thm body font
		{}                           %%% Indent amount (empty = no indent)
		{}                         %%% Thm head font
		{}                          %%% Punctuation after thm head
		{0em}                          %%% Space after thm head
		{}%%% Thm head spec
\theoremstyle{Question}
\newtheorem*{question}{}






\renewcommand{\tilde}{\widetilde}
\renewcommand{\bar}{\overline}
\renewcommand{\hat}{\widehat}
\newcommand{\N}{\mathbb N}
\newcommand{\Z}{\mathbb Z}
\newcommand{\R}{\mathbb R}
\newcommand{\Q}{\mathbb Q}
\newcommand{\C}{\mathbb C}
\newcommand{\V}{\mathbb V}
\newcommand{\I}{\mathbb I}
\newcommand{\A}{\mathbb A}
\renewcommand{\o}{\mathbf o}
\newcommand{\iso}{\simeq}
\newcommand{\ph}{\varphi}
\newcommand{\Cf}{\mathcal{C}}
\newcommand{\IZ}{\mathrm{Int}(\Z)}
\newcommand{\dsum}{\oplus}
\newcommand{\directsum}{\bigoplus}
\newcommand{\union}{\bigcup}
\newcommand{\subgp}{\leq}
\newcommand{\normal}{\trianglelefteq}
\renewcommand{\i}{\mathfrak}
\renewcommand{\a}{\mathfrak{a}}
\renewcommand{\b}{\mathfrak{b}}
\newcommand{\m}{\mathfrak{m}}
\newcommand{\p}{\mathfrak{p}}
\newcommand{\q}{\mathfrak{q}}
\newcommand{\dfn}[1]{\textbf{#1}\index{#1}}
\let\hom\relax
\DeclareMathOperator{\mat}{Mat}
\DeclareMathOperator{\ann}{Ann}
\DeclareMathOperator{\h}{ht}
\DeclareMathOperator{\tr}{tr}
\DeclareMathOperator{\hom}{Hom}
\DeclareMathOperator{\Span}{Span}
\DeclareMathOperator{\spec}{Spec}
\DeclareMathOperator{\maxspec}{MaxSpec}
\DeclareMathOperator{\aut}{Aut}
\DeclareMathOperator{\ass}{Ass}
\DeclareMathOperator{\lcm}{lcm}
\DeclareMathOperator{\ff}{Frac}
\DeclareMathOperator{\im}{Im}
\DeclareMathOperator{\syz}{Syz}
\DeclareMathOperator{\gr}{Gr}
\DeclareMathOperator{\multideg}{multideg}
\renewcommand{\ker}{\mathop{\mathrm{Ker}}\nolimits}
\newcommand{\coker}{\mathop{\mathrm{Coker}}\nolimits}
\newcommand{\lps}{[\hspace{-0.25ex}[}
\newcommand{\rps}{]\hspace{-0.25ex}]}
\newcommand{\into}{\hookrightarrow}
\newcommand{\onto}{\twoheadrightarrow}
\newcommand{\tensor}{\otimes}
\newcommand{\x}{\mathbf{x}}
\newcommand{\X}{\mathbf X}
\newcommand{\Y}{\mathbf Y}
\renewcommand{\k}{\boldsymbol{\kappa}}
\renewcommand{\emptyset}{\varnothing}
\renewcommand{\qedsymbol}{$\blacksquare$}
\renewcommand{\l}{\ell}
\newcommand{\1}{\mathds{1}}
\newcommand{\lto}{\mathop{\longrightarrow\,}\limits}
\newcommand{\rad}{\sqrt}
\newcommand{\hf}{H}
\newcommand{\hs}{H\!S}
\newcommand{\hp}{H\!P}
\renewcommand{\vec}{\mathbf}
\let\temp\phi
\let\phi\varphi
\let\eulerphi\temp


\renewcommand{\epsilon}{\varepsilon}
\renewcommand{\subset}{\subseteq}
\renewcommand{\supset}{\supseteq}
\newcommand{\macaulay}{\normalfont\textsl{Macaulay2}}
\newcommand{\GAP}{\normalfont\textsf{GAP}}
\newcommand{\invlim}{\varprojlim}
\renewcommand{\le}{\leqslant}
\renewcommand{\ge}{\geqslant}
\newcommand{\valpha}{{\boldsymbol\alpha}}
\newcommand{\vbeta}{{\boldsymbol\beta}}
\newcommand{\vgamma}{{\boldsymbol\gamma}}
\newcommand{\dotp}{\bullet}
\newcommand{\lc}{\normalfont\textsc{lc}}
\newcommand{\lt}{\normalfont\textsc{lt}}
\newcommand{\lm}{\normalfont\textsc{lm}}
\newcommand{\from}{\leftarrow}
\newcommand{\transpose}{\intercal}
\newcommand{\grad}{\boldsymbol\nabla}
\newcommand{\curl}{\boldsymbol{\nabla\times}}
\renewcommand{\d}{\, d}
\newcommand{\<}{\langle}
\renewcommand{\>}{\rangle}

%\renewcommand{\proofname}{Sketch of Proof}


\renewenvironment{proof}[1][Proof]
  {\begin{trivlist}\item[\hskip \labelsep \itshape \bfseries #1{}\hspace{2ex}]\upshape}
{\qed\end{trivlist}}

\newenvironment{sketch}[1][Sketch of Proof]
  {\begin{trivlist}\item[\hskip \labelsep \itshape \bfseries #1{}\hspace{2ex}]\upshape}
{\qed\end{trivlist}}



\makeatletter
\renewcommand\section{\@startsection{paragraph}{10}{\z@}%
                                     {-3.25ex\@plus -1ex \@minus -.2ex}%
                                     {1.5ex \@plus .2ex}%
                                     {\normalfont\large\sffamily\bfseries}}
\renewcommand\subsection{\@startsection{subparagraph}{10}{\z@}%
                                    {3.25ex \@plus1ex \@minus.2ex}%
                                    {-1em}%
                                    {\normalfont\normalsize\sffamily\bfseries}}
\makeatother

%% Fix weird index/bib issue.
\makeatletter
\gdef\ttl@savemark{\sectionmark{}}
\makeatother


\makeatletter
%% no number for refs
\newcommand\frontstyle{%
  \def\activitystyle{activity-chapter}
  \def\maketitle{%
    \addtocounter{titlenumber}{1}%
                    {\flushleft\small\sffamily\bfseries\@pretitle\par\vspace{-1.5em}}%
                    {\flushleft\LARGE\sffamily\bfseries\@title \par }%
                    {\vskip .6em\noindent\textit\theabstract\setcounter{problem}{0}\setcounter{sectiontitlenumber}{0}}%
                    \par\vspace{2em}
                    \phantomsection\addcontentsline{toc}{section}{\textbf{\@title}}%
                  }}
\makeatother



\NewEnviron{annotate}{\vspace{-.3cm}\small \itshape \BODY \vspace{.3cm}}


%%%% TIKZ STUFF

%% N-GON code
\tikzset{
    pics/tikzngon/.style={
        code={
        \tikzmath{\xx = #1;\rr=1.7;}
        \draw[ultra thick,rounded corners=.05mm] ({\rr*sin(0*360/\xx)},{\rr*cos(0*360/\xx)})
        \foreach \x in {-1,0,...,\xx}
        {
        -- ({\rr*sin(\x*360/\xx)},{\rr*cos(\x*360/\xx)})
        }
           -- cycle;
  }}}

%% N-GON code (even)
\tikzset{
    pics/tikzegon/.style={
        code={
        \tikzmath{\xx = #1;\rr=1.7;}
        \draw[ultra thick,rounded corners=.05mm] ({\rr*sin(0*360/\xx+180/\xx)},{\rr*cos(0*360/\xx+180/\xx)})
        \foreach \x in {-1,0,...,\xx}
           {
           -- ({\rr*sin(\x*360/\xx+180/\xx)},{\rr*cos(\x*360/\xx+180/\xx)}) 
           }
           -- cycle;
  }}}




%% N-CLOCK code
\tikzset{
    pics/tikznclock/.style={
        code={
        \tikzmath{\xx = #1;\rr=1.7;\dd=.4;}
        \foreach \x in {1,...,\xx}
        \pgfmathtruncatemacro{\xy}{\x-1}
           {
             \node[circle,fill=black,inner sep=0pt, minimum size=13pt,text=white]
             at ({(\rr-\dd)*sin((\x-1)*360/(\xx)},{(\rr-\dd)*cos((\x-1)*360/\xx}) {\normalfont\bfseries\sffamily\small {\xy}};
           }
  \draw[thick] (0,0) circle (\rr);
  }}}



%% barcode from
%% https://tex.stackexchange.com/questions/6895/is-there-a-good-latex-package-for-generating-barcodes
%% NOT CURRENTLY USED!


\def\barcode#1#2#3#4#5#6#7{\begingroup%
  \dimen0=0.1em
  \def\stack##1##2{\oalign{##1\cr\hidewidth##2\hidewidth}}%
  \def\0##1{\kern##1\dimen0}%
  \def\1##1{\vrule height10ex width##1\dimen0}%
  \def\L##1{\ifcase##1\bc3211##1\or\bc2221##1\or\bc2122##1\or\bc1411##1%
    \or\bc1132##1\or\bc1231##1\or\bc1114##1\or\bc1312##1\or\bc1213##1%
    \or\bc3112##1\fi}%
  \def\R##1{\bgroup\let\next\1\let\1\0\let\0\next\L##1\egroup}%
  \def\G##1{\bgroup\let\bc\bcg\L##1\egroup}% reverse
  \def\bc##1##2##3##4##5{\stack{\0##1\1##2\0##3\1##4}##5}%
  \def\bcg##1##2##3##4##5{\stack{\0##4\1##3\0##2\1##1}##5}%
  \def\bcR##1##2##3##4##5##6{\R##1\R##2\R##3\R##4\R##5\R##6\11\01\11\09%
    \endgroup}%
  \stack{\09}#1\11\01\11\L#2%
  \ifcase#1\L#3\L#4\L#5\L#6\L#7\or\L#3\G#4\L#5\G#6\G#7%
    \or\L#3\G#4\G#5\L#6\G#7\or\L#3\G#4\G#5\G#6\L#7%
    \or\G#3\L#4\L#5\G#6\G#7\or\G#3\G#4\L#5\L#6\G#7%
    \or\G#3\G#4\G#5\L#6\L#7\or\G#3\L#4\G#5\L#6\G#7%
    \or\G#3\L#4\G#5\G#6\L#7\or\G#3\G#4\L#5\G#6\L#7%
  \fi\01\11\01\11\01\bcR}


\author{Bart Snapp}

\title{Matrix groups}

\begin{document}
\begin{abstract}
  Sets of matrices can form groups.
\end{abstract}
\maketitle

Matrices are functions between vector spaces. Let's prove that the
linear transformations of $\R$-vector spaces are matrices. Our proof
will be general enough that one could change the set of scalars.

\begin{lemma}[Matricies are linear transformations]\label{L:MT}
  The function $T: V \to W$ from an $m$-dimensional vector space
  $V$ to a $n$-dimensional vector space $W$ is a linear transformation
  if and only if can be represented by a $n\times m$ matrix.
  \begin{proof}
    $(\Rightarrow)$ Recall the definition of a \index{linear transformation}linear transformation,
    for $\vec{u},\vec{v}\in V$ and $s\in \R$:
    \begin{enumerate}
    \item $T(\vec{u}+\vec{v}) = T(\vec{u})+T(\vec{v})$.
    \item $T(s \vec{v}) = sT(\vec{v})$.
    \end{enumerate}
    We will show that any function on vector spaces having these
    properties can be expressed as a matrix over the set of scalars.

    The image of $T$ is completely determined by the action of
    $T$ on a basis. Let $\{\vec{b}_1,\dots,\vec{b}_m\}$ be a
    basis for $V$. Then if $\vec{v}\in V$, we may write
    \begin{align*}
      \vec{v} &= \begin{bmatrix}
        a_1\\
        \vdots \\
        a_m
        \end{bmatrix}\\
      &=a_1\vec{b}_1 + \dots + a_m\vec{b}_m.
    \end{align*}
    Now,
    \begin{align*}
      T(\vec{v})&=T(a_1\vec{b}_1 + \dots + a_m\vec{b}_m)\\
      &= T(a_1\vec{b}_1) + \dots + T(a_m\vec{b}_m) \\
      &= a_1T(\vec{b}_1) + \dots + a_mT(\vec{b}_m).
    \end{align*}
    Hence, to define a linear transformation $T$, we only need
    to know where $T$ maps each basis element. In this case, we
    can represent
    \[
    T = \begin{bmatrix}
      T(\vec{b}_1) & \cdots & T(\vec{b}_m)
    \end{bmatrix}
    \]
    where each $T(\vec{b}_i)$ is a column vector of length $n$.
    So, we may write
    \[
    T(\vec{v}) = \begin{bmatrix}
      T(\vec{b}_1) & \cdots & T(\vec{b}_m)
    \end{bmatrix} \begin{bmatrix}
        a_1\\
        \vdots \\
        a_m
        \end{bmatrix}.
    \]
    We have now shown that every linear transformation can be thought
    of as an $n\times m$ matrix.

      $(\Leftarrow)$ Since matrix multiplication distributes over
      vector addition,
      \[
      M(\vec{u}+\vec{v})  = M\cdot \vec{u} + M\cdot \vec{v}.
      \]
      Also, we know that with matrix multiplication,
      \[
      s M \vec{v} = M(s\vec{v}).
      \]
      Hence any $n\times m$ matrix is a linear transformation from an
      $m$-dimensional vector space to an $n$-dimensional vector
      space.
  \end{proof}
\end{lemma}



\begin{definition}
  The set $GL(n)$ is the set of invertible $n\times n$ matrices with
  entries in $\R$. This group is called the \dfn{general linear
    group}.
\end{definition}

\begin{exercise}
  Prove that $GL(n)$ is a group under matrix multiplication.
\end{exercise}



\begin{definition}
  A matrix $M:\R^n\to \R^n$ satisfying
  \[
  M^\transpose \cdot M=I
  \]
  is called an \dfn{orthogonal} matrix. We denote the set of $n\times
  n$ orthogonal matrices by $O(n)$.
\end{definition}


\begin{exercise}
  Prove that if $M\in O(n)$, then $\det(M) = \pm 1$.
\end{exercise}


\begin{exercise}
  Prove that $O(n)$ is a group under matrix multiplication.
\end{exercise}




\begin{lemma}[Isometries and orthogonality]
  A matrix $M$ is orthogonal if and only if it defines an isometry via
  \[
  \begin{bmatrix}
    x' \\ y' \\ z'
  \end{bmatrix}
  = M \begin{bmatrix} x \\ y \\ z\end{bmatrix}.
  \]
  \begin{sketch}
    Note that the square of the distance between the points $x_{1}$
    and $x_{2}$ is the dot product of the vector%
    \[
    \vec{v}=x_{2}-x_{1}%
    \]
    with itself.  Also recall the identity
    $(AB)^\transpose=B^\transpose A^\transpose$.
    
    $(\Rightarrow)$ If $M$ is orthogonal, write
    \[
    (M\vec{v}) \bullet (M\vec{v})
    \]
    and deduce that this equals ${\vec v}\bullet{\vec v}$.

    $(\Leftarrow)$ Suppose that $M$ defines an isometry. Explain why
    this means that
    \[
    ( M{\vec v}) \bullet ( M{\vec v})=
    {\vec v} \bullet {\vec v}
    \]
    for every ${\vec v}$.  Now rewrite as:
    \[
    \vec{v}^\transpose M^\transpose \cdot M\vec{v}=\vec{v}^\transpose
    \cdot{\vec v}.
    \]
    Write
    \[
    {\vec v} =
    \begin{bmatrix}
      a_1 \\ a_2 \\ \vdots \\ a_n
    \end{bmatrix}
    \]
     and view the equation 
    \[
    \vec{v}^\transpose M^\transpose \cdot M\vec{v}=\vec{v}^\transpose
    \cdot{\vec v}
    \]
    as a polynomial equation in the variables $a_1,\dots,a_n$. Since
    polynomials are equal if and only if their coefficients are equal
    this should finish the proof.
  \end{sketch}
\end{lemma}





\begin{definition}
  The set $SO(n)$ is the set of $n\times n$ matrices with entries in
  $\R$ and determinant equal to $1$. This is called the \dfn{special
    orthogonal group}.
\end{definition}

\begin{exercise}
  Prove that $SO(n)$ is a group under matrix multiplication.
\end{exercise}


\begin{exercise}
  Prove that:
  \[
  SO(n) \subset O(n) \subset GL(n)
  \]
\end{exercise}

\begin{lemma}[Classification of \textit{SO}(2)]\label{L:SO}
  If $M\in SO(2)$, then
  \[
  M =
  \begin{bmatrix}
    \cos(\theta) & -\sin(\theta) \\
    \sin(\theta) & \cos(\theta)
  \end{bmatrix}
  \]
  and $M$ is a rotation about the origin.
  \begin{proof}
    Let
    \[
    M=
    \begin{bmatrix}
      a & b \\
      c & d
    \end{bmatrix}.
    \]
    Since $SO(2) \subset O(2)$, we have that
    \[
    M^\transpose M =
    \begin{bmatrix}
      a^2+c^2 & ab+cd\\
      ab + cd & b^2+d^2
    \end{bmatrix}
    \]
    This means that $b= \pm c$ and $d = \mp a$. Hence
    \[
    M = \begin{bmatrix}
      a & -c \\
      c & a
    \end{bmatrix}
    \quad\text{or}\quad
    M = \begin{bmatrix}
      a & c \\
      c & -a
    \end{bmatrix}.
    \]
    However, we also know that
    \[
    \det(M)  = ad-bc = 1.
    \]
    Thus, 
    \[
     M = \begin{bmatrix}
      a & -c \\
      c & a
    \end{bmatrix}
     \]
     where $a^2 + c^2 = 1$. This means that
     \[
     M =
     \begin{bmatrix}
       \cos(\theta) & -\sin(\theta) \\
       \sin(\theta) & \cos(\theta)
     \end{bmatrix}
     \]
     and is a rotation. Note that we could have switched sine and
     cosine, changing the direction of the rotation.
  \end{proof}
\end{lemma}





\begin{theorem}[Classification of \textit{O}(2)]
  If $M\in O(2)$, then $M$ is a rotation about the origin or a
  reflection across a line through the origin.
  \begin{proof}
    Working exactly as in the proof of the classification of $SO(n)$,
    Lemma~\ref{L:SO}, we find
    \[
    M = \begin{bmatrix}
      a & -c \\
      c & a
    \end{bmatrix} \quad\text{or}\quad
    M = \begin{bmatrix}
      a & c \\
      c & -a
    \end{bmatrix}.
    \]
    The first is a rotation. The second can be viewed as the product
    \[
    \begin{bmatrix}
      \cos(\theta) & -\sin(\theta) \\
      \sin(\theta) & \cos(\theta)
    \end{bmatrix}
    \underbrace{\begin{bmatrix}
      1 & 0 \\
      0 & -1
    \end{bmatrix}}_{\text{reflection across $x$-axis}}
    \]
    for some value of $\theta$. Thus every matrix in $O(2)$ is a
    rotation about the origin or a reflection across a line through
    the origin.
  \end{proof}
\end{theorem}


\section{The quaternions}

Now we introduce a group that is fundamental for physics and computer
graphics called the \textit{quaternions}.

\begin{definition}
  The \dfn{quaternions} are elements in $GL(4)$:
  \begin{align*}
    1 &:=
    \begin{bmatrix}
      1 & 0 & 0 & 0 \\
      0 & 1 & 0 & 0 \\
      0 & 0 & 1 & 0 \\
      0 & 0 & 0 & 1
    \end{bmatrix}
    &
    i &:=
    \begin{bmatrix}
      0 & -1 & 0 &  0 \\
      1 &  0 & 0 &  0 \\
      0 &  0 & 0 & -1 \\
      0 &  0 & 1 &  0
    \end{bmatrix} \\
    j &:=
    \begin{bmatrix}
      0 &  0 & -1 &  0 \\
      0 &  0 &  0 &  1 \\
      1 &  0 &  0 &  0 \\
      0 & -1 &  0 &  0 
    \end{bmatrix}
    &
    k &:=
    \begin{bmatrix}
      0 &  0 &  0 & -1 \\
      0 &  0 & -1 &  0 \\
      0 &  1 &  0 &  0 \\
      1 &  0 &  0 &  0 
    \end{bmatrix}
  \end{align*}
\end{definition}

  With this set up, the quaternions follow the rules:
  \[
  i^2 = j^2 = k^2 = -1,
  \]
and
\begin{align*}
  ij &= k,  & jk &= i, & ki &= j, \\
  ji &= -k, & kj &= -i, & ik &= -j. \\
\end{align*}
As you go around counterclockwise in the picture below
\[
\begin{tikzpicture}
  \draw[ultra thick,->] (.94,-.34) arc (-20:80:1cm);
  \draw[ultra thick,->] (-.17,.98) arc (100:200:1cm);
  \draw[ultra thick,->] (-.77,-.64) arc (220:320:1cm);
  \node at (0,1) {$i$};
  \node at (-.87,-.5) {$j$};
  \node at (.87,-.5) {$k$};
\end{tikzpicture}
\]
the products are positive.

\begin{example}[The quaternion group]
  The eight elements
  \[
  \{e,1,-1,i,-i,j,-j,k,-k\}
  \]
  form a group under (matrix) multiplication. This group is known as
  the \dfn{quaternion group}, and is also called (and denoted)
  $Q_{8}$.
  \begin{proof}
    Let's check the conditions for this to be a group. We know that
    the operation is associative, since functional composition is
    associative (and matrix multiplication is function composition),
    Lemma~\ref{L:funCompAss}.

    There is an identity element, $1$.
    
    We must show that this set is closed under function composition.
    To see this we will make a Cayley table. Write with me:
    \begin{gather*}
    \renewcommand{\arraystretch}{1.6}
    \begin{array}{c!{\vline width 2pt}cccccccc}
      (Q_8,\cdot) & 1 & -1 \cellcolor{black!6!white} & i \cellcolor{red!6!white} & -i \cellcolor{red!12!white} & j \cellcolor{blue!6!white} & -j \cellcolor{blue!12!white} & k \cellcolor{yellow!6!white} & -k \cellcolor{yellow!12!white} \\ \Xhline{2pt}
      1 & 1 & -1 \cellcolor{black!6!white} & i \cellcolor{red!6!white} & -i \cellcolor{red!12!white} & j \cellcolor{blue!6!white} & -j \cellcolor{blue!12!white} & k \cellcolor{yellow!6!white} & -k \cellcolor{yellow!12!white} \\
      -1 \cellcolor{black!6!white}& -1 \cellcolor{black!6!white} & 1 & -i \cellcolor{red!12!white} & i \cellcolor{red!6!white} & -j \cellcolor{blue!12!white} & j \cellcolor{blue!6!white} & -k \cellcolor{yellow!12!white} & k \cellcolor{yellow!6!white} \\
      i \cellcolor{red!6!white}& i \cellcolor{red!6!white} & -i \cellcolor{red!12!white} & -1 \cellcolor{black!6!white}  & 1 & k \cellcolor{yellow!6!white} & -k \cellcolor{yellow!12!white} & -j \cellcolor{blue!12!white} & j \cellcolor{blue!6!white} \\
      -i \cellcolor{red!12!white}& -i \cellcolor{red!12!white} & i \cellcolor{red!6!white} & 1  & -1  \cellcolor{black!6!white}& -k \cellcolor{yellow!12!white} & k \cellcolor{yellow!6!white} & j \cellcolor{blue!6!white} & -j \cellcolor{blue!12!white} \\
      j \cellcolor{blue!6!white}& j \cellcolor{blue!6!white} & -j \cellcolor{blue!12!white} & -k \cellcolor{yellow!12!white}  & k\cellcolor{yellow!6!white} & -1 \cellcolor{black!6!white} & 1  & i \cellcolor{red!6!white} & -i \cellcolor{red!12!white} \\
      -j \cellcolor{blue!12!white}& -j \cellcolor{blue!12!white} & j \cellcolor{blue!6!white} & k \cellcolor{yellow!6!white}  & -k\cellcolor{yellow!12!white} & 1  & -1 \cellcolor{black!6!white} & -i \cellcolor{red!12!white} & i \cellcolor{red!6!white} \\
      k \cellcolor{yellow!6!white}& k \cellcolor{yellow!6!white} & -k \cellcolor{yellow!12!white} & j \cellcolor{blue!6!white}  & -j\cellcolor{blue!12!white} & -i \cellcolor{red!12!white} & i\cellcolor{red!6!white}  & -1 \cellcolor{black!6!white} & 1  \\
      -k \cellcolor{yellow!12!white}& -k \cellcolor{yellow!12!white} & k \cellcolor{yellow!6!white} & -j \cellcolor{blue!12!white}  & j\cellcolor{blue!6!white} & i \cellcolor{red!6!white} & -i\cellcolor{red!12!white}  & 1 & -1 \cellcolor{black!6!white}
    \end{array}
    \end{gather*}
    From this, we see that $Q_8$ is closed under function composition,
    and that every element has an inverse. Hence $Q_8$ is a group.
  \end{proof}
\end{example}

\begin{exercise}
  Find the order of every element of $Q_8$.
\end{exercise}


\begin{exercise}
   Show that the quaternion group $Q_8$ is not isomorphic to the
   dihedral group $D_4$.
\end{exercise}

\begin{exercise}
  Prove that $Q_8$ is generated by $\{i, j\}$ and make a Cayley table
  for $Q_8$ with entries
  \[
  e, i, i^2, i^3, j, ij, i^2j, i^3j.
  \]
\end{exercise}


For some interesting extra reading check out:
\begin{itemize}
\item \link[\textit{The origin of quaternions}, T.\ Bannon, The College Mathematics Journal Volume 46, 2015]{https://doi-org.proxy.lib.ohio-state.edu/10.4169/college.math.j.46.1.43}.
\item \link[\textit{Finite groups of $2\times 2$ integer matrices}, G.\ Mackiw, Mathematics Magazine, December, 1996]{https://www.maa.org/sites/default/files/George_Mackiw20823.pdf}.
\end{itemize}




\end{document}
