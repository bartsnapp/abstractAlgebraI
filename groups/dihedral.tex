\documentclass{ximera}

\usepackage[T1]{fontenc}
\usepackage{stix2}
\usepackage{gillius}
\usepackage{resizegather}
%\usepackage{rsfso} fancy cal
\DeclareMathAlphabet{\mathcal}{OMS}{cmsy}{m}{n} %less fancy cal


\usepackage{multicol}


\usepackage{tikz-cd}
\usepackage{tkz-euclide} %% compass
\usetkzobj{all}  %% tkzCompass
\tikzset{>=stealth}
\tikzcdset{arrow style=tikz}
\usetikzlibrary{math} %% for assigning variables
%\usetikzlibrary{fadings}

\usepackage{colortbl,boldline,makecell} %% group tables


\usepackage[sans]{dsfont}

\usepackage{stmaryrd,pifont}

\graphicspath{
  {./}
  {fields/}
  }     



\let\oldbibliography\thebibliography%% to compact bib
\renewcommand{\thebibliography}[1]{%
  \oldbibliography{#1}%
  \setlength{\itemsep}{0pt}%
}
\renewcommand\refname{} %% no name needed!


\DefineVerbatimEnvironment{macaulay2}{Verbatim}{numbers=left,frame=lines,label=Macaulay2,labelposition=topline}

\DefineVerbatimEnvironment{gap}{Verbatim}{numbers=left,frame=lines,label=GAP,labelposition=topline}

%%% This next bit of code defines all our theorem environments
\makeatletter
\let\c@theorem\relax
\let\c@corollary\relax
%\let\c@example\relax
\makeatother

\let\definition\relax
\let\enddefinition\relax

\let\theorem\relax
\let\endtheorem\relax

\let\proposition\relax
\let\endproposition\relax

\let\exercise\relax
\let\endexercise\relax

\let\question\relax
\let\endquestion\relax

\let\remark\relax
\let\endremark\relax

\let\corollary\relax
\let\endcorollary\relax


\let\example\relax
\let\endexample\relax

\let\warning\relax
\let\endwarning\relax

\let\lemma\relax
\let\endlemma\relax


\let\algorithm\relax
\let\endalgorithm\relax
\usepackage{algpseudocode}

\newtheoremstyle{SlantTheorem}{\topsep}{\topsep}%%% space between body and thm
		{\slshape}                      %%% Thm body font
		{}                              %%% Indent amount (empty = no indent)
		{\bfseries\sffamily}            %%% Thm head font
		{}                              %%% Punctuation after thm head
		{3ex}                           %%% Space after thm head
		{\thmname{#1}\thmnumber{ #2}\thmnote{ \bfseries(#3)}}%%% Thm head spec
\theoremstyle{SlantTheorem}
\newtheorem{theorem}{Theorem}
%\newtheorem{definition}[theorem]{Definition}
%\newtheorem{proposition}[theorem]{Proposition}
%% \newtheorem*{dfnn}{Definition}
%% \newtheorem{ques}{Question}[theorem]
%% \newtheorem*{war}{WARNING}
%% \newtheorem*{cor}{Corollary}
%% \newtheorem*{eg}{Example}
\newtheorem*{remark}{Remark}
\newtheorem*{touchstone}{Touchstone}
\newtheorem{corollary}{Corollary}[theorem]
\newtheorem*{warning}{WARNING}
\newtheorem{example}[corollary]{Example}
\newtheorem{lemma}[theorem]{Lemma}




\newtheoremstyle{Definition}{\topsep}{\topsep}%%% space between body and thm
		{}                              %%% Thm body font
		{}                              %%% Indent amount (empty = no indent)
		{\bfseries\sffamily}            %%% Thm head font
		{}                              %%% Punctuation after thm head
		{3ex}                           %%% Space after thm head
		{\thmname{#1}\thmnumber{ #2}\thmnote{ \bfseries(#3)}}%%% Thm head spec
\theoremstyle{Definition}
\newtheorem{definition}[theorem]{Definition}



\let\algorithm\relax
\let\endalgorithm\relax
\newtheoremstyle{Alg}{\topsep}{\topsep}%%% space between body and thm
		{}                              %%% Thm body font
		{}                              %%% Indent amount (empty = no indent)
		{\bfseries\sffamily}            %%% Thm head font
		{}                              %%% Punctuation after thm head
		{3ex}                           %%% Space after thm head
		{\thmname{#1}\thmnumber{ #2}\thmnote{ \bfseries(#3)}}%%% Thm head spec
\theoremstyle{Alg}
\newtheorem*{algorithm}{Algorithm}
\newtheorem*{construction}{Construction}




\newtheoremstyle{Exercise}{\topsep}{\topsep} %%% space between body and thm
		{}                           %%% Thm body font
		{}                           %%% Indent amount (empty = no indent)
		{\bfseries\sffamily}         %%% Thm head font
		{)}                          %%% Punctuation after thm head
		{ }                          %%% Space after thm head
		{\thmnumber{#2}\thmnote{ \bfseries(#3)}}%%% Thm head spec
\theoremstyle{Exercise}
\newtheorem{exercise}[corollary]{}%[theorem]

%% \newtheoremstyle{Question}{\topsep}{\topsep} %%% space between body and thm
%% 		{\bfseries}                  %%% Thm body font
%% 		{3ex}                        %%% Indent amount (empty = no indent)
%% 		{}                           %%% Thm head font
%% 		{}                           %%% Punctuation after thm head
%% 		{}                           %%% Space after thm head
%% 		{\thmnumber{#2}\thmnote{ \bfseries(#3)}}%%% Thm head spec
\newtheoremstyle{Question}{3em}{3em} %%% space between body and thm
		{\large\bfseries}                           %%% Thm body font
		{}                           %%% Indent amount (empty = no indent)
		{}                         %%% Thm head font
		{}                          %%% Punctuation after thm head
		{0em}                          %%% Space after thm head
		{}%%% Thm head spec
\theoremstyle{Question}
\newtheorem*{question}{}






\renewcommand{\tilde}{\widetilde}
\renewcommand{\bar}{\overline}
\renewcommand{\hat}{\widehat}
\newcommand{\N}{\mathbb N}
\newcommand{\Z}{\mathbb Z}
\newcommand{\R}{\mathbb R}
\newcommand{\Q}{\mathbb Q}
\newcommand{\C}{\mathbb C}
\newcommand{\V}{\mathbb V}
\newcommand{\I}{\mathbb I}
\newcommand{\A}{\mathbb A}
\renewcommand{\o}{\mathbf o}
\newcommand{\iso}{\simeq}
\newcommand{\ph}{\varphi}
\newcommand{\Cf}{\mathcal{C}}
\newcommand{\IZ}{\mathrm{Int}(\Z)}
\newcommand{\dsum}{\oplus}
\newcommand{\directsum}{\bigoplus}
\newcommand{\union}{\bigcup}
\newcommand{\subgp}{\leq}
\newcommand{\normal}{\trianglelefteq}
\renewcommand{\i}{\mathfrak}
\renewcommand{\a}{\mathfrak{a}}
\renewcommand{\b}{\mathfrak{b}}
\newcommand{\m}{\mathfrak{m}}
\newcommand{\p}{\mathfrak{p}}
\newcommand{\q}{\mathfrak{q}}
\newcommand{\dfn}[1]{\textbf{#1}\index{#1}}
\let\hom\relax
\DeclareMathOperator{\mat}{Mat}
\DeclareMathOperator{\ann}{Ann}
\DeclareMathOperator{\h}{ht}
\DeclareMathOperator{\tr}{tr}
\DeclareMathOperator{\hom}{Hom}
\DeclareMathOperator{\Span}{Span}
\DeclareMathOperator{\spec}{Spec}
\DeclareMathOperator{\maxspec}{MaxSpec}
\DeclareMathOperator{\aut}{Aut}
\DeclareMathOperator{\ass}{Ass}
\DeclareMathOperator{\lcm}{lcm}
\DeclareMathOperator{\ff}{Frac}
\DeclareMathOperator{\im}{Im}
\DeclareMathOperator{\syz}{Syz}
\DeclareMathOperator{\gr}{Gr}
\DeclareMathOperator{\multideg}{multideg}
\renewcommand{\ker}{\mathop{\mathrm{Ker}}\nolimits}
\newcommand{\coker}{\mathop{\mathrm{Coker}}\nolimits}
\newcommand{\lps}{[\hspace{-0.25ex}[}
\newcommand{\rps}{]\hspace{-0.25ex}]}
\newcommand{\into}{\hookrightarrow}
\newcommand{\onto}{\twoheadrightarrow}
\newcommand{\tensor}{\otimes}
\newcommand{\x}{\mathbf{x}}
\newcommand{\X}{\mathbf X}
\newcommand{\Y}{\mathbf Y}
\renewcommand{\k}{\boldsymbol{\kappa}}
\renewcommand{\emptyset}{\varnothing}
\renewcommand{\qedsymbol}{$\blacksquare$}
\renewcommand{\l}{\ell}
\newcommand{\1}{\mathds{1}}
\newcommand{\lto}{\mathop{\longrightarrow\,}\limits}
\newcommand{\rad}{\sqrt}
\newcommand{\hf}{H}
\newcommand{\hs}{H\!S}
\newcommand{\hp}{H\!P}
\renewcommand{\vec}{\mathbf}
\let\temp\phi
\let\phi\varphi
\let\eulerphi\temp


\renewcommand{\epsilon}{\varepsilon}
\renewcommand{\subset}{\subseteq}
\renewcommand{\supset}{\supseteq}
\newcommand{\macaulay}{\normalfont\textsl{Macaulay2}}
\newcommand{\GAP}{\normalfont\textsf{GAP}}
\newcommand{\invlim}{\varprojlim}
\renewcommand{\le}{\leqslant}
\renewcommand{\ge}{\geqslant}
\newcommand{\valpha}{{\boldsymbol\alpha}}
\newcommand{\vbeta}{{\boldsymbol\beta}}
\newcommand{\vgamma}{{\boldsymbol\gamma}}
\newcommand{\dotp}{\bullet}
\newcommand{\lc}{\normalfont\textsc{lc}}
\newcommand{\lt}{\normalfont\textsc{lt}}
\newcommand{\lm}{\normalfont\textsc{lm}}
\newcommand{\from}{\leftarrow}
\newcommand{\transpose}{\intercal}
\newcommand{\grad}{\boldsymbol\nabla}
\newcommand{\curl}{\boldsymbol{\nabla\times}}
\renewcommand{\d}{\, d}
\newcommand{\<}{\langle}
\renewcommand{\>}{\rangle}

%\renewcommand{\proofname}{Sketch of Proof}


\renewenvironment{proof}[1][Proof]
  {\begin{trivlist}\item[\hskip \labelsep \itshape \bfseries #1{}\hspace{2ex}]\upshape}
{\qed\end{trivlist}}

\newenvironment{sketch}[1][Sketch of Proof]
  {\begin{trivlist}\item[\hskip \labelsep \itshape \bfseries #1{}\hspace{2ex}]\upshape}
{\qed\end{trivlist}}



\makeatletter
\renewcommand\section{\@startsection{paragraph}{10}{\z@}%
                                     {-3.25ex\@plus -1ex \@minus -.2ex}%
                                     {1.5ex \@plus .2ex}%
                                     {\normalfont\large\sffamily\bfseries}}
\renewcommand\subsection{\@startsection{subparagraph}{10}{\z@}%
                                    {3.25ex \@plus1ex \@minus.2ex}%
                                    {-1em}%
                                    {\normalfont\normalsize\sffamily\bfseries}}
\makeatother

%% Fix weird index/bib issue.
\makeatletter
\gdef\ttl@savemark{\sectionmark{}}
\makeatother


\makeatletter
%% no number for refs
\newcommand\frontstyle{%
  \def\activitystyle{activity-chapter}
  \def\maketitle{%
    \addtocounter{titlenumber}{1}%
                    {\flushleft\small\sffamily\bfseries\@pretitle\par\vspace{-1.5em}}%
                    {\flushleft\LARGE\sffamily\bfseries\@title \par }%
                    {\vskip .6em\noindent\textit\theabstract\setcounter{problem}{0}\setcounter{sectiontitlenumber}{0}}%
                    \par\vspace{2em}
                    \phantomsection\addcontentsline{toc}{section}{\textbf{\@title}}%
                  }}
\makeatother



\NewEnviron{annotate}{\vspace{-.3cm}\small \itshape \BODY \vspace{.3cm}}


%%%% TIKZ STUFF

%% N-GON code
\tikzset{
    pics/tikzngon/.style={
        code={
        \tikzmath{\xx = #1;\rr=1.7;}
        \draw[ultra thick,rounded corners=.05mm] ({\rr*sin(0*360/\xx)},{\rr*cos(0*360/\xx)})
        \foreach \x in {-1,0,...,\xx}
        {
        -- ({\rr*sin(\x*360/\xx)},{\rr*cos(\x*360/\xx)})
        }
           -- cycle;
  }}}

%% N-GON code (even)
\tikzset{
    pics/tikzegon/.style={
        code={
        \tikzmath{\xx = #1;\rr=1.7;}
        \draw[ultra thick,rounded corners=.05mm] ({\rr*sin(0*360/\xx+180/\xx)},{\rr*cos(0*360/\xx+180/\xx)})
        \foreach \x in {-1,0,...,\xx}
           {
           -- ({\rr*sin(\x*360/\xx+180/\xx)},{\rr*cos(\x*360/\xx+180/\xx)}) 
           }
           -- cycle;
  }}}




%% N-CLOCK code
\tikzset{
    pics/tikznclock/.style={
        code={
        \tikzmath{\xx = #1;\rr=1.7;\dd=.4;}
        \foreach \x in {1,...,\xx}
        \pgfmathtruncatemacro{\xy}{\x-1}
           {
             \node[circle,fill=black,inner sep=0pt, minimum size=13pt,text=white]
             at ({(\rr-\dd)*sin((\x-1)*360/(\xx)},{(\rr-\dd)*cos((\x-1)*360/\xx}) {\normalfont\bfseries\sffamily\small {\xy}};
           }
  \draw[thick] (0,0) circle (\rr);
  }}}



%% barcode from
%% https://tex.stackexchange.com/questions/6895/is-there-a-good-latex-package-for-generating-barcodes
%% NOT CURRENTLY USED!


\def\barcode#1#2#3#4#5#6#7{\begingroup%
  \dimen0=0.1em
  \def\stack##1##2{\oalign{##1\cr\hidewidth##2\hidewidth}}%
  \def\0##1{\kern##1\dimen0}%
  \def\1##1{\vrule height10ex width##1\dimen0}%
  \def\L##1{\ifcase##1\bc3211##1\or\bc2221##1\or\bc2122##1\or\bc1411##1%
    \or\bc1132##1\or\bc1231##1\or\bc1114##1\or\bc1312##1\or\bc1213##1%
    \or\bc3112##1\fi}%
  \def\R##1{\bgroup\let\next\1\let\1\0\let\0\next\L##1\egroup}%
  \def\G##1{\bgroup\let\bc\bcg\L##1\egroup}% reverse
  \def\bc##1##2##3##4##5{\stack{\0##1\1##2\0##3\1##4}##5}%
  \def\bcg##1##2##3##4##5{\stack{\0##4\1##3\0##2\1##1}##5}%
  \def\bcR##1##2##3##4##5##6{\R##1\R##2\R##3\R##4\R##5\R##6\11\01\11\09%
    \endgroup}%
  \stack{\09}#1\11\01\11\L#2%
  \ifcase#1\L#3\L#4\L#5\L#6\L#7\or\L#3\G#4\L#5\G#6\G#7%
    \or\L#3\G#4\G#5\L#6\G#7\or\L#3\G#4\G#5\G#6\L#7%
    \or\G#3\L#4\L#5\G#6\G#7\or\G#3\G#4\L#5\L#6\G#7%
    \or\G#3\G#4\G#5\L#6\L#7\or\G#3\L#4\G#5\L#6\G#7%
    \or\G#3\L#4\G#5\G#6\L#7\or\G#3\G#4\L#5\G#6\L#7%
  \fi\01\11\01\11\01\bcR}


\author{Bart Snapp}

\title{Dihedral groups}

\begin{document}
\begin{abstract}
  We introduce dihedral groups.
\end{abstract}
\maketitle

\begin{theorem}[Automorphisms form a group]\label{E:aut0}
  Given a set $X$, prove that the collection of all bijections from
  $X$ to itself is a group under function composition.  This group is
  called the \dfn{automorphism group} of $X$, and is denoted
  \[
  \aut(X).
  \]
  \begin{sketch}
    Just check the definition of a group. First show $\aut(X)$ is
    closed under $\circ$. Next explain why the group operation is
    associative. Next find the identity element. Finally, explain why
    every element has an inverse.
  \end{sketch}
\end{theorem}


If you find $\aut(X)$ difficult to think about, don't worry. We'll
take small steps to understand this. 

To start, imagine an equilateral triangle sitting
on an empty $2$D-plane:
\[
\begin{tikzpicture}
  \pic {tikzngon={3}};
\end{tikzpicture}
\]
We are thinking about the points that lie on the lines that contain
the edges of the triangle.  Call this set of points, this
\dfn{geometric set}, $T$. In symbols we could write
\[
T = \{(x,y):\text{the point $(x,y)$ is on the triangle}\}
\]
Unfortunatly the group $\aut(T)$ seems much to hard (at least at this
point, to this author) to tackle. Instead, let me tell you something I
tell myself in hard times:

\begin{quote}
  For every problem I cannot solve, there's an easier problem that I
  cannot solve.
\end{quote}



So here's my proposal for an eaiser problem. Let's study a subset of
$\aut(T)$. Let's roll it way back to middle school. We'll study the
\textit{symmetries} of $T$. An informal definition of symmetry might be:
\begin{quote}
  Symmetry is immunity to a possible change.
\end{quote}
There are six symmetries of the equilateral triangle. Let's list
them. There are $3$ flips (reflections across lines) $f_1$, $f_2$,
$f_3$:
\[
\begin{tikzpicture}
  \tikzmath{\rrr=1.7;}
  \pic {tikzngon={3}};
  \draw[fill] ({\rrr*sin(0)},{\rrr*cos(0)}) circle (1mm) node[right] {$1$};
  \draw[fill] ({\rrr*sin(360/3)},{\rrr*cos(360/3)}) circle (1mm) node[right] {$3$};
  \draw[fill] ({\rrr*sin(2*360/3)},{\rrr*cos(2*360/3)}) circle (1mm) node[left] {$2$};

  \pic at (-4,-4) {tikzngon={3}};
  \draw[dashed] ({-4+sqrt(3)*\rrr/4},{\rrr/4-4}) --  ({-\rrr*sin(360/3)-4},{\rrr*cos(360/3)-4});
  \draw[fill] ({\rrr*sin(0)-4},{\rrr*cos(0)-4}) circle (1mm) node[right] {$3$};
  \draw[fill] ({\rrr*sin(360/3)-4},{\rrr*cos(360/3)-4}) circle (1mm) node[right] {$1$};
  \draw[fill] ({\rrr*sin(2*360/3)-4},{\rrr*cos(2*360/3)-4}) circle (1mm) node[left] {$2$};
  
  \pic at (0,-4) {tikzngon={3}};
  \draw[dashed] (0,{\rrr-4}) -- (0,{\rrr*cos(360/3)-4});
  \draw[fill] ({\rrr*sin(0)},{\rrr*cos(0)-4}) circle (1mm) node[right] {$1$};
  \draw[fill] ({\rrr*sin(360/3)},{\rrr*cos(360/3)-4}) circle (1mm) node[right] {$2$};
  \draw[fill] ({\rrr*sin(2*360/3)},{\rrr*cos(2*360/3)-4}) circle (1mm) node[left] {$3$};

  \pic at (4,-4) {tikzngon={3}};
  \draw[dashed] ({4-sqrt(3)*\rrr/4},{\rrr/4-4}) --  ({\rrr*sin(360/3)+4},{\rrr*cos(360/3)-4});
  \draw[fill] ({\rrr*sin(0)+4},{\rrr*cos(0)-4}) circle (1mm) node[right] {$2$};
  \draw[fill] ({\rrr*sin(360/3)+4},{\rrr*cos(360/3)-4}) circle (1mm) node[right] {$3$};
  \draw[fill] ({\rrr*sin(2*360/3)+4},{\rrr*cos(2*360/3)-4}) circle (1mm) node[left] {$1$};

  
  \draw[->] (0,-1.3) -- (0,-1.9); 
  \draw[->] (-2,-1.5) -- (-3,-2.5);
  \draw[->] (2,-1.5) -- (3,-2.5);

  \node[left] at (0,-1.6) {$f_2$};
  \node[above left] at (-2.5,-2) {$f_1$};
  \node[above right] at (2.5,-2) {$f_3$};
\end{tikzpicture}
\]


There are two counterclockwise rotations, $r_{120}$, $r_{240}$:
\[
\begin{tikzpicture}
  \tikzmath{\rrr=1.7;}
  \pic at (0,0) {tikzngon={3}};
  \draw[fill] ({\rrr*sin(0)},{\rrr*cos(0)}) circle (1mm) node[right] {$1$};
  \draw[fill] ({\rrr*sin(360/3)},{\rrr*cos(360/3)}) circle (1mm) node[right] {$3$};
  \draw[fill] ({\rrr*sin(2*360/3)},{\rrr*cos(2*360/3)}) circle (1mm) node[left] {$2$};


  \node[above] at (-2,.5) {$r_{120}$};
  \node[above] at (2,.5) {$r_{240}$};

  \pic at (-4,0) {tikzngon={3}};
  \draw[fill] ({\rrr*sin(0)-4},{\rrr*cos(0)}) circle (1mm) node[right] {$3$};
  \draw[fill] ({\rrr*sin(360/3)-4},{\rrr*cos(360/3)}) circle (1mm) node[right] {$2$};
  \draw[fill] ({\rrr*sin(2*360/3)-4},{\rrr*cos(2*360/3)}) circle (1mm) node[left] {$1$};
  \draw[dashed,->] (-3.5,0) arc (0:120:.5);
  
  \pic at (4,0) {tikzngon={3}};
  \draw[fill] ({\rrr*sin(0)+4},{\rrr*cos(0)}) circle (1mm) node[right] {$2$};
  \draw[fill] ({\rrr*sin(360/3)+4},{\rrr*cos(360/3)}) circle (1mm) node[right] {$1$};
  \draw[fill] ({\rrr*sin(2*360/3)+4},{\rrr*cos(2*360/3)}) circle (1mm) node[left] {$3$};
  \draw[dashed,->] (4.5,0) arc (0:240:.5);
  
  \draw[->] (-1,.5) -- (-3,.5);
  \draw[->] (1,.5) -- (3,.5);  

  
\end{tikzpicture}
\]
and an identity symmetry, the ``do nothing'' symmetry. Let's call it
$e$.

\begin{example}[The dihedral group of order six]
  The six symmetries
  \[
  \{e,r_{120},r_{240},f_1,f_2,f_3\}
  \]
  of the equilateral triangle form a group under function
  composition. This group is known as the \dfn{dihedral group}\textbf{
    of order six}, the \dfn{symmetries of the equilateral triangle},
  and is also called (and denoted) $D_3$.
  \begin{proof}
    Let's check the conditions for this to be a group. We know that
    the operation is associative, since functional composition is
    associative, Lemma~\ref{L:funCompAss}.

    There is an identity element, $e$. Every element we've seen so far
    has an inverse, namely the rotations are inverses of each other,
    and the flips are their own inverse.
    
    We must show that this set is closed under function composition.
    To see this we will make a Cayley table. We will express all the
    symmetries in terms of $e$, $r=r_{120}$, $f=f_2$, with the
    stipulation that when the rotation and flip are applied, we always
    apply the flip first. Hence
    \[
    r_{240} = r^2, \quad  f_1 = r^2\circ f, \quad  f_3 = r\circ f.
    \]
    In other words, $\{r,f\}$ generate $D_3$.  In our Cayley table
    below, let's ommit the $\circ$, so we will write something like
    $f\circ r$ as simply $fr$.
    \[
    \renewcommand{\arraystretch}{1.6}
    \begin{array}{c!{\vline width 2pt}cccccc}
      (D_3,\circ)& e     & r\cellcolor{blue!12!white}     & r^2\cellcolor{blue!24!white}   & f \cellcolor{red!12!white}    & r f \cellcolor{blue!33!red!24!white}  & r^2 f\cellcolor{blue!66!red!24!white} \\  \Xhline{2pt}
      e          & e     & r\cellcolor{blue!12!white}    & r^2\cellcolor{blue!24!white}   & f \cellcolor{red!12!white}    & r f \cellcolor{blue!33!red!24!white}   & r^2f \cellcolor{blue!66!red!24!white} \\  
      r\cellcolor{blue!12!white}         & r\cellcolor{blue!12!white}    & r^2\cellcolor{blue!24!white}   & e     & rf\cellcolor{blue!33!red!24!white} & r^2f \cellcolor{blue!66!red!24!white}    & f \cellcolor{red!12!white}   \\  
      r^2\cellcolor{blue!24!white}        & r^2\cellcolor{blue!24!white}   & e     & r\cellcolor{blue!12!white}    & r^2 f \cellcolor{blue!66!red!24!white}   & f \cellcolor{red!12!white} & rf \cellcolor{blue!33!red!24!white}    \\  
      f \cellcolor{red!12!white}         & f \cellcolor{red!12!white}    & r^2f \cellcolor{blue!66!red!24!white}   & r f \cellcolor{blue!33!red!24!white} & e     & r^2\cellcolor{blue!24!white}    & r\cellcolor{blue!12!white}   \\  
      r f \cellcolor{blue!33!red!24!white}        & rf \cellcolor{blue!33!red!24!white}   & f \cellcolor{red!12!white} & r^2f \cellcolor{blue!66!red!24!white}    & r\cellcolor{blue!12!white}   & e     & r^2\cellcolor{blue!24!white}    \\  
      r^2 f\cellcolor{blue!66!red!24!white}      & r^2 f\cellcolor{blue!66!red!24!white} & rf \cellcolor{blue!33!red!24!white}    & f \cellcolor{red!12!white}   & r^2\cellcolor{blue!24!white}    & r\cellcolor{blue!12!white}   & e     \\  
    \end{array}
    \]
    Note, from the table above we can see that $D_3$ is not Abelian.
  \end{proof}
\end{example}


If a set has symmetry, then that symmetry is a function that doesn't
change the set. In our case, the symmetries map $T$ to itself, and
they maintain the distance between points.  Automorphisms that
maintain the distance between points are called
\index{isometries}\textit{isometries}.

We can define a dihedral group for any regular $n$-gon.

\begin{definition}
  A \dfn{dihedral group} a group of symmetries of the regular
  $n$-gon. We denote this group as $D_n$.
\end{definition}



\begin{exercise}
  Make a Cayley table for $D_4$, the symmetries of the regular
  $4$-gon. Write the elements so that $f$ is applied before $r$, that is
  $r^nf$:
  \[
  \begin{tikzpicture} %% N-GON
    \pic {tikzegon={4}};
  \end{tikzpicture}
  \]
\end{exercise}


\begin{exercise}
  Make a Cayley table for $D_5$, the symmetries of the regular
  $5$-gon. Write the elements so that $f$ is applied before $r$, that is
  $r^nf$:
  \[
  \begin{tikzpicture} %% N-GON
    \pic {tikzngon={5}};
  \end{tikzpicture}
  \]
\end{exercise}


\begin{exercise}
  Make a Cayley table for $D_6$, the symmetries of the regular
  $6$-gon. Write the elements so that $f$ is applied before $r$, that
  is $r^nf$:
  \[
  \begin{tikzpicture} %% N-GON
    \pic {tikzegon={6}};
  \end{tikzpicture}
  \]
\end{exercise}

\begin{exercise}
  Give an argument explaining why in $2$D, every flip or rotation can
  be written in terms of one flip, and compostions of rotations.
  \begin{hint}
    Apply the flip first. 
  \end{hint}
\end{exercise}

\begin{exercise}
  Prove that $D_n$ is never Abelian.
\end{exercise}


\begin{exercise}\label{E:D2n}
  Let $S$ be a geometric set for a regular $n$-gon.
  \begin{enumerate}
  \item How many rotations are there in $\aut(S)$? 
  \item How many flips (reflections across lines) are there in
    $\aut(S)$?
  \item Find the order of $D_n$. 
  \end{enumerate}
  In each case, prove your answer is correct.
\end{exercise}





\section{Symmetries of the tetrahedron}

Now let's study the symmetries of the tetrahedron in $3$ dimensions.
As it turns out, these symmetries will form a group (though not a
dihedral group). This group will have some interesting properties.


A regular \dfn{tetrahedron} is a triangular based pyramid where each
side is an equilateral triangle.

\[
\begin{tikzpicture}
  \draw[ultra thick,rounded corners=.05mm] (0,0) -- (3,0) -- (1.5,2.8) -- cycle;
  \draw[ultra thick,rounded corners=.05mm] (3,1) -- (3,0) -- (1.5,2.8) -- cycle;
  \draw[ultra thick,dashed] (0,0) -- (3,1);
  %% \draw[fill] (3,1) circle (1mm) node[above right] {$1$};
  %% \draw[fill] (0,0) circle (1mm) node[left] {$2$};
  %% \draw[fill] (3,0) circle (1mm) node[right] {$3$};
  %% \draw[fill] (1.5,2.8) circle (1mm) node[above] {$4$};
\end{tikzpicture}
\]
Let's think about the symmetries of this tetrahedron. The first one
will be a counterclockwise rotation of the base:
\[
\begin{tikzpicture}
  \draw[ultra thick,rounded corners=.05mm] (0,0) -- (3,0) -- (1.5,2.8) -- cycle;
  \draw[ultra thick,rounded corners=.05mm] (3,1) -- (3,0) -- (1.5,2.8) -- cycle;
  \draw[ultra thick,dashed] (0,0) -- (3,1);
  \draw[fill] (3,1) circle (1mm) node[above right] {$1$};
  \draw[fill] (0,0) circle (1mm) node[left] {$2$};
  \draw[fill] (3,0) circle (1mm) node[right] {$3$};
  \draw[fill] (1.5,2.8) circle (1mm) node[above] {$4$};

  \draw[->] (3,2) -- (5,2);
  \node[above] at (4,2) {$r$};
  
  \draw[ultra thick,rounded corners=.05mm] (5,0) -- (8,0) -- (6.5,2.8) -- cycle;
  \draw[ultra thick,rounded corners=.05mm] (8,1) -- (8,0) -- (6.5,2.8) -- cycle;
  \draw[ultra thick,dashed] (5,0) -- (8,1);
  \draw[fill] (8,1) circle (1mm) node[above right] {$3$};
  \draw[fill] (5,0) circle (1mm) node[left] {$1$};
  \draw[fill] (8,0) circle (1mm) node[right] {$2$};
  \draw[fill] (6.5,2.8) circle (1mm) node[above] {$4$};
\end{tikzpicture}
\]
The next one will be a counterclockwise piviot around the far vertex. 
\[
\begin{tikzpicture}
  \draw[ultra thick,rounded corners=.05mm] (0,0) -- (3,0) -- (1.5,2.8) -- cycle;
  \draw[ultra thick,rounded corners=.05mm] (3,1) -- (3,0) -- (1.5,2.8) -- cycle;
  \draw[ultra thick,dashed] (0,0) -- (3,1);
  \draw[fill] (3,1) circle (1mm) node[above right] {$1$};
  \draw[fill] (0,0) circle (1mm) node[left] {$2$};
  \draw[fill] (3,0) circle (1mm) node[right] {$3$};
  \draw[fill] (1.5,2.8) circle (1mm) node[above] {$4$};

  \draw[->] (3,2) -- (5,2);
  \node[above] at (4,2) {$p$};
  
  \draw[ultra thick,rounded corners=.05mm] (5,0) -- (8,0) -- (6.5,2.8) -- cycle;
  \draw[ultra thick,rounded corners=.05mm] (8,1) -- (8,0) -- (6.5,2.8) -- cycle;
  \draw[ultra thick,dashed] (5,0) -- (8,1);
  \draw[fill] (8,1) circle (1mm) node[above right] {$1$};
  \draw[fill] (5,0) circle (1mm) node[left] {$4$};
  \draw[fill] (8,0) circle (1mm) node[right] {$2$};
  \draw[fill] (6.5,2.8) circle (1mm) node[above] {$3$};
\end{tikzpicture}
\]
From these two functions, we can write all the (orientation
preserving) symmetries of the tetrahedron:
\[
\{\underbrace{e,r,r^2}_{\text{$4$ on top}}, \underbrace{p,rp,r^2p}_{\text{$3$ on top}}, \underbrace{p^2,rp^2,r^2p^2}_{\text{$2$ on top}},\underbrace{pr^2,rpr^2,r^2pr^2}_{\text{$1$ on top}}\}
\]
Thus there are $12$ orientation preserving symmetries of the
tetrahedron.

\begin{example}[The tetrahedral group]
  The twelve symmetries
  \[
  \{e,r,r^2,p,rp,r^2p,p^2,rp^2,r^2p^2,pr^2,rpr^2,r^2pr^2\}
  \]
  of the regular tetrahedron form a group under function
  composition. This group is known as the \dfn{tetrahedral group}, the
  \dfn{symmetries of the regular tetrahedron}, and is also called (and
  denoted) $T_{12}$.
  \begin{proof}
    Let's check the conditions for this to be a group. We know that
    the operation is associative, since functional composition is
    associative, Lemma~\ref{L:funCompAss}.

    There is an identity element, $e$. Every element we've seen so far
    has an inverse, namely $r^3 = e$ and $p^3= e$. Any composition of
    $r$ and $p$ will have an inverse.
    
    We must show that this set is closed under function composition.
    To see this we will make a Cayley table. We will express all the
    symmetries in terms of
    \[
    \{e,r,r^2,p,rp,r^2p,p^2,rp^2,r^2p^2,pr^2,rpr^2,r^2pr^2\}
    \]
    In other words, $\{r,p\}$ generate $T_{12}$.  
    \begin{gather*}
    \renewcommand{\arraystretch}{1.6}
      \begin{array}{c!{\vline width 2pt}cccccccccccc}
      (T_{12},\circ)& e\cellcolor{red!6!white}     & r\cellcolor{red!12!white}     & r^2\cellcolor{red!24!white}   & p \cellcolor{yellow!6!white}    & rp \cellcolor{yellow!12!white}  & r^2 p\cellcolor{yellow!24!white} & p^2\cellcolor{green!6!white}    &  r p^2\cellcolor{green!12!white}  &  r^2 p^2\cellcolor{green!24!white} & pr^2 \cellcolor{blue!6!white} & rpr^2 \cellcolor{blue!12!white}& r^2 p r^2\cellcolor{blue!24!white}\\  \Xhline{2pt}
      e \cellcolor{red!6!white}         & e\cellcolor{red!6!white}         & r\cellcolor{red!12!white}           & r^2\cellcolor{red!24!white}   & p \cellcolor{yellow!6!white}    & rp \cellcolor{yellow!12!white}  & r^2 p\cellcolor{yellow!24!white} & p^2\cellcolor{green!6!white}    &  r p^2\cellcolor{green!12!white}  &  r^2 p^2\cellcolor{green!24!white} & pr^2 \cellcolor{blue!6!white} & rpr^2 \cellcolor{blue!12!white}& r^2 p r^2\cellcolor{blue!24!white}\\ 
      r\cellcolor{red!12!white}         & r\cellcolor{red!12!white}        & r^2\cellcolor{red!24!white}            & e\cellcolor{red!6!white}     & rp\cellcolor{yellow!12!white} & r^2p \cellcolor{yellow!24!white}    & p \cellcolor{yellow!6!white}   & rp^2 \cellcolor{green!12!white}  & r^2p^2 \cellcolor{green!24!white} & p^2 \cellcolor{green!6!white} & r pr^2 \cellcolor{blue!12!white} & r^2 pr^2 \cellcolor{blue!24!white} & pr^2 \cellcolor{blue!6!white}\\  
      r^2\cellcolor{red!24!white}       & r^2\cellcolor{red!24!white}      & e\cellcolor{red!6!white}           & r\cellcolor{red!12!white}    & r^2 p \cellcolor{yellow!24!white}   & p \cellcolor{yellow!6!white} & rp \cellcolor{yellow!12!white}   &  r^2p^2 \cellcolor{green!24!white}  & p^2 \cellcolor{green!6!white} & rp^2\cellcolor{green!12!white}  & r^2pr^2\cellcolor{blue!24!white} & pr^2\cellcolor{blue!6!white} & rpr^2\cellcolor{blue!12!white}\\  
      p \cellcolor{yellow!6!white}      & p \cellcolor{yellow!6!white}     & r^2p^2 \cellcolor{green!24!white}   & pr^2 \cellcolor{blue!6!white} & p^2\cellcolor{green!6!white}    & r^2\cellcolor{red!24!white}    & r^2pr^2\cellcolor{blue!24!white}  & e \cellcolor{red!6!white} & r^2p \cellcolor{yellow!24!white} &  rpr^2\cellcolor{blue!12!white}  &  rp \cellcolor{yellow!12!white}  &  r \cellcolor{red!12!white}  &  rp^2 \cellcolor{green!12!white}   \\
      r p \cellcolor{yellow!12!white}   & rp \cellcolor{yellow!12!white}   & p^2 \cellcolor{green!6!white}   & rpr^2 \cellcolor{blue!12!white}    & rp^2\cellcolor{green!12!white}   & e\cellcolor{red!6!white}     & pr^2\cellcolor{blue!6!white}  & r\cellcolor{red!12!white}  &  p \cellcolor{yellow!6!white}  & r^2pr^2\cellcolor{blue!24!white} & r^2p \cellcolor{yellow!24!white}  &  ??\cellcolor{white}  & ?? \cellcolor{white}  \\  
      r^2 p\cellcolor{yellow!24!white}  & r^2 p\cellcolor{yellow!24!white} & rp^2 \cellcolor{green!12!white}    & r^2 pr^2\cellcolor{blue!24!white}   & r^2p^2\cellcolor{green!24!white}    & r\cellcolor{red!12!white}   & rpr^2\cellcolor{blue!12!white} & r^2\cellcolor{red!24!white} &  rp \cellcolor{yellow!12!white}  &  pr^2 \cellcolor{blue!6!white}  & p \cellcolor{yellow!6!white}  &  ?? \cellcolor{white}  & ?? \cellcolor{white}    \\
      p^2\cellcolor{green!6!white}      & p^2\cellcolor{green!6!white} & rpr^2 \cellcolor{blue!12!white} & rp \cellcolor{yellow!12!white} & e\cellcolor{red!6!white} &  ?? \cellcolor{white}  &  ?? \cellcolor{white} &  p \cellcolor{yellow!6!white}  & ?? \cellcolor{white}  & r \cellcolor{red!12!white}   & r^2\cellcolor{red!24!white} &  ?? \cellcolor{white}  & ?? \cellcolor{white}\\
      rp^2\cellcolor{green!12!white}    & rp^2\cellcolor{green!12!white} & r^2pr^2\cellcolor{blue!24!white} & r^2p \cellcolor{yellow!24!white} & r\cellcolor{red!12!white}&  ?? \cellcolor{white}  &  ?? \cellcolor{white} &  rp \cellcolor{yellow!12!white}  & ?? \cellcolor{white}  & r^2 \cellcolor{red!24!white}   & e\cellcolor{red!6!white} &  ?? \cellcolor{white}  & ?? \cellcolor{white}\\
      r^2p^2\cellcolor{green!24!white}  & r^2p^2\cellcolor{green!24!white}& pr^2 \cellcolor{blue!6!white} & p\cellcolor{yellow!6!white} & r^2\cellcolor{red!24!white}&  ?? \cellcolor{white}  &  ?? \cellcolor{white} &  r^2p \cellcolor{yellow!24!white}  & ?? \cellcolor{white}  & e \cellcolor{red!6!white}   & r\cellcolor{red!12!white} &  ?? \cellcolor{white}  & ?? \cellcolor{white}\\
      pr^2\cellcolor{blue!6!white}      & pr^2\cellcolor{blue!6!white} & p\cellcolor{yellow!6!white} & r^2p^2 \cellcolor{green!24!white} & r^2pr^2\cellcolor{blue!24!white}& p^2\cellcolor{green!6!white}& r^2\cellcolor{red!24!white} & rpr^2\cellcolor{blue!12!white} & e\cellcolor{red!6!white}  &  r^2p \cellcolor{yellow!24!white}  & rp^2 \cellcolor{green!12!white}  & rp \cellcolor{yellow!12!white}  & r \cellcolor{red!12!white}\\
      rpr^2\cellcolor{blue!12!white}    & rpr^2\cellcolor{blue!12!white} & rp\cellcolor{yellow!12!white} & p^2\cellcolor{green!6!white} & pr^2\cellcolor{blue!6!white} & rp^2\cellcolor{green!12!white}& e\cellcolor{red!6!white} & r^2pr^2\cellcolor{blue!24!white} & r\cellcolor{red!12!white}  &  p \cellcolor{yellow!6!white}  & r^2p^2 \cellcolor{green!24!white}  &  r^2p \cellcolor{yellow!24!white}  & r^2 \cellcolor{red!24!white}\\
      r^2pr^2\cellcolor{blue!24!white}  & r^2pr^2\cellcolor{blue!24!white}& r^2p\cellcolor{yellow!24!white} & rp^2\cellcolor{green!12!white} & rpr^2\cellcolor{blue!12!white} & r^2p^2\cellcolor{green!24!white}& r\cellcolor{red!12!white} & pr^2\cellcolor{blue!6!white} & r^2\cellcolor{red!24!white}  &  rp \cellcolor{yellow!12!white}  & p^2 \cellcolor{green!6!white}  &  p \cellcolor{yellow!6!white}  & e \cellcolor{red!6!white}\\
    \end{array}
    \end{gather*}
    Note, from the table above we can see that $T_{12}$ is not Abelian.
  \end{proof}
\end{example}



MAYBE DEFINE ORDER HERE

\begin{exercise}
  Find the order of each element of $T_{12}$.
\end{exercise}


\begin{exercise}
  NON ORIENTATION PRESERVING 
\end{exercise}


\begin{question}
  This seems hard. What happens if we only study rotations in 2D?
\end{question}



\end{document}
