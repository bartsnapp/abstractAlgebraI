\documentclass{ximera}

\usepackage[T1]{fontenc}
\usepackage{stix2}
\usepackage{gillius}
\usepackage{resizegather}
%\usepackage{rsfso} fancy cal
\DeclareMathAlphabet{\mathcal}{OMS}{cmsy}{m}{n} %less fancy cal


\usepackage{multicol}


\usepackage{tikz-cd}
\tikzset{>=stealth}
\tikzcdset{arrow style=tikz}
\usetikzlibrary{math} %% for assigning variables
%\usetikzlibrary{fadings}

\usepackage{colortbl,boldline,makecell} %% group tables


\usepackage[sans]{dsfont}

\usepackage{stmaryrd,pifont}


\let\oldbibliography\thebibliography%% to compact bib
\renewcommand{\thebibliography}[1]{%
  \oldbibliography{#1}%
  \setlength{\itemsep}{0pt}%
}
\renewcommand\refname{} %% no name needed!


\DefineVerbatimEnvironment{macaulay2}{Verbatim}{numbers=left,frame=lines,label=Macaulay2,labelposition=topline}

\DefineVerbatimEnvironment{gap}{Verbatim}{numbers=left,frame=lines,label=GAP,labelposition=topline}

%%% This next bit of code defines all our theorem environments
\makeatletter
\let\c@theorem\relax
\let\c@corollary\relax
\let\c@example\relax
\makeatother

\let\definition\relax
\let\enddefinition\relax

\let\theorem\relax
\let\endtheorem\relax

\let\proposition\relax
\let\endproposition\relax

\let\exercise\relax
\let\endexercise\relax

\let\question\relax
\let\endquestion\relax

\let\remark\relax
\let\endremark\relax

\let\corollary\relax
\let\endcorollary\relax


\let\example\relax
\let\endexample\relax

\let\warning\relax
\let\endwarning\relax

\let\lemma\relax
\let\endlemma\relax


\let\algorithm\relax
\let\endalgorithm\relax
\usepackage{algpseudocode}

\newtheoremstyle{SlantTheorem}{\topsep}{\topsep}%%% space between body and thm
		{\slshape}                      %%% Thm body font
		{}                              %%% Indent amount (empty = no indent)
		{\bfseries\sffamily}            %%% Thm head font
		{}                              %%% Punctuation after thm head
		{3ex}                           %%% Space after thm head
		{\thmname{#1}\thmnumber{ #2}\thmnote{ \bfseries(#3)}}%%% Thm head spec
\theoremstyle{SlantTheorem}
\newtheorem{theorem}{Theorem}
%\newtheorem{definition}[theorem]{Definition}
%\newtheorem{proposition}[theorem]{Proposition}
%% \newtheorem*{dfnn}{Definition}
%% \newtheorem{ques}{Question}[theorem]
%% \newtheorem*{war}{WARNING}
%% \newtheorem*{cor}{Corollary}
%% \newtheorem*{eg}{Example}
\newtheorem*{remark}{Remark}
\newtheorem*{touchstone}{Touchstone}
\newtheorem{corollary}{Corollary}[theorem]
\newtheorem*{warning}{WARNING}
\newtheorem{example}{Example}[theorem]
\newtheorem{lemma}[theorem]{Lemma}




\newtheoremstyle{Definition}{\topsep}{\topsep}%%% space between body and thm
		{}                              %%% Thm body font
		{}                              %%% Indent amount (empty = no indent)
		{\bfseries\sffamily}            %%% Thm head font
		{}                              %%% Punctuation after thm head
		{3ex}                           %%% Space after thm head
		{\thmname{#1}\thmnumber{ #2}\thmnote{ \bfseries(#3)}}%%% Thm head spec
\theoremstyle{Definition}
\newtheorem{definition}[theorem]{Definition}



\let\algorithm\relax
\let\endalgorithm\relax
\newtheoremstyle{Alg}{\topsep}{\topsep}%%% space between body and thm
		{}                              %%% Thm body font
		{}                              %%% Indent amount (empty = no indent)
		{\bfseries\sffamily}            %%% Thm head font
		{}                              %%% Punctuation after thm head
		{3ex}                           %%% Space after thm head
		{\thmname{#1}\thmnumber{ #2}\thmnote{ \bfseries(#3)}}%%% Thm head spec
\theoremstyle{Alg}
\newtheorem*{algorithm}{Algorithm}




\newtheoremstyle{Exercise}{\topsep}{\topsep} %%% space between body and thm
		{}                           %%% Thm body font
		{}                           %%% Indent amount (empty = no indent)
		{\bfseries\sffamily}         %%% Thm head font
		{)}                          %%% Punctuation after thm head
		{ }                          %%% Space after thm head
		{\thmnumber{#2}\thmnote{ \bfseries(#3)}}%%% Thm head spec
\theoremstyle{Exercise}
\newtheorem{exercise}{}[theorem]

%% \newtheoremstyle{Question}{\topsep}{\topsep} %%% space between body and thm
%% 		{\bfseries}                  %%% Thm body font
%% 		{3ex}                        %%% Indent amount (empty = no indent)
%% 		{}                           %%% Thm head font
%% 		{}                           %%% Punctuation after thm head
%% 		{}                           %%% Space after thm head
%% 		{\thmnumber{#2}\thmnote{ \bfseries(#3)}}%%% Thm head spec
\newtheoremstyle{Question}{3em}{3em} %%% space between body and thm
		{\large\bfseries}                           %%% Thm body font
		{}                           %%% Indent amount (empty = no indent)
		{}                         %%% Thm head font
		{}                          %%% Punctuation after thm head
		{0em}                          %%% Space after thm head
		{}%%% Thm head spec
\theoremstyle{Question}
\newtheorem*{question}{}






\renewcommand{\tilde}{\widetilde}
\renewcommand{\bar}{\overline}
\renewcommand{\hat}{\widehat}
\newcommand{\N}{\mathbb N}
\newcommand{\Z}{\mathbb Z}
\newcommand{\R}{\mathbb R}
\newcommand{\Q}{\mathbb Q}
\newcommand{\C}{\mathbb C}
\newcommand{\V}{\mathbb V}
\newcommand{\I}{\mathbb I}
\newcommand{\A}{\mathbb A}
\renewcommand{\o}{\mathbf o}
\newcommand{\iso}{\simeq}
\newcommand{\ph}{\varphi}
\newcommand{\Cf}{\mathcal{C}}
\newcommand{\IZ}{\mathrm{Int}(\Z)}
\newcommand{\dsum}{\oplus}
\newcommand{\directsum}{\bigoplus}
\newcommand{\union}{\bigcup}
\newcommand{\subgp}{\leq}
\newcommand{\normal}{\trianglelefteq}
\renewcommand{\i}{\mathfrak}
\renewcommand{\a}{\mathfrak{a}}
\renewcommand{\b}{\mathfrak{b}}
\newcommand{\m}{\mathfrak{m}}
\newcommand{\p}{\mathfrak{p}}
\newcommand{\q}{\mathfrak{q}}
\newcommand{\dfn}[1]{\textbf{#1}\index{#1}}
\let\hom\relax
\DeclareMathOperator{\ann}{Ann}
\DeclareMathOperator{\h}{ht}
\DeclareMathOperator{\hom}{Hom}
\DeclareMathOperator{\Span}{Span}
\DeclareMathOperator{\spec}{Spec}
\DeclareMathOperator{\maxspec}{MaxSpec}
\DeclareMathOperator{\aut}{Aut}
\DeclareMathOperator{\ass}{Ass}
\DeclareMathOperator{\lcm}{lcm}
\DeclareMathOperator{\ff}{Frac}
\DeclareMathOperator{\im}{Im}
\DeclareMathOperator{\syz}{Syz}
\DeclareMathOperator{\gr}{Gr}
\DeclareMathOperator{\multideg}{multideg}
\renewcommand{\ker}{\mathop{\mathrm{Ker}}\nolimits}
\newcommand{\coker}{\mathop{\mathrm{Coker}}\nolimits}
\newcommand{\lps}{[\hspace{-0.25ex}[}
\newcommand{\rps}{]\hspace{-0.25ex}]}
\newcommand{\into}{\hookrightarrow}
\newcommand{\onto}{\twoheadrightarrow}
\newcommand{\tensor}{\otimes}
\newcommand{\x}{\mathbf{x}}
\newcommand{\X}{\mathbf X}
\newcommand{\Y}{\mathbf Y}
\renewcommand{\k}{\boldsymbol{\kappa}}
\renewcommand{\emptyset}{\varnothing}
\renewcommand{\qedsymbol}{$\blacksquare$}
\renewcommand{\l}{\ell}
\newcommand{\1}{\mathds{1}}
\newcommand{\lto}{\mathop{\longrightarrow\,}\limits}
\newcommand{\rad}{\sqrt}
\newcommand{\hf}{H}
\newcommand{\hs}{H\!S}
\newcommand{\hp}{H\!P}
\renewcommand{\vec}{\mathbf}
\let\temp\phi
\let\phi\varphi
\let\eulerphi\temp


\renewcommand{\epsilon}{\varepsilon}
\renewcommand{\subset}{\subseteq}
\renewcommand{\supset}{\supseteq}
\newcommand{\macaulay}{\normalfont\textsl{Macaulay2}}
\newcommand{\GAP}{\normalfont\textsf{GAP}}
\newcommand{\invlim}{\varprojlim}
\renewcommand{\le}{\leqslant}
\renewcommand{\ge}{\geqslant}
\newcommand{\valpha}{{\boldsymbol\alpha}}
\newcommand{\vbeta}{{\boldsymbol\beta}}
\newcommand{\vgamma}{{\boldsymbol\gamma}}
\newcommand{\dotp}{\bullet}
\newcommand{\lc}{\normalfont\textsc{lc}}
\newcommand{\lt}{\normalfont\textsc{lt}}
\newcommand{\lm}{\normalfont\textsc{lm}}
\newcommand{\from}{\leftarrow}
\newcommand{\transpose}{\intercal}
\newcommand{\grad}{\boldsymbol\nabla}
\newcommand{\curl}{\boldsymbol{\nabla\times}}
\renewcommand{\d}{\, d}
\newcommand{\<}{\langle}
\renewcommand{\>}{\rangle}

%\renewcommand{\proofname}{Sketch of Proof}


\renewenvironment{proof}[1][Proof]
  {\begin{trivlist}\item[\hskip \labelsep \itshape \bfseries #1{}\hspace{2ex}]\upshape}
{\qed\end{trivlist}}

\newenvironment{sketch}[1][Sketch of Proof]
  {\begin{trivlist}\item[\hskip \labelsep \itshape \bfseries #1{}\hspace{2ex}]\upshape}
{\qed\end{trivlist}}



\makeatletter
\renewcommand\section{\@startsection{paragraph}{10}{\z@}%
                                     {-3.25ex\@plus -1ex \@minus -.2ex}%
                                     {1.5ex \@plus .2ex}%
                                     {\normalfont\large\sffamily\bfseries}}
\renewcommand\subsection{\@startsection{subparagraph}{10}{\z@}%
                                    {3.25ex \@plus1ex \@minus.2ex}%
                                    {-1em}%
                                    {\normalfont\normalsize\sffamily\bfseries}}
\makeatother

%% Fix weird index/bib issue.
\makeatletter
\gdef\ttl@savemark{\sectionmark{}}
\makeatother


\makeatletter
%% no number for refs
\newcommand\frontstyle{%
  \def\activitystyle{activity-chapter}
  \def\maketitle{%
    \addtocounter{titlenumber}{1}%
                    {\flushleft\small\sffamily\bfseries\@pretitle\par\vspace{-1.5em}}%
                    {\flushleft\LARGE\sffamily\bfseries\@title \par }%
                    {\vskip .6em\noindent\textit\theabstract\setcounter{problem}{0}\setcounter{sectiontitlenumber}{0}}%
                    \par\vspace{2em}
                    \phantomsection\addcontentsline{toc}{section}{\textbf{\@title}}%
                  }}
\makeatother



\NewEnviron{annotate}{\vspace{-.3cm}\small \itshape \BODY \vspace{.3cm}}


%%%% TIKZ STUFF

%% N-GON code
\tikzset{
    pics/tikzngon/.style={
        code={
        \tikzmath{\xx = #1;\rr=1.7;}
        \draw[ultra thick,rounded corners=.05mm] ({\rr*sin(0*360/\xx)},{\rr*cos(0*360/\xx)})
        \foreach \x in {0,1,...,\xx+1}
           {
           -- ({\rr*sin(\x*360/\xx)},{\rr*cos(\x*360/\xx)}) 
           }
           -- cycle;
  }}}

%% N-GON code (even)
\tikzset{
    pics/tikzegon/.style={
        code={
        \tikzmath{\xx = #1;\rr=1.7;}
        \draw[ultra thick,rounded corners=.05mm] ({\rr*sin(0*360/\xx+180/\xx)},{\rr*cos(0*360/\xx+180/\xx)})
        \foreach \x in {0,1,...,\xx+1}
           {
           -- ({\rr*sin(\x*360/\xx+180/\xx)},{\rr*cos(\x*360/\xx+180/\xx)}) 
           }
           -- cycle;
  }}}




%% N-CLOCK code
\tikzset{
    pics/tikznclock/.style={
        code={
        \tikzmath{\xx = #1;\rr=1.7;\dd=.4;}
        \foreach \x in {0,1,...,\xx-1}
           {
             \node[circle,fill=black,inner sep=0pt, minimum size=13pt,text=white]
             at ({(\rr-\dd)*sin(\x*360/\xx)},{(\rr-\dd)*cos(\x*360/\xx}) {\normalfont\bfseries\sffamily\small \x};
           }
  \draw[thick] (0,0) circle (\rr);
  }}}



%% barcode from
%% https://tex.stackexchange.com/questions/6895/is-there-a-good-latex-package-for-generating-barcodes
%% NOT CURRENTLY USED!


\def\barcode#1#2#3#4#5#6#7{\begingroup%
  \dimen0=0.1em
  \def\stack##1##2{\oalign{##1\cr\hidewidth##2\hidewidth}}%
  \def\0##1{\kern##1\dimen0}%
  \def\1##1{\vrule height10ex width##1\dimen0}%
  \def\L##1{\ifcase##1\bc3211##1\or\bc2221##1\or\bc2122##1\or\bc1411##1%
    \or\bc1132##1\or\bc1231##1\or\bc1114##1\or\bc1312##1\or\bc1213##1%
    \or\bc3112##1\fi}%
  \def\R##1{\bgroup\let\next\1\let\1\0\let\0\next\L##1\egroup}%
  \def\G##1{\bgroup\let\bc\bcg\L##1\egroup}% reverse
  \def\bc##1##2##3##4##5{\stack{\0##1\1##2\0##3\1##4}##5}%
  \def\bcg##1##2##3##4##5{\stack{\0##4\1##3\0##2\1##1}##5}%
  \def\bcR##1##2##3##4##5##6{\R##1\R##2\R##3\R##4\R##5\R##6\11\01\11\09%
    \endgroup}%
  \stack{\09}#1\11\01\11\L#2%
  \ifcase#1\L#3\L#4\L#5\L#6\L#7\or\L#3\G#4\L#5\G#6\G#7%
    \or\L#3\G#4\G#5\L#6\G#7\or\L#3\G#4\G#5\G#6\L#7%
    \or\G#3\L#4\L#5\G#6\G#7\or\G#3\G#4\L#5\L#6\G#7%
    \or\G#3\G#4\G#5\L#6\L#7\or\G#3\L#4\G#5\L#6\G#7%
    \or\G#3\L#4\G#5\G#6\L#7\or\G#3\G#4\L#5\G#6\L#7%
  \fi\01\11\01\11\01\bcR}


\author{Bart Snapp}

\title{Dihedral groups}

\begin{document}
\begin{abstract}
  We introduce dihedral groups.
\end{abstract}
\maketitle

So here's my proposal for an eaiser problem. Let's study a subset of
$\aut(T)$. Let's roll it way back to middle school. We'll study the
\textit{symmetries} of $T$. There are six symmetries of the
equilateral triangle. Let's list them. There are $3$ flips
(reflections across lines) $f_1$, $f_2$, $f_3$:
\[
\begin{tikzpicture}
  \tikzmath{\rrr=1.7;}
  \pic {tikzngon={3}};
  \draw[fill] ({\rrr*sin(0)},{\rrr*cos(0)}) circle (1mm) node[right] {$A$};
  \draw[fill] ({\rrr*sin(360/3)},{\rrr*cos(360/3)}) circle (1mm) node[right] {$C$};
  \draw[fill] ({\rrr*sin(2*360/3)},{\rrr*cos(2*360/3)}) circle (1mm) node[left] {$B$};

  \pic at (-4,-4) {tikzngon={3}};
  \draw[dashed] ({-4+sqrt(3)*\rrr/4},{\rrr/4-4}) --  ({-\rrr*sin(360/3)-4},{\rrr*cos(360/3)-4});
  \draw[fill] ({\rrr*sin(0)-4},{\rrr*cos(0)-4}) circle (1mm) node[right] {$C$};
  \draw[fill] ({\rrr*sin(360/3)-4},{\rrr*cos(360/3)-4}) circle (1mm) node[right] {$A$};
  \draw[fill] ({\rrr*sin(2*360/3)-4},{\rrr*cos(2*360/3)-4}) circle (1mm) node[left] {$B$};
  
  \pic at (0,-4) {tikzngon={3}};
  \draw[dashed] (0,{\rrr-4}) -- (0,{\rrr*cos(360/3)-4});
  \draw[fill] ({\rrr*sin(0)},{\rrr*cos(0)-4}) circle (1mm) node[right] {$A$};
  \draw[fill] ({\rrr*sin(360/3)},{\rrr*cos(360/3)-4}) circle (1mm) node[right] {$B$};
  \draw[fill] ({\rrr*sin(2*360/3)},{\rrr*cos(2*360/3)-4}) circle (1mm) node[left] {$C$};

  \pic at (4,-4) {tikzngon={3}};
  \draw[dashed] ({4-sqrt(3)*\rrr/4},{\rrr/4-4}) --  ({\rrr*sin(360/3)+4},{\rrr*cos(360/3)-4});
  \draw[fill] ({\rrr*sin(0)+4},{\rrr*cos(0)-4}) circle (1mm) node[right] {$B$};
  \draw[fill] ({\rrr*sin(360/3)+4},{\rrr*cos(360/3)-4}) circle (1mm) node[right] {$C$};
  \draw[fill] ({\rrr*sin(2*360/3)+4},{\rrr*cos(2*360/3)-4}) circle (1mm) node[left] {$A$};

  
  \draw[->] (0,-1.3) -- (0,-1.9); 
  \draw[->] (-2,-1.5) -- (-3,-2.5);
  \draw[->] (2,-1.5) -- (3,-2.5);

  \node[left] at (0,-1.6) {$f_2$};
  \node[above left] at (-2.5,-2) {$f_1$};
  \node[above right] at (2.5,-2) {$f_3$};
\end{tikzpicture}
\]


There are two counterclockwise rotations, $r_{120}$, $r_{240}$:
\[
\begin{tikzpicture}
  \tikzmath{\rrr=1.7;}
  \pic at (0,0) {tikzngon={3}};
  \draw[fill] ({\rrr*sin(0)},{\rrr*cos(0)}) circle (1mm) node[right] {$A$};
  \draw[fill] ({\rrr*sin(360/3)},{\rrr*cos(360/3)}) circle (1mm) node[right] {$C$};
  \draw[fill] ({\rrr*sin(2*360/3)},{\rrr*cos(2*360/3)}) circle (1mm) node[left] {$B$};


  \node[above] at (-2,.5) {$r_{120}$};
  \node[above] at (2,.5) {$r_{240}$};

  \pic at (-4,0) {tikzngon={3}};
  \draw[fill] ({\rrr*sin(0)-4},{\rrr*cos(0)}) circle (1mm) node[right] {$C$};
  \draw[fill] ({\rrr*sin(360/3)-4},{\rrr*cos(360/3)}) circle (1mm) node[right] {$B$};
  \draw[fill] ({\rrr*sin(2*360/3)-4},{\rrr*cos(2*360/3)}) circle (1mm) node[left] {$A$};
  \draw[dashed,->] (-3.5,0) arc (0:120:.5);
  
  \pic at (4,0) {tikzngon={3}};
  \draw[fill] ({\rrr*sin(0)+4},{\rrr*cos(0)}) circle (1mm) node[right] {$B$};
  \draw[fill] ({\rrr*sin(360/3)+4},{\rrr*cos(360/3)}) circle (1mm) node[right] {$A$};
  \draw[fill] ({\rrr*sin(2*360/3)+4},{\rrr*cos(2*360/3)}) circle (1mm) node[left] {$C$};
  \draw[dashed,->] (4.5,0) arc (0:240:.5);
  
  \draw[->] (-1,.5) -- (-3,.5);
  \draw[->] (1,.5) -- (3,.5);  

  
\end{tikzpicture}
\]
and an identity symmetry, the ``do nothing'' symmetry. Let's call it
$e$.

\begin{theorem}[The Dihedral group of order six]
  The six symmetries
  \[
  \{e,r_{120},r_{240},f_1,f_2,f_3\}
  \]
  of the equilateral triangle form a group under function
  composition. This group is known as the \dfn{dihedral group}\textbf{
    of order six}, the \dfn{symmetries of the equilateral triangle},
  and is also called (and denoted) $D_3$.
  \begin{proof}
    Let's check the conditions for this to be a group. We know that
    the operation is associative, since functional composition is
    associative, Lemma~\ref{L:funCompAss}.

    There is an identity element, $e$. Every element we've seen so far
    has an inverse, namely the rotations are inverses of each other,
    and the flips are their own inverse.
    
    We must show that this set is closed under function composition.
    To see this we will make a Cayley table, also known as a
    \dfn{group table}. We will express all the symmetries in terms of
    $e$, $r=r_{120}$, $f=f_2$, with the stipulation that when the
    rotation and flip are applied, we always apply the rotation
    first. Hence
    \[
    r_{240} = r^2, \quad  f_1 = f\circ r, \quad  f_3 = f\circ r^2.
    \]
    In our complete multiplication below, let's ommit the $\circ$, so
    we will write something like $f\circ r$ as simply $fr$.
    \[
    \renewcommand{\arraystretch}{1.6}
    \begin{array}{|c||c|c|c|c|c|c|}\hline
      (D_3,\circ)& e     & r     & r^2   & f     & f r   & f r^2 \\  \hline\hline
      e          & e     & r     & r^2   & f     & f r   & f r^2 \\  \hline
      r          & r     & r^2   & e     & f r^2 & f     & f r   \\  \hline
      r^2        & r^2   & e     & r     & f r   & f r^2 & f     \\  \hline
      f          & f     & f r   & f r^2 & e     & r     & r^2   \\  \hline
      f r        & f r   & f r^2 & f     & r^2   & e     & r     \\  \hline
      f r^2      & f r^2 & f     & f r   & r     & r^2   & e     \\  \hline
    \end{array}
    \]
  \end{proof}
\end{theorem}



Roughly speaking, \textit{symmetry} breaks down as \textit{sym}
meaning \textit{with} and \textit{metry} meaning
\textit{measure}. Symmetries are automorphisms that maintain the
distance between points---also known as
\textit{isometries}\index{isometries}. To illustrate by example in
$\R^2$, we have a distance function
\[
d((x,y),(a,b)) = \sqrt{(x-a)^2+(y-b)^2}.
\]
So suppose you have an symmetry of the equilateral triangle above. Let
$(x,y)$ and $(a,b)$ be a points on the triangle.
\[
\begin{tikzpicture}
  \draw[ultra thick,rounded corners=.5] (0,0) -- (3,0) -- (1.5,2.6) -- cycle;
  \draw[fill] (1,1.72) circle (1mm) node[above left] {$(x,y)$};
  \draw[fill] (2.5,0.87) circle (1mm) node[right] {$(a,b)$};
\end{tikzpicture}
\]
suppose that $s$ is a symmetry on $T$, meaning $s:T\to T$
bijectively, and that 
\[
d((x,y),(a,b)) = d((s(x),s(y)),(s(a),s(b))).
\]
This means that if $s$ is a symmetry on $T$ we could have
\[
\begin{tikzpicture}
  \draw[ultra thick,rounded corners=.5] (0,0) -- (3,0) -- (1.5,2.6) -- cycle;
  \draw[fill] (1,1.72) circle (1mm) node[above left] {$(x,y)$};
  \draw[fill] (2.5,0.87) circle (1mm) node[right] {$(a,b)$};

  \draw[dashed] (1,1.72) -- (2.5,0.87);
  
  \draw[fill] (2,0) circle (1mm) node[below] {$(s(a),s(b))$};
  \draw[fill] (.5,0.87) circle (1mm) node[left] {$(s(x),s(y))$};

  \draw[dashed] (2,0) -- (.5,0.87);
\end{tikzpicture}
\]
as the distance as measured by the dashed lines in the diagram above
is the same. The function mapping $(x,y)$ and $(a,b)$ to different
verticies of the triangle would \textbf{not} be an symmetry, the
distance between the points in the image is greater than the distance
between the points $(x,y)$ and $(a,b)$. In fact, 


Now call the distance between these two points


Let me tell you what these are.

\begin{definition}
  Consider a set $X$ with a notion of distance. 
\end{definition}


Let's make it our short-term goal to understand a subset of
$\aut(T)$. Why a subset?


We'll start with an subgroup of
$\aut(X)$. Speaking of subgroups, 

\begin{exercise}
Here is a problem related to electronics. Given two resistors, one of
resistance $r$ and another of resistance $s$, the resistance of these
two resistors wired in parallel is given by:
\[
r \star s = \frac{1}{\frac{1}{r} + \frac{1}{s}}
\]
Prove that this operation is an associative operation on $\R-\{0\}$. 
\end{exercise}


Do subgroups? NO!
\end{document}
