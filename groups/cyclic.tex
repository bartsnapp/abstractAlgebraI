\documentclass{ximera}

\usepackage[T1]{fontenc}
\usepackage{stix2}
\usepackage{gillius}
\usepackage{resizegather}
%\usepackage{rsfso} fancy cal
\DeclareMathAlphabet{\mathcal}{OMS}{cmsy}{m}{n} %less fancy cal


\usepackage{multicol}


\usepackage{tikz-cd}
\tikzset{>=stealth}
\tikzcdset{arrow style=tikz}
\usetikzlibrary{math} %% for assigning variables
%\usetikzlibrary{fadings}

\usepackage{colortbl,boldline,makecell} %% group tables


\usepackage[sans]{dsfont}

\usepackage{stmaryrd,pifont}


\let\oldbibliography\thebibliography%% to compact bib
\renewcommand{\thebibliography}[1]{%
  \oldbibliography{#1}%
  \setlength{\itemsep}{0pt}%
}
\renewcommand\refname{} %% no name needed!


\DefineVerbatimEnvironment{macaulay2}{Verbatim}{numbers=left,frame=lines,label=Macaulay2,labelposition=topline}

\DefineVerbatimEnvironment{gap}{Verbatim}{numbers=left,frame=lines,label=GAP,labelposition=topline}

%%% This next bit of code defines all our theorem environments
\makeatletter
\let\c@theorem\relax
\let\c@corollary\relax
\let\c@example\relax
\makeatother

\let\definition\relax
\let\enddefinition\relax

\let\theorem\relax
\let\endtheorem\relax

\let\proposition\relax
\let\endproposition\relax

\let\exercise\relax
\let\endexercise\relax

\let\question\relax
\let\endquestion\relax

\let\remark\relax
\let\endremark\relax

\let\corollary\relax
\let\endcorollary\relax


\let\example\relax
\let\endexample\relax

\let\warning\relax
\let\endwarning\relax

\let\lemma\relax
\let\endlemma\relax


\let\algorithm\relax
\let\endalgorithm\relax
\usepackage{algpseudocode}

\newtheoremstyle{SlantTheorem}{\topsep}{\topsep}%%% space between body and thm
		{\slshape}                      %%% Thm body font
		{}                              %%% Indent amount (empty = no indent)
		{\bfseries\sffamily}            %%% Thm head font
		{}                              %%% Punctuation after thm head
		{3ex}                           %%% Space after thm head
		{\thmname{#1}\thmnumber{ #2}\thmnote{ \bfseries(#3)}}%%% Thm head spec
\theoremstyle{SlantTheorem}
\newtheorem{theorem}{Theorem}
%\newtheorem{definition}[theorem]{Definition}
%\newtheorem{proposition}[theorem]{Proposition}
%% \newtheorem*{dfnn}{Definition}
%% \newtheorem{ques}{Question}[theorem]
%% \newtheorem*{war}{WARNING}
%% \newtheorem*{cor}{Corollary}
%% \newtheorem*{eg}{Example}
\newtheorem*{remark}{Remark}
\newtheorem*{touchstone}{Touchstone}
\newtheorem{corollary}{Corollary}[theorem]
\newtheorem*{warning}{WARNING}
\newtheorem{example}{Example}[theorem]
\newtheorem{lemma}[theorem]{Lemma}




\newtheoremstyle{Definition}{\topsep}{\topsep}%%% space between body and thm
		{}                              %%% Thm body font
		{}                              %%% Indent amount (empty = no indent)
		{\bfseries\sffamily}            %%% Thm head font
		{}                              %%% Punctuation after thm head
		{3ex}                           %%% Space after thm head
		{\thmname{#1}\thmnumber{ #2}\thmnote{ \bfseries(#3)}}%%% Thm head spec
\theoremstyle{Definition}
\newtheorem{definition}[theorem]{Definition}



\let\algorithm\relax
\let\endalgorithm\relax
\newtheoremstyle{Alg}{\topsep}{\topsep}%%% space between body and thm
		{}                              %%% Thm body font
		{}                              %%% Indent amount (empty = no indent)
		{\bfseries\sffamily}            %%% Thm head font
		{}                              %%% Punctuation after thm head
		{3ex}                           %%% Space after thm head
		{\thmname{#1}\thmnumber{ #2}\thmnote{ \bfseries(#3)}}%%% Thm head spec
\theoremstyle{Alg}
\newtheorem*{algorithm}{Algorithm}




\newtheoremstyle{Exercise}{\topsep}{\topsep} %%% space between body and thm
		{}                           %%% Thm body font
		{}                           %%% Indent amount (empty = no indent)
		{\bfseries\sffamily}         %%% Thm head font
		{)}                          %%% Punctuation after thm head
		{ }                          %%% Space after thm head
		{\thmnumber{#2}\thmnote{ \bfseries(#3)}}%%% Thm head spec
\theoremstyle{Exercise}
\newtheorem{exercise}{}[theorem]

%% \newtheoremstyle{Question}{\topsep}{\topsep} %%% space between body and thm
%% 		{\bfseries}                  %%% Thm body font
%% 		{3ex}                        %%% Indent amount (empty = no indent)
%% 		{}                           %%% Thm head font
%% 		{}                           %%% Punctuation after thm head
%% 		{}                           %%% Space after thm head
%% 		{\thmnumber{#2}\thmnote{ \bfseries(#3)}}%%% Thm head spec
\newtheoremstyle{Question}{3em}{3em} %%% space between body and thm
		{\large\bfseries}                           %%% Thm body font
		{}                           %%% Indent amount (empty = no indent)
		{}                         %%% Thm head font
		{}                          %%% Punctuation after thm head
		{0em}                          %%% Space after thm head
		{}%%% Thm head spec
\theoremstyle{Question}
\newtheorem*{question}{}






\renewcommand{\tilde}{\widetilde}
\renewcommand{\bar}{\overline}
\renewcommand{\hat}{\widehat}
\newcommand{\N}{\mathbb N}
\newcommand{\Z}{\mathbb Z}
\newcommand{\R}{\mathbb R}
\newcommand{\Q}{\mathbb Q}
\newcommand{\C}{\mathbb C}
\newcommand{\V}{\mathbb V}
\newcommand{\I}{\mathbb I}
\newcommand{\A}{\mathbb A}
\renewcommand{\o}{\mathbf o}
\newcommand{\iso}{\simeq}
\newcommand{\ph}{\varphi}
\newcommand{\Cf}{\mathcal{C}}
\newcommand{\IZ}{\mathrm{Int}(\Z)}
\newcommand{\dsum}{\oplus}
\newcommand{\directsum}{\bigoplus}
\newcommand{\union}{\bigcup}
\newcommand{\subgp}{\leq}
\newcommand{\normal}{\trianglelefteq}
\renewcommand{\i}{\mathfrak}
\renewcommand{\a}{\mathfrak{a}}
\renewcommand{\b}{\mathfrak{b}}
\newcommand{\m}{\mathfrak{m}}
\newcommand{\p}{\mathfrak{p}}
\newcommand{\q}{\mathfrak{q}}
\newcommand{\dfn}[1]{\textbf{#1}\index{#1}}
\let\hom\relax
\DeclareMathOperator{\ann}{Ann}
\DeclareMathOperator{\h}{ht}
\DeclareMathOperator{\hom}{Hom}
\DeclareMathOperator{\Span}{Span}
\DeclareMathOperator{\spec}{Spec}
\DeclareMathOperator{\maxspec}{MaxSpec}
\DeclareMathOperator{\aut}{Aut}
\DeclareMathOperator{\ass}{Ass}
\DeclareMathOperator{\lcm}{lcm}
\DeclareMathOperator{\ff}{Frac}
\DeclareMathOperator{\im}{Im}
\DeclareMathOperator{\syz}{Syz}
\DeclareMathOperator{\gr}{Gr}
\DeclareMathOperator{\multideg}{multideg}
\renewcommand{\ker}{\mathop{\mathrm{Ker}}\nolimits}
\newcommand{\coker}{\mathop{\mathrm{Coker}}\nolimits}
\newcommand{\lps}{[\hspace{-0.25ex}[}
\newcommand{\rps}{]\hspace{-0.25ex}]}
\newcommand{\into}{\hookrightarrow}
\newcommand{\onto}{\twoheadrightarrow}
\newcommand{\tensor}{\otimes}
\newcommand{\x}{\mathbf{x}}
\newcommand{\X}{\mathbf X}
\newcommand{\Y}{\mathbf Y}
\renewcommand{\k}{\boldsymbol{\kappa}}
\renewcommand{\emptyset}{\varnothing}
\renewcommand{\qedsymbol}{$\blacksquare$}
\renewcommand{\l}{\ell}
\newcommand{\1}{\mathds{1}}
\newcommand{\lto}{\mathop{\longrightarrow\,}\limits}
\newcommand{\rad}{\sqrt}
\newcommand{\hf}{H}
\newcommand{\hs}{H\!S}
\newcommand{\hp}{H\!P}
\renewcommand{\vec}{\mathbf}
\let\temp\phi
\let\phi\varphi
\let\eulerphi\temp


\renewcommand{\epsilon}{\varepsilon}
\renewcommand{\subset}{\subseteq}
\renewcommand{\supset}{\supseteq}
\newcommand{\macaulay}{\normalfont\textsl{Macaulay2}}
\newcommand{\GAP}{\normalfont\textsf{GAP}}
\newcommand{\invlim}{\varprojlim}
\renewcommand{\le}{\leqslant}
\renewcommand{\ge}{\geqslant}
\newcommand{\valpha}{{\boldsymbol\alpha}}
\newcommand{\vbeta}{{\boldsymbol\beta}}
\newcommand{\vgamma}{{\boldsymbol\gamma}}
\newcommand{\dotp}{\bullet}
\newcommand{\lc}{\normalfont\textsc{lc}}
\newcommand{\lt}{\normalfont\textsc{lt}}
\newcommand{\lm}{\normalfont\textsc{lm}}
\newcommand{\from}{\leftarrow}
\newcommand{\transpose}{\intercal}
\newcommand{\grad}{\boldsymbol\nabla}
\newcommand{\curl}{\boldsymbol{\nabla\times}}
\renewcommand{\d}{\, d}
\newcommand{\<}{\langle}
\renewcommand{\>}{\rangle}

%\renewcommand{\proofname}{Sketch of Proof}


\renewenvironment{proof}[1][Proof]
  {\begin{trivlist}\item[\hskip \labelsep \itshape \bfseries #1{}\hspace{2ex}]\upshape}
{\qed\end{trivlist}}

\newenvironment{sketch}[1][Sketch of Proof]
  {\begin{trivlist}\item[\hskip \labelsep \itshape \bfseries #1{}\hspace{2ex}]\upshape}
{\qed\end{trivlist}}



\makeatletter
\renewcommand\section{\@startsection{paragraph}{10}{\z@}%
                                     {-3.25ex\@plus -1ex \@minus -.2ex}%
                                     {1.5ex \@plus .2ex}%
                                     {\normalfont\large\sffamily\bfseries}}
\renewcommand\subsection{\@startsection{subparagraph}{10}{\z@}%
                                    {3.25ex \@plus1ex \@minus.2ex}%
                                    {-1em}%
                                    {\normalfont\normalsize\sffamily\bfseries}}
\makeatother

%% Fix weird index/bib issue.
\makeatletter
\gdef\ttl@savemark{\sectionmark{}}
\makeatother


\makeatletter
%% no number for refs
\newcommand\frontstyle{%
  \def\activitystyle{activity-chapter}
  \def\maketitle{%
    \addtocounter{titlenumber}{1}%
                    {\flushleft\small\sffamily\bfseries\@pretitle\par\vspace{-1.5em}}%
                    {\flushleft\LARGE\sffamily\bfseries\@title \par }%
                    {\vskip .6em\noindent\textit\theabstract\setcounter{problem}{0}\setcounter{sectiontitlenumber}{0}}%
                    \par\vspace{2em}
                    \phantomsection\addcontentsline{toc}{section}{\textbf{\@title}}%
                  }}
\makeatother



\NewEnviron{annotate}{\vspace{-.3cm}\small \itshape \BODY \vspace{.3cm}}


%%%% TIKZ STUFF

%% N-GON code
\tikzset{
    pics/tikzngon/.style={
        code={
        \tikzmath{\xx = #1;\rr=1.7;}
        \draw[ultra thick,rounded corners=.05mm] ({\rr*sin(0*360/\xx)},{\rr*cos(0*360/\xx)})
        \foreach \x in {0,1,...,\xx+1}
           {
           -- ({\rr*sin(\x*360/\xx)},{\rr*cos(\x*360/\xx)}) 
           }
           -- cycle;
  }}}

%% N-GON code (even)
\tikzset{
    pics/tikzegon/.style={
        code={
        \tikzmath{\xx = #1;\rr=1.7;}
        \draw[ultra thick,rounded corners=.05mm] ({\rr*sin(0*360/\xx+180/\xx)},{\rr*cos(0*360/\xx+180/\xx)})
        \foreach \x in {0,1,...,\xx+1}
           {
           -- ({\rr*sin(\x*360/\xx+180/\xx)},{\rr*cos(\x*360/\xx+180/\xx)}) 
           }
           -- cycle;
  }}}




%% N-CLOCK code
\tikzset{
    pics/tikznclock/.style={
        code={
        \tikzmath{\xx = #1;\rr=1.7;\dd=.4;}
        \foreach \x in {0,1,...,\xx-1}
           {
             \node[circle,fill=black,inner sep=0pt, minimum size=13pt,text=white]
             at ({(\rr-\dd)*sin(\x*360/\xx)},{(\rr-\dd)*cos(\x*360/\xx}) {\normalfont\bfseries\sffamily\small \x};
           }
  \draw[thick] (0,0) circle (\rr);
  }}}



%% barcode from
%% https://tex.stackexchange.com/questions/6895/is-there-a-good-latex-package-for-generating-barcodes
%% NOT CURRENTLY USED!


\def\barcode#1#2#3#4#5#6#7{\begingroup%
  \dimen0=0.1em
  \def\stack##1##2{\oalign{##1\cr\hidewidth##2\hidewidth}}%
  \def\0##1{\kern##1\dimen0}%
  \def\1##1{\vrule height10ex width##1\dimen0}%
  \def\L##1{\ifcase##1\bc3211##1\or\bc2221##1\or\bc2122##1\or\bc1411##1%
    \or\bc1132##1\or\bc1231##1\or\bc1114##1\or\bc1312##1\or\bc1213##1%
    \or\bc3112##1\fi}%
  \def\R##1{\bgroup\let\next\1\let\1\0\let\0\next\L##1\egroup}%
  \def\G##1{\bgroup\let\bc\bcg\L##1\egroup}% reverse
  \def\bc##1##2##3##4##5{\stack{\0##1\1##2\0##3\1##4}##5}%
  \def\bcg##1##2##3##4##5{\stack{\0##4\1##3\0##2\1##1}##5}%
  \def\bcR##1##2##3##4##5##6{\R##1\R##2\R##3\R##4\R##5\R##6\11\01\11\09%
    \endgroup}%
  \stack{\09}#1\11\01\11\L#2%
  \ifcase#1\L#3\L#4\L#5\L#6\L#7\or\L#3\G#4\L#5\G#6\G#7%
    \or\L#3\G#4\G#5\L#6\G#7\or\L#3\G#4\G#5\G#6\L#7%
    \or\G#3\L#4\L#5\G#6\G#7\or\G#3\G#4\L#5\L#6\G#7%
    \or\G#3\G#4\G#5\L#6\L#7\or\G#3\L#4\G#5\L#6\G#7%
    \or\G#3\L#4\G#5\G#6\L#7\or\G#3\G#4\L#5\G#6\L#7%
  \fi\01\11\01\11\01\bcR}


\author{Bart Snapp}

\title{Cyclic groups}

\begin{document}
\begin{abstract}
  We introduce finite cyclic groups.
\end{abstract}
\maketitle

Now we will study groups that are generated by a single element. 

\begin{definition}
  A group $G$ is \dfn{cyclic} if it is generated by one element.


  If the operation for $G$ is $\star$, this means
  \[
  \< a\> := \left\{\bigstar_{i=1}^n a :n\in \Z\right\} = (G,\star).
  \]


  If the operation for $G$ is $\cdot$, this means
  \[
  \< a\> := \left\{\prod_{i=1}^n a :n\in \Z\right\} = (G,\cdot).
  \]



  If the operation for $G$ is $+$, this means
  \[
  \< a\> := \left\{\sum_{i=1}^n a :n\in \Z\right\} = (G,+).
  \]
\end{definition}


\begin{exercise}
  Prove that the group $(\Z,+)$ is a cyclic group by finding a
  generator.
\end{exercise}


\begin{exercise}
  Prove that the group $(C,\cdot)$ where $C=\{1,i,-1,-i\}$ is a cyclic
  group by finding a generator. See Example~\ref{E:C} and
  Example~\ref{E:CC}.
\end{exercise}


\begin{exercise}
  Prove that a finite group $G$ is cyclic if and only if there is an element
  $g\in G$ with $\o(g) = |G|$.
\end{exercise}



\begin{lemma}[Cyclic groups are Abelian]
  Prove that if $G$ is a cyclic group, then $G$ is Abelian.
  \begin{sketch}
    Write the elements in terms of the generator.
  \end{sketch}
\end{lemma}

In light of the last lemma, let us define a cyclic group using $+$,
the prototypical Abelian operation and focus on finite cyclic
groups. In this case, there is a \dfn{canonical} example, meaning
most-obvious example: The integers modulo a natural number. This may
sound fancy but it isn't. Imagine the numbers of a clock, where we
replace $12$ with $0$.
\[
\begin{tikzpicture} 
    \pic {tikznclock={12}};
\end{tikzpicture}
\]
We can add numbers on this clock, for example on this clock $3+4 = 7$
(what else would it be?) as we start at $3$ and then count clockwise
\[
\begin{tikzpicture} 
  \pic {tikznclock={12}};
  \foreach \x in {1,2,...,4}
           {
             \node[]
             at ({2*sin(\x*360/12+3*360/12)},{2*cos(\x*360/12 + 3*360/12}) {\small \x};
           }
\end{tikzpicture}
\]
$4$ more spaces to land on $7$. However, if we want to compute $7+9$,
we need start at $7$ and count clockwise 
\[
\begin{tikzpicture} 
  \pic {tikznclock={12}};
  \foreach \x in {1,2,...,9}
           {
             \node[]
             at ({2*sin(\x*360/12+7*360/12)},{2*cos(\x*360/12 + 7*360/12}) {\small \x};
           }
\end{tikzpicture}
\]
hence, in this case we would say $7+9$ is $4$, modulo $12$. Since
$7+9\ne 4$ in the integers, we usually write
\[
7+9 \equiv 4 \pmod{12}
\]
and say $7+9$ is \index{congruent modulo a!number}\textbf{congruent} to $4$ modulo $12$. We denote the group
of integers $\{0,1,2,\dots,11\}$ under modular addition as $\Z_{12}$.





\begin{exercise} Consider a clock with $2$ numbers:
  \[
  \begin{tikzpicture} 
    \pic {tikznclock={2}};
  \end{tikzpicture}
  \]
  Arithmetic on this clock is given by the group $(\Z_2,+)$. Make a
  Cayley table for $(\Z_2,+)$.
\end{exercise}

\begin{exercise}
  Use Cayley tables to show that $\Z_2\iso S$, where $S$ is the
  so-called ``sock group'' from Exercise~\ref{E:SG}.
\end{exercise}




\begin{exercise} Consider a clock with $3$ numbers:
  \[
  \begin{tikzpicture} 
    \pic {tikznclock={3}};
  \end{tikzpicture}
  \]
  Arithmetic on this clock is given by the group $(\Z_3,+)$. Make a
  Cayley table for $(\Z_3,+)$.
\end{exercise}

\begin{exercise} Consider a clock with $4$ numbers:
  \[
  \begin{tikzpicture} 
    \pic {tikznclock={4}};
  \end{tikzpicture}
  \]
  Arithmetic on this clock is given by the group $(\Z_4,+)$. Make a
  Cayley table for $(\Z_4,+)$.
\end{exercise}

\begin{exercise}
  Use Cayley tables to show that $\Z_4\iso C$, where $C$ is the group
  from Exercise~\ref{E:C}.
\end{exercise}


\begin{exercise} Consider a clock with $5$ numbers:
  \[
  \begin{tikzpicture} 
    \pic {tikznclock={5}};
  \end{tikzpicture}
  \]
  Arithmetic on this clock is given by the group $(\Z_5,+)$. Make a
  Cayley table for $(\Z_5,+)$.
\end{exercise}

\begin{exercise} Consider a clock with $6$ numbers:
  \[
  \begin{tikzpicture} 
    \pic {tikznclock={6}};
  \end{tikzpicture}
  \]
  Arithmetic on this clock is given by the group $(\Z_6,+)$. Make a
  Cayley table for $(\Z_6,+)$.
\end{exercise}


\begin{exercise}
  Both $\Z_6$ and $D_3$ are groups with $6$ elements. Is $\Z_6$
  isomorphic to $D_3$?
\end{exercise}




Modular arithmetic is closely related to division. The division
theorem below, establishes \textbf{the uniqueness and existence} of a
solution when computing grade-school long-division:
\[
d\,\begin{tabular}[b]{@{}r@{} r} $q$ &\, R$r$\\ \cline{1-1}
\big)\begin{tabular}[t]{@{}l@{}} $n$
\end{tabular}
\end{tabular}
\qquad\text{where}\qquad
\begin{tabular}{l}
$d$ is the divisor, \\
$n$ is the dividend, \\
$q$ is the quotient, \\
$r$ is the remainder.
\end{tabular}
\]


\begin{theorem}[Division theorem]\label{T:DT}\index{division theorem}
  Given any integer $n$ and a nonzero integer $d$, there exist unique
  integers $q$ and $r$ such that
  \[
  n = d\cdot q+r\quad\text{where $0\le r< |d|$.}
  \]
  If $r=0$, then we say $d$ \dfn{divides} $n$ and write $d|n$.
  \begin{sketch}
    To establish existence, consider the set
    \[
    \mathcal{S} = \{n -  dx : x\in \Z\}\cap \N.
    \]
    \begin{enumerate}
    \item Explain why $\mathcal S$ is not empty.
    \item Explain why $\mathcal S$ has a least element.
    \item Call the least element found above $r$, explain why $r < |d|$.
    \item Explain how to choose $q$ satisfying the conditions of the
      Division Theorem.
    \end{enumerate}
    
    To establish uniqueness, seeking a contradiction, suppose that
    $(q_1,r_1)$ and $(q_2,r_2)$ both satisfy the conditions of the
    division theorem for a divisor $n$ and dividend $d$. Subtract
    these two equations to produce a third equation relating $d$,
    $q_1$, $q_2$, $r_1$, and $r_2$.


    If $q_1 \ne q_2$, explain why $|r_1 - r_2| \ge |d|$.  Explain how
    this shows uniqueness of the quotient and remainder in the
    division theorem.
  \end{sketch}
\end{theorem}


The proof above basically establishes the existence of a smallest
remainder and finds the quotient from there. This is not how anyone
really divides. In practice, we find the quotient first and then the
remainder.  This actual algorithm used to solve a division problem can
also be a proof. In fact, the essence of the grade school algorithm is
built into the proof by picture below.

\begin{exercise}
  Consider the following picture:
  \[
  \begin{tikzpicture}
    \foreach \i in {0,...,2}
    {
      \draw (\i,0) rectangle (1+\i,.5);
      \node at (.5+\i,.25) {$d$};
    }

    \foreach \i in {4,...,6}
    {
      \draw (\i,0) rectangle (1+\i,.5);
      \node at (.5+\i,.25) {$d$};
    }

    \node at(.5+3, .25) {$\cdots$};

    \draw[fill=white!50!gray] (7,0) rectangle (7.4,.5);

    \node at(7.2, .25) {$r$};
    
    \draw[dotted] (7,0) rectangle (8,.5);

    \draw [thick,decoration={brace,mirror,raise=0.2cm},decorate] (0,0) -- (7.4,0)
    node [pos=0.5,anchor=north,yshift=-0.3cm] {$n$};

    \draw [thick,decoration={brace,raise=0.2cm},decorate] (0,.5) -- (7,.5)
    node [pos=0.5,anchor=south,yshift=0.3cm] {$q~d$'s}; 
  \end{tikzpicture}
  \]
  Give a proof of the division theorem when $n\ge 0$ based on this
  picture.
\end{exercise}


\begin{exercise}
  Explain why we can always find an $0\le r<|d|$, such that
  \[
  n \equiv r\pmod{d}.
  \]
\end{exercise}





To be more rigorous with \index{modular arithmetic}modular arithmetic,
we should show that addition modulo $n$ is well-defined. This will be
our next lemma.


\begin{lemma}[Modular addition is well-defined]\label{L:mawd}\index{well-defined}
  Addition on $\Z_n$ is well-defined. This means that if
  \begin{align*}
    a &\equiv a' \pmod{n}\\
    b &\equiv b' \pmod{n},
  \end{align*}
  then
  \[
  a+b \equiv a'+b' \pmod{n}.
  \]
  \begin{sketch}
    Use the fact that
    \[
    x \equiv x'\pmod{n} \quad \Leftrightarrow \quad x -x' = n\cdot q
    \]
    for some number $q$.
  \end{sketch}
\end{lemma}



\begin{exercise}
  Today is Saturday. What day will it be in $3281$ days? Explain your
  reasoning.
\end{exercise}

\begin{exercise}
  It is now December. What month will it be in $219$ months? What about
  $111$ months ago? Explain your reasoning.
\end{exercise}


\begin{exercise}
  Look up ``UPC-A'' numbers. What is their connection to modular
  arithmetic?
\end{exercise}


\begin{exercise}
  Look up ``ISBN-10'' numbers. What is their connection to modular
  arithmetic?
\end{exercise}

\begin{definition}
  An \dfn{equivalence relation} is a binary relation $\sim$ on a set
  $X$ with the following properties:
  \begin{description}
  \item[Reflexive] Meaning that for all $a\in X$, $a\sim a$.
  \item[Symmetric] Meaning that for all $a,b\in X$ if $a\sim b$, then
    $b\sim a$.
  \item[Transitive] Meaning that for all $a,b,c\in X$ if $a\sim b$ and
    $b \sim c$, then $a\sim c$.
  \end{description}
\end{definition}


\begin{exercise}
  Which of the following are equivalence relations:
  \begin{selectAll}
    \choice{``$\le$'' on $\Z$.}
    \choice[correct]{$a\sim b$ if and only if $|a| = |b|$, on $\R$.}
    \choice[correct]{$(a,b) \sim (c,d)$ if and only if $ad = bc$, on $\Z\times \Z$.}
    \choice[correct]{$\alpha \sim\beta$ if and only if $\sin(\alpha) = \sin(\beta)$.}
    \choice{$a/b \sim c/d$ if and only if $a+d=b+c$, on $\Q$.}
  \end{selectAll}
\end{exercise}

\begin{exercise}
  Prove that congruence modulo $n$ is an equivalence relation on $\Z$.
\end{exercise}


\begin{exercise}
  Consider the equation $x^2+1 \equiv 0$. What are the solutions to this
  equation modulo $n$, $n=2,3,4,5,6$?
\end{exercise}


\begin{exercise}
  Prove that $\Z_{6}$ is a cyclic group, find all of its generators,
  find the order of every element. Organize this information into a
  table, \textbf{make wild conjectures.}
\end{exercise}

\begin{exercise}
  Prove that $\Z_{7}$ is a cyclic group, find all of its generators,
  find the order of every element. Organize this information into a
  table, \textbf{make wild conjectures.}
\end{exercise}


\begin{exercise}
  Prove that $\Z_{11}$ is a cyclic group, find all of its generators,
  find the order of every element. Organize this information into a
  table, \textbf{make wild conjectures.}
\end{exercise}



\begin{exercise}
  Prove that $\Z_{12}$ is a cyclic group, find all of its generators,
  find the order of every element. Organize this information into a
  table, \textbf{make wild conjectures.}
\end{exercise}


\begin{exercise}
  Prove that $\Z_{15}$ is a cyclic group, find all of its generators,
  find the order of every element. Organize this information into a
  table, \textbf{make wild conjectures.}
\end{exercise}


At this point, we should be thinking that the structure of a cyclic
group is determined by number theoretic properties based on the order
of the group.


\begin{question}
  What does the order of a cyclic group tell us?
\end{question}


\end{document}
