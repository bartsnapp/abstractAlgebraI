\documentclass{ximera}

\usepackage[T1]{fontenc}
\usepackage{stix2}
\usepackage{gillius}
\usepackage{resizegather}
%\usepackage{rsfso} fancy cal
\DeclareMathAlphabet{\mathcal}{OMS}{cmsy}{m}{n} %less fancy cal


\usepackage{multicol}


\usepackage{tikz-cd}
\usepackage{tkz-euclide} %% compass
\usetkzobj{all}  %% tkzCompass
\tikzset{>=stealth}
\tikzcdset{arrow style=tikz}
\usetikzlibrary{math} %% for assigning variables
%\usetikzlibrary{fadings}

\usepackage{colortbl,boldline,makecell} %% group tables


\usepackage[sans]{dsfont}

\usepackage{stmaryrd,pifont}

\graphicspath{
  {./}
  {fields/}
  }     



\let\oldbibliography\thebibliography%% to compact bib
\renewcommand{\thebibliography}[1]{%
  \oldbibliography{#1}%
  \setlength{\itemsep}{0pt}%
}
\renewcommand\refname{} %% no name needed!


\DefineVerbatimEnvironment{macaulay2}{Verbatim}{numbers=left,frame=lines,label=Macaulay2,labelposition=topline}

\DefineVerbatimEnvironment{gap}{Verbatim}{numbers=left,frame=lines,label=GAP,labelposition=topline}

%%% This next bit of code defines all our theorem environments
\makeatletter
\let\c@theorem\relax
\let\c@corollary\relax
%\let\c@example\relax
\makeatother

\let\definition\relax
\let\enddefinition\relax

\let\theorem\relax
\let\endtheorem\relax

\let\proposition\relax
\let\endproposition\relax

\let\exercise\relax
\let\endexercise\relax

\let\question\relax
\let\endquestion\relax

\let\remark\relax
\let\endremark\relax

\let\corollary\relax
\let\endcorollary\relax


\let\example\relax
\let\endexample\relax

\let\warning\relax
\let\endwarning\relax

\let\lemma\relax
\let\endlemma\relax


\let\algorithm\relax
\let\endalgorithm\relax
\usepackage{algpseudocode}

\newtheoremstyle{SlantTheorem}{\topsep}{\topsep}%%% space between body and thm
		{\slshape}                      %%% Thm body font
		{}                              %%% Indent amount (empty = no indent)
		{\bfseries\sffamily}            %%% Thm head font
		{}                              %%% Punctuation after thm head
		{3ex}                           %%% Space after thm head
		{\thmname{#1}\thmnumber{ #2}\thmnote{ \bfseries(#3)}}%%% Thm head spec
\theoremstyle{SlantTheorem}
\newtheorem{theorem}{Theorem}
%\newtheorem{definition}[theorem]{Definition}
%\newtheorem{proposition}[theorem]{Proposition}
%% \newtheorem*{dfnn}{Definition}
%% \newtheorem{ques}{Question}[theorem]
%% \newtheorem*{war}{WARNING}
%% \newtheorem*{cor}{Corollary}
%% \newtheorem*{eg}{Example}
\newtheorem*{remark}{Remark}
\newtheorem*{touchstone}{Touchstone}
\newtheorem{corollary}{Corollary}[theorem]
\newtheorem*{warning}{WARNING}
\newtheorem{example}[corollary]{Example}
\newtheorem{lemma}[theorem]{Lemma}




\newtheoremstyle{Definition}{\topsep}{\topsep}%%% space between body and thm
		{}                              %%% Thm body font
		{}                              %%% Indent amount (empty = no indent)
		{\bfseries\sffamily}            %%% Thm head font
		{}                              %%% Punctuation after thm head
		{3ex}                           %%% Space after thm head
		{\thmname{#1}\thmnumber{ #2}\thmnote{ \bfseries(#3)}}%%% Thm head spec
\theoremstyle{Definition}
\newtheorem{definition}[theorem]{Definition}



\let\algorithm\relax
\let\endalgorithm\relax
\newtheoremstyle{Alg}{\topsep}{\topsep}%%% space between body and thm
		{}                              %%% Thm body font
		{}                              %%% Indent amount (empty = no indent)
		{\bfseries\sffamily}            %%% Thm head font
		{}                              %%% Punctuation after thm head
		{3ex}                           %%% Space after thm head
		{\thmname{#1}\thmnumber{ #2}\thmnote{ \bfseries(#3)}}%%% Thm head spec
\theoremstyle{Alg}
\newtheorem*{algorithm}{Algorithm}
\newtheorem*{construction}{Construction}




\newtheoremstyle{Exercise}{\topsep}{\topsep} %%% space between body and thm
		{}                           %%% Thm body font
		{}                           %%% Indent amount (empty = no indent)
		{\bfseries\sffamily}         %%% Thm head font
		{)}                          %%% Punctuation after thm head
		{ }                          %%% Space after thm head
		{\thmnumber{#2}\thmnote{ \bfseries(#3)}}%%% Thm head spec
\theoremstyle{Exercise}
\newtheorem{exercise}[corollary]{}%[theorem]

%% \newtheoremstyle{Question}{\topsep}{\topsep} %%% space between body and thm
%% 		{\bfseries}                  %%% Thm body font
%% 		{3ex}                        %%% Indent amount (empty = no indent)
%% 		{}                           %%% Thm head font
%% 		{}                           %%% Punctuation after thm head
%% 		{}                           %%% Space after thm head
%% 		{\thmnumber{#2}\thmnote{ \bfseries(#3)}}%%% Thm head spec
\newtheoremstyle{Question}{3em}{3em} %%% space between body and thm
		{\large\bfseries}                           %%% Thm body font
		{}                           %%% Indent amount (empty = no indent)
		{}                         %%% Thm head font
		{}                          %%% Punctuation after thm head
		{0em}                          %%% Space after thm head
		{}%%% Thm head spec
\theoremstyle{Question}
\newtheorem*{question}{}






\renewcommand{\tilde}{\widetilde}
\renewcommand{\bar}{\overline}
\renewcommand{\hat}{\widehat}
\newcommand{\N}{\mathbb N}
\newcommand{\Z}{\mathbb Z}
\newcommand{\R}{\mathbb R}
\newcommand{\Q}{\mathbb Q}
\newcommand{\C}{\mathbb C}
\newcommand{\V}{\mathbb V}
\newcommand{\I}{\mathbb I}
\newcommand{\A}{\mathbb A}
\renewcommand{\o}{\mathbf o}
\newcommand{\iso}{\simeq}
\newcommand{\ph}{\varphi}
\newcommand{\Cf}{\mathcal{C}}
\newcommand{\IZ}{\mathrm{Int}(\Z)}
\newcommand{\dsum}{\oplus}
\newcommand{\directsum}{\bigoplus}
\newcommand{\union}{\bigcup}
\newcommand{\subgp}{\leq}
\newcommand{\normal}{\trianglelefteq}
\renewcommand{\i}{\mathfrak}
\renewcommand{\a}{\mathfrak{a}}
\renewcommand{\b}{\mathfrak{b}}
\newcommand{\m}{\mathfrak{m}}
\newcommand{\p}{\mathfrak{p}}
\newcommand{\q}{\mathfrak{q}}
\newcommand{\dfn}[1]{\textbf{#1}\index{#1}}
\let\hom\relax
\DeclareMathOperator{\mat}{Mat}
\DeclareMathOperator{\ann}{Ann}
\DeclareMathOperator{\h}{ht}
\DeclareMathOperator{\tr}{tr}
\DeclareMathOperator{\hom}{Hom}
\DeclareMathOperator{\Span}{Span}
\DeclareMathOperator{\spec}{Spec}
\DeclareMathOperator{\maxspec}{MaxSpec}
\DeclareMathOperator{\aut}{Aut}
\DeclareMathOperator{\ass}{Ass}
\DeclareMathOperator{\lcm}{lcm}
\DeclareMathOperator{\ff}{Frac}
\DeclareMathOperator{\im}{Im}
\DeclareMathOperator{\syz}{Syz}
\DeclareMathOperator{\gr}{Gr}
\DeclareMathOperator{\multideg}{multideg}
\renewcommand{\ker}{\mathop{\mathrm{Ker}}\nolimits}
\newcommand{\coker}{\mathop{\mathrm{Coker}}\nolimits}
\newcommand{\lps}{[\hspace{-0.25ex}[}
\newcommand{\rps}{]\hspace{-0.25ex}]}
\newcommand{\into}{\hookrightarrow}
\newcommand{\onto}{\twoheadrightarrow}
\newcommand{\tensor}{\otimes}
\newcommand{\x}{\mathbf{x}}
\newcommand{\X}{\mathbf X}
\newcommand{\Y}{\mathbf Y}
\renewcommand{\k}{\boldsymbol{\kappa}}
\renewcommand{\emptyset}{\varnothing}
\renewcommand{\qedsymbol}{$\blacksquare$}
\renewcommand{\l}{\ell}
\newcommand{\1}{\mathds{1}}
\newcommand{\lto}{\mathop{\longrightarrow\,}\limits}
\newcommand{\rad}{\sqrt}
\newcommand{\hf}{H}
\newcommand{\hs}{H\!S}
\newcommand{\hp}{H\!P}
\renewcommand{\vec}{\mathbf}
\let\temp\phi
\let\phi\varphi
\let\eulerphi\temp


\renewcommand{\epsilon}{\varepsilon}
\renewcommand{\subset}{\subseteq}
\renewcommand{\supset}{\supseteq}
\newcommand{\macaulay}{\normalfont\textsl{Macaulay2}}
\newcommand{\GAP}{\normalfont\textsf{GAP}}
\newcommand{\invlim}{\varprojlim}
\renewcommand{\le}{\leqslant}
\renewcommand{\ge}{\geqslant}
\newcommand{\valpha}{{\boldsymbol\alpha}}
\newcommand{\vbeta}{{\boldsymbol\beta}}
\newcommand{\vgamma}{{\boldsymbol\gamma}}
\newcommand{\dotp}{\bullet}
\newcommand{\lc}{\normalfont\textsc{lc}}
\newcommand{\lt}{\normalfont\textsc{lt}}
\newcommand{\lm}{\normalfont\textsc{lm}}
\newcommand{\from}{\leftarrow}
\newcommand{\transpose}{\intercal}
\newcommand{\grad}{\boldsymbol\nabla}
\newcommand{\curl}{\boldsymbol{\nabla\times}}
\renewcommand{\d}{\, d}
\newcommand{\<}{\langle}
\renewcommand{\>}{\rangle}

%\renewcommand{\proofname}{Sketch of Proof}


\renewenvironment{proof}[1][Proof]
  {\begin{trivlist}\item[\hskip \labelsep \itshape \bfseries #1{}\hspace{2ex}]\upshape}
{\qed\end{trivlist}}

\newenvironment{sketch}[1][Sketch of Proof]
  {\begin{trivlist}\item[\hskip \labelsep \itshape \bfseries #1{}\hspace{2ex}]\upshape}
{\qed\end{trivlist}}



\makeatletter
\renewcommand\section{\@startsection{paragraph}{10}{\z@}%
                                     {-3.25ex\@plus -1ex \@minus -.2ex}%
                                     {1.5ex \@plus .2ex}%
                                     {\normalfont\large\sffamily\bfseries}}
\renewcommand\subsection{\@startsection{subparagraph}{10}{\z@}%
                                    {3.25ex \@plus1ex \@minus.2ex}%
                                    {-1em}%
                                    {\normalfont\normalsize\sffamily\bfseries}}
\makeatother

%% Fix weird index/bib issue.
\makeatletter
\gdef\ttl@savemark{\sectionmark{}}
\makeatother


\makeatletter
%% no number for refs
\newcommand\frontstyle{%
  \def\activitystyle{activity-chapter}
  \def\maketitle{%
    \addtocounter{titlenumber}{1}%
                    {\flushleft\small\sffamily\bfseries\@pretitle\par\vspace{-1.5em}}%
                    {\flushleft\LARGE\sffamily\bfseries\@title \par }%
                    {\vskip .6em\noindent\textit\theabstract\setcounter{problem}{0}\setcounter{sectiontitlenumber}{0}}%
                    \par\vspace{2em}
                    \phantomsection\addcontentsline{toc}{section}{\textbf{\@title}}%
                  }}
\makeatother



\NewEnviron{annotate}{\vspace{-.3cm}\small \itshape \BODY \vspace{.3cm}}


%%%% TIKZ STUFF

%% N-GON code
\tikzset{
    pics/tikzngon/.style={
        code={
        \tikzmath{\xx = #1;\rr=1.7;}
        \draw[ultra thick,rounded corners=.05mm] ({\rr*sin(0*360/\xx)},{\rr*cos(0*360/\xx)})
        \foreach \x in {-1,0,...,\xx}
        {
        -- ({\rr*sin(\x*360/\xx)},{\rr*cos(\x*360/\xx)})
        }
           -- cycle;
  }}}

%% N-GON code (even)
\tikzset{
    pics/tikzegon/.style={
        code={
        \tikzmath{\xx = #1;\rr=1.7;}
        \draw[ultra thick,rounded corners=.05mm] ({\rr*sin(0*360/\xx+180/\xx)},{\rr*cos(0*360/\xx+180/\xx)})
        \foreach \x in {-1,0,...,\xx}
           {
           -- ({\rr*sin(\x*360/\xx+180/\xx)},{\rr*cos(\x*360/\xx+180/\xx)}) 
           }
           -- cycle;
  }}}




%% N-CLOCK code
\tikzset{
    pics/tikznclock/.style={
        code={
        \tikzmath{\xx = #1;\rr=1.7;\dd=.4;}
        \foreach \x in {1,...,\xx}
        \pgfmathtruncatemacro{\xy}{\x-1}
           {
             \node[circle,fill=black,inner sep=0pt, minimum size=13pt,text=white]
             at ({(\rr-\dd)*sin((\x-1)*360/(\xx)},{(\rr-\dd)*cos((\x-1)*360/\xx}) {\normalfont\bfseries\sffamily\small {\xy}};
           }
  \draw[thick] (0,0) circle (\rr);
  }}}



%% barcode from
%% https://tex.stackexchange.com/questions/6895/is-there-a-good-latex-package-for-generating-barcodes
%% NOT CURRENTLY USED!


\def\barcode#1#2#3#4#5#6#7{\begingroup%
  \dimen0=0.1em
  \def\stack##1##2{\oalign{##1\cr\hidewidth##2\hidewidth}}%
  \def\0##1{\kern##1\dimen0}%
  \def\1##1{\vrule height10ex width##1\dimen0}%
  \def\L##1{\ifcase##1\bc3211##1\or\bc2221##1\or\bc2122##1\or\bc1411##1%
    \or\bc1132##1\or\bc1231##1\or\bc1114##1\or\bc1312##1\or\bc1213##1%
    \or\bc3112##1\fi}%
  \def\R##1{\bgroup\let\next\1\let\1\0\let\0\next\L##1\egroup}%
  \def\G##1{\bgroup\let\bc\bcg\L##1\egroup}% reverse
  \def\bc##1##2##3##4##5{\stack{\0##1\1##2\0##3\1##4}##5}%
  \def\bcg##1##2##3##4##5{\stack{\0##4\1##3\0##2\1##1}##5}%
  \def\bcR##1##2##3##4##5##6{\R##1\R##2\R##3\R##4\R##5\R##6\11\01\11\09%
    \endgroup}%
  \stack{\09}#1\11\01\11\L#2%
  \ifcase#1\L#3\L#4\L#5\L#6\L#7\or\L#3\G#4\L#5\G#6\G#7%
    \or\L#3\G#4\G#5\L#6\G#7\or\L#3\G#4\G#5\G#6\L#7%
    \or\G#3\L#4\L#5\G#6\G#7\or\G#3\G#4\L#5\L#6\G#7%
    \or\G#3\G#4\G#5\L#6\L#7\or\G#3\L#4\G#5\L#6\G#7%
    \or\G#3\L#4\G#5\G#6\L#7\or\G#3\G#4\L#5\G#6\L#7%
  \fi\01\11\01\11\01\bcR}


\author{Bart Snapp}

\title{Permutation groups}

\begin{document}
\begin{abstract}
  We introduce groups of permutations. 
\end{abstract}
\maketitle

In mathematics, there are special types of functions called
\textit{permutations}. These functions take a list of symbols (usually
the symbols are numbers) and rearrange the symbols.  

\begin{definition}
  A \dfn{permutation} is a bijection $\sigma$ from a set $X$ to
  itself, this means that $\sigma\in\aut(X)$. If $X$ is finite,
  we may represent the elements of $X$ using natural numbers
  $1,2,\dots,n$. In this case, we denote $\sigma$ one of two ways,
  either in \dfn{permutation notation}
  \[
  \sigma = \left(\begin{smallmatrix}
    1 & 2 & 3 & \cdots & n-1 & n \\
    \sigma(1) & \sigma(2) & \sigma(3) & \cdots &  \sigma(n-1)  & \sigma(n)
  \end{smallmatrix}\right)
  \]
  or in \dfn{cycle notation}
  \[
  \sigma = (x_1\ \sigma(x_1)\ \sigma^2(x_1)\ \cdots\ \sigma^{n_1}(x_1)) \cdots (x_m \ \sigma(x_m)\ \sigma^2(x_m)\ \cdots\ \sigma^{n_m}(x_m))
  \]
  where $x_i\in X$ is usually a natural number.
\end{definition}


It helps to see an example of permutations in action.

\begin{example}[Permutation versus cycle notation]
  Let $\sigma$ be a permutation on the set $\{1,2,3\}$ represented by
  \[
  \sigma = \left(\begin{smallmatrix}
    1 & 2 & 3 \\
    3 & 1 & 2
  \end{smallmatrix}\right)
  \quad\text{or}\quad \sigma = (1 \ 3 \ 2).
  \]
  These two representations tell us how $\sigma$ acts on $\{1,2,3\}$
  in the following ways. In permutation notation, you look for the
  number \textit{below} the number from the argument. So
  \begin{align*}
    \left(\begin{smallmatrix}
    \textcolor{blue!60!black}1 & 2 & 3 \\
    \textcolor{red!60!black}{3} & 1 & 2
    \end{smallmatrix}\right) (\textcolor{blue!60!black}1) &= \textcolor{red!60!black}{3}, \\
    \left(\begin{smallmatrix}
    1 & \textcolor{blue!60!black}2 & 3 \\
    3 & \textcolor{red!60!black}{1} & 2
    \end{smallmatrix}\right) (\textcolor{blue!60!black}2) &= \textcolor{red!60!black}{1}, \\
    \left(\begin{smallmatrix}
    1 & 2 & \textcolor{blue!60!black}3 \\
    3 & 1 & \textcolor{red!60!black}{2}
    \end{smallmatrix}\right) (\textcolor{blue!60!black}3) &= \textcolor{red!60!black}{2}.
  \end{align*}

  In cycle notation we look to the right of the number in the
  argument, or if there is none to the right, to the furthest left,
  \begin{align*}
    ( \textcolor{blue!60!black}1\  \textcolor{red!60!black}{3}\  2) (\textcolor{blue!60!black}1) &= \textcolor{red!60!black}{3}, \\
    ( 1\  \textcolor{blue!60!black}{3}\   \textcolor{red!60!black}2) (\textcolor{blue!60!black}3) &= \textcolor{red!60!black}{2}, \\
    ( \textcolor{red!60!black}1\  3\   \textcolor{blue!60!black}2) (\textcolor{blue!60!black}2)   &= \textcolor{red!60!black}{1}.
  \end{align*}
\end{example}

If one example is good, two are better.

\begin{example}[Permutation versus cycle notation]
  Let $\sigma$ be a permutation on the set $\{1,2,3,4,5\}$ represented by
  \[
  \sigma = \left(\begin{smallmatrix}
    1 & 2 & 3 & 4 & 5\\
    4 & 3 & 5 & 1 & 2
  \end{smallmatrix}\right)
  \quad\text{or}\quad \sigma = (1 \ 4) (2 \ 3 \ 5).
  \]
  These two representations tell us how $\sigma$ acts on $\{1,2,3,4,5\}$
  in the following ways. In permutation notation, you look for the
  number \textit{below} the number from the argument. So
  \begin{align*}
    \left(\begin{smallmatrix}
    \textcolor{blue!60!black}1 & 2 & 3 & 4 & 5\\
    \textcolor{red!60!black}{4} & 3 & 5 & 1 & 2
    \end{smallmatrix}\right) (\textcolor{blue!60!black}1) &= \textcolor{red!60!black}{4}, \\
    \left(\begin{smallmatrix}
    1 & \textcolor{blue!60!black}2 & 3 & 4 & 5\\
    4 & \textcolor{red!60!black}3 & 5 & 1 & 2
    \end{smallmatrix}\right) (\textcolor{blue!60!black}2) &= \textcolor{red!60!black}{3}, \\
    \left(\begin{smallmatrix}
      1 & 2 & \textcolor{blue!60!black}3 & 4 & 5\\
      4 & 3 & \textcolor{red!60!black}5  & 1 & 2
    \end{smallmatrix}\right) (\textcolor{blue!60!black}3) &= \textcolor{red!60!black}{5}, \\
    \left(\begin{smallmatrix}
      1 & 2 & 3 & \textcolor{blue!60!black}4 & 5\\
      4 & 3 & 5 & \textcolor{red!60!black}1 & 2
    \end{smallmatrix}\right) (\textcolor{blue!60!black}4) &= \textcolor{red!60!black}{1}, \\
    \left(\begin{smallmatrix}
      1 & 2 & 3 & 4 & \textcolor{blue!60!black}5\\
      4 & 3 & 5 & 1 & \textcolor{red!60!black}2
    \end{smallmatrix}\right) (\textcolor{blue!60!black}5) &= \textcolor{red!60!black}{2}.
  \end{align*}

  In cycle notation we look to the right of the number in the
  argument, or if there is none to the right, to the furthest left,
  \begin{align*}
    (\textcolor{blue!60!black}1 \ \textcolor{red!60!black}4) (2 \ 3 \ 5)(\textcolor{blue!60!black}1) &= \textcolor{red!60!black}4, \\
    (1 \ 4) (\textcolor{blue!60!black}2 \ \textcolor{red!60!black}3 \ 5)(\textcolor{blue!60!black}2) &= \textcolor{red!60!black}3, \\
    (1 \ 4) (2 \ \textcolor{blue!60!black}3 \ \textcolor{red!60!black}5)(\textcolor{blue!60!black}3) &= \textcolor{red!60!black}5, \\
    (\textcolor{red!60!black}1 \ \textcolor{blue!60!black}4) (2 \ 3 \ 5)(\textcolor{blue!60!black}4) &= \textcolor{red!60!black}1, \\
    (1 \ 4) (\textcolor{red!60!black}2 \ 3 \ \textcolor{blue!60!black}5)(\textcolor{blue!60!black}5) &= \textcolor{red!60!black}2.
  \end{align*}
\end{example}



\begin{exercise}
  Consider the set $\{1,2,3\}$ and let $\sigma$ map
  \[
  1 \mapsto 2, \quad 2 \mapsto 3, \quad 3\mapsto 1.
  \]
  Express $\sigma$ in both permutation and cycle notation.
\end{exercise}



\begin{exercise}
  Consider the set $\{1,2,3,4\}$ and let $\sigma$ map
  \[
  1 \mapsto 2, \quad 2 \mapsto 1, \quad 3\mapsto 4, \quad 4\mapsto 3.
  \]
  Express $\sigma$ in both permutation and cycle notation.
\end{exercise}

\begin{exercise}
  Consider the set $\{1,2,3,4\}$ and let $\sigma$ map
  \[
  1 \mapsto 4, \quad 2 \mapsto 2, \quad 3\mapsto 1, \quad 4\mapsto 3.
  \]
  Express $\sigma$ in both permutation and cycle notation.
\end{exercise}

\begin{exercise}
  Consider the set $\{1,2,3,4\}$ and let $\sigma$ map
  \[
  1 \mapsto 1, \quad 2 \mapsto 2, \quad 3\mapsto 3, \quad 4\mapsto 4.
  \]
  Express $\sigma$ in both permutation and cycle notation.
\end{exercise}


\begin{example}[Composition]
  Composing permutations is similar to how you evaluate permutations.
  Write with me
  \[
  \left(\begin{smallmatrix}
    1 & 2 & 3 & 4 & 5\\
    4 & 3 & 5 & 1 & 2
  \end{smallmatrix}\right)
  \left(\begin{smallmatrix}
    1 & 2 & 3\\
    3 & 1 & 2
  \end{smallmatrix}\right)
  = \left(\begin{smallmatrix}
    1 & 2 & 3 & 4 & 5\\
    5 & 4 & 3 & 1 & 2
  \end{smallmatrix}\right),
  \]
  or in cycle notation
  \[
  (1\ 4)(2\ 3\ 5)(1\ 3\ 2) = (1\ 5\ 2\ 4).
  \]
\end{example}


\begin{exercise}
  Find the inverse of the permutation $\left(\begin{smallmatrix}
    1 & 2 & 3\\
    3 & 1 & 2
  \end{smallmatrix}\right)$.
\end{exercise}

\begin{exercise}
  Find the inverse of the permutation $\left(\begin{smallmatrix}
    1 & 2 & 3 & 4 & 5\\
    4 & 3 & 5 & 1 & 2
  \end{smallmatrix}\right)$.
\end{exercise}

\begin{exercise}
  Find the inverse of the cycle $(1 \ 2 \ 3 )$.
\end{exercise}

\begin{exercise}
  Find the inverse of the cycle $(1 \ 3 \ 2 )$.
\end{exercise}

\begin{exercise}
  Find the inverse of the cycle $(1\ 3\ 4 \ 2 \ 5 )$.
\end{exercise}


\begin{exercise}
  Find the inverse of the cycle $(1 \ 2 \ 5) (3\ 4)$.
\end{exercise}



\begin{definition}
  The \dfn{symmetric group} \textbf{on $\boldsymbol{n}$ symbols} is
  the set of bijections from a set of order $n$ to itself along with
  function composition. We call (and denote) such a group as $S_n$.
\end{definition}

\begin{remark}
  While $S_n$ could be the set of functions on any set of order $n$,
  we usually assume that the set is $\{1,2,3,\dots,n\}$. In
  particular, we could say
  \[
  S_n = \aut(\{1,2,3,\dots,n\}).
  \]
  By Theorem~\ref{E:aut0}, we see immediately that $S_n$ is a group.
\end{remark}

\begin{exercise}
  Prove that $S_n$ has $n!$ elements.
\end{exercise}



\begin{example}[Symmetric group on three symbols]
  The bijections from $\{1,2,3\}$ to $\{1,2,3\}$ form a group under
  function composition. This group is known as the \dfn{symmetric
    group}\textbf{ on three symbols} and is also called (and denoted) $S_3$.
  \begin{proof}
    Note, $S_3$ is generated by $(1\ 2\ 3)$ and $(2\ 3)$, you can show
    this by composing these permutations and finding all of $S_3$,
    \[
    S_3 = \{e,(1\ 2),(1\ 3), (2\ 3), (1 \ 2 \ 3), (1 \ 3 \ 2)\}. 
    \]
    Since function composition, this is associative. The identity
    exists, it's $e$. Now we'll make a Cayley table for $S_3$.
    \begin{gather*}
      \renewcommand{\arraystretch}{1.6}
      \begin{array}{c!{\vline width 2pt}cccccc}
        (S_3,\circ)& e     & (1\ 2\ 3)\cellcolor{blue!12!white}     & (1\ 2\ 3)^2\cellcolor{blue!24!white}   & (2\ 3) \cellcolor{red!12!white}    & (1\ 2\ 3) (2\ 3) \cellcolor{blue!33!red!24!white}  & (1\ 2\ 3)^2 (2\ 3)\cellcolor{blue!66!red!24!white} \\  \Xhline{2pt}
        e          & e     & (1\ 2\ 3)\cellcolor{blue!12!white}    & (1\ 2\ 3)^2\cellcolor{blue!24!white}   & (2\ 3) \cellcolor{red!12!white}    & (1\ 2\ 3) (2\ 3) \cellcolor{blue!33!red!24!white}   & (1\ 2\ 3)^2(2\ 3) \cellcolor{blue!66!red!24!white} \\  
        (1\ 2\ 3)\cellcolor{blue!12!white}         & (1\ 2\ 3)\cellcolor{blue!12!white}    & (1\ 2\ 3)^2\cellcolor{blue!24!white}   & e     & (1\ 2\ 3)(2\ 3)\cellcolor{blue!33!red!24!white} & (1\ 2\ 3)^2(2\ 3) \cellcolor{blue!66!red!24!white}    & (2\ 3) \cellcolor{red!12!white}   \\  
        (1\ 2\ 3)^2\cellcolor{blue!24!white}        & (1\ 2\ 3)^2\cellcolor{blue!24!white}   & e     & (1\ 2\ 3)\cellcolor{blue!12!white}    & (1\ 2\ 3)^2 (2\ 3) \cellcolor{blue!66!red!24!white}   & (2\ 3) \cellcolor{red!12!white} & (1\ 2\ 3)(2\ 3) \cellcolor{blue!33!red!24!white}    \\  
        (2\ 3) \cellcolor{red!12!white}         & (2\ 3) \cellcolor{red!12!white}    & (1\ 2\ 3)^2(2\ 3) \cellcolor{blue!66!red!24!white}   & (1\ 2\ 3) (2\ 3) \cellcolor{blue!33!red!24!white} & e     & (1\ 2\ 3)^2\cellcolor{blue!24!white}    & (1\ 2\ 3)\cellcolor{blue!12!white}   \\  
        (1\ 2\ 3) (2\ 3) \cellcolor{blue!33!red!24!white}        & (1\ 2\ 3)(2\ 3) \cellcolor{blue!33!red!24!white}   & (2\ 3) \cellcolor{red!12!white} & (1\ 2\ 3)^2(2\ 3) \cellcolor{blue!66!red!24!white}    & (1\ 2\ 3)\cellcolor{blue!12!white}   & e     & (1\ 2\ 3)^2\cellcolor{blue!24!white}    \\  
        (1\ 2\ 3)^2 (2\ 3)\cellcolor{blue!66!red!24!white}      & (1\ 2\ 3)^2 (2\ 3)\cellcolor{blue!66!red!24!white} & (1\ 2\ 3)(2\ 3) \cellcolor{blue!33!red!24!white}    & (2\ 3) \cellcolor{red!12!white}   & (1\ 2\ 3)^2\cellcolor{blue!24!white}    & (1\ 2\ 3)\cellcolor{blue!12!white}   & e     \\  
      \end{array}
    \end{gather*}
    From the table we see that every element of $S_3$ has an inverse
    and that $S_3$ is closed under composition.
  \end{proof}
\end{example}

\begin{exercise}
  Is $S_4$ isomorphic to $D_4$?
\end{exercise}


\begin{exercise}
  Find the order of: $(1 \ 3)$, $(1 \ 5 \ 2)$, $(4 \ 8 \ 7 \ 5)$ in
  $S_8$.
\end{exercise}



\begin{definition}\index{n-cycle@$n$-cycle}
  An \textbf{$\boldsymbol{n}$-cycle} is a permutation $\sigma$ that
  can be represented as a single cycle with $n$ entries
  \[
  \sigma = (x_1\ \sigma(x_1)\ \sigma^2(x_1)\ \cdots\ \sigma^{n-1}(x_1))
  \]
  where $\sigma^n(x_1) = x_1$.  If the permutation is a $2$-cycle, it
  is sometimes called a \dfn{transposition}.
\end{definition}


\begin{example}[$\boldsymbol{3}$-cycle]
  The permutation $\sigma = \left(\begin{smallmatrix}
    1 & 2 & 3\\
    3 & 1 & 2
  \end{smallmatrix}\right)$ can be represented by the cycle $(1\ 3 \ 2)$. However,
  the permutation $\tau = \left(\begin{smallmatrix}
    1 & 2 & 3 & 4 & 5\\
    4 & 3 & 5 & 1 & 2
  \end{smallmatrix}\right)$ cannot be represented by a (single) cycle, since in cycle notation $\tau$ is the composition of \textbf{two} cycles, $\tau = (1\ 4)(2\ 3\ 5)$.
\end{example}




\begin{exercise}
  Write down the elements of $S_4$ that are cycles.
\end{exercise}

\begin{exercise}
  Write down the elements of $S_4$ that are are not single cycles, but
  can only be represented as a composition of cycles.
\end{exercise}



\begin{exercise}
  Prove that an $n$-cycle has order $n$.
\end{exercise}


\begin{exercise}
  Find the order of an element of maximal order in $S_3$, $S_4$,
  $S_5$, $S_6$, $S_7$, and $S_8$.
\end{exercise}


\begin{exercise}
  Consider the $5$-cycle $\sigma=(1 \ 2\ 3\ 4\ 5 )$. For which values
  of $n$ is $\sigma^n$ also a $5$-cycle?
\end{exercise}


\begin{exercise}
  Consider the $6$-cycle $\sigma=(1 \ 2\ 3\ 4\ 5 \ 6)$. For which
  values of $n$ is $\sigma^n$ also a $6$-cycle?
\end{exercise}


\begin{exercise}
  Consider the $8$-cycle $\sigma=(1 \ 2\ 3\ 4\ 5 \ 6 \ 7 \ 8)$. For
  which values of $n$ is $\sigma^n$ also an $8$-cycle?
\end{exercise}





\begin{exercise}\label{E:tcd}
  Let $(a_1 \ a_2 \ a_3 \ \cdots \ a_{n})\in S_n$. Prove that
  \[
  (a_1 \ a_2 \ a_3 \ \cdots \ a_{n}) = (a_2 \ a_3 \ \cdots \ a_{n}) (a_1 \ a_{n}). 
  \]
  See Lemma~\ref{L:td}.
\end{exercise}


\begin{lemma}[Representation as a composition  of cycles]\label{L:CC}
  Every permutation can be represented as a composition of disjoint
  cycles.
  \begin{sketch}
    Write out the cycle as described in the definition of permutations
    and prove that the cycles are disjoint.
  \end{sketch}
\end{lemma}

\begin{lemma}[Disjoint cycles commute]
  If the cycle representation of $\sigma,\tau\in S_n$ are disjoint,
  meaning none of the entries of the cycle representation of $\sigma$
  are in the cycle representation of $\tau$, and vice versa, then
  \[
  \sigma \circ \tau = \tau \circ \sigma.
  \]
  \begin{sketch}
    Show that both $\sigma \circ \tau$ and $\tau \circ \sigma$ act the
    same on natural numbers.
  \end{sketch} 
\end{lemma}


\begin{exercise}
  Prove that a permutation $\sigma$ has order $2$ if and only if it
  can be represented by a composition of disjoint $2$-cycles.
\end{exercise}

\begin{exercise}
  Find the order of $(1\ 4\ 3)(2 \ 7 \ 5 \ 6) \in S_7$.
\end{exercise}

\begin{exercise}
  Find the order of $\left(\begin{smallmatrix}
    1 & 2 & 3 & 4 & 5\\
    4 & 3 & 5 & 1 & 2
  \end{smallmatrix}\right)\in S_5$.
\end{exercise}



\section{Permutation matrices}

\begin{definition}
  Let $\sigma \in S_n$. A \dfn{permutation matrix} is a matrix
  $P_\sigma\in GL(n)$ whose entries are zero or one and whose columns
  describe the map
  \[
  P_\sigma : \begin{bmatrix}
    1 \\
    2 \\
    \vdots \\
    n
  \end{bmatrix}
  \mapsto
  \begin{bmatrix}
    \sigma(1) \\
    \sigma(2) \\
    \vdots \\
    \sigma(n)
  \end{bmatrix}
  \]
\end{definition}



\begin{example}[Permutation matrices]
  If $\sigma =\left(\begin{smallmatrix}
    1 & 2 & 3 \\
    3 & 1 & 2
  \end{smallmatrix}\right)$, then
  \[
  P_\sigma =
  \begin{bmatrix}
    0 & 0 & 1 \\
    1 & 0 & 0 \\
    0  & 1 & 0
  \end{bmatrix}.
  \]
  Indeed,
  \[
  \begin{bmatrix}
    0 & 0 & 1 \\
    1 & 0 & 0 \\
    0  & 1 & 0
  \end{bmatrix}
  \begin{bmatrix}
    1 \\
    2 \\
    3
  \end{bmatrix}
  =
  \begin{bmatrix}
    3 \\
    1 \\
    2
  \end{bmatrix}.
  \]
\end{example}


\begin{exercise}
  Find the permutation matrix for $\sigma = \left(\begin{smallmatrix}
    1 & 2 & 3 & 4 & 5\\
    4 & 3 & 5 & 1 & 2
  \end{smallmatrix}\right)$.
\end{exercise}


\begin{exercise}
  Given a permutation $\sigma = \left(\begin{smallmatrix}
    1 & 2 & 3 & \cdots & n\\
    a_1 & a_2 & a_3 & \cdots & a_n
  \end{smallmatrix}\right)$, devise an algorithm for producing the permutation matrix $P_{\sigma}$.
\end{exercise}

\begin{exercise}
  If $P_\sigma$ is a permutation matrix for $\sigma$, prove that
  $P_\sigma^\transpose$ is a permutation matrix for $\sigma^{-1}$.
\end{exercise}



\begin{lemma}[Permutations matrices are orthogonal]
  If $\sigma \in S_n$, then $P_\sigma \in O(n)$.
  \begin{sketch}
    The matrix $P_\sigma^\transpose$ is the inverse of $P_\sigma$.
  \end{sketch}
\end{lemma}

\begin{corollary}[Determinant of transpositions]
  If $\tau$ is a $2$-cycle, then $\det(P_\tau) = -1$.
\end{corollary}





\begin{lemma}[Transposition decomposition]\label{L:td}
  Every permutation is a composition of $2$-cycles.
  \begin{sketch}
    First note $e = (12)(12)$. Now use induction on the length of a
    cycle to prove that every (single) cycle can be represented as a
    composition of $2$-cycles. Note that from Exercise~\ref{E:tcd}
    \[
    (a_1 \ a_2 \ a_3 \ \cdots \ a_{n}) = (a_2 \ a_3 \ \cdots \ a_{n+1}) (a_1 \ a_{n}). 
    \]
    Now use Lemma~\ref{L:CC}.
  \end{sketch}
\end{lemma}


\begin{exercise}
  Write each element of $S_3$ as a composition of $2$-cycles.
\end{exercise}

\begin{exercise}
  Write each element of $S_4$ as a composition of $2$-cycles.
\end{exercise}

\begin{exercise}
  Write $\left(\begin{smallmatrix}
    1 & 2 & 3 & 4 & 5 & 6 & 7 \\
    5 & 2 & 6 & 1 & 7 & 4 & 3
  \end{smallmatrix}\right)$ as a composition of $2$-cycles.
\end{exercise}



\section{Alternating groups}


We'll need to work a bit to get to the definition of an
\textit{alternating group}. Basically, an alternating group is a group
of permutations, each of which can be written as a composition of an
even number of $2$-cycles.




\begin{definition}
  A permutation is called an \dfn{even permutation} if it is the
  composition of an even number of $2$-cycles. A permutation is called
  an \dfn{odd permutation} if it is the composition of an odd number
  of $2$-cycles.
\end{definition}










\begin{lemma}[Identity is even]
  If the identity $e\in S_n$ is represented as a composition of $2$-cycles, $\tau_i$, 
  \[
  e = \tau_1 \tau_2 \cdots \tau_n,
  \]
  then $n$ is even.
  \begin{sketch}
    Note that $\det(P_{e})=1$ and that $\det(P_{\tau_i})=-1$.
  \end{sketch}
\end{lemma}

\begin{lemma}[Even or odd]
  If a permutation $\sigma$ can be represented as an even number of
  $2$-cycles, then every $2$-cycle representation of $\sigma$ will
  have an even number of $2$-cycles.

  If a permutation $\sigma$ can be represented as an odd number of
  $2$-cycles, then every $2$-cycle representation of $\sigma$ will
  have an odd number of $2$-cycles.
  \begin{proof}
    Consider two different $2$-cycle representations of some
    permutation $\sigma$,
    \[
    \sigma = \alpha_1\alpha_2\cdots\alpha_m \quad\text{and}\quad \sigma = \beta_1\beta_2\cdots\beta_n 
    \]
    where each $\alpha_i$ and $\beta_i$ are $2$-cycles. Write with me
    \[
    \alpha_1\alpha_2\cdots\alpha_m = \beta_1\beta_2\cdots\beta_n
    \]
    so, we may multiply both sides by
    \begin{align*}
      (\alpha_1\alpha_2\cdots\alpha_m)^{-1} &= \alpha_m^{-1}\alpha_{m-1}^{-1}\cdots\alpha_1^{-1}\\
      &= \alpha_m\alpha_{m-1}\cdots\alpha_1
    \end{align*}
    to see
    \[
    e = \beta_1\beta_2\cdots\beta_n\alpha_m\alpha_{m-1}\cdots\alpha_1.
    \]
    Since $e$ is always expressed as an even number of $2$-cycles, if
    $m$ is even, then $n$ must be even. If $m$ is odd, then $n$ is
    odd. This proves that a permutation is either even or odd and not
    both.
  \end{proof}
\end{lemma}






\begin{definition}
  The set of even permutations in $S_n$ form what is known as a
  \dfn{alternating group}. We call (and denote) such a group as $A_n$.
\end{definition}

\begin{remark}
  $A_n$ is a group under composition via the lemmas above.
\end{remark}

\begin{exercise}
  Prove that $A_2 = \{e\}$.
\end{exercise}

\begin{exercise}
  Prove that $A_3 = \{e,(1\ 2\ 3), (1\ 3\ 2)\}$, and show that
  $A_3\iso \Z_3$.
\end{exercise}

\begin{exercise}
  Prove that
  \begin{align*}
    A_4 =\{e, &(1\ 2\ 3), (1\ 2\ 4), (1\ 3 \ 2), (1\ 3\ 4),  (1\ 4\ 2),  \\
    &(1\ 4\ 3),(2 \ 3 \ 4), (2 \ 4 \ 3), (1\ 2)(3 \ 4), (1\ 3)(2\ 4),(1\ 4)(2\ 3)\}.
  \end{align*}
\end{exercise}


\begin{exercise}
  Prove that $|A_n| = n!/2$ for $n\ge 2$.
  \begin{hint}
    Prove that the number of even permutations equals the number of
    odd permutations.
  \end{hint}
\end{exercise}


\begin{example}[Alternating group on four symbols]\label{EG:A4}
  The bijections $\{1,2,3,4\}$ to $\{1,2,3,4\}$ that can be represented
  as even permutations form a group under function composition. This
  group is known as the \dfn{alternating group}\textbf{ on four symbols} and is
  also called (and denoted) $A_4$.


  At this point we know $A_4$ is a group, let's check out it's Cayley
  table.

  We leave it to the reader to verify that $A_4$ is generated by
  \[
  \sigma=(1\ 2\ 3)\quad \text{and}\quad \tau = (2\ 3\ 4).
  \]
  \begin{gather*}
    \renewcommand{\arraystretch}{1.6}
      \begin{array}{c!{\vline width 2pt}cccccccccccc}
      (A_{4},\circ)& e\cellcolor{red!6!white}     & \sigma\cellcolor{red!12!white}     & \sigma^2\cellcolor{red!24!white}   & \tau \cellcolor{yellow!6!white}    & \sigma\tau \cellcolor{yellow!12!white}  & \sigma^2 \tau\cellcolor{yellow!24!white} & \tau^2\cellcolor{green!6!white}    &  \sigma \tau^2\cellcolor{green!12!white}  &  \sigma^2 \tau^2\cellcolor{green!24!white} & \tau\sigma^2 \cellcolor{blue!6!white} & \sigma\tau\sigma^2 \cellcolor{blue!12!white}& \sigma^2 \tau \sigma^2\cellcolor{blue!24!white}\\  \Xhline{2pt}
      e \cellcolor{red!6!white}         & e\cellcolor{red!6!white}         & \sigma\cellcolor{red!12!white}           & \sigma^2\cellcolor{red!24!white}   & \tau \cellcolor{yellow!6!white}    & \sigma\tau \cellcolor{yellow!12!white}  & \sigma^2 \tau\cellcolor{yellow!24!white} & \tau^2\cellcolor{green!6!white}    &  \sigma \tau^2\cellcolor{green!12!white}  &  \sigma^2 \tau^2\cellcolor{green!24!white} & \tau\sigma^2 \cellcolor{blue!6!white} & \sigma\tau\sigma^2 \cellcolor{blue!12!white}& \sigma^2 \tau \sigma^2\cellcolor{blue!24!white}\\ 
      \sigma\cellcolor{red!12!white}         & \sigma\cellcolor{red!12!white}        & \sigma^2\cellcolor{red!24!white}            & e\cellcolor{red!6!white}     & \sigma\tau\cellcolor{yellow!12!white} & \sigma^2\tau \cellcolor{yellow!24!white}    & \tau \cellcolor{yellow!6!white}   & \sigma\tau^2 \cellcolor{green!12!white}  & \sigma^2\tau^2 \cellcolor{green!24!white} & \tau^2 \cellcolor{green!6!white} & \sigma \tau\sigma^2 \cellcolor{blue!12!white} & \sigma^2 \tau\sigma^2 \cellcolor{blue!24!white} & \tau\sigma^2 \cellcolor{blue!6!white}\\  
      \sigma^2\cellcolor{red!24!white}       & \sigma^2\cellcolor{red!24!white}      & e\cellcolor{red!6!white}           & \sigma\cellcolor{red!12!white}    & \sigma^2 \tau \cellcolor{yellow!24!white}   & \tau \cellcolor{yellow!6!white} & \sigma\tau \cellcolor{yellow!12!white}   &  \sigma^2\tau^2 \cellcolor{green!24!white}  & \tau^2 \cellcolor{green!6!white} & \sigma\tau^2\cellcolor{green!12!white}  & \sigma^2\tau\sigma^2\cellcolor{blue!24!white} & \tau\sigma^2\cellcolor{blue!6!white} & \sigma\tau\sigma^2\cellcolor{blue!12!white}\\  
      \tau \cellcolor{yellow!6!white}      & \tau \cellcolor{yellow!6!white}     & \sigma^2\tau^2 \cellcolor{green!24!white}   & \tau\sigma^2 \cellcolor{blue!6!white} & \tau^2\cellcolor{green!6!white}    & \sigma^2\cellcolor{red!24!white}    & \sigma^2\tau\sigma^2\cellcolor{blue!24!white}  & e \cellcolor{red!6!white} & \sigma^2\tau \cellcolor{yellow!24!white} &  \sigma\tau\sigma^2\cellcolor{blue!12!white}  &  \sigma\tau \cellcolor{yellow!12!white}  &  \sigma \cellcolor{red!12!white}  &  \sigma\tau^2 \cellcolor{green!12!white}   \\
      \sigma \tau \cellcolor{yellow!12!white}   & \sigma\tau \cellcolor{yellow!12!white}   & \tau^2 \cellcolor{green!6!white}   & \sigma\tau\sigma^2 \cellcolor{blue!12!white}    & \sigma\tau^2\cellcolor{green!12!white}   & e\cellcolor{red!6!white}     & \tau\sigma^2\cellcolor{blue!6!white}  & \sigma\cellcolor{red!12!white}  &  \tau \cellcolor{yellow!6!white}  & \sigma^2\tau\sigma^2\cellcolor{blue!24!white} & \sigma^2\tau \cellcolor{yellow!24!white}  &  \sigma^2 \cellcolor{red!24!white}  & \sigma^2\tau^2 \cellcolor{green!24!white}  \\  
      \sigma^2 \tau\cellcolor{yellow!24!white}  & \sigma^2 \tau\cellcolor{yellow!24!white} & \sigma\tau^2 \cellcolor{green!12!white}    & \sigma^2 \tau\sigma^2\cellcolor{blue!24!white}   & \sigma^2\tau^2\cellcolor{green!24!white}    & \sigma\cellcolor{red!12!white}   & \sigma\tau\sigma^2\cellcolor{blue!12!white} & \sigma^2\cellcolor{red!24!white} &  \sigma\tau \cellcolor{yellow!12!white}  &  \tau\sigma^2 \cellcolor{blue!6!white}  & \tau \cellcolor{yellow!6!white}  &  e \cellcolor{red!6!white}  & \tau^2 \cellcolor{green!6!white}    \\
      \tau^2\cellcolor{green!6!white}      & \tau^2\cellcolor{green!6!white} & \sigma\tau\sigma^2 \cellcolor{blue!12!white} & \sigma\tau \cellcolor{yellow!12!white} & e\cellcolor{red!6!white} &  \tau\sigma^2 \cellcolor{blue!6!white}  &  \sigma\tau^2 \cellcolor{green!12!white} &  \tau \cellcolor{yellow!6!white}  & \sigma^2\tau\sigma^2 \cellcolor{blue!24!white}  & \sigma \cellcolor{red!12!white}   & \sigma^2\cellcolor{red!24!white} &  \sigma^2\tau^2 \cellcolor{green!24!white}  & \sigma^2\tau \cellcolor{yellow!24!white}\\
      \sigma\tau^2\cellcolor{green!12!white}    & \sigma\tau^2\cellcolor{green!12!white} & \sigma^2\tau\sigma^2\cellcolor{blue!24!white} & \sigma^2\tau \cellcolor{yellow!24!white} & \sigma\cellcolor{red!12!white}& \sigma\tau\sigma^2 \cellcolor{blue!12!white}  & \sigma^2\tau^2 \cellcolor{green!24!white} &  \sigma\tau \cellcolor{yellow!12!white}  & \tau\sigma^2 \cellcolor{blue!6!white}  & \sigma^2 \cellcolor{red!24!white}   & e\cellcolor{red!6!white} &  \tau^2 \cellcolor{green!6!white}  & \tau \cellcolor{yellow!6!white}\\
      \sigma^2\tau^2\cellcolor{green!24!white}  & \sigma^2\tau^2\cellcolor{green!24!white}& \tau\sigma^2 \cellcolor{blue!6!white} & \tau\cellcolor{yellow!6!white} & \sigma^2\cellcolor{red!24!white}&  \sigma^2\tau\sigma^2 \cellcolor{blue!24!white}  &  \tau^2 \cellcolor{green!6!white} &  \sigma^2\tau \cellcolor{yellow!24!white}  & \sigma\tau\sigma^2 \cellcolor{blue!12!white}  & e \cellcolor{red!6!white}   & \sigma\cellcolor{red!12!white} &  \sigma\tau^2 \cellcolor{green!12!white}  & \sigma\tau \cellcolor{yellow!12!white}\\
      \tau\sigma^2\cellcolor{blue!6!white}      & \tau\sigma^2\cellcolor{blue!6!white} & \tau\cellcolor{yellow!6!white} & \sigma^2\tau^2 \cellcolor{green!24!white} & \sigma^2\tau\sigma^2\cellcolor{blue!24!white}& \tau^2\cellcolor{green!6!white}& \sigma^2\cellcolor{red!24!white} & \sigma\tau\sigma^2\cellcolor{blue!12!white} & e\cellcolor{red!6!white}  &  \sigma^2\tau \cellcolor{yellow!24!white}  & \sigma\tau^2 \cellcolor{green!12!white}  & \sigma\tau \cellcolor{yellow!12!white}  & \sigma \cellcolor{red!12!white}\\
      \sigma\tau\sigma^2\cellcolor{blue!12!white}    & \sigma\tau\sigma^2\cellcolor{blue!12!white} & \sigma\tau\cellcolor{yellow!12!white} & \tau^2\cellcolor{green!6!white} & \tau\sigma^2\cellcolor{blue!6!white} & \sigma\tau^2\cellcolor{green!12!white}& e\cellcolor{red!6!white} & \sigma^2\tau\sigma^2\cellcolor{blue!24!white} & \sigma\cellcolor{red!12!white}  &  \tau \cellcolor{yellow!6!white}  & \sigma^2\tau^2 \cellcolor{green!24!white}  &  \sigma^2\tau \cellcolor{yellow!24!white}  & \sigma^2 \cellcolor{red!24!white}\\
      \sigma^2\tau\sigma^2\cellcolor{blue!24!white}  & \sigma^2\tau\sigma^2\cellcolor{blue!24!white}& \sigma^2\tau\cellcolor{yellow!24!white} & \sigma\tau^2\cellcolor{green!12!white} & \sigma\tau\sigma^2\cellcolor{blue!12!white} & \sigma^2\tau^2\cellcolor{green!24!white}& \sigma\cellcolor{red!12!white} & \tau\sigma^2\cellcolor{blue!6!white} & \sigma^2\cellcolor{red!24!white}  &  \sigma\tau \cellcolor{yellow!12!white}  & \tau^2 \cellcolor{green!6!white}  &  \tau \cellcolor{yellow!6!white}  & e \cellcolor{red!6!white}\\
    \end{array}
    \end{gather*}

From this Cayley table we see that $A_4\iso \mathcal{T}_{12}$.
  
  
\end{example}



\begin{question}
  Groups of permutations seem really expressive.  Is every (finite)
  group isomorphic to a group of permutations?
\end{question}



\end{document}
