\documentclass{ximera}

\usepackage[T1]{fontenc}
\usepackage{stix2}
\usepackage{gillius}
\usepackage{resizegather}
%\usepackage{rsfso} fancy cal
\DeclareMathAlphabet{\mathcal}{OMS}{cmsy}{m}{n} %less fancy cal


\usepackage{multicol}


\usepackage{tikz-cd}
\tikzset{>=stealth}
\tikzcdset{arrow style=tikz}
\usetikzlibrary{math} %% for assigning variables
%\usetikzlibrary{fadings}

\usepackage{colortbl,boldline,makecell} %% group tables


\usepackage[sans]{dsfont}

\usepackage{stmaryrd,pifont}


\let\oldbibliography\thebibliography%% to compact bib
\renewcommand{\thebibliography}[1]{%
  \oldbibliography{#1}%
  \setlength{\itemsep}{0pt}%
}
\renewcommand\refname{} %% no name needed!


\DefineVerbatimEnvironment{macaulay2}{Verbatim}{numbers=left,frame=lines,label=Macaulay2,labelposition=topline}

\DefineVerbatimEnvironment{gap}{Verbatim}{numbers=left,frame=lines,label=GAP,labelposition=topline}

%%% This next bit of code defines all our theorem environments
\makeatletter
\let\c@theorem\relax
\let\c@corollary\relax
\let\c@example\relax
\makeatother

\let\definition\relax
\let\enddefinition\relax

\let\theorem\relax
\let\endtheorem\relax

\let\proposition\relax
\let\endproposition\relax

\let\exercise\relax
\let\endexercise\relax

\let\question\relax
\let\endquestion\relax

\let\remark\relax
\let\endremark\relax

\let\corollary\relax
\let\endcorollary\relax


\let\example\relax
\let\endexample\relax

\let\warning\relax
\let\endwarning\relax

\let\lemma\relax
\let\endlemma\relax


\let\algorithm\relax
\let\endalgorithm\relax
\usepackage{algpseudocode}

\newtheoremstyle{SlantTheorem}{\topsep}{\topsep}%%% space between body and thm
		{\slshape}                      %%% Thm body font
		{}                              %%% Indent amount (empty = no indent)
		{\bfseries\sffamily}            %%% Thm head font
		{}                              %%% Punctuation after thm head
		{3ex}                           %%% Space after thm head
		{\thmname{#1}\thmnumber{ #2}\thmnote{ \bfseries(#3)}}%%% Thm head spec
\theoremstyle{SlantTheorem}
\newtheorem{theorem}{Theorem}
%\newtheorem{definition}[theorem]{Definition}
%\newtheorem{proposition}[theorem]{Proposition}
%% \newtheorem*{dfnn}{Definition}
%% \newtheorem{ques}{Question}[theorem]
%% \newtheorem*{war}{WARNING}
%% \newtheorem*{cor}{Corollary}
%% \newtheorem*{eg}{Example}
\newtheorem*{remark}{Remark}
\newtheorem*{touchstone}{Touchstone}
\newtheorem{corollary}{Corollary}[theorem]
\newtheorem*{warning}{WARNING}
\newtheorem{example}{Example}[theorem]
\newtheorem{lemma}[theorem]{Lemma}




\newtheoremstyle{Definition}{\topsep}{\topsep}%%% space between body and thm
		{}                              %%% Thm body font
		{}                              %%% Indent amount (empty = no indent)
		{\bfseries\sffamily}            %%% Thm head font
		{}                              %%% Punctuation after thm head
		{3ex}                           %%% Space after thm head
		{\thmname{#1}\thmnumber{ #2}\thmnote{ \bfseries(#3)}}%%% Thm head spec
\theoremstyle{Definition}
\newtheorem{definition}[theorem]{Definition}



\let\algorithm\relax
\let\endalgorithm\relax
\newtheoremstyle{Alg}{\topsep}{\topsep}%%% space between body and thm
		{}                              %%% Thm body font
		{}                              %%% Indent amount (empty = no indent)
		{\bfseries\sffamily}            %%% Thm head font
		{}                              %%% Punctuation after thm head
		{3ex}                           %%% Space after thm head
		{\thmname{#1}\thmnumber{ #2}\thmnote{ \bfseries(#3)}}%%% Thm head spec
\theoremstyle{Alg}
\newtheorem*{algorithm}{Algorithm}




\newtheoremstyle{Exercise}{\topsep}{\topsep} %%% space between body and thm
		{}                           %%% Thm body font
		{}                           %%% Indent amount (empty = no indent)
		{\bfseries\sffamily}         %%% Thm head font
		{)}                          %%% Punctuation after thm head
		{ }                          %%% Space after thm head
		{\thmnumber{#2}\thmnote{ \bfseries(#3)}}%%% Thm head spec
\theoremstyle{Exercise}
\newtheorem{exercise}{}[theorem]

%% \newtheoremstyle{Question}{\topsep}{\topsep} %%% space between body and thm
%% 		{\bfseries}                  %%% Thm body font
%% 		{3ex}                        %%% Indent amount (empty = no indent)
%% 		{}                           %%% Thm head font
%% 		{}                           %%% Punctuation after thm head
%% 		{}                           %%% Space after thm head
%% 		{\thmnumber{#2}\thmnote{ \bfseries(#3)}}%%% Thm head spec
\newtheoremstyle{Question}{3em}{3em} %%% space between body and thm
		{\large\bfseries}                           %%% Thm body font
		{}                           %%% Indent amount (empty = no indent)
		{}                         %%% Thm head font
		{}                          %%% Punctuation after thm head
		{0em}                          %%% Space after thm head
		{}%%% Thm head spec
\theoremstyle{Question}
\newtheorem*{question}{}






\renewcommand{\tilde}{\widetilde}
\renewcommand{\bar}{\overline}
\renewcommand{\hat}{\widehat}
\newcommand{\N}{\mathbb N}
\newcommand{\Z}{\mathbb Z}
\newcommand{\R}{\mathbb R}
\newcommand{\Q}{\mathbb Q}
\newcommand{\C}{\mathbb C}
\newcommand{\V}{\mathbb V}
\newcommand{\I}{\mathbb I}
\newcommand{\A}{\mathbb A}
\renewcommand{\o}{\mathbf o}
\newcommand{\iso}{\simeq}
\newcommand{\ph}{\varphi}
\newcommand{\Cf}{\mathcal{C}}
\newcommand{\IZ}{\mathrm{Int}(\Z)}
\newcommand{\dsum}{\oplus}
\newcommand{\directsum}{\bigoplus}
\newcommand{\union}{\bigcup}
\newcommand{\subgp}{\leq}
\newcommand{\normal}{\trianglelefteq}
\renewcommand{\i}{\mathfrak}
\renewcommand{\a}{\mathfrak{a}}
\renewcommand{\b}{\mathfrak{b}}
\newcommand{\m}{\mathfrak{m}}
\newcommand{\p}{\mathfrak{p}}
\newcommand{\q}{\mathfrak{q}}
\newcommand{\dfn}[1]{\textbf{#1}\index{#1}}
\let\hom\relax
\DeclareMathOperator{\ann}{Ann}
\DeclareMathOperator{\h}{ht}
\DeclareMathOperator{\hom}{Hom}
\DeclareMathOperator{\Span}{Span}
\DeclareMathOperator{\spec}{Spec}
\DeclareMathOperator{\maxspec}{MaxSpec}
\DeclareMathOperator{\aut}{Aut}
\DeclareMathOperator{\ass}{Ass}
\DeclareMathOperator{\lcm}{lcm}
\DeclareMathOperator{\ff}{Frac}
\DeclareMathOperator{\im}{Im}
\DeclareMathOperator{\syz}{Syz}
\DeclareMathOperator{\gr}{Gr}
\DeclareMathOperator{\multideg}{multideg}
\renewcommand{\ker}{\mathop{\mathrm{Ker}}\nolimits}
\newcommand{\coker}{\mathop{\mathrm{Coker}}\nolimits}
\newcommand{\lps}{[\hspace{-0.25ex}[}
\newcommand{\rps}{]\hspace{-0.25ex}]}
\newcommand{\into}{\hookrightarrow}
\newcommand{\onto}{\twoheadrightarrow}
\newcommand{\tensor}{\otimes}
\newcommand{\x}{\mathbf{x}}
\newcommand{\X}{\mathbf X}
\newcommand{\Y}{\mathbf Y}
\renewcommand{\k}{\boldsymbol{\kappa}}
\renewcommand{\emptyset}{\varnothing}
\renewcommand{\qedsymbol}{$\blacksquare$}
\renewcommand{\l}{\ell}
\newcommand{\1}{\mathds{1}}
\newcommand{\lto}{\mathop{\longrightarrow\,}\limits}
\newcommand{\rad}{\sqrt}
\newcommand{\hf}{H}
\newcommand{\hs}{H\!S}
\newcommand{\hp}{H\!P}
\renewcommand{\vec}{\mathbf}
\let\temp\phi
\let\phi\varphi
\let\eulerphi\temp


\renewcommand{\epsilon}{\varepsilon}
\renewcommand{\subset}{\subseteq}
\renewcommand{\supset}{\supseteq}
\newcommand{\macaulay}{\normalfont\textsl{Macaulay2}}
\newcommand{\GAP}{\normalfont\textsf{GAP}}
\newcommand{\invlim}{\varprojlim}
\renewcommand{\le}{\leqslant}
\renewcommand{\ge}{\geqslant}
\newcommand{\valpha}{{\boldsymbol\alpha}}
\newcommand{\vbeta}{{\boldsymbol\beta}}
\newcommand{\vgamma}{{\boldsymbol\gamma}}
\newcommand{\dotp}{\bullet}
\newcommand{\lc}{\normalfont\textsc{lc}}
\newcommand{\lt}{\normalfont\textsc{lt}}
\newcommand{\lm}{\normalfont\textsc{lm}}
\newcommand{\from}{\leftarrow}
\newcommand{\transpose}{\intercal}
\newcommand{\grad}{\boldsymbol\nabla}
\newcommand{\curl}{\boldsymbol{\nabla\times}}
\renewcommand{\d}{\, d}
\newcommand{\<}{\langle}
\renewcommand{\>}{\rangle}

%\renewcommand{\proofname}{Sketch of Proof}


\renewenvironment{proof}[1][Proof]
  {\begin{trivlist}\item[\hskip \labelsep \itshape \bfseries #1{}\hspace{2ex}]\upshape}
{\qed\end{trivlist}}

\newenvironment{sketch}[1][Sketch of Proof]
  {\begin{trivlist}\item[\hskip \labelsep \itshape \bfseries #1{}\hspace{2ex}]\upshape}
{\qed\end{trivlist}}



\makeatletter
\renewcommand\section{\@startsection{paragraph}{10}{\z@}%
                                     {-3.25ex\@plus -1ex \@minus -.2ex}%
                                     {1.5ex \@plus .2ex}%
                                     {\normalfont\large\sffamily\bfseries}}
\renewcommand\subsection{\@startsection{subparagraph}{10}{\z@}%
                                    {3.25ex \@plus1ex \@minus.2ex}%
                                    {-1em}%
                                    {\normalfont\normalsize\sffamily\bfseries}}
\makeatother

%% Fix weird index/bib issue.
\makeatletter
\gdef\ttl@savemark{\sectionmark{}}
\makeatother


\makeatletter
%% no number for refs
\newcommand\frontstyle{%
  \def\activitystyle{activity-chapter}
  \def\maketitle{%
    \addtocounter{titlenumber}{1}%
                    {\flushleft\small\sffamily\bfseries\@pretitle\par\vspace{-1.5em}}%
                    {\flushleft\LARGE\sffamily\bfseries\@title \par }%
                    {\vskip .6em\noindent\textit\theabstract\setcounter{problem}{0}\setcounter{sectiontitlenumber}{0}}%
                    \par\vspace{2em}
                    \phantomsection\addcontentsline{toc}{section}{\textbf{\@title}}%
                  }}
\makeatother



\NewEnviron{annotate}{\vspace{-.3cm}\small \itshape \BODY \vspace{.3cm}}


%%%% TIKZ STUFF

%% N-GON code
\tikzset{
    pics/tikzngon/.style={
        code={
        \tikzmath{\xx = #1;\rr=1.7;}
        \draw[ultra thick,rounded corners=.05mm] ({\rr*sin(0*360/\xx)},{\rr*cos(0*360/\xx)})
        \foreach \x in {0,1,...,\xx+1}
           {
           -- ({\rr*sin(\x*360/\xx)},{\rr*cos(\x*360/\xx)}) 
           }
           -- cycle;
  }}}

%% N-GON code (even)
\tikzset{
    pics/tikzegon/.style={
        code={
        \tikzmath{\xx = #1;\rr=1.7;}
        \draw[ultra thick,rounded corners=.05mm] ({\rr*sin(0*360/\xx+180/\xx)},{\rr*cos(0*360/\xx+180/\xx)})
        \foreach \x in {0,1,...,\xx+1}
           {
           -- ({\rr*sin(\x*360/\xx+180/\xx)},{\rr*cos(\x*360/\xx+180/\xx)}) 
           }
           -- cycle;
  }}}




%% N-CLOCK code
\tikzset{
    pics/tikznclock/.style={
        code={
        \tikzmath{\xx = #1;\rr=1.7;\dd=.4;}
        \foreach \x in {0,1,...,\xx-1}
           {
             \node[circle,fill=black,inner sep=0pt, minimum size=13pt,text=white]
             at ({(\rr-\dd)*sin(\x*360/\xx)},{(\rr-\dd)*cos(\x*360/\xx}) {\normalfont\bfseries\sffamily\small \x};
           }
  \draw[thick] (0,0) circle (\rr);
  }}}



%% barcode from
%% https://tex.stackexchange.com/questions/6895/is-there-a-good-latex-package-for-generating-barcodes
%% NOT CURRENTLY USED!


\def\barcode#1#2#3#4#5#6#7{\begingroup%
  \dimen0=0.1em
  \def\stack##1##2{\oalign{##1\cr\hidewidth##2\hidewidth}}%
  \def\0##1{\kern##1\dimen0}%
  \def\1##1{\vrule height10ex width##1\dimen0}%
  \def\L##1{\ifcase##1\bc3211##1\or\bc2221##1\or\bc2122##1\or\bc1411##1%
    \or\bc1132##1\or\bc1231##1\or\bc1114##1\or\bc1312##1\or\bc1213##1%
    \or\bc3112##1\fi}%
  \def\R##1{\bgroup\let\next\1\let\1\0\let\0\next\L##1\egroup}%
  \def\G##1{\bgroup\let\bc\bcg\L##1\egroup}% reverse
  \def\bc##1##2##3##4##5{\stack{\0##1\1##2\0##3\1##4}##5}%
  \def\bcg##1##2##3##4##5{\stack{\0##4\1##3\0##2\1##1}##5}%
  \def\bcR##1##2##3##4##5##6{\R##1\R##2\R##3\R##4\R##5\R##6\11\01\11\09%
    \endgroup}%
  \stack{\09}#1\11\01\11\L#2%
  \ifcase#1\L#3\L#4\L#5\L#6\L#7\or\L#3\G#4\L#5\G#6\G#7%
    \or\L#3\G#4\G#5\L#6\G#7\or\L#3\G#4\G#5\G#6\L#7%
    \or\G#3\L#4\L#5\G#6\G#7\or\G#3\G#4\L#5\L#6\G#7%
    \or\G#3\G#4\G#5\L#6\L#7\or\G#3\L#4\G#5\L#6\G#7%
    \or\G#3\L#4\G#5\G#6\L#7\or\G#3\G#4\L#5\G#6\L#7%
  \fi\01\11\01\11\01\bcR}


\title{Number theory and cyclic groups}

\begin{document}
\begin{abstract}
  We further explore cyclic groups.
\end{abstract}
\maketitle


\begin{definition} \index{greatest common divisor}\index{GCD}
An integer $g$ is called a \textbf{greatest common divisor} of two
integer $a$ and $b$ provided that
\begin{enumerate}
\item $g| a$ and $g | b$.
\item If $d$ is an element where $d| a$ and $d | b$, then $d\le g$.
\end{enumerate}
We denote the GCD of $a$ and $b$ as $\gcd(a,b)$. If $\gcd(a,b) = 1$ we
say that $a$ and $b$ are \dfn{coprime} or \dfn{relatively prime}.
\end{definition}


\begin{exercise} 
Given $n\in \Z$, what is $\gcd(n,0)$? Prove your conclusion.
\end{exercise}

\begin{exercise} 
Prove that if $n$ is any integer, then $\gcd(a + n\cdot b, b) = \gcd(a, b)$.
\end{exercise}




\begin{theorem}[Euclid's lemma, version 1]\label{T:EL1}\index{Euclid's lemma}
  Given nonzero $a,b\in\Z$, we will prove that $g=\gcd(a,b)$ is the
  \textit{smallest} positive integer such that
  \[
  g = a\cdot m +b\cdot n
  \]
  for some integers $m$ and $n$.
  \begin{enumerate}
  \item Let $\mathcal S = \{x\in \N: x = a\cdot m +b\cdot n\text{ for } m,n\in
    \Z\}$. Prove that $\mathcal S$ has a least element, call it $d$.
  \item Prove that $d| x$ for all $x\in \mathcal S$.
  \item Prove that $d| a$ and $d| b$. Explain why $1 \le d \le g$.
  \item Recall that $d = a\cdot m +b\cdot n$, and prove that $g |
    d$. Explain why we must conclude that $d = g$.
  \end{enumerate}
\end{theorem}



\begin{corollary}[Euclid's lemma, version 2]\label{C:EL2}\index{Euclid's lemma}
  Let $a$, $b$, and $c$ be nonzero integers. Suppose that 
  \[
  a |bc\qquad\text{and}\qquad \gcd(a,b)=1.
  \]
  Prove that $a|c$. 
\end{corollary}


The statements of Theorem~\ref{T:EL1} and Corollary~\ref{C:EL2} are
each sometimes referred to as \textbf{Euclid's lemma}.\index{Euclid's lemma}



\begin{corollary}[A characterization of prime numbers]\label{E:EL}\index{prime number}
  A natural number $p\in \N$ has exactly two factors, $1$ and $p$ if
  and only if for all $a,b\in\N$
  \[
  p|ab \quad \Rightarrow \quad p|a \text{ or } p|b.
  \]
  \begin{sketch}
  \begin{enumerate}
  \item Suppose that $p\nmid a$, explain why $\gcd(a,p) =1$.
  \item Write $1 = am + pn$, multiply both sides by $b$.
  \item Can you finish it from here?
  \end{enumerate}
  \end{sketch}
\end{corollary}






\begin{exercise} Prove that if $\gcd(m,n) = 1$, then 
\[
in \equiv jn \pmod m \qquad \Rightarrow\qquad i\equiv j\pmod m.
\]
\end{exercise}



\begin{exercise} Study the following calculations:
\begin{align*}
  22 &= \boldsymbol{6}\cdot 3 + \boldsymbol{4}\\
  \boldsymbol{6} &= \boldsymbol{4} \cdot 1 + \fbox{$\boldsymbol{2}$}\\
  \boldsymbol{4} &= \boldsymbol{2} \cdot 2 + 0 \qquad
\therefore \gcd(22,6) = 2
\end{align*}

\begin{align*}
33 &= \boldsymbol{24}\cdot 1 + \boldsymbol{9}\\
\boldsymbol{24} &= \boldsymbol{9} \cdot 2 + \boldsymbol{6}\\
\boldsymbol{9} &= \boldsymbol{6} \cdot 1 + \fbox{$\boldsymbol{3}$}\\
\boldsymbol{6} &= \boldsymbol{3} \cdot 2 + 0 \qquad \therefore \gcd(33,24) = 3 
\end{align*}

\begin{align*}
42 &= \boldsymbol{16}\cdot 2 + \boldsymbol{10}\\
\boldsymbol{16} &= \boldsymbol{10} \cdot 1 + \boldsymbol{6}\\
\boldsymbol{10} &= \boldsymbol{6} \cdot 1 + \boldsymbol{4}\\
\boldsymbol{6} &= \boldsymbol{4} \cdot 1 + \fbox{$\boldsymbol{2}$}\\
\boldsymbol{4} &= \boldsymbol{2} \cdot 2 + 0 \qquad \therefore \gcd(42,16) = 2 
\end{align*}

Explain how the above algorithm works and write it under the Euclidean
algorithm below.
\end{exercise}


\begin{theorem}[Euclidean algorithm] 
We can easily compute the GCD of two numbers using the following
algorithm:
\vspace{2in}

\end{theorem}
\noindent The above space has intentionally been left blank for you to
fill in.

\begin{exercise} 
We will prove that when working with integers, the Euclidean algorithm
will always produce the GCD of two numbers.
\begin{enumerate}
\item Prove that the remainders found in the Euclidean algorithm form a
  decreasing sequence.
\item Prove that this sequence must terminate with a final remainder
  of zero.
\item Proceed by induction on the number of steps in the Euclidean
  Algorithm. If there are two steps:
\begin{align*}
a &= b\cdot q_1 + g \\
b &= g\cdot q_2 +0
\end{align*}
Prove that any divisor of both $a$ and $b$ is necessarily a divisor
of $g$. Explain why this proves that $g = \gcd(a,b)$.
\item Now suppose that any time we have $n+1$ equations:
\begin{align*}
a &= b\cdot q_1 + r_1 \\
b &= r_1\cdot q_2 + r_2 \\
  &\hspace{.5em}\vdots\\
r_{n-2} &= r_{n-1}\cdot q_{n} + r_n\\
r_{n-1} &= r_{n}\cdot q_{n+1} + 0
\end{align*}
that $r_n = \gcd(a,b)$. Prove that when we have $n+2$ equations, $r_{n+1}
= \gcd(a,b)$.
\item Explain how we have proved that when working with integers, the
  Euclidean algorithm will always produce the GCD of two numbers.
\end{enumerate}
\end{exercise}


\begin{exercise} 
Prove that if $x = am +bn$ and $d$ is a common divisor of $a$ and $b$,
then $d|x$. What does this say about the GCD of $a$ and $b$?
\end{exercise}



\begin{corollary}[Generators of cyclic groups]
  An element $g\in \Z_n$ generates $\Z_n$ if and only if $\gcd(g,n) =
  1$.
\end{corollary}






\section{Direct products of cyclic groups}

\[
|G\times H| = |G|\cdot |H|
\]
When is $\Z_m\times\Z_n$ cyclic?



Here's another famous group you might run into. The \textit{Klein
  four-group}.

\begin{definition}
  The Klein four-group is $V_4 = \Z_2\times\Z_2$ with modular addition
  being the operation.
\end{definition}

\begin{exercise}
  Make a Cayley table for $V_4$.
\end{exercise}

\begin{exercise}
  What are the generators of $V_4$?
\end{exercise}








\section{Multiplicative groups}


The cyclic groups $\Z_n$ are groups under addition. However, we also
can multiply elements in $\Z_n$. We now show that modular
multiplication is well defined.

\begin{lemma}[Modular multiplication is well defined]
  Addition on $\Z_n$ is well defined. This means that if
  \begin{align*}
    a &\equiv a' \pmod{n}\\
    b &\equiv b' \pmod{n},
  \end{align*}
  then
  \[
  a\cdot b \equiv a'\cdot b' \pmod{n}.
  \]
  \begin{sketch}
    Use the fact that
    \[
    x \equiv x'\pmod{n} \quad \Leftrightarrow \quad x -x' = n\cdot q
    \]
    for some number $q$.
  \end{sketch}
\end{lemma}

However, $\Z_n$ is \textbf{not a group under multiplication.} The
issue is that $0$ has no inverse. Moreover, any number $x|n$ will also
not have an inverse. We should state this as a lemma.

\begin{lemma}[Multiplicative inverses]
  An element $a\in\Z_n$ has a multiplicative inverse if and only if
  $\gcd(a,n) = 1$.
  \begin{sketch}
    Use Euclid's lemma, Theorem~\ref{T:EL1}.
  \end{sketch}
\end{lemma}

\begin{definition}
  The \dfn{Euler totient function}, $\eulerphi:\N\to\N$, is defined as
  follows:
  \[
  \eulerphi(n) = |\Z_n^\times|.
  \]
\end{definition}

\begin{exercise}
  Compute $\eulerphi(n)$ for $n = 1,2,\dots, 10$.
\end{exercise}


\begin{exercise}
  Make a Cayley table for $\Z_n^\times$ for $n = 3,4,5,6,7,8,9$.
\end{exercise}




\end{document}
