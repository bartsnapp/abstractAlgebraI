\documentclass{ximera}

\usepackage[T1]{fontenc}
\usepackage{stix2}
\usepackage{gillius}
\usepackage{resizegather}
%\usepackage{rsfso} fancy cal
\DeclareMathAlphabet{\mathcal}{OMS}{cmsy}{m}{n} %less fancy cal


\usepackage{multicol}


\usepackage{tikz-cd}
\usepackage{tkz-euclide} %% compass
\usetkzobj{all}  %% tkzCompass
\tikzset{>=stealth}
\tikzcdset{arrow style=tikz}
\usetikzlibrary{math} %% for assigning variables
%\usetikzlibrary{fadings}

\usepackage{colortbl,boldline,makecell} %% group tables


\usepackage[sans]{dsfont}

\usepackage{stmaryrd,pifont}

\graphicspath{
  {./}
  {fields/}
  }     



\let\oldbibliography\thebibliography%% to compact bib
\renewcommand{\thebibliography}[1]{%
  \oldbibliography{#1}%
  \setlength{\itemsep}{0pt}%
}
\renewcommand\refname{} %% no name needed!


\DefineVerbatimEnvironment{macaulay2}{Verbatim}{numbers=left,frame=lines,label=Macaulay2,labelposition=topline}

\DefineVerbatimEnvironment{gap}{Verbatim}{numbers=left,frame=lines,label=GAP,labelposition=topline}

%%% This next bit of code defines all our theorem environments
\makeatletter
\let\c@theorem\relax
\let\c@corollary\relax
%\let\c@example\relax
\makeatother

\let\definition\relax
\let\enddefinition\relax

\let\theorem\relax
\let\endtheorem\relax

\let\proposition\relax
\let\endproposition\relax

\let\exercise\relax
\let\endexercise\relax

\let\question\relax
\let\endquestion\relax

\let\remark\relax
\let\endremark\relax

\let\corollary\relax
\let\endcorollary\relax


\let\example\relax
\let\endexample\relax

\let\warning\relax
\let\endwarning\relax

\let\lemma\relax
\let\endlemma\relax


\let\algorithm\relax
\let\endalgorithm\relax
\usepackage{algpseudocode}

\newtheoremstyle{SlantTheorem}{\topsep}{\topsep}%%% space between body and thm
		{\slshape}                      %%% Thm body font
		{}                              %%% Indent amount (empty = no indent)
		{\bfseries\sffamily}            %%% Thm head font
		{}                              %%% Punctuation after thm head
		{3ex}                           %%% Space after thm head
		{\thmname{#1}\thmnumber{ #2}\thmnote{ \bfseries(#3)}}%%% Thm head spec
\theoremstyle{SlantTheorem}
\newtheorem{theorem}{Theorem}
%\newtheorem{definition}[theorem]{Definition}
%\newtheorem{proposition}[theorem]{Proposition}
%% \newtheorem*{dfnn}{Definition}
%% \newtheorem{ques}{Question}[theorem]
%% \newtheorem*{war}{WARNING}
%% \newtheorem*{cor}{Corollary}
%% \newtheorem*{eg}{Example}
\newtheorem*{remark}{Remark}
\newtheorem*{touchstone}{Touchstone}
\newtheorem{corollary}{Corollary}[theorem]
\newtheorem*{warning}{WARNING}
\newtheorem{example}[corollary]{Example}
\newtheorem{lemma}[theorem]{Lemma}




\newtheoremstyle{Definition}{\topsep}{\topsep}%%% space between body and thm
		{}                              %%% Thm body font
		{}                              %%% Indent amount (empty = no indent)
		{\bfseries\sffamily}            %%% Thm head font
		{}                              %%% Punctuation after thm head
		{3ex}                           %%% Space after thm head
		{\thmname{#1}\thmnumber{ #2}\thmnote{ \bfseries(#3)}}%%% Thm head spec
\theoremstyle{Definition}
\newtheorem{definition}[theorem]{Definition}



\let\algorithm\relax
\let\endalgorithm\relax
\newtheoremstyle{Alg}{\topsep}{\topsep}%%% space between body and thm
		{}                              %%% Thm body font
		{}                              %%% Indent amount (empty = no indent)
		{\bfseries\sffamily}            %%% Thm head font
		{}                              %%% Punctuation after thm head
		{3ex}                           %%% Space after thm head
		{\thmname{#1}\thmnumber{ #2}\thmnote{ \bfseries(#3)}}%%% Thm head spec
\theoremstyle{Alg}
\newtheorem*{algorithm}{Algorithm}
\newtheorem*{construction}{Construction}




\newtheoremstyle{Exercise}{\topsep}{\topsep} %%% space between body and thm
		{}                           %%% Thm body font
		{}                           %%% Indent amount (empty = no indent)
		{\bfseries\sffamily}         %%% Thm head font
		{)}                          %%% Punctuation after thm head
		{ }                          %%% Space after thm head
		{\thmnumber{#2}\thmnote{ \bfseries(#3)}}%%% Thm head spec
\theoremstyle{Exercise}
\newtheorem{exercise}[corollary]{}%[theorem]

%% \newtheoremstyle{Question}{\topsep}{\topsep} %%% space between body and thm
%% 		{\bfseries}                  %%% Thm body font
%% 		{3ex}                        %%% Indent amount (empty = no indent)
%% 		{}                           %%% Thm head font
%% 		{}                           %%% Punctuation after thm head
%% 		{}                           %%% Space after thm head
%% 		{\thmnumber{#2}\thmnote{ \bfseries(#3)}}%%% Thm head spec
\newtheoremstyle{Question}{3em}{3em} %%% space between body and thm
		{\large\bfseries}                           %%% Thm body font
		{}                           %%% Indent amount (empty = no indent)
		{}                         %%% Thm head font
		{}                          %%% Punctuation after thm head
		{0em}                          %%% Space after thm head
		{}%%% Thm head spec
\theoremstyle{Question}
\newtheorem*{question}{}






\renewcommand{\tilde}{\widetilde}
\renewcommand{\bar}{\overline}
\renewcommand{\hat}{\widehat}
\newcommand{\N}{\mathbb N}
\newcommand{\Z}{\mathbb Z}
\newcommand{\R}{\mathbb R}
\newcommand{\Q}{\mathbb Q}
\newcommand{\C}{\mathbb C}
\newcommand{\V}{\mathbb V}
\newcommand{\I}{\mathbb I}
\newcommand{\A}{\mathbb A}
\renewcommand{\o}{\mathbf o}
\newcommand{\iso}{\simeq}
\newcommand{\ph}{\varphi}
\newcommand{\Cf}{\mathcal{C}}
\newcommand{\IZ}{\mathrm{Int}(\Z)}
\newcommand{\dsum}{\oplus}
\newcommand{\directsum}{\bigoplus}
\newcommand{\union}{\bigcup}
\newcommand{\subgp}{\leq}
\newcommand{\normal}{\trianglelefteq}
\renewcommand{\i}{\mathfrak}
\renewcommand{\a}{\mathfrak{a}}
\renewcommand{\b}{\mathfrak{b}}
\newcommand{\m}{\mathfrak{m}}
\newcommand{\p}{\mathfrak{p}}
\newcommand{\q}{\mathfrak{q}}
\newcommand{\dfn}[1]{\textbf{#1}\index{#1}}
\let\hom\relax
\DeclareMathOperator{\mat}{Mat}
\DeclareMathOperator{\ann}{Ann}
\DeclareMathOperator{\h}{ht}
\DeclareMathOperator{\tr}{tr}
\DeclareMathOperator{\hom}{Hom}
\DeclareMathOperator{\Span}{Span}
\DeclareMathOperator{\spec}{Spec}
\DeclareMathOperator{\maxspec}{MaxSpec}
\DeclareMathOperator{\aut}{Aut}
\DeclareMathOperator{\ass}{Ass}
\DeclareMathOperator{\lcm}{lcm}
\DeclareMathOperator{\ff}{Frac}
\DeclareMathOperator{\im}{Im}
\DeclareMathOperator{\syz}{Syz}
\DeclareMathOperator{\gr}{Gr}
\DeclareMathOperator{\multideg}{multideg}
\renewcommand{\ker}{\mathop{\mathrm{Ker}}\nolimits}
\newcommand{\coker}{\mathop{\mathrm{Coker}}\nolimits}
\newcommand{\lps}{[\hspace{-0.25ex}[}
\newcommand{\rps}{]\hspace{-0.25ex}]}
\newcommand{\into}{\hookrightarrow}
\newcommand{\onto}{\twoheadrightarrow}
\newcommand{\tensor}{\otimes}
\newcommand{\x}{\mathbf{x}}
\newcommand{\X}{\mathbf X}
\newcommand{\Y}{\mathbf Y}
\renewcommand{\k}{\boldsymbol{\kappa}}
\renewcommand{\emptyset}{\varnothing}
\renewcommand{\qedsymbol}{$\blacksquare$}
\renewcommand{\l}{\ell}
\newcommand{\1}{\mathds{1}}
\newcommand{\lto}{\mathop{\longrightarrow\,}\limits}
\newcommand{\rad}{\sqrt}
\newcommand{\hf}{H}
\newcommand{\hs}{H\!S}
\newcommand{\hp}{H\!P}
\renewcommand{\vec}{\mathbf}
\let\temp\phi
\let\phi\varphi
\let\eulerphi\temp


\renewcommand{\epsilon}{\varepsilon}
\renewcommand{\subset}{\subseteq}
\renewcommand{\supset}{\supseteq}
\newcommand{\macaulay}{\normalfont\textsl{Macaulay2}}
\newcommand{\GAP}{\normalfont\textsf{GAP}}
\newcommand{\invlim}{\varprojlim}
\renewcommand{\le}{\leqslant}
\renewcommand{\ge}{\geqslant}
\newcommand{\valpha}{{\boldsymbol\alpha}}
\newcommand{\vbeta}{{\boldsymbol\beta}}
\newcommand{\vgamma}{{\boldsymbol\gamma}}
\newcommand{\dotp}{\bullet}
\newcommand{\lc}{\normalfont\textsc{lc}}
\newcommand{\lt}{\normalfont\textsc{lt}}
\newcommand{\lm}{\normalfont\textsc{lm}}
\newcommand{\from}{\leftarrow}
\newcommand{\transpose}{\intercal}
\newcommand{\grad}{\boldsymbol\nabla}
\newcommand{\curl}{\boldsymbol{\nabla\times}}
\renewcommand{\d}{\, d}
\newcommand{\<}{\langle}
\renewcommand{\>}{\rangle}

%\renewcommand{\proofname}{Sketch of Proof}


\renewenvironment{proof}[1][Proof]
  {\begin{trivlist}\item[\hskip \labelsep \itshape \bfseries #1{}\hspace{2ex}]\upshape}
{\qed\end{trivlist}}

\newenvironment{sketch}[1][Sketch of Proof]
  {\begin{trivlist}\item[\hskip \labelsep \itshape \bfseries #1{}\hspace{2ex}]\upshape}
{\qed\end{trivlist}}



\makeatletter
\renewcommand\section{\@startsection{paragraph}{10}{\z@}%
                                     {-3.25ex\@plus -1ex \@minus -.2ex}%
                                     {1.5ex \@plus .2ex}%
                                     {\normalfont\large\sffamily\bfseries}}
\renewcommand\subsection{\@startsection{subparagraph}{10}{\z@}%
                                    {3.25ex \@plus1ex \@minus.2ex}%
                                    {-1em}%
                                    {\normalfont\normalsize\sffamily\bfseries}}
\makeatother

%% Fix weird index/bib issue.
\makeatletter
\gdef\ttl@savemark{\sectionmark{}}
\makeatother


\makeatletter
%% no number for refs
\newcommand\frontstyle{%
  \def\activitystyle{activity-chapter}
  \def\maketitle{%
    \addtocounter{titlenumber}{1}%
                    {\flushleft\small\sffamily\bfseries\@pretitle\par\vspace{-1.5em}}%
                    {\flushleft\LARGE\sffamily\bfseries\@title \par }%
                    {\vskip .6em\noindent\textit\theabstract\setcounter{problem}{0}\setcounter{sectiontitlenumber}{0}}%
                    \par\vspace{2em}
                    \phantomsection\addcontentsline{toc}{section}{\textbf{\@title}}%
                  }}
\makeatother



\NewEnviron{annotate}{\vspace{-.3cm}\small \itshape \BODY \vspace{.3cm}}


%%%% TIKZ STUFF

%% N-GON code
\tikzset{
    pics/tikzngon/.style={
        code={
        \tikzmath{\xx = #1;\rr=1.7;}
        \draw[ultra thick,rounded corners=.05mm] ({\rr*sin(0*360/\xx)},{\rr*cos(0*360/\xx)})
        \foreach \x in {-1,0,...,\xx}
        {
        -- ({\rr*sin(\x*360/\xx)},{\rr*cos(\x*360/\xx)})
        }
           -- cycle;
  }}}

%% N-GON code (even)
\tikzset{
    pics/tikzegon/.style={
        code={
        \tikzmath{\xx = #1;\rr=1.7;}
        \draw[ultra thick,rounded corners=.05mm] ({\rr*sin(0*360/\xx+180/\xx)},{\rr*cos(0*360/\xx+180/\xx)})
        \foreach \x in {-1,0,...,\xx}
           {
           -- ({\rr*sin(\x*360/\xx+180/\xx)},{\rr*cos(\x*360/\xx+180/\xx)}) 
           }
           -- cycle;
  }}}




%% N-CLOCK code
\tikzset{
    pics/tikznclock/.style={
        code={
        \tikzmath{\xx = #1;\rr=1.7;\dd=.4;}
        \foreach \x in {1,...,\xx}
        \pgfmathtruncatemacro{\xy}{\x-1}
           {
             \node[circle,fill=black,inner sep=0pt, minimum size=13pt,text=white]
             at ({(\rr-\dd)*sin((\x-1)*360/(\xx)},{(\rr-\dd)*cos((\x-1)*360/\xx}) {\normalfont\bfseries\sffamily\small {\xy}};
           }
  \draw[thick] (0,0) circle (\rr);
  }}}



%% barcode from
%% https://tex.stackexchange.com/questions/6895/is-there-a-good-latex-package-for-generating-barcodes
%% NOT CURRENTLY USED!


\def\barcode#1#2#3#4#5#6#7{\begingroup%
  \dimen0=0.1em
  \def\stack##1##2{\oalign{##1\cr\hidewidth##2\hidewidth}}%
  \def\0##1{\kern##1\dimen0}%
  \def\1##1{\vrule height10ex width##1\dimen0}%
  \def\L##1{\ifcase##1\bc3211##1\or\bc2221##1\or\bc2122##1\or\bc1411##1%
    \or\bc1132##1\or\bc1231##1\or\bc1114##1\or\bc1312##1\or\bc1213##1%
    \or\bc3112##1\fi}%
  \def\R##1{\bgroup\let\next\1\let\1\0\let\0\next\L##1\egroup}%
  \def\G##1{\bgroup\let\bc\bcg\L##1\egroup}% reverse
  \def\bc##1##2##3##4##5{\stack{\0##1\1##2\0##3\1##4}##5}%
  \def\bcg##1##2##3##4##5{\stack{\0##4\1##3\0##2\1##1}##5}%
  \def\bcR##1##2##3##4##5##6{\R##1\R##2\R##3\R##4\R##5\R##6\11\01\11\09%
    \endgroup}%
  \stack{\09}#1\11\01\11\L#2%
  \ifcase#1\L#3\L#4\L#5\L#6\L#7\or\L#3\G#4\L#5\G#6\G#7%
    \or\L#3\G#4\G#5\L#6\G#7\or\L#3\G#4\G#5\G#6\L#7%
    \or\G#3\L#4\L#5\G#6\G#7\or\G#3\G#4\L#5\L#6\G#7%
    \or\G#3\G#4\G#5\L#6\L#7\or\G#3\L#4\G#5\L#6\G#7%
    \or\G#3\L#4\G#5\G#6\L#7\or\G#3\G#4\L#5\G#6\L#7%
  \fi\01\11\01\11\01\bcR}


\title{Number theory and cyclic groups}

\begin{document}
\begin{abstract}
  We further explore cyclic groups.
\end{abstract}
\maketitle


\section{Euclid's lemma}

We start by introducing the GCD of two numbers and by proving
fundamental theorems about the GCD.


\begin{definition}\index{GCD}
An integer $g$ is called a \dfn{greatest common divisor} of two
integers $m$ and $n$ provided that:
\begin{enumerate}
\item $g| m$ and $g | n$.
\item If $d$ is an element where $d| m$ and $d | n$, then $d\le g$.
\end{enumerate}
We denote the GCD of $m$ and $n$ as $\gcd(m,n)$. If $\gcd(m,n) = 1$ we
say that $m$ and $n$ are \dfn{coprime} or \dfn{relatively prime}.
\end{definition}


\begin{exercise} 
Given $n\in \Z$, what is $\gcd(n,0)$? Prove your conclusion.
\end{exercise}

\begin{exercise} 
Prove that if $a$ is any integer, then $\gcd(m + a\cdot n, n) =
\gcd(m, n)$.
\end{exercise}




\begin{theorem}[Euclid's lemma, version 1]\label{T:EL1}\index{Euclid's lemma}
  Given nonzero $m,n\in\Z$, $g=\gcd(m,n)$ is the \textit{smallest}
  positive integer such that
  \[
  g = a\cdot m +b\cdot n
  \]
  for some integers $a$ and $b$.
  \begin{sketch} We will prove this in several steps.
  \begin{enumerate}
  \item Let $\mathcal S = \{x\in \N: x = a\cdot m +b\cdot n\text{ for } a,b\in
    \Z\}$. Prove that $\mathcal S$ has a least element, call it $d$.
  \item Prove that $d| x$ for all $x\in \mathcal S$.
  \item Prove that $d| m$ and $d| n$. Explain why $1 \le d \le \gcd(m,n)$.
  \item Recall that $d = a\cdot m +b\cdot n$, and prove that $\gcd(m,n) |
    d$. Explain why we must conclude that $d = \gcd(m,n)$.
  \end{enumerate}
  \end{sketch}
\end{theorem}

At this point let's pause to catch our breath. It might seem strange
that the $\gcd(m,n)$ is simultaneously the \textbf{largest divisor} and
\textbf{smallest positive value} of $a\cdot m + b\cdot n$. However
consider this
\[
\gcd(m,n) \in \{\text{common divisors of $m$ and $n$}\}\cap \mathcal{S}
\]
where $\mathcal{S}$ is the set from the proof of Euclid's
lemma. Here is a picture that attempts to convey the situation:
\[
\begin{tikzpicture}
  \draw[white,thin,shading = axis, top color = white, bottom color = gray] (-2,2) -- (0,0)-- (2,2) -- (-2,2) -- cycle;
  \draw[white,thin,shading = axis, bottom color = white, top color = gray] (-2,-2) -- (0,0)-- (2,-2) -- (-2,-2) -- cycle;
  \filldraw (0,0) circle (3pt);
  \node[right] at (0,0) {$\scriptstyle\gcd(m,n)$};
  \node at (0,1.5) {$\scriptstyle a\cdot m + b\cdot n\in\mathcal{S}$};
  \node at (0,-1.5) {\text{\tiny common divisors of $m$ and $n$}};
  \draw[->,ultra thick] (-3,-2) -- (-3,2);
  \node[left] at (-3,-1.5) {\text{\tiny smaller numbers}};
  \node[left] at (-3,1.5) {\text{\tiny larger numbers}};
\end{tikzpicture}
\]
The cone above $\gcd(m,n)$ is $\mathcal{S}$. The cone below
$\gcd(m,n)$ is the set of common divisors of $m$ and $n$.




\begin{corollary}[Euclid's lemma, version 2]\label{C:EL2}\index{Euclid's lemma}
  Let $\l$, $m$, and $n$ be nonzero integers. If
  \[
  \l|mn\qquad\text{and}\qquad \gcd(\l,m)=1,
  \]
  then $\l|n$. 
\end{corollary}


The statements of Theorem~\ref{T:EL1} and Corollary~\ref{C:EL2} are
each sometimes referred to as \dfn{Euclid's lemma}. This fact appears as \link[Proposition 30 in Book VII]{https://mathcs.clarku.edu/~djoyce/java/elements/bookVII/propVII30.html} of
\textit{Euclid's elements}.



\begin{corollary}[Generators of cyclic groups]
  An element $g\in \Z_n$ generates $\Z_n$ if and only if $\gcd(g,n) =
  1$.
  \begin{sketch}
    $(\Rightarrow)$ If $g\in\Z_n$ generates $\Z_n$, then there is
    $a\in\N$ such that
    \[
    ag \equiv 1\pmod{n}.
    \]
    Work from here to show that $\gcd(g,n) = 1$.
    

    $(\Leftarrow)$ If $\gcd(g,n) = 1$, use Euclid's lemma,
    Theorem~\ref{T:EL1}.
  \end{sketch}
\end{corollary}

\begin{exercise}
  Find all the generators of $\Z_{12}$. Explain why your solution is correct.
\end{exercise}


\begin{exercise}
  Find all the generators of $\Z_{24}$. Explain why your solution is correct.
\end{exercise}





\begin{corollary}[A characterization of prime numbers]\label{E:EL}\index{prime number}
  A natural number $p\in \N$ has exactly two factors, $1$ and $p$ if
  and only if for all $m,n\in\N$
  \[
  p|mn \quad \Rightarrow \quad p|m \text{ or } p|n.
  \]
  \begin{sketch}
    Follows from Euclids's lemma, Corollary~\ref{C:EL2}.
  \end{sketch}
\end{corollary}


\begin{exercise} Prove that if $\gcd(m,n) = 1$, then 
\[
in \equiv jn \pmod m \qquad \Rightarrow\qquad i\equiv j\pmod m.
\]
\end{exercise}



\begin{corollary}[Unique factorization of natrual numbers]\label{C:UFNN}
  Every natural number greater than $1$ can be expressed as a unique
  (up to order) product of prime numbers.
  \begin{proof}
    We'll write this proof in two of steps.

    %% \textbf{First we'll show that every natural number bigger than one
    %%   is divisible by a prime number.} Let
    %% \[
    %% \mathcal{S} = \{n\in\N: \text{$n>1$ and $n$ has no prime divisors}\}\subset \N.
    %% \]
    %% Seeking a contradiction, suppose that $\mathcal{S} \ne
    %% \emptyset$. By the \index{well-ordering principle}well-ordering
    %% principle, $\mathcal{S}$ has a lesst element, call it $\l\in
    %% \mathcal{S}$. By the definition of $\mathcal{S}$, $\l$ is not
    %% prime. Hence $\l = a\cdot b$, where $a,b\in \N$ with $a<\l$ and
    %% $b\le \l$. Since neither $a$ nor $b$ are in $\mathcal{S}$, both
    %% must have prime divisors, and thus so must $\l$, a
    %% contradiction. Hence every natural number bigger than one is
    %% divisible by a prime number.

    \textbf{First we'll show that every natural number bigger than one
      is a product of prime numbers.} Let 
    \[
    \mathcal{S} = \{n\in\N: \text{$n>1$ and $n$ is not a product of
      prime numbers}\}\subset \N.
      \]
      Seeking a contradiction, suppose that $\mathcal{S} \ne
      \emptyset$. By the \index{well-ordering principle}well-ordering
      principle, $\mathcal{S}$ has a least element, call it $\l\in
      \mathcal{S}$. By the definition of $\mathcal{S}$, $\l$ is not
      prime. Hence $\l = a\cdot b$, where $a,b\in \N$ with $a<\l$ and
      $b\le \l$ and both $a>1$ and $b>1$. Since neither $a$ nor $b$
      are in $\mathcal{S}$, both must be a product of primes, and so
      $\l$ must be a product of prime numbers.


      \textbf{Next we'll show that this factorization is unique.}
      Let 
      \[
      \mathcal{S} = \{n\in\N: \text{$n>1$ and $n$ has two prime
        factorizations}\}\subset \N.
       \]
      Seeking a contradiction, suppose that $\mathcal{S} \ne
      \emptyset$. By the \index{well-ordering principle}well-ordering
      principle, $\mathcal{S}$ has a least element, call it $\l\in
      \mathcal{S}$. Write
      \[
      \l = p_1 \cdot p_2 \cdot \cdots \cdot p_m = q_1 \cdot q_2 \cdot \cdots \cdot p_n
      \]
      where the $p_i$s and the $q_i$s are not necessarily distinct.
      By definition,
      \[
      p_1 |   q_1 \cdot q_2 \cdot \cdots \cdot p_n
      \]
      By Euclid's lemma, Corollary~\ref{C:EL2}, we see WLOG,
      relabeling as needed,
      \[
      p_1 | q_1.
      \]
      However, this can only happen if $p_1 = q_1$. Thus
      \[
      \l' = p_2 \cdot \cdots \cdot p_m = q_2 \cdot \cdots \cdot p_n
      \]
      is a number smaller than $\l$, and hence
      $\l'\notin\mathcal{S}$. This means that the factorization of
      $\l'$ is unique, meaning the factorization of $\l$ is a also
      unique, up to reordering. This is a contradiction, and we see
      $\mathcal{S}$ is empty. Thus every natural number greater than
      $1$ can be expressed as a unique (up to order) product of prime
      numbers.
  \end{proof}
\end{corollary}


\begin{exercise}
  Let each $p_i$ be a distinct prime number.  How many divisors does
  \[
  p_1^{\alpha_1} \cdot p_2^{\alpha_2} \cdot \cdots \cdot p_n^{\alpha_n}
  \]
  have? Prove your answer is correct. What number(s) between $0$ and
  $100$ have the most divisors?
\end{exercise}





\section{The Euclidean algorithm}

Now we introduce a fantastic algorithm, called the \textit{Euclidean
  algorithm}. We do this by example, so study the following
calculations:
\begin{align*}
  22 &= \boldsymbol{6}\cdot 3 + \boldsymbol{4}\\
  \boldsymbol{6} &= \boldsymbol{4} \cdot 1 + \fbox{$\boldsymbol{2}$}\\
  \boldsymbol{4} &= \boldsymbol{2} \cdot 2 + 0 \qquad
\therefore \gcd(22,6) = 2
\end{align*}

\begin{align*}
33 &= \boldsymbol{24}\cdot 1 + \boldsymbol{9}\\
\boldsymbol{24} &= \boldsymbol{9} \cdot 2 + \boldsymbol{6}\\
\boldsymbol{9} &= \boldsymbol{6} \cdot 1 + \fbox{$\boldsymbol{3}$}\\
\boldsymbol{6} &= \boldsymbol{3} \cdot 2 + 0 \qquad \therefore \gcd(33,24) = 3 
\end{align*}

\begin{align*}
42 &= \boldsymbol{16}\cdot 2 + \boldsymbol{10}\\
\boldsymbol{16} &= \boldsymbol{10} \cdot 1 + \boldsymbol{6}\\
\boldsymbol{10} &= \boldsymbol{6} \cdot 1 + \boldsymbol{4}\\
\boldsymbol{6} &= \boldsymbol{4} \cdot 1 + \fbox{$\boldsymbol{2}$}\\
\boldsymbol{4} &= \boldsymbol{2} \cdot 2 + 0 \qquad \therefore \gcd(42,16) = 2 
\end{align*}

Explain how the above algorithm works and write it under the Euclidean
algorithm below.


\begin{theorem}[Euclidean algorithm] 
We can easily compute the GCD of two numbers $m,n\in\Z$ using the
following algorithm:
\vspace{2in}

\end{theorem}
\noindent The above space has intentionally been left blank for you to
fill in.

\begin{exercise} 
We will prove that when working with integers, the Euclidean algorithm
will always produce the GCD of two numbers.
\begin{enumerate}
\item Prove that the remainders found in the Euclidean algorithm form a
  decreasing sequence.
\item Prove that this sequence must terminate with a final remainder
  of zero.
\item Proceed by induction on the number of steps in the Euclidean
  Algorithm. If there are two steps:
\begin{align*}
m &= n\cdot q_1 + g \\
n &= g\cdot q_2 +0
\end{align*}
Prove that any divisor of both $m$ and $n$ is necessarily a divisor
of $g$. Explain why this proves that $g = \gcd(m,n)$.
\item Now suppose that any time we have $i+1$ equations:
\begin{align*}
m &= n\cdot q_1 + r_1 \\
n &= r_1\cdot q_2 + r_2 \\
  &\hspace{.5em}\vdots\\
r_{i-2} &= r_{i-1}\cdot q_{i} + r_i\\
r_{i-1} &= r_{i}\cdot q_{i+1} + 0
\end{align*}
that $r_i = \gcd(m,n)$. Prove that when we have $i+2$ equations, $r_{i+1}
= \gcd(m,n)$.
\item Explain how we have proved that when working with integers, the
  Euclidean algorithm will always produce the GCD of two numbers.
\end{enumerate}
\end{exercise}


\begin{exercise} 
Prove that if $x = am +bn$ and $d$ is a common divisor of $m$ and $n$,
then $d|x$. Explain how this shows that every common divisor of $m$
and $n$ also divides $\gcd(m,n)$.
\end{exercise}







\section{Direct products of cyclic groups}


\begin{lemma}[Order of direct product groups]
  Let $G$ and $H$ be finite groups. In this case,
  \[
  |G\times H| = |G|\cdot |H|.
  \]
  \begin{sketch}
    Recall that
    \[
    G\times H := \{(g,h): \text{$g\in G$ and $h\in H$}\}.
    \]
  \end{sketch}
\end{lemma}

\begin{exercise}
  What is the order of $\Z_6\times\Z_7$. Is this the
  ``\link[ultimate question]{https://en.wikipedia.org/wiki/42\_(number)}?''
\end{exercise}



\begin{theorem}[Criterion for cyclic direct product]
  The group $\Z_p\times \Z_q$ is cyclic if and only if $\gcd(p,q) =
  1$.
  \begin{proof}
    $(\Rightarrow)$ We will prove this direction by way of
    contraposition. Suppose that $\gcd(p,q) = g>1$. We must show that
    $\Z_p\times \Z_q$ is not cyclic. Consider $(a,b)\in\Z_p\times\Z_q$
    and note that $\frac{p}{g}\in\N$ and $\frac{q}{g}\in\N$. Write
    with me:
    \begin{align*}
      (a,b) \cdot \frac{pq}{g} &= \left(ap\left(\frac{q}{g}\right),bq\left(\frac{p}{g}\right)\right)\\
      &\equiv (0,0).
    \end{align*}
    Hence $\o((a,b)) \le \frac{pq}{g} < |\Z_p\times\Z_q|$, and so
    $\Z_p\times\Z_q$ cannot be cyclic.

    $(\Leftarrow)$ Suppose that $\gcd(p,q) = 1$ and that
    \[
    (1,1)\cdot n = (n,n) \equiv (0,0).
    \]
    This means that $p|n$ and $q|n$, so
    \[
    n = pa = qb.
    \]
    Since $\gcd(p,q) = 1$, and $p|qb$, we see by Euclid's lemma,
    Corollary~\ref{C:EL2}, that $p|b$. The smallest $b$ with this
    property is $p$. A similar argument shows that $a = q$, hence
    $\o((1,1)) = pq$, and so $\Z_p\times\Z_q$ is cyclic.
  \end{proof}
\end{theorem}


\begin{exercise}
  Use a Cayley table to show that $\Z_2\times\Z_3 \iso \Z_6$.
\end{exercise}



Here's another famous group you might run into. The \textit{Klein
  four-group}.

\begin{definition}
  The \dfn{Klein four-group} is $V_4 = \Z_2\times\Z_2$ with
  componentwise modular addition being the operation.
\end{definition}

\begin{exercise}
  Make a Cayley table for $V_4$.
\end{exercise}

\begin{exercise}
  What are the generators of $V_4$?
\end{exercise}

\begin{exercise}
  Is the group $V_4$ isomorphic to the group $\Z_4$?
\end{exercise}



\section{Multiplicative groups}


The cyclic groups $\Z_n$ are groups under addition. However, we also
can multiply elements in $\Z_n$. We now show that modular
multiplication is well-defined.

\begin{lemma}[Modular multiplication is well-defined]\label{L:mmwd}\index{well-defined}
  Multiplication on $\Z_n$ is well-defined. This means that if
  \begin{align*}
    a &\equiv a' \pmod{n}\\
    b &\equiv b' \pmod{n},
  \end{align*}
  then
  \[
  a\cdot b \equiv a'\cdot b' \pmod{n}.
  \]
  \begin{sketch}
    Use the fact that
    \[
    x \equiv x'\pmod{n} \quad \Leftrightarrow \quad x -x' = n\cdot q
    \]
    for some number $q$.
  \end{sketch}
\end{lemma}

However, $\Z_n$ is \textbf{not a group under multiplication.} The
issue is that $0$ has no inverse. Moreover, any number $x|n$ will also
not have an inverse. We should state this as a lemma.

\begin{lemma}[Multiplicative inverses]\label{L:mi}
  An element $a\in\Z_n$ has a multiplicative inverse if and only if
  $\gcd(a,n) = 1$.
  \begin{sketch}
    Use Euclid's lemma, Theorem~\ref{T:EL1}.
  \end{sketch}
\end{lemma}

\begin{definition}
  The \dfn{multiplicative group} of $\Z_n$ is the set of elements that
  have multiplicative inverses modulo $n$. It is denoted
  $\Z_n^\times$.
\end{definition}


\begin{exercise}
  Let $p$ be a prime number. Prove that $|\Z_p^\times| = p-1$.
\end{exercise}


\begin{exercise}
  Make a Cayley table for $\Z_n^\times$ for $n = 3,4,5,6,7,8,9$.
\end{exercise}

\begin{exercise}
  Is $\Z_2$ isomorphic to $\Z_3^\times$?
\end{exercise}


\begin{exercise}
  Is $\Z_4$ isomorphic to $\Z_5^\times$?
\end{exercise}

\begin{exercise}
  Is $\Z_6$ isomorphic to $\Z_7^\times$?
\end{exercise}


\begin{exercise}
  Prove that $\Z_{12}^\times$ is not cyclic.
\end{exercise}

\begin{exercise}
  Use a Cayley table to show that $\Z^\times_{12}\iso V_4$.
\end{exercise}


\begin{definition}
  The \dfn{Euler totient function}, $\eulerphi:\N\to\N$, is defined as
  follows:
  \[
  \eulerphi(n) = |\Z_n^\times|.
  \]
\end{definition}


\begin{exercise}
  Compute $\eulerphi(n)$ for $n = 1,2,\dots, 10$.
\end{exercise}




\begin{exercise}
  Compute $\eulerphi(n)$ for $n = 2,3,6$. Now compute $\eulerphi(n)$
  for $n = 2,5,10$. Finally compute $\eulerphi(n)$ for $n=
  3,5,15$. \textbf{Make wild conjectures.}
\end{exercise}


\begin{exercise}
  Prove that $|\Z^\times_{12}| = 4$ and that $\Z^\times_{12}$ is not
  cyclic.
\end{exercise}








\subsection{Finding inverses with the Euclidean algorithm}



Now, suppose you have an element in $\Z_n^\times$. How do you find the
multiplicative inverse of this element? We use the Euclidean
algorithm.\index{Euclidean algorithm} Working by example, let's find
the multiplicative inverse of $16$ in $\Z_{21}^\times$. Study
the following calculations:
\begin{align*}
  21 &= 16\cdot 1 + 5 &\Leftrightarrow & &                21-16\cdot 1 &= \boldsymbol{5}\\ 
  16 &= 5\cdot 3 + 1 &\Leftrightarrow  & &   16 - \boldsymbol{5}\cdot 3 &= 1\\ 
  5 &= 1 \cdot 5 + 0 &  & & &
\end{align*}

Now, using back substitution, write:

\begin{align*}
16 - \boldsymbol{5}\cdot 3 &= 1 \\
16 - (21-16\cdot 1)\cdot 3 &= 1 \\
16(4) + 21(-3) &= 1
\end{align*}

This tells me that in $\Z_{21}^\times$,
\[
16\cdot 4 \equiv 1 \pmod{21}.
\]


Now let's do it again. This time we will find the multiplicative
inverse of $49$ in $\Z_{96}^\times$.
%% https://www.calculatorsoup.com/calculators/math/gcf-euclids-algorithm.php
\begin{align*}
  96 &= 49\cdot 1 + 47 &\Leftrightarrow  & &                96 - 49\cdot 1 &= \boldsymbol{47}\\ 
  49 &= 47\cdot 1 + 2  &\Leftrightarrow  & &   49 - \boldsymbol{47}\cdot 1 &= \boldsymbol{2}\\ 
  47 &= 2 \cdot 23 + 1 &\Leftrightarrow  & &   47 - \boldsymbol{2}\cdot 23 &= 1 \\
  2 &= 1\cdot  2+ 0 & & & & 
\end{align*}

Now, using back substitution, write:

\begin{align*}
47 - \boldsymbol{2}\cdot 23 &= 1 \\
\boldsymbol{47} - (49-\boldsymbol{47}\cdot 1)\cdot 23 &= 1 \\
96-49 - (49- (96-49))\cdot 23 &=1 \\
96(24) +49(-47) &=1
\end{align*}

This tells me that in $\Z_{96}^\times$,
\[
49\cdot (-47) \equiv 49\cdot 49 \equiv 1 \pmod{96}.
\]

\begin{exercise} %% 4 steps
  Quick! Find the inverse of $98$ in $\Z_{99}^\times$.
  \begin{hint}
    Note that $98 \equiv -1 \pmod{99}$.
  \end{hint}
\end{exercise}

\begin{exercise}
  Find the inverse of $13$ in $\Z_{56}^\times$.
\end{exercise}

\begin{exercise}
  Find the inverse of $21$ in $\Z_{73}^\times$.
\end{exercise}

\begin{exercise} %% 4 steps
  Find the inverse of $37$ in $\Z_{62}^\times$.
\end{exercise}

\begin{exercise} %% 4 steps
  Find the inverse of $41$ in $\Z_{88}^\times$.
\end{exercise}


\begin{exercise}
  An abstract algebra teacher intended to list the nine integers that
  form a group under multiplication modulo $91$. Instead, one of the
  nine integers was inadvertently left out, so that the list appeared
  as $1$, $9$, $16$, $22$, $53$, $74$, $79$, $81$. Which integer was
  left out? Explain how you reached your conclusion.
\end{exercise}



\section{A nod toward Lagrange}

We will end with statements that will follow from Lagrange's
celebrated theorem. Until then, we will make due with these more
specialized theorems that follow from the division theorem,
Theorem~\ref{T:DT}.


\begin{lemma}[Element's order is at most the group's order]
  Prove that if $G$ is a group and $a\in G$, then $\o(a)\le |G|$.
  \begin{sketch}
    Use the pigeon-hole principle.
  \end{sketch}
\end{lemma}


\begin{theorem}[Element's order divides cyclic group's order]\label{T:eodvgo}
  Let $G$ be a finite cyclic group with $a\in G$. In this case,
  $\o(a)$ divides $|G|$.
  \begin{sketch}
    Use the division theorem.
  \end{sketch}
\end{theorem}

\begin{corollary}[An element for each divisor]
  Let $G$ be a finite cyclic group of order $n$. Then for all $d|n$,
  there exists $a_d\in G$ such that $\o(a_d) = d$.
  \begin{sketch}
    Let $g\in G$ be a generator for $G$. Set $a_d = g^{n/d}$. Now
    explain why the order of $a_d$ must be $d$.
  \end{sketch}
\end{corollary}

We can continue on and prove a little more.



\begin{lemma}[Order divides other exponents]\label{L:odoe}
  Let $G$ be a group with $a\in G$. If $a^n = 1$, then $o(a)|n$.
  \begin{sketch}
    Use the division theorem, Theorem~\ref{T:DT}.
  \end{sketch}
\end{lemma}


\begin{lemma}[Coprime element order]\label{L:cpeo}
  Let $G$ be a finite Abelian group. If $\gcd(\o(a),\o(b)) = 1$, then
  $\o(ab) = \o(a)\cdot \o(b)$.
  \begin{proof}
    If $(ab)^n = 1$ then $(a)^{n\o(b)} = 1$. By Lemma~\ref{L:odoe},
    \[
    \o(a) | n\o(b).
    \]
    By Euclid's lemma, Corollary~\ref{C:EL2}, we see $\o(a) | n$. The
    same argument with $a$ and $b$ switched will show $\o(b)|n$. Thus
    \[
    (\o(a)\o(b))|n.
    \]
    Since $(ab)^(\o(a)\o(b)) = 1$, we are done.
  \end{proof}
\end{lemma}



\begin{corollary}[Order divides maximal order]\label{C:odmo}
  Let $G$ be a finite Abelian group and let $m\in G$ such that $\o(m)
  \ge \o(a)$ for all $a\in G$. In this case,
  \[
  \o(a) | \o(m).
  \]
  \begin{proof}
    Seeking a contradiction, suppose that $\o(a) \nmid \o(m)$. By
    unique factorization, Corollary~\ref{C:UFNN}, this means that
    there is a prime $p$ where a higher power of $p$ divides $\o(a)$
    than $\o(m)$,
    \[
    \o(m) = p^\alpha q_1 \quad\text{and}\quad \o(a) = p^\beta q_2  
    \]
    where $\beta>\alpha$ and $p\nmid q_1$ and $p\nmid q_2$. In this case,
    \[
    \o(m^{p^\alpha}) = q_1\quad\text{and}\quad \o(a^{q_2}) = p^\beta
    \]
    and $\gcd(q_1,p^\beta) = 1$. Thus by Lemma~\ref{L:cpeo},
    \[
    \o(m^{p^\alpha}a^{q_2}) = p^\beta\cdot q_1> p^\alpha q_1 = \o(m).
    \]
    This is a contradiction, since $m$ has maximal order by
    assumption.
  \end{proof}
\end{corollary}






%% \begin{corollary}[Fermat's little theorem]\index{Fermat's little theorem}
%%   Let $p\in \N$ be a prime number. Given $a\in\Z_p$,
%%   \[
%%   a^{p-1}\equiv 1 \pmod{p}.
%%   \]
%% \end{corollary}


%% \begin{corollary}[Euler's little theorem]\index{Euler's little theorem}
%%   Let $n\in \N$. Given $a\in\Z_n$,
%%   \[
%%   a^{\eulerphi(n)}\equiv 1 \pmod{n}.
%%   \]
%% \end{corollary}



\section{Solving equations}

Our final section serves to explicitly remind us that: \textbf{Algebra
  is about solving equations}. The equation we continually solved was
the following,
\[
ax + by = c
\]
where $a,b,c\in\N$ and $x,y\in\Z$. The key point here is that the
solutions, $x$ and $y$, must be \textbf{integers}.

\begin{definition}
  A \textbf{linear} \dfn{Diophantine equation} is an equation of the
  form
  \[
  ax + by = c
  \]
  where $a,b,c,x,y\in\Z$.
\end{definition}

Diophantine equations are named after the Greek mathematician who
studied them, \index{Diophantus of Alexandria}\link[Diophantus of Alexandria]{https://en.wikipedia.org/wiki/Diophantus}.

\begin{exercise}
  Prove that the Diophantine equation $ax + by = c$ has solutions if
  and only if $\gcd(a,b)|c$.
\end{exercise}

\begin{exercise}
  Find integers $x$ and $y$ satisfying the following Diophantine
  equations:
  \begin{enumerate}
  \item $671x + 715 y = 11$ 
  \item $667x + 713 y = 69$ 
  \item $671x + 713 y = 1$
  \item $682x + 715 y = 55$
  \item $601x + 735 y = 4$
  \item $701x + 835 y = 15$
  \end{enumerate}
\end{exercise}

\begin{exercise}
  Consider the Diophantine equation
  \[
  15x + 4y = 1.
  \]
  \begin{enumerate}
  \item Use the Euclidean algorithm to find a solution to this
    equation. Explain your reasoning.
  \item Plot the line $15x + 4y = 1$, adding points at the solutions
    to the equation.
  \item Explain why a Diophantine equation 
    \[
    ax + by = c
    \]
  has either an infinite number of solutions or zero solutions.
\item Using your plot and the slope of the line, parameterize the
  solutions, meaning, give a formula in terms of $t$, that outputs all
  of the solutions as $t$ runs through the integers.
  \end{enumerate}
\end{exercise}


So we've seen that this algebra is about solving equations. On the
other hand, \textbf{algebra is geometry}. We've seen this a little bit
with the dihedral groups.


\begin{question}
  So what is another connection to geometry?
\end{question}


For some interesting extra reading check out:
\begin{itemize}
\item \link[\textit{Automotive repair by number theory},
  B.\ Snapp and C.\ Snapp, Math Horizons. September (2009), 26--27.]{https://people.math.osu.edu/snapp.14/SnappFiat.pdf}.
\end{itemize}







\end{document}
