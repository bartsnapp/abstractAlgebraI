\documentclass{ximera}

\usepackage[T1]{fontenc}
\usepackage{stix2}
\usepackage{gillius}
\usepackage{resizegather}
%\usepackage{rsfso} fancy cal
\DeclareMathAlphabet{\mathcal}{OMS}{cmsy}{m}{n} %less fancy cal


\usepackage{multicol}


\usepackage{tikz-cd}
\usepackage{tkz-euclide} %% compass
\usetkzobj{all}  %% tkzCompass
\tikzset{>=stealth}
\tikzcdset{arrow style=tikz}
\usetikzlibrary{math} %% for assigning variables
%\usetikzlibrary{fadings}

\usepackage{colortbl,boldline,makecell} %% group tables


\usepackage[sans]{dsfont}

\usepackage{stmaryrd,pifont}

\graphicspath{
  {./}
  {fields/}
  }     



\let\oldbibliography\thebibliography%% to compact bib
\renewcommand{\thebibliography}[1]{%
  \oldbibliography{#1}%
  \setlength{\itemsep}{0pt}%
}
\renewcommand\refname{} %% no name needed!


\DefineVerbatimEnvironment{macaulay2}{Verbatim}{numbers=left,frame=lines,label=Macaulay2,labelposition=topline}

\DefineVerbatimEnvironment{gap}{Verbatim}{numbers=left,frame=lines,label=GAP,labelposition=topline}

%%% This next bit of code defines all our theorem environments
\makeatletter
\let\c@theorem\relax
\let\c@corollary\relax
%\let\c@example\relax
\makeatother

\let\definition\relax
\let\enddefinition\relax

\let\theorem\relax
\let\endtheorem\relax

\let\proposition\relax
\let\endproposition\relax

\let\exercise\relax
\let\endexercise\relax

\let\question\relax
\let\endquestion\relax

\let\remark\relax
\let\endremark\relax

\let\corollary\relax
\let\endcorollary\relax


\let\example\relax
\let\endexample\relax

\let\warning\relax
\let\endwarning\relax

\let\lemma\relax
\let\endlemma\relax


\let\algorithm\relax
\let\endalgorithm\relax
\usepackage{algpseudocode}

\newtheoremstyle{SlantTheorem}{\topsep}{\topsep}%%% space between body and thm
		{\slshape}                      %%% Thm body font
		{}                              %%% Indent amount (empty = no indent)
		{\bfseries\sffamily}            %%% Thm head font
		{}                              %%% Punctuation after thm head
		{3ex}                           %%% Space after thm head
		{\thmname{#1}\thmnumber{ #2}\thmnote{ \bfseries(#3)}}%%% Thm head spec
\theoremstyle{SlantTheorem}
\newtheorem{theorem}{Theorem}
%\newtheorem{definition}[theorem]{Definition}
%\newtheorem{proposition}[theorem]{Proposition}
%% \newtheorem*{dfnn}{Definition}
%% \newtheorem{ques}{Question}[theorem]
%% \newtheorem*{war}{WARNING}
%% \newtheorem*{cor}{Corollary}
%% \newtheorem*{eg}{Example}
\newtheorem*{remark}{Remark}
\newtheorem*{touchstone}{Touchstone}
\newtheorem{corollary}{Corollary}[theorem]
\newtheorem*{warning}{WARNING}
\newtheorem{example}[corollary]{Example}
\newtheorem{lemma}[theorem]{Lemma}




\newtheoremstyle{Definition}{\topsep}{\topsep}%%% space between body and thm
		{}                              %%% Thm body font
		{}                              %%% Indent amount (empty = no indent)
		{\bfseries\sffamily}            %%% Thm head font
		{}                              %%% Punctuation after thm head
		{3ex}                           %%% Space after thm head
		{\thmname{#1}\thmnumber{ #2}\thmnote{ \bfseries(#3)}}%%% Thm head spec
\theoremstyle{Definition}
\newtheorem{definition}[theorem]{Definition}



\let\algorithm\relax
\let\endalgorithm\relax
\newtheoremstyle{Alg}{\topsep}{\topsep}%%% space between body and thm
		{}                              %%% Thm body font
		{}                              %%% Indent amount (empty = no indent)
		{\bfseries\sffamily}            %%% Thm head font
		{}                              %%% Punctuation after thm head
		{3ex}                           %%% Space after thm head
		{\thmname{#1}\thmnumber{ #2}\thmnote{ \bfseries(#3)}}%%% Thm head spec
\theoremstyle{Alg}
\newtheorem*{algorithm}{Algorithm}
\newtheorem*{construction}{Construction}




\newtheoremstyle{Exercise}{\topsep}{\topsep} %%% space between body and thm
		{}                           %%% Thm body font
		{}                           %%% Indent amount (empty = no indent)
		{\bfseries\sffamily}         %%% Thm head font
		{)}                          %%% Punctuation after thm head
		{ }                          %%% Space after thm head
		{\thmnumber{#2}\thmnote{ \bfseries(#3)}}%%% Thm head spec
\theoremstyle{Exercise}
\newtheorem{exercise}[corollary]{}%[theorem]

%% \newtheoremstyle{Question}{\topsep}{\topsep} %%% space between body and thm
%% 		{\bfseries}                  %%% Thm body font
%% 		{3ex}                        %%% Indent amount (empty = no indent)
%% 		{}                           %%% Thm head font
%% 		{}                           %%% Punctuation after thm head
%% 		{}                           %%% Space after thm head
%% 		{\thmnumber{#2}\thmnote{ \bfseries(#3)}}%%% Thm head spec
\newtheoremstyle{Question}{3em}{3em} %%% space between body and thm
		{\large\bfseries}                           %%% Thm body font
		{}                           %%% Indent amount (empty = no indent)
		{}                         %%% Thm head font
		{}                          %%% Punctuation after thm head
		{0em}                          %%% Space after thm head
		{}%%% Thm head spec
\theoremstyle{Question}
\newtheorem*{question}{}






\renewcommand{\tilde}{\widetilde}
\renewcommand{\bar}{\overline}
\renewcommand{\hat}{\widehat}
\newcommand{\N}{\mathbb N}
\newcommand{\Z}{\mathbb Z}
\newcommand{\R}{\mathbb R}
\newcommand{\Q}{\mathbb Q}
\newcommand{\C}{\mathbb C}
\newcommand{\V}{\mathbb V}
\newcommand{\I}{\mathbb I}
\newcommand{\A}{\mathbb A}
\renewcommand{\o}{\mathbf o}
\newcommand{\iso}{\simeq}
\newcommand{\ph}{\varphi}
\newcommand{\Cf}{\mathcal{C}}
\newcommand{\IZ}{\mathrm{Int}(\Z)}
\newcommand{\dsum}{\oplus}
\newcommand{\directsum}{\bigoplus}
\newcommand{\union}{\bigcup}
\newcommand{\subgp}{\leq}
\newcommand{\normal}{\trianglelefteq}
\renewcommand{\i}{\mathfrak}
\renewcommand{\a}{\mathfrak{a}}
\renewcommand{\b}{\mathfrak{b}}
\newcommand{\m}{\mathfrak{m}}
\newcommand{\p}{\mathfrak{p}}
\newcommand{\q}{\mathfrak{q}}
\newcommand{\dfn}[1]{\textbf{#1}\index{#1}}
\let\hom\relax
\DeclareMathOperator{\mat}{Mat}
\DeclareMathOperator{\ann}{Ann}
\DeclareMathOperator{\h}{ht}
\DeclareMathOperator{\tr}{tr}
\DeclareMathOperator{\hom}{Hom}
\DeclareMathOperator{\Span}{Span}
\DeclareMathOperator{\spec}{Spec}
\DeclareMathOperator{\maxspec}{MaxSpec}
\DeclareMathOperator{\aut}{Aut}
\DeclareMathOperator{\ass}{Ass}
\DeclareMathOperator{\lcm}{lcm}
\DeclareMathOperator{\ff}{Frac}
\DeclareMathOperator{\im}{Im}
\DeclareMathOperator{\syz}{Syz}
\DeclareMathOperator{\gr}{Gr}
\DeclareMathOperator{\multideg}{multideg}
\renewcommand{\ker}{\mathop{\mathrm{Ker}}\nolimits}
\newcommand{\coker}{\mathop{\mathrm{Coker}}\nolimits}
\newcommand{\lps}{[\hspace{-0.25ex}[}
\newcommand{\rps}{]\hspace{-0.25ex}]}
\newcommand{\into}{\hookrightarrow}
\newcommand{\onto}{\twoheadrightarrow}
\newcommand{\tensor}{\otimes}
\newcommand{\x}{\mathbf{x}}
\newcommand{\X}{\mathbf X}
\newcommand{\Y}{\mathbf Y}
\renewcommand{\k}{\boldsymbol{\kappa}}
\renewcommand{\emptyset}{\varnothing}
\renewcommand{\qedsymbol}{$\blacksquare$}
\renewcommand{\l}{\ell}
\newcommand{\1}{\mathds{1}}
\newcommand{\lto}{\mathop{\longrightarrow\,}\limits}
\newcommand{\rad}{\sqrt}
\newcommand{\hf}{H}
\newcommand{\hs}{H\!S}
\newcommand{\hp}{H\!P}
\renewcommand{\vec}{\mathbf}
\let\temp\phi
\let\phi\varphi
\let\eulerphi\temp


\renewcommand{\epsilon}{\varepsilon}
\renewcommand{\subset}{\subseteq}
\renewcommand{\supset}{\supseteq}
\newcommand{\macaulay}{\normalfont\textsl{Macaulay2}}
\newcommand{\GAP}{\normalfont\textsf{GAP}}
\newcommand{\invlim}{\varprojlim}
\renewcommand{\le}{\leqslant}
\renewcommand{\ge}{\geqslant}
\newcommand{\valpha}{{\boldsymbol\alpha}}
\newcommand{\vbeta}{{\boldsymbol\beta}}
\newcommand{\vgamma}{{\boldsymbol\gamma}}
\newcommand{\dotp}{\bullet}
\newcommand{\lc}{\normalfont\textsc{lc}}
\newcommand{\lt}{\normalfont\textsc{lt}}
\newcommand{\lm}{\normalfont\textsc{lm}}
\newcommand{\from}{\leftarrow}
\newcommand{\transpose}{\intercal}
\newcommand{\grad}{\boldsymbol\nabla}
\newcommand{\curl}{\boldsymbol{\nabla\times}}
\renewcommand{\d}{\, d}
\newcommand{\<}{\langle}
\renewcommand{\>}{\rangle}

%\renewcommand{\proofname}{Sketch of Proof}


\renewenvironment{proof}[1][Proof]
  {\begin{trivlist}\item[\hskip \labelsep \itshape \bfseries #1{}\hspace{2ex}]\upshape}
{\qed\end{trivlist}}

\newenvironment{sketch}[1][Sketch of Proof]
  {\begin{trivlist}\item[\hskip \labelsep \itshape \bfseries #1{}\hspace{2ex}]\upshape}
{\qed\end{trivlist}}



\makeatletter
\renewcommand\section{\@startsection{paragraph}{10}{\z@}%
                                     {-3.25ex\@plus -1ex \@minus -.2ex}%
                                     {1.5ex \@plus .2ex}%
                                     {\normalfont\large\sffamily\bfseries}}
\renewcommand\subsection{\@startsection{subparagraph}{10}{\z@}%
                                    {3.25ex \@plus1ex \@minus.2ex}%
                                    {-1em}%
                                    {\normalfont\normalsize\sffamily\bfseries}}
\makeatother

%% Fix weird index/bib issue.
\makeatletter
\gdef\ttl@savemark{\sectionmark{}}
\makeatother


\makeatletter
%% no number for refs
\newcommand\frontstyle{%
  \def\activitystyle{activity-chapter}
  \def\maketitle{%
    \addtocounter{titlenumber}{1}%
                    {\flushleft\small\sffamily\bfseries\@pretitle\par\vspace{-1.5em}}%
                    {\flushleft\LARGE\sffamily\bfseries\@title \par }%
                    {\vskip .6em\noindent\textit\theabstract\setcounter{problem}{0}\setcounter{sectiontitlenumber}{0}}%
                    \par\vspace{2em}
                    \phantomsection\addcontentsline{toc}{section}{\textbf{\@title}}%
                  }}
\makeatother



\NewEnviron{annotate}{\vspace{-.3cm}\small \itshape \BODY \vspace{.3cm}}


%%%% TIKZ STUFF

%% N-GON code
\tikzset{
    pics/tikzngon/.style={
        code={
        \tikzmath{\xx = #1;\rr=1.7;}
        \draw[ultra thick,rounded corners=.05mm] ({\rr*sin(0*360/\xx)},{\rr*cos(0*360/\xx)})
        \foreach \x in {-1,0,...,\xx}
        {
        -- ({\rr*sin(\x*360/\xx)},{\rr*cos(\x*360/\xx)})
        }
           -- cycle;
  }}}

%% N-GON code (even)
\tikzset{
    pics/tikzegon/.style={
        code={
        \tikzmath{\xx = #1;\rr=1.7;}
        \draw[ultra thick,rounded corners=.05mm] ({\rr*sin(0*360/\xx+180/\xx)},{\rr*cos(0*360/\xx+180/\xx)})
        \foreach \x in {-1,0,...,\xx}
           {
           -- ({\rr*sin(\x*360/\xx+180/\xx)},{\rr*cos(\x*360/\xx+180/\xx)}) 
           }
           -- cycle;
  }}}




%% N-CLOCK code
\tikzset{
    pics/tikznclock/.style={
        code={
        \tikzmath{\xx = #1;\rr=1.7;\dd=.4;}
        \foreach \x in {1,...,\xx}
        \pgfmathtruncatemacro{\xy}{\x-1}
           {
             \node[circle,fill=black,inner sep=0pt, minimum size=13pt,text=white]
             at ({(\rr-\dd)*sin((\x-1)*360/(\xx)},{(\rr-\dd)*cos((\x-1)*360/\xx}) {\normalfont\bfseries\sffamily\small {\xy}};
           }
  \draw[thick] (0,0) circle (\rr);
  }}}



%% barcode from
%% https://tex.stackexchange.com/questions/6895/is-there-a-good-latex-package-for-generating-barcodes
%% NOT CURRENTLY USED!


\def\barcode#1#2#3#4#5#6#7{\begingroup%
  \dimen0=0.1em
  \def\stack##1##2{\oalign{##1\cr\hidewidth##2\hidewidth}}%
  \def\0##1{\kern##1\dimen0}%
  \def\1##1{\vrule height10ex width##1\dimen0}%
  \def\L##1{\ifcase##1\bc3211##1\or\bc2221##1\or\bc2122##1\or\bc1411##1%
    \or\bc1132##1\or\bc1231##1\or\bc1114##1\or\bc1312##1\or\bc1213##1%
    \or\bc3112##1\fi}%
  \def\R##1{\bgroup\let\next\1\let\1\0\let\0\next\L##1\egroup}%
  \def\G##1{\bgroup\let\bc\bcg\L##1\egroup}% reverse
  \def\bc##1##2##3##4##5{\stack{\0##1\1##2\0##3\1##4}##5}%
  \def\bcg##1##2##3##4##5{\stack{\0##4\1##3\0##2\1##1}##5}%
  \def\bcR##1##2##3##4##5##6{\R##1\R##2\R##3\R##4\R##5\R##6\11\01\11\09%
    \endgroup}%
  \stack{\09}#1\11\01\11\L#2%
  \ifcase#1\L#3\L#4\L#5\L#6\L#7\or\L#3\G#4\L#5\G#6\G#7%
    \or\L#3\G#4\G#5\L#6\G#7\or\L#3\G#4\G#5\G#6\L#7%
    \or\G#3\L#4\L#5\G#6\G#7\or\G#3\G#4\L#5\L#6\G#7%
    \or\G#3\G#4\G#5\L#6\L#7\or\G#3\L#4\G#5\L#6\G#7%
    \or\G#3\L#4\G#5\G#6\L#7\or\G#3\G#4\L#5\G#6\L#7%
  \fi\01\11\01\11\01\bcR}


\author{Bart Snapp}

\title{Groups}

\begin{document}
\begin{abstract}
  We introduce the notion of a group.
\end{abstract}
\maketitle

Let us not mince words:

\begin{definition}
  A \dfn{group} is a set $G$ with an operation $\star$ such that the
  following properties hold:
  \begin{description}
  \item[Closure] If $a,b\in G$, then $a\star b\in G$.
  \item[Associativity] If $a,b,c\in G$, then
    \[
    a\star(b\star c)  = (a\star b)\star c.
    \]
  \item[Identity] There exists $e\in G$ such that for all $a\in G$, 
    \[
    e\star a = a \star e  = a.
    \]
  \item[Inverse] For all $a\in G$, there exists $a^{-1}$ such that
    \[
    a\star a^{-1} = a^{-1}\star a = e.
    \]
  \end{description}
  We will often write a group as $G$, or $(G,\star)$ when the
  operation is not clear.
\end{definition}

\begin{exercise} %% suppose to test closure
  Which of the following are groups? Select all that apply.
  \begin{selectAll}
    \choice[correct]{$(\Z,+)$ you may assume addition is associative.}
    \choice{$(\{1,2,\dots,n\},+)$.}
    \choice{$(\N,\cdot)$ where $\N = \{1,2,3,\dots\}$, you may assume multiplication is associative.}
    \choice[correct]{$(\Z\times\Z,+)$ where
      \[
      \Z\times \Z := \{(a,b): a,b\in \Z\}
      \]
      and
      \[
      (a_1,b_1) + (a_2,b_2) := (a_1+a_2,b_1+b_2).
      \]}
    \choice[correct]{$(\Q-\{0\},\cdot)$}
  \end{selectAll}
\end{exercise}

\begin{exercise} %% supposed to test associativity
  Which of the following are groups? Select all that apply.
  \begin{selectAll}
    \choice{$(\{x^n:n\in \N\},\div)$}
    \choice{$(\{x^n:n\in \Z\},\cdot)$}
    \choice[correct]{$(\{x^n:n\in \Z\},\circ)$}
    \choice[correct]{$(\{x^n:n\in\Q\},\cdot)$}
    \choice{$(\{x^n:n\in\Q\},\circ)$}
  \end{selectAll}
\end{exercise}


\begin{exercise} %% harder constructions
  Which of the following are groups? Select all that apply.
  \begin{selectAll}
    \choice[correct]{$(2\Z,+)$ where $2\Z$ is the set of even integers.}
    \choice{$(\Z_{\mathrm{odd}},\cdot)$ where $\Z_{\mathrm{odd}}$ is
      the set of odd integers.}
    \choice[correct]{$(I(\x),+)$ where $\x$ is a set of points in $\R^2$ and
        \[
        I(\x) = \{F:\text{where $F:\R^2\to\R$ and $F(\x) = 0$}\}.
        \]}
    \choice{$(\Q[x],\cdot)$ where $\Q[x]$ is the set of polynomials in
      the variable $x$ with coefficients in $\Q$.}
    \choice[correct]{$(C^\infty(\R,\R), +)$ where $C^\infty(\R,\R)$ is
      the set of continuous functions mapping $\R\to\R$ such that
      every derivative of the function is also continuous.}
  \end{selectAll}
\end{exercise}

\begin{exercise}
  Let $(G,\star)$ and $(H,\diamond)$ be groups. Prove that
  \[
  G\times H := \{(g,h): \text{$g\in G$ and $h\in H$}\}
  \]
  is a group under
  \[
  (g_1,h_1) \circledast (g_2,h_2) := (g_1\star g_2, h_1\diamond h_2).
  \]
\end{exercise}

\begin{exercise}
  Consider the set $\Z\times \Z$ under the operation
  \[
  (a,b) \star(c,d) := (ad+bc,bd).
  \]
  Is this a group? If so, give a proof. If not, explain why not.
\end{exercise}


\begin{exercise}
  Find the largest subset $S\subset \Q$ such that $(S,\star)$ is a
  group where
  \[
  x\star y = x\cdot y + x + y
  \]
  and prove your answer is correct.
\end{exercise}




\begin{lemma}[Identity is unique]
  Let $G$ be a group. There is a unique identity element $e\in G$.
  \begin{sketch}
    Suppose there are two identities.
  \end{sketch}
\end{lemma}


\begin{lemma}[Inverses are unique]
  Let $G$ be a group. Every element in $G$ has a unique inverse.
  \begin{sketch}
    Suppose that an element has two inverses.
  \end{sketch}
\end{lemma}





\section{Finite groups}


So far, every group we've seen has had an infinite number of
elements. Groups can be finite too.

\begin{definition}
  If $G$ is a group, the \dfn{order} of $G$, denoted $|G|$ is equal to
  the number of distinct elements in $G$.
\end{definition}


\begin{example}\label{E:C}
  Let $C = \{1,i,-1,-i\}$ where $i =\sqrt{-1}$. We claim that $C$ is a
  group under multiplication.
  \begin{proof}
    First note that $C$ is closed under multiplication. Now also note
    that we know that multiplication is a associative. The identity
    element is $1$, and each element has an inverse, behold:
    \[
    i\cdot (-i) = 1, \quad\text{and}\quad (-1)\cdot(-1) = 1.
    \]
    So $C$ is a group of order $4$.
  \end{proof}
\end{example}


\begin{definition}
  Given a finite group $(G,\star)$, a \dfn{group table}, or
  \dfn{Cayley table}, is a complete $\star$-``multiplication'' table
  for $G$.
\end{definition}

\begin{example}
  Let's consider the group $C =\{1,i,-1,-i\}$ under multiplication once more.
  Write with me  
  \[
  \renewcommand{\arraystretch}{1.6}
  \begin{array}{c!{\vline width 2pt}cccccc}
    (C,\cdot)& 1     & i\cellcolor{blue!36!white}     & -1\cellcolor{blue!24!white}    & -i\cellcolor{blue!12!white}  \\  \Xhline{2pt}
    1        & 1     & i\cellcolor{blue!36!white}     & -1\cellcolor{blue!24!white}    & -i\cellcolor{blue!12!white}  \\  
    i\cellcolor{blue!36!white}        & i\cellcolor{blue!36!white}     & -1\cellcolor{blue!24!white}    & -i\cellcolor{blue!12!white}    & 1   \\  
    -1\cellcolor{blue!24!white}       & -1\cellcolor{blue!24!white}    & -i\cellcolor{blue!12!white}    & 1     & i\cellcolor{blue!36!white}   \\  
    -i\cellcolor{blue!12!white}       & -i\cellcolor{blue!12!white}    & 1     & i\cellcolor{blue!36!white}     & -1\cellcolor{blue!24!white}  
  \end{array}
  \]
  We colored in the cells of the table for dramatic effect. You don't
  need to do that, but it might help to see symmetry.  Sometimes, it
  is best to write a Cayley table using fewer symbols.  Noting that
  \[
  i^2 = -1 \quad\text{and}\quad i^3 = -i,
  \]
  we can also write:
  \[
  \renewcommand{\arraystretch}{1.6}
  \begin{array}{c!{\vline width 2pt}cccccc}
    (C,\cdot)& 1     & i\cellcolor{blue!36!white}     & i^2\cellcolor{blue!24!white}    & i^3\cellcolor{blue!12!white}  \\  \Xhline{2pt}
    1        & 1     & i\cellcolor{blue!36!white}     & i^2\cellcolor{blue!24!white}    & i^3\cellcolor{blue!12!white}  \\  
    i\cellcolor{blue!36!white}        & i\cellcolor{blue!36!white}     & i^2\cellcolor{blue!24!white}    & i^3\cellcolor{blue!12!white}    & 1   \\  
    i^2\cellcolor{blue!24!white}       & i^2\cellcolor{blue!24!white}    & i^3\cellcolor{blue!12!white}    & 1     & i\cellcolor{blue!36!white}   \\  
    i^3\cellcolor{blue!12!white}       & i^3\cellcolor{blue!12!white}    & 1     & i\cellcolor{blue!36!white}     & i^2\cellcolor{blue!24!white}  
  \end{array}
  \]
  You must write the entries of the Cayley table using elements from
  the top-most row; those same elements should also appear in the
  left-most column.
\end{example}



%% \begin{exercise}
%%   The following operation table is \textbf{not} a Cayley table for a
%%   group:
%%   \[
%%   \renewcommand{\arraystretch}{1.6}
%%   \begin{array}{|c||c|c|c|c|c|c|}\hline
%%      * & a & b & c & d & e & f  \\  \hline\hline
%%     a  & a & b & c & d & e & f  \\  \hline
%%     b  & b & c & d & e & f & a  \\  \hline
%%     c  & c & d & a & f & b & e  \\  \hline
%%     d  & d & e & f & a & c & b  \\  \hline
%%     e  & e & f & b & c & a & d  \\  \hline
%%     f  & f & a & e & b & d & c  \\  \hline
%%     \end{array}
%%   \]
%%   Which properties of a group fail?
%% \end{exercise}

\section{Groups of functions}

Groups are absolutely fundamental to all of modern mathematics; their
importance cannot be overstated. I remember reading a similar line to
the one that precedes this one as an undergraduate. I didn't believe
it. However, make no mistake. Group theory permeates all of modern
mathematics. Let me explain.

One of the most powerful insights of 20th-Century mathematics is that
to understand an object in mathematics, you should understand the
functions on the object.

At this point, you may be wondering, ``First I'm told groups are
important, then I'm told that functions are important, well\dots which
is it?''

And the answer is, ``both.'' This is because if you have a set $X$
then the set of all bijections from $X$ to $X$ is a group under
composition, see Exercise~\ref{E:aut0}. Now we will prove a
fundamental lemma.



\begin{lemma}[Function composition is associative]\label{L:funCompAss}
  Suppose that $A$, $B$, $C$, and $D$ are sets and that
  \begin{align*}
    \alpha:A &\to B\\
    \beta:B &\to C\\
    \gamma:C &\to D
  \end{align*}
  are functions. Then,
  \[
  \gamma\circ(\beta\circ\alpha) = (\gamma\circ\beta)\circ\alpha.
  \]
  \begin{proof}
    Let's put our maps in a row and represent the functions as arrows
    between the sets:
    \[
    \begin{tikzcd}[row sep=-1ex]%, column sep=-5ex]
      A \ar[r,"\alpha"] & B \ar[r,"\beta"] & C \ar[r,"\gamma"] & D\\
      a \ar[mapsto,r] & \alpha(a) = b  \ar[mapsto,r] & \beta(b) = c  \ar[mapsto,r] & \gamma(c)=d 
    \end{tikzcd}
    \]
    The second row above shows where the elements of these sets map
    to. The funny arrow with a base, ``$\mapsto$,'' is how we show
    where elements map. The diagram above represents
    $\gamma\circ\beta\circ\alpha$. In particular, with our diagram, we have set
    \[
    \gamma\circ\beta\circ\alpha(a) = d.
    \]
    Now let's make a similar diagram to
    represent $\gamma\circ(\beta\circ\alpha)$:
    \[
    \begin{tikzcd}[row sep=-1ex]%, column sep=-5ex]
      A \ar[r,"\alpha"] & B \ar[r,"\beta"] & C \ar[r,"\gamma"] & D\\
      a \ar[mapsto,rr,"\beta\circ\alpha",swap] &  & c  \ar[mapsto,r] & d 
    \end{tikzcd}
    \]
    We see that
    \[
    \gamma\circ(\beta\circ\alpha)(a) = d.
    \]
    Finally, let's make a similar diagram to
    represent $(\gamma\circ\beta)\circ\alpha$:
    \[
    \begin{tikzcd}[row sep=-1ex]%, column sep=-5ex]
      A \ar[r,"\alpha"] & B \ar[r,"\beta"] & C \ar[r,"\gamma"] & D\\
      a \ar[mapsto,r] & b \ar[rr,mapsto,"\gamma\circ\beta",swap] &   & d 
    \end{tikzcd}
    \]
    Again, we see
    \[
    (\gamma\circ\beta)\circ\alpha(a) = d.
    \]
    Since $a$ is an arbitrary label, and $d$ is simply a label for
    $\gamma\circ\beta\circ\alpha(a)$, we have shown
    \[
    \gamma\circ(\beta\circ\alpha)(a) = \gamma\circ\beta\circ\alpha(a) = (\gamma\circ\beta)\circ\alpha(a).
    \]
    We conclude that functional composition is always associative.
  \end{proof}
\end{lemma}


\begin{exercise}
  Prove that addition is an associative operation  over $\Z$.
\end{exercise}

\begin{exercise}
  Prove that multiplication is an associative operation  over $\Z$.
\end{exercise}



\begin{exercise}
  We write functions on the left, $f(x)$, meaning that the ``$f$'' is
  left of the ``$(x)$.''  Some authors write functions on the right,
  meaning they would write ``$(x)f$.'' Why would an author do this? Hint:
  Rephrase the lemma above, writing the functions on the right, and
  see if you notice anything nice.
  \begin{hint}
    Think about this:
    \[
    \underbrace{\gamma\circ\beta\circ\alpha}_{\text{left}}(a) = (a)\underbrace{\alpha\circ\beta\circ\gamma}_{\text{right}}
    \]
  \end{hint}
\end{exercise}



\begin{exercise}
  Prove that matrix multiplication is associative.
  \begin{hint}
    An $n\times m$ matrix is a function mapping $\R^m\to \R^n$ when
    you write your vectors vertically and functions on the left.
  \end{hint}
\end{exercise}




\begin{exercise}
  Pick up a sock. Now choose one:
  \begin{enumerate}
  \item[$(n)$] Do nothing to the sock.
  \item[$(i)$] Pull the toe of the sock out the inside of the foot
    opening. Hence if the sock was rightside out, it is now inside
    out, and vice versa.
  \end{enumerate}
  Someone says,
  \begin{quote}
    Ah! The sock group! The set $\{n,i\}$ is a group of order $2$ with
    \begin{align*}
      r\circ r &= r\\
  r\circ i &= i\\
  i\circ i &= r.
  \end{align*}
  \end{quote}
  Explain what this ``groupie'' is taking about. Be very
  detailed. Make a Cayley table with entries $n$ and $r$. 
\end{exercise}

\begin{exercise}
  In the Ohio State University Marching Band, there is a fundamental
  movement called a ``right-face.'' With this turn, the band-member
  spins to the right, $90^\circ$, ending up facing right. Denote a
  right-face by $R$. Explain how
  \[
  \{R^0,R^1,R^2,R^3\}
  \]
  forms a group of order $4$. Make a Cayley table for this group where the entries
  are $R^0$, $R^1$, $R^2$, and $R^3$.
\end{exercise}

\begin{exercise}
  In the Ohio State University Marching Band, there is a fundamental
  movement called a ``270-spin-turn.'' With this turn, the band-member
  spins to the left, $270^\circ$, thus ends up facing right. Denote a
  $270$-spin-turn by $T$. Explain how
  \[
  \{T^0,T^1,T^2,T^3\}
  \]
  forms a group of order $4$. Make a Cayley table for this group where
  the entries are $T^0$, $T^1$, $T^2$, and $T^3$.
\end{exercise}



\begin{exercise}\label{E:invg}
  Let
  \begin{align*}
    i(x) &= x,             & \l(x) &= \frac{1}{x}, \\
    f(x) &= \frac{1}{1-x}, &  m(x) &= 1-x, \\
    g(x) &= \frac{x-1}{x}, &  n(x) &= \frac{x}{x-1}.
  \end{align*}
  be functions from $\R-{0,1}\to\R$.  Prove that $\{i,f,g,\l,m,n\}$
  forms a group under function composition.
\end{exercise}


\begin{definition}
  Let $G$ be group, a subset $\{g_1,\dots, g_n,\dots\}\subset G$ is a
  set of \dfn{generators} for $G$ if for all $a\in G$
  \[
  a = \prod_{\mathrm{finite}} g_i^{\alpha_i}
  \]
  where $\alpha_i\in \Z$ and the $g_i$'s are not necessarily distinct.
\end{definition}


\begin{exercise}
  Prove that $C =\{1,i,-1,-i\}$ is generated by $i$. See Example~\ref{E:C}
\end{exercise}

\begin{exercise}
  Prove that $(\Z,+)$ is generated by $1$.
\end{exercise}

\begin{exercise}
  Prove that $(\Z,+)$ is generated by $3$ and $5$.
\end{exercise}

\begin{exercise}
  If $G$ is a group, and $g\in G$, prove that
  \[
  g^m\cdot g^n = g^{m+n}.
  \]
\end{exercise}



\begin{exercise}
  Prove that the group $\{i,f,g,\l,m,n\}$, as defined in
  Exercise~\ref{E:invg}, is generated by $f$ and $\l$. Make a Cayley
  table with entries
  \[
  e, f, f\circ f, \l, \l\circ f,\l \circ f^2
  \]
  where $e$ is the identity function.
\end{exercise}


\section{Abelian groups}


Finally, note that the group operation need not be commutative.

\begin{definition}
  A group $(G,\star)$ is called \dfn{Abelian} if
  \[
  a\star b = b\star a
  \]
  for all $a,b\in G$.
\end{definition}

\begin{exercise}
  Give examples of Abelian groups. Can you give an example of a group
  that is not Abelian? Hint: Think matrix multiplication.
\end{exercise}

\begin{exercise}
  Prove that a group $G$ is Abelian if and only if
  \[
  aba^{-1}b^{-1} = e
  \]
  for all $a,b\in G$.
\end{exercise}


\begin{exercise}
  Suppose that $G$ is a group where for all $x\in G$, $x^2 =e$. Prove
  that $G$ is Abelian.
\end{exercise}


%% \section{Solving equations}

%% When you ask the pros what algebra is ``about,'' a common answer is:
%% \begin{quote}
%%   Algebra is about solving equations.
%% \end{quote}

%% Let me see if I can convince you of this. Suppose you have a group $G$
%% with elements $a,b\in G$. If you wish to solve this equation
%% \[
%% ax = b
%% \]
%% the definition of a group allows you \textit{always} solve this
%% equation by multiplying by $a^{-1}$ on the left, obtaining
%% \[
%% x = a^{-1}b.
%% \]
%% The definition of a group allows you to always solve linear equations
%% consisting of group elements.

%% On the other hand,














\end{document}
