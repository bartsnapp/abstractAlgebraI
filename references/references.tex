\documentclass{ximera}


\title{References and further reading}

\begin{document}
\begin{abstract}
  These are the sources I used to guide me in the construction of this
  document.
\end{abstract}
\maketitle

\vspace{-1in}

\begin{thebibliography}{MMMM2002}

\bibitem[dB1976]{aC1984} Burton, \textit{Elementary number theory}.
  
  \begin{annotate}
    A wonderful introduction to number theory. Very clear proofs and
    exposition.
  \end{annotate}

  

\bibitem[aC1984]{aC1984} Clark, \textit{Elements of abstract algebra}.
  
  \begin{annotate}
    An inspirational book, student focused and active in its
    presentation.
  \end{annotate}
   
  
\bibitem[DF2004]{DF2004} Dummit and Foote, \textit{Abstract algebra}.

  \begin{annotate}
    An excellent encyclopedic presentation with many examples,
    sometimes used as a graduate text.
  \end{annotate}

\bibitem[jG2017]{jG2017} Gallian, \textit{Contemporary abstract algebra}.

  \begin{annotate}
    A popular undergraduate text on abstract algebra.
  \end{annotate}
   
\bibitem[tJ2018]{tJ2018} Judson, \textit{Abstract algebra: Theory and applications}.

  \begin{annotate}
    A free, open-source, interactive text with many computer-algebra
    examples.
  \end{annotate}


\bibitem[mL2006]{mL2006} Livio, \textit{The equation that couldn't be solved:
  How mathematical genius discovered the language of symmetry}.

  \begin{annotate}
    Not a textbook, but a popular mathematics book. Excellent
    presentation of the history, people, and basic ideas of abstract
    algebra.
  \end{annotate}


%% \bibitem[RG2003]{RG2003} Rainbolt and Gallian, \textit{Abstract algebra with \GAP}.
  
%%   \begin{annotate}
%%     A companion book to \cite{jG2017}. Many interesting examples and
%%     exercises.
%%   \end{annotate}

   
\bibitem[rS2019]{rS2019} Solomon, \textit{An introduction to abstract algebra}.

  \begin{annotate}
    A fascinating text with many intriguing examples and results.
  \end{annotate}
  
\end{thebibliography}
\end{document}
