\documentclass{ximera}

\usepackage[T1]{fontenc}
\usepackage{stix2}
\usepackage{gillius}
\usepackage{resizegather}
%\usepackage{rsfso} fancy cal
\DeclareMathAlphabet{\mathcal}{OMS}{cmsy}{m}{n} %less fancy cal


\usepackage{multicol}


\usepackage{tikz-cd}
\usepackage{tkz-euclide} %% compass
\usetkzobj{all}  %% tkzCompass
\tikzset{>=stealth}
\tikzcdset{arrow style=tikz}
\usetikzlibrary{math} %% for assigning variables
%\usetikzlibrary{fadings}

\usepackage{colortbl,boldline,makecell} %% group tables


\usepackage[sans]{dsfont}

\usepackage{stmaryrd,pifont}

\graphicspath{
  {./}
  {fields/}
  }     



\let\oldbibliography\thebibliography%% to compact bib
\renewcommand{\thebibliography}[1]{%
  \oldbibliography{#1}%
  \setlength{\itemsep}{0pt}%
}
\renewcommand\refname{} %% no name needed!


\DefineVerbatimEnvironment{macaulay2}{Verbatim}{numbers=left,frame=lines,label=Macaulay2,labelposition=topline}

\DefineVerbatimEnvironment{gap}{Verbatim}{numbers=left,frame=lines,label=GAP,labelposition=topline}

%%% This next bit of code defines all our theorem environments
\makeatletter
\let\c@theorem\relax
\let\c@corollary\relax
%\let\c@example\relax
\makeatother

\let\definition\relax
\let\enddefinition\relax

\let\theorem\relax
\let\endtheorem\relax

\let\proposition\relax
\let\endproposition\relax

\let\exercise\relax
\let\endexercise\relax

\let\question\relax
\let\endquestion\relax

\let\remark\relax
\let\endremark\relax

\let\corollary\relax
\let\endcorollary\relax


\let\example\relax
\let\endexample\relax

\let\warning\relax
\let\endwarning\relax

\let\lemma\relax
\let\endlemma\relax


\let\algorithm\relax
\let\endalgorithm\relax
\usepackage{algpseudocode}

\newtheoremstyle{SlantTheorem}{\topsep}{\topsep}%%% space between body and thm
		{\slshape}                      %%% Thm body font
		{}                              %%% Indent amount (empty = no indent)
		{\bfseries\sffamily}            %%% Thm head font
		{}                              %%% Punctuation after thm head
		{3ex}                           %%% Space after thm head
		{\thmname{#1}\thmnumber{ #2}\thmnote{ \bfseries(#3)}}%%% Thm head spec
\theoremstyle{SlantTheorem}
\newtheorem{theorem}{Theorem}
%\newtheorem{definition}[theorem]{Definition}
%\newtheorem{proposition}[theorem]{Proposition}
%% \newtheorem*{dfnn}{Definition}
%% \newtheorem{ques}{Question}[theorem]
%% \newtheorem*{war}{WARNING}
%% \newtheorem*{cor}{Corollary}
%% \newtheorem*{eg}{Example}
\newtheorem*{remark}{Remark}
\newtheorem*{touchstone}{Touchstone}
\newtheorem{corollary}{Corollary}[theorem]
\newtheorem*{warning}{WARNING}
\newtheorem{example}[corollary]{Example}
\newtheorem{lemma}[theorem]{Lemma}




\newtheoremstyle{Definition}{\topsep}{\topsep}%%% space between body and thm
		{}                              %%% Thm body font
		{}                              %%% Indent amount (empty = no indent)
		{\bfseries\sffamily}            %%% Thm head font
		{}                              %%% Punctuation after thm head
		{3ex}                           %%% Space after thm head
		{\thmname{#1}\thmnumber{ #2}\thmnote{ \bfseries(#3)}}%%% Thm head spec
\theoremstyle{Definition}
\newtheorem{definition}[theorem]{Definition}



\let\algorithm\relax
\let\endalgorithm\relax
\newtheoremstyle{Alg}{\topsep}{\topsep}%%% space between body and thm
		{}                              %%% Thm body font
		{}                              %%% Indent amount (empty = no indent)
		{\bfseries\sffamily}            %%% Thm head font
		{}                              %%% Punctuation after thm head
		{3ex}                           %%% Space after thm head
		{\thmname{#1}\thmnumber{ #2}\thmnote{ \bfseries(#3)}}%%% Thm head spec
\theoremstyle{Alg}
\newtheorem*{algorithm}{Algorithm}
\newtheorem*{construction}{Construction}




\newtheoremstyle{Exercise}{\topsep}{\topsep} %%% space between body and thm
		{}                           %%% Thm body font
		{}                           %%% Indent amount (empty = no indent)
		{\bfseries\sffamily}         %%% Thm head font
		{)}                          %%% Punctuation after thm head
		{ }                          %%% Space after thm head
		{\thmnumber{#2}\thmnote{ \bfseries(#3)}}%%% Thm head spec
\theoremstyle{Exercise}
\newtheorem{exercise}[corollary]{}%[theorem]

%% \newtheoremstyle{Question}{\topsep}{\topsep} %%% space between body and thm
%% 		{\bfseries}                  %%% Thm body font
%% 		{3ex}                        %%% Indent amount (empty = no indent)
%% 		{}                           %%% Thm head font
%% 		{}                           %%% Punctuation after thm head
%% 		{}                           %%% Space after thm head
%% 		{\thmnumber{#2}\thmnote{ \bfseries(#3)}}%%% Thm head spec
\newtheoremstyle{Question}{3em}{3em} %%% space between body and thm
		{\large\bfseries}                           %%% Thm body font
		{}                           %%% Indent amount (empty = no indent)
		{}                         %%% Thm head font
		{}                          %%% Punctuation after thm head
		{0em}                          %%% Space after thm head
		{}%%% Thm head spec
\theoremstyle{Question}
\newtheorem*{question}{}






\renewcommand{\tilde}{\widetilde}
\renewcommand{\bar}{\overline}
\renewcommand{\hat}{\widehat}
\newcommand{\N}{\mathbb N}
\newcommand{\Z}{\mathbb Z}
\newcommand{\R}{\mathbb R}
\newcommand{\Q}{\mathbb Q}
\newcommand{\C}{\mathbb C}
\newcommand{\V}{\mathbb V}
\newcommand{\I}{\mathbb I}
\newcommand{\A}{\mathbb A}
\renewcommand{\o}{\mathbf o}
\newcommand{\iso}{\simeq}
\newcommand{\ph}{\varphi}
\newcommand{\Cf}{\mathcal{C}}
\newcommand{\IZ}{\mathrm{Int}(\Z)}
\newcommand{\dsum}{\oplus}
\newcommand{\directsum}{\bigoplus}
\newcommand{\union}{\bigcup}
\newcommand{\subgp}{\leq}
\newcommand{\normal}{\trianglelefteq}
\renewcommand{\i}{\mathfrak}
\renewcommand{\a}{\mathfrak{a}}
\renewcommand{\b}{\mathfrak{b}}
\newcommand{\m}{\mathfrak{m}}
\newcommand{\p}{\mathfrak{p}}
\newcommand{\q}{\mathfrak{q}}
\newcommand{\dfn}[1]{\textbf{#1}\index{#1}}
\let\hom\relax
\DeclareMathOperator{\mat}{Mat}
\DeclareMathOperator{\ann}{Ann}
\DeclareMathOperator{\h}{ht}
\DeclareMathOperator{\tr}{tr}
\DeclareMathOperator{\hom}{Hom}
\DeclareMathOperator{\Span}{Span}
\DeclareMathOperator{\spec}{Spec}
\DeclareMathOperator{\maxspec}{MaxSpec}
\DeclareMathOperator{\aut}{Aut}
\DeclareMathOperator{\ass}{Ass}
\DeclareMathOperator{\lcm}{lcm}
\DeclareMathOperator{\ff}{Frac}
\DeclareMathOperator{\im}{Im}
\DeclareMathOperator{\syz}{Syz}
\DeclareMathOperator{\gr}{Gr}
\DeclareMathOperator{\multideg}{multideg}
\renewcommand{\ker}{\mathop{\mathrm{Ker}}\nolimits}
\newcommand{\coker}{\mathop{\mathrm{Coker}}\nolimits}
\newcommand{\lps}{[\hspace{-0.25ex}[}
\newcommand{\rps}{]\hspace{-0.25ex}]}
\newcommand{\into}{\hookrightarrow}
\newcommand{\onto}{\twoheadrightarrow}
\newcommand{\tensor}{\otimes}
\newcommand{\x}{\mathbf{x}}
\newcommand{\X}{\mathbf X}
\newcommand{\Y}{\mathbf Y}
\renewcommand{\k}{\boldsymbol{\kappa}}
\renewcommand{\emptyset}{\varnothing}
\renewcommand{\qedsymbol}{$\blacksquare$}
\renewcommand{\l}{\ell}
\newcommand{\1}{\mathds{1}}
\newcommand{\lto}{\mathop{\longrightarrow\,}\limits}
\newcommand{\rad}{\sqrt}
\newcommand{\hf}{H}
\newcommand{\hs}{H\!S}
\newcommand{\hp}{H\!P}
\renewcommand{\vec}{\mathbf}
\let\temp\phi
\let\phi\varphi
\let\eulerphi\temp


\renewcommand{\epsilon}{\varepsilon}
\renewcommand{\subset}{\subseteq}
\renewcommand{\supset}{\supseteq}
\newcommand{\macaulay}{\normalfont\textsl{Macaulay2}}
\newcommand{\GAP}{\normalfont\textsf{GAP}}
\newcommand{\invlim}{\varprojlim}
\renewcommand{\le}{\leqslant}
\renewcommand{\ge}{\geqslant}
\newcommand{\valpha}{{\boldsymbol\alpha}}
\newcommand{\vbeta}{{\boldsymbol\beta}}
\newcommand{\vgamma}{{\boldsymbol\gamma}}
\newcommand{\dotp}{\bullet}
\newcommand{\lc}{\normalfont\textsc{lc}}
\newcommand{\lt}{\normalfont\textsc{lt}}
\newcommand{\lm}{\normalfont\textsc{lm}}
\newcommand{\from}{\leftarrow}
\newcommand{\transpose}{\intercal}
\newcommand{\grad}{\boldsymbol\nabla}
\newcommand{\curl}{\boldsymbol{\nabla\times}}
\renewcommand{\d}{\, d}
\newcommand{\<}{\langle}
\renewcommand{\>}{\rangle}

%\renewcommand{\proofname}{Sketch of Proof}


\renewenvironment{proof}[1][Proof]
  {\begin{trivlist}\item[\hskip \labelsep \itshape \bfseries #1{}\hspace{2ex}]\upshape}
{\qed\end{trivlist}}

\newenvironment{sketch}[1][Sketch of Proof]
  {\begin{trivlist}\item[\hskip \labelsep \itshape \bfseries #1{}\hspace{2ex}]\upshape}
{\qed\end{trivlist}}



\makeatletter
\renewcommand\section{\@startsection{paragraph}{10}{\z@}%
                                     {-3.25ex\@plus -1ex \@minus -.2ex}%
                                     {1.5ex \@plus .2ex}%
                                     {\normalfont\large\sffamily\bfseries}}
\renewcommand\subsection{\@startsection{subparagraph}{10}{\z@}%
                                    {3.25ex \@plus1ex \@minus.2ex}%
                                    {-1em}%
                                    {\normalfont\normalsize\sffamily\bfseries}}
\makeatother

%% Fix weird index/bib issue.
\makeatletter
\gdef\ttl@savemark{\sectionmark{}}
\makeatother


\makeatletter
%% no number for refs
\newcommand\frontstyle{%
  \def\activitystyle{activity-chapter}
  \def\maketitle{%
    \addtocounter{titlenumber}{1}%
                    {\flushleft\small\sffamily\bfseries\@pretitle\par\vspace{-1.5em}}%
                    {\flushleft\LARGE\sffamily\bfseries\@title \par }%
                    {\vskip .6em\noindent\textit\theabstract\setcounter{problem}{0}\setcounter{sectiontitlenumber}{0}}%
                    \par\vspace{2em}
                    \phantomsection\addcontentsline{toc}{section}{\textbf{\@title}}%
                  }}
\makeatother



\NewEnviron{annotate}{\vspace{-.3cm}\small \itshape \BODY \vspace{.3cm}}


%%%% TIKZ STUFF

%% N-GON code
\tikzset{
    pics/tikzngon/.style={
        code={
        \tikzmath{\xx = #1;\rr=1.7;}
        \draw[ultra thick,rounded corners=.05mm] ({\rr*sin(0*360/\xx)},{\rr*cos(0*360/\xx)})
        \foreach \x in {-1,0,...,\xx}
        {
        -- ({\rr*sin(\x*360/\xx)},{\rr*cos(\x*360/\xx)})
        }
           -- cycle;
  }}}

%% N-GON code (even)
\tikzset{
    pics/tikzegon/.style={
        code={
        \tikzmath{\xx = #1;\rr=1.7;}
        \draw[ultra thick,rounded corners=.05mm] ({\rr*sin(0*360/\xx+180/\xx)},{\rr*cos(0*360/\xx+180/\xx)})
        \foreach \x in {-1,0,...,\xx}
           {
           -- ({\rr*sin(\x*360/\xx+180/\xx)},{\rr*cos(\x*360/\xx+180/\xx)}) 
           }
           -- cycle;
  }}}




%% N-CLOCK code
\tikzset{
    pics/tikznclock/.style={
        code={
        \tikzmath{\xx = #1;\rr=1.7;\dd=.4;}
        \foreach \x in {1,...,\xx}
        \pgfmathtruncatemacro{\xy}{\x-1}
           {
             \node[circle,fill=black,inner sep=0pt, minimum size=13pt,text=white]
             at ({(\rr-\dd)*sin((\x-1)*360/(\xx)},{(\rr-\dd)*cos((\x-1)*360/\xx}) {\normalfont\bfseries\sffamily\small {\xy}};
           }
  \draw[thick] (0,0) circle (\rr);
  }}}



%% barcode from
%% https://tex.stackexchange.com/questions/6895/is-there-a-good-latex-package-for-generating-barcodes
%% NOT CURRENTLY USED!


\def\barcode#1#2#3#4#5#6#7{\begingroup%
  \dimen0=0.1em
  \def\stack##1##2{\oalign{##1\cr\hidewidth##2\hidewidth}}%
  \def\0##1{\kern##1\dimen0}%
  \def\1##1{\vrule height10ex width##1\dimen0}%
  \def\L##1{\ifcase##1\bc3211##1\or\bc2221##1\or\bc2122##1\or\bc1411##1%
    \or\bc1132##1\or\bc1231##1\or\bc1114##1\or\bc1312##1\or\bc1213##1%
    \or\bc3112##1\fi}%
  \def\R##1{\bgroup\let\next\1\let\1\0\let\0\next\L##1\egroup}%
  \def\G##1{\bgroup\let\bc\bcg\L##1\egroup}% reverse
  \def\bc##1##2##3##4##5{\stack{\0##1\1##2\0##3\1##4}##5}%
  \def\bcg##1##2##3##4##5{\stack{\0##4\1##3\0##2\1##1}##5}%
  \def\bcR##1##2##3##4##5##6{\R##1\R##2\R##3\R##4\R##5\R##6\11\01\11\09%
    \endgroup}%
  \stack{\09}#1\11\01\11\L#2%
  \ifcase#1\L#3\L#4\L#5\L#6\L#7\or\L#3\G#4\L#5\G#6\G#7%
    \or\L#3\G#4\G#5\L#6\G#7\or\L#3\G#4\G#5\G#6\L#7%
    \or\G#3\L#4\L#5\G#6\G#7\or\G#3\G#4\L#5\L#6\G#7%
    \or\G#3\G#4\G#5\L#6\L#7\or\G#3\L#4\G#5\L#6\G#7%
    \or\G#3\L#4\G#5\G#6\L#7\or\G#3\G#4\L#5\G#6\L#7%
  \fi\01\11\01\11\01\bcR}


\author{Bart Snapp}

\title{Field extensions}

\begin{document}
\begin{abstract}
  We introduce field extensions.
\end{abstract}
\maketitle

To understand a group, it helps to understand the subgroup
structure. To understand a field, it helps to understand its field
extensions.


\begin{definition}
  Given fields $K\subset L$, $L$ is a \dfn{field extension} of $K$ if
  the field operations and identities $L$ restrict to those of $K$. If
  there are several field extensions, say $k \subset K \subset L$ we
  often write
  \[
  \begin{tikzcd}
    L\ar[d,-] \\
    K \ar[d,-]\\
    k
  \end{tikzcd}
  \]
  and call this a \dfn{tower of field extensions}. In this case, $k$ is called the \dfn{ground field}.
\end{definition}

\begin{example}[Reals and rationals]
  The real numbers are a field extension of the rational numbers,
  \[
  \begin{tikzcd}
    \R\ar[d,-] \\
    \Q
  \end{tikzcd}
  \]
  and here $\Q$ is the ground field. In particular, if $K$ is a field
  of characteristic zero, $K$ is (isomorphic to) a field extension of $\Q$.
\end{example}



\begin{example}[Fields adjoin an element]\index{K[alpha]@$K[\alpha]$}
  Given a field $K$, we may adjoin an element $\alpha$ by writing
  \[
  K[\alpha] :=\{\text{polynomials in terms of $\alpha$ with
    coefficients in $K$}\}.
  \]
  This may or may not be a field depending on $\alpha$. In a somewhat
  similar way,
  \[
  K(\alpha) := \bigcap_{\substack{\alpha \in L\\ K\subseteq L}} L\index{K(alpha)@$K(\alpha)$}
  \]
  where $L$ ranges over all field extensions of $K$ containing
  $\alpha$.
\end{example}

\begin{exercise}
  Prove if $K[\alpha]$ is a field, then $K[\alpha] = K(\alpha)$.
\end{exercise}


\begin{exercise}
  Define:
  \[
  \Q[\sqrt{2}] = \{ a+ b\sqrt{2}: a,b,\in\Q\}.
  \]
  Prove that $\Q[\sqrt{2}]$ is a field, and so $\Q[\sqrt{2}] = \Q(\sqrt{2})$.
\end{exercise}

\begin{exercise}
  Prove or disprove that
  \[
  \Z_5[i] = \{a + b i: a,b\in\Z_5\}
  \]
  is a field, and so $\Z_5[i] = \Z_5(i)$.
\end{exercise}


\begin{example}[Rationals and some irrationals]
  The structure of field extensions can be more complicated than the
  first example. Consider
  \[
  \begin{tikzcd}
    & \Q[\sqrt{2},\sqrt{3}]\ar[rd,-] \ar[d,-] \ar[ld,-]& \\
    \Q[\sqrt{2}] \ar[rd,-]& \Q[\sqrt{6}] \ar[d,-]& \Q[\sqrt{3}]\ar[ld,-]\\
    &  \Q & 
  \end{tikzcd}
  \]
  again, $\Q$ is the ground field.  The reader should verify that each
  of
  \begin{align*}
    \Q[\sqrt{2}] &=\{a + b\sqrt{2}: a,b\in \Q\},\\
    \Q[\sqrt{3}] &=\{a + b\sqrt{3}: a,b\in \Q\},\\
    \Q[\sqrt{6}] &=\{a + b\sqrt{6}: a,b\in \Q\},\\
    \Q[\sqrt{2},\sqrt{3}] &=\{a + b\sqrt{2} + c\sqrt{3} + d\sqrt{6}: a,b,c,d\in \Q\},
  \end{align*}
  are in fact fields.
\end{example}



\begin{lemma}[Field extensions are vector spaces]\label{L:fevs}
  If $L$ is a field extension of $K$, then $L$ is a $K$-vector space.
  \begin{sketch}
    Check the conditions of the definition of a vector space.
  \end{sketch}
\end{lemma}

\begin{definition}\index{degree!of a field extension}
  If $L$ is a field extension of $K$, then we define the
  \textbf{degree} of a $L$ over $K$ to be the dimension of $L$ as a
  $K$-vector space. We denote the degree of $L$ over $K$ with
  $[L:K]$. In short,
  \[
  [L:K] = \dim_K(L).
  \]
\end{definition}

\begin{example}[Complex numbers]
  The complex numbers
  \[
  \C = \{a+bi:a,b\in\R\}
  \]
  are a field extension of $\R$. Since a basis of $\C$ over $\R$ is
  $\{1,i\}$, we see $\dim_\R(\C)=2$ and so we say $[\C:\R]=2$. The
  field $\C$ is a degree two extension of $\R$.
\end{example}

\begin{example}[{$\boldsymbol{\pmb{[\Q(}\sqrt{2}\pmb{):\Q]}}$}]
  The degree of $\Q(\sqrt{3})$ over $\Q$ is $2$, since
  $\{1,\sqrt{3}\}$ is a basis for $\Q(\sqrt{3})$ as a $\Q$-vector
  space.
\end{example}


\begin{example}[$\boldsymbol{\pmb{[\Q(}\sqrt{2},\sqrt{3}\pmb{):\Q]}}$]
  The degree of $\Q(\sqrt{2},\sqrt{3})$ over $\Q$ is $4$, since
  $\{1,\sqrt{2},\sqrt{3},\sqrt{6}\}$ is a basis for
  $\Q(\sqrt{2},\sqrt{3})$ as a $\Q$-vector space.
\end{example}

\begin{example}[{$\boldsymbol{\pmb{[\Q(}\sqrt{2},\sqrt{3}\pmb{):\Q(}\sqrt{6}\pmb{)]}}$}]
  The degree of $\Q(\sqrt{2},\sqrt{3})$ over $\Q(\sqrt{6})$ is $2$, since
  $\{1,\sqrt{2}\}$ is a basis for
  $\Q(\sqrt{2},\sqrt{3})$ as a $\Q(\sqrt{6})$-vector space.
\end{example}

\begin{lemma}[Degree is multiplicative]\label{L:dm}
  Let $k\subset K\subset L$ be a tower of field extensions,
  \[
  \begin{tikzcd}[column sep=0pt]
    L\ar[d,-,"{[L:K]}",swap]  &             &  L\ar[dd,-,"{[L:K]\cdot [K:k]}"]   \\
    K \ar[d,-,"{[K:k]}",swap] & \Rightarrow &     \\
    k                         &             &  k
  \end{tikzcd}
  \]
  In this case $[L:K]\cdot [K:k] = [L:k]$.
  \begin{proof}
    Let
    \[
    A = \{\alpha_1,\alpha_2,\dots,\alpha_m\}\subset L
    \]
    be a basis for $L$ as a $K$-vector space. Let
    \[
    B = \{\beta_1,\beta_2,\dots,\beta_n\}\subset K
    \]
    be a basis for $K$ as a $k$-vector space. We'll show that
    \[
    C = \{\alpha_i\cdot \beta_j:\alpha_i\in A \text{ and }\beta_j\in B\} \subset L
    \]
    is a basis for $L$ as a $k$-vector space. Since $C$ has
    $[L:K]\cdot [K:k]$ elements, this will prove the result.

    We must show that $C$ spans $L$ as a $k$-vector space and that $C$
    is a linearly independent set of vectors in a $k$-vector space.

    \textbf{$\boldsymbol C$ spans $\boldsymbol L$ as a $\boldsymbol k$-vector space.} Suppose
    that $\l\in L$. Since $L$ is a $K$-vector space, we may write
    \[
    \l = a_1\alpha_1 + a_2 \alpha_2 + \cdots + a_m\alpha_m,
    \]
    where each $a_i\in K$. However, $K$ is a $k$-vector space, so for
    each $i$, we may write
    \[
    a_i = b_{i,1}\beta_1 +  b_{i,2}\beta_2 + \cdots +   b_{i,n}\beta_n
    \]
    where each $b_{i,j}\in k$. Substituting in, we can express $\l$ as
    a linear combination of vectors of the form $\alpha_i\cdot
    \beta_j$. Hence $C$ spans $L$ as a $k$-vector space.



    \textbf{$\boldsymbol C$ is a linear independent set of vectors.}
    Suppose that there is a linear relation
    \begin{align*}
      0 = &c_{1,1} \alpha_1\beta_1 + c_{1,2} \alpha_1\beta_2 + \cdots + c_{1,n} \alpha_1\beta_n\\
      +&c_{2,1} \alpha_2\beta_1 + c_{2,2} \alpha_2\beta_2 + \cdots + c_{2,n} \alpha_2\beta_n\\
      &\vdots\\
      +&c_{m,1} \alpha_m\beta_1 + c_{m,2} \alpha_m\beta_2 + \cdots + c_{m,n} \alpha_m\beta_n,
    \end{align*}
    where $c_{i,j}\in k$. Factor out $\alpha_i$ to find
    \begin{align*}
      0 = &\alpha_1(c_{1,1} \beta_1 + c_{1,2} \beta_2 + \cdots + c_{1,n} \beta_n)\\
      +&\alpha_2(c_{2,1} \beta_1 + c_{2,2} \beta_2 + \cdots + c_{2,n} \beta_n)\\
      &\vdots\\
      +&\alpha_m(c_{m,1} \beta_1 + c_{m,2} \beta_2 + \cdots + c_{m,n} \beta_n),
    \end{align*}
    where each 
    \[
    (c_{i,1} \beta_1 + c_{i,2} \beta_2 + \cdots + c_{i,n} \beta_n)\in K.
    \]
    Since $A$ is a set of linearly independent vectors, we see that
    for each $i= 1,\dots, m$
    \[
    (c_{i,1} \beta_1 + c_{i,2} \beta_2 + \cdots + c_{i,n} \beta_n) = 0.
    \]
    However, since $B$ is a basis for $K$ as a $k$-vector space, we
    see each $c_{i,j}=0$. This completes the proof.
  \end{proof}
\end{lemma}

\begin{corollary}[Finite extensions of finite extensions]\label{C:fe}
  If $k\subset K\subset L$ is a tower of field extensions
  \[
  \begin{tikzcd}[column sep=0pt]
    L\ar[d,-,"\text{finite}",swap] & & L\ar[dd,-,"\text{finite}"]\\
    K \ar[d,-,"\text{finite}",swap] & \Rightarrow& \\
    k & \vphantom{K} & k
  \end{tikzcd}
  \]
  and $[K:k]<\infty$ and $[L:K]<\infty$, then $[L:k]<\infty$.
\end{corollary}



\section{Algebraic extensions}

\begin{definition}
  Let $K\subset L$ be a field extension.
  \begin{itemize}
  \item An element $\alpha\in L$ is said to be \dfn{algebraic} over
    $K$ if there exists $f(x) \in K[x]$ such that $f(\alpha) = 0$.
  \item The field $L$ is an \dfn{algebraic extension} over $K$ if for
    every element $\alpha\in L$, $\alpha$ is algebraic over $K$.
  \item If $\alpha\in L$ is not the root of any polynomial with
    coefficients in $K$, then we say $\alpha$ is \dfn{transcendental}
    over $K$.
  \end{itemize}
\end{definition}

\begin{example}[$\pmb{\C}$ is algebraic over $\pmb{\R}$]
  Let $a+ bi$ be any complex number. It's the root of
  \[
  x^2 -2ax + a^2 + b^2 \in \R[x].
  \]
  Hence, $\C$ is algebraic over $\R$.
\end{example}

\begin{exercise}
  Prove that $\cos(r\pi)$ is algebraic over $\Q$ for all $r\in\Q$.
\end{exercise}

\begin{exercise}
  A field $K$ is called \textbf{algebraically closed} if every
  polynomial of positive degree in $K[x]$ has a root in $K$. For
  example, $\C$ is algebraically closed but $\R$ is not. Prove that
  every algebraically closed field has an infinite number of elements.
\end{exercise}


\begin{definition}
  Let $K\subset L$ be an algebraic extension of fields with
  $\alpha\in L$. A polynomial $f(x)\in K[x]$ is called a \dfn{minimal
    polynomial} for $\alpha$ if for all $g(x)\in K[x]$,
  \[
  g(\alpha) = 0 \quad\Rightarrow\quad \deg(f) \le \deg(g).
  \]
\end{definition}

\begin{example}[Minimal polynomials]
  Let $a+bi$ be any complex number where $a,b\in\R$ and $b\ne 0$. A
  minimal polynomial for this complex number is
  \[
  (x-(a+bi))(x-(a-bi)) = x^2 -2ax +a^2 + b^2\in\R[x].
  \]
\end{example}

\begin{lemma}[Minimal polynomials and division]
  Let $K$ be a field and $f(x)\in K[x]$. If $f(x),g(x)\in K[x]$ and
  $f(x)$ is a minimal polynomial for $\alpha$ over $K$, and $g(\alpha)
  = 0$, then $f(x) | g(x)$.
  \begin{proof}
    Seeking a contradiction, suppose that $g(\alpha) = 0$ and $f(x)
    \nmid g(x)$. Then
    \[
    g(x) = f(x) \cdot q(x) + r(x), \quad 0<\deg(r)< \deg(f).
    \]
    Now evaluate at $x = \alpha$,
    \begin{align*}
      g(\alpha) &= f(\alpha) \cdot q(\alpha) + r(\alpha),\\
      0 &= 0 + r(\alpha),
    \end{align*}
    so $r(\alpha)=0$, but this is a contradiction, as the degree of
    $r(x)$ is less than the degree of the minimal polynomial.
  \end{proof}
\end{lemma}

\begin{lemma}[Minimal polynomials are irreducible]\label{L:ipi}
  Let $K$ be a field and $f(x)\in K[x]$. In this case $f(x)$ is a
  minimal polynomial for $\alpha$ over $K$ if and only if $f(x)$ is
  irreducible in $K[x]$.
  \begin{proof}
    $(\Rightarrow)$ Seeking a contradiction, suppose that $f(x)$ is a
    minimal polynomial for $\alpha$ over $K$ and that
    \[
    f(x) = g(x) \cdot h(x) 
    \]
    where
    \[
    \deg(g) <\deg(f)\quad\text{and}\quad\deg(h) <\deg(f).
    \]
    Now $f(\alpha) = 0$ implies that $g(\alpha) =0$ or $h(\alpha) =
    0$, but the degrees of both $g(x)$ and $h(x)$ are both less than
    the degree of $f(x)$, the minimal polynomial. This is a
    contradiction.

    $(\Leftarrow)$ Suppose that $f(\alpha) = 0$ and that $f(x)$ is
    irreducible in $K[x]$. We claim $f$ has minimal degree, and is
    hence the minimal polynomial for $\alpha$. Seeking a
    contradiction, suppose that a minimal polynomial for $\alpha$ is
    $g(x)\in K[x]$, where $\deg(g) < \deg(f)$. Since $f(x)$ is irreducible,
    \[
    f(x) = g(x) \cdot h(x) + r(x) \qquad \deg(r)<\deg(g).
    \]
    Evaluating at $\alpha$, we see that $r(x)$ has $\alpha$ as a root,
    contradicting the minimality of $g(x)$. Hence if $f(x)$ is
    irreducible, and $f(\alpha) = 0$, then $f(x)$ is a minimal
    polynomial for $\alpha$.
  \end{proof}
\end{lemma}


\begin{lemma}[Field extensions and congruence]
  Let $K$ be a field and $f(x)\in K[x]$ be a minimal polynomial for
  $\alpha$. In this case
  \[
  K[x]/(f(x))\iso K(\alpha). 
  \]
  \begin{proof}
    By Lemma~\ref{L:ipi}, $f(x)$ is irreducible, and hence by
    Theorem~\ref{T:cmip}, $K[x]/(f(x))$ is a field. Define a map
    \begin{align*}
    \phi:K[x]/(f(x)) &\to K(\alpha)\\
    g(x) &\mapsto g(\alpha).
    \end{align*}
    We must prove that this map is \index{well-defined}well-defined.
    Suppose that
    \[
    g(x) \equiv h(x) \pmod{f(x)}
    \]
    We must show that $\phi(g(x)) = \phi(h(x))$. Since $g(x) \equiv
    h(x) \pmod{f(x)}$, we may write
    \[
    g(x) = h(x) + n(x) \cdot f(x).
    \]
    If we evaluate this at $x= \alpha$, we see
    \[
    g(\alpha) = h(\alpha)
    \]
    as $f(\alpha) = 0$.

    We claim that $\phi$ is an isomorphism of fields. First note
    $\phi(0) = 0$, $\phi(1) = 1$ and that if $g(x),h(x)\in
    K[x]/(f(x))$, then
    \begin{align*}
      \phi(g(x) + h(x)) &= g(\alpha) + h(\alpha) & &\text{and}\\
      \phi(g(x) \cdot h(x)) &= g(\alpha)\cdot h(\alpha). & &
    \end{align*}


    Finally, we claim that this map is a surjective homomorphism of
    fields, and hence is bijective. Consider
    \[
    c_m\alpha^m + c_{m-1}\alpha^{m-1} + \cdots + c_1\alpha + c_0\in
    K(\alpha).
    \]
    The polynomial
    \[
    h(x) = c_m x^m + c_{m-1} x^{m-1} + \cdots + c_1x  + c_0\in
    K[x]
    \]
    will map to this polynomial. Moreover, if $m\ge \deg(f)$, then
    we may write
    \[
    h(x) \equiv g(x) \pmod{f(x)}
    \]
    where $\deg(g)< \deg(f)$. Thus $\phi(g(x)) =\phi(h(x))$ and
    we are done.
  \end{proof}
\end{lemma}


\begin{remark}
From the proof of the lemma above, we see that if $\alpha$ is
algebraic over $K$ of degree $n$, then every element of $K(\alpha)$ can be
expressed in the form
\[
c_0 + c_1\alpha_1 + \cdots + c_{n-1}\alpha^{n-1}.
\]
\end{remark}



\begin{theorem}[Degree of an algebraic extension]\label{T:dae}
  Let $K$ be a field and let $\alpha$ be algebraic over $K$ with
  minimal polynomial $f(x)\in K[x]$. Then
  \[
  \dim_K(K(\alpha)) = [K(\alpha):K] = \deg(f) = n,
  \]
  and
  \[
  \{1,\alpha,\alpha^2,\dots,\alpha^{n-1}\}
  \]
  is a basis for $K(\alpha)$ as a $K$-vector space.
  \begin{proof}
    Since $K(\alpha)$ is a field that contains $\alpha$, it also contains
    \[
    1, \alpha, \alpha^2, \dots, \alpha^{n-1}.
    \]
    Set $V$ to be the $K$-vector space spanned by
    \[
    A = \{1,\alpha,\alpha^2,\dots,\alpha^{n-1}\}.
    \]
    By our lemma above, $V = K(\alpha)$.  We claim that $A$ is a basis for
    $V$ over $K$. By our setup, $A$ necessarily spans $V$. Now we'll
    show that the set $A$ is a set of linearly independent vectors.

    Seeking a contradiction, suppose that $A$ consists of linearly
    dependent vectors, so we may write
    \[
    c_0 + c_1\alpha_1 + \cdots + c_{n-1}\alpha^{n-1} =0
    \]
    where $c_i\ne 0$ for some $i$. But now $\alpha$ is a root of a
    polynomial of degree $n-1$, a contradiction, since the degree of
    the minimal polynomial for $\alpha$ is $n$. Thus
    \[
    \{1,\alpha,\alpha^2,\dots,\alpha^{n-1}\}
    \]
    is a linearly independent set of vectors, and hence a basis.
  \end{proof}
\end{theorem}


\begin{example}[Cube root of unity]
  Consider $x^3-1 \in \Q[x]$. By the fundamental theorem of algebra,
  Theorem~\ref{T:fta}, this polynomial has three roots in $\C$, all of
  whose cube root is one. Here is a plot of them in the complex plane:
\[
  \begin{tikzpicture}  
    \begin{axis}[  
        xmin=-1.2,  
        xmax=1.2,  
        ymin=-1.2,  
        ymax=1.2,  
        axis lines=center,
        ticks=none,
        xlabel=$\Re$,  
        ylabel=$\Im$,  
        every axis y label/.style={at=(current axis.above origin),anchor=south},  
        every axis x label/.style={at=(current axis.right of origin),anchor=west},
        axis equal image,clip mode=individual
      ]
      \draw[dashed] (axis cs: 0,0) circle[radius=.935in];
      
      \draw[fill=black] (axis cs: {1},{0}) circle[radius=.03in];
      \node[anchor=south west] at (axis cs:{1},{0}) {\scriptsize$1$};
      
      \draw[fill=black] (axis cs: {cos(360/3)},{sin(360/3)}) circle[radius=.03in];
      \node[anchor=east] at (axis cs:{cos(360/3)},{sin(360/3)}) {\scriptsize$e^{2\pi i/5} = \cos(2\pi/3) + i \sin(2\pi/3)$};

      \draw[fill=black] (axis cs: {cos(2*360/3)},{sin(2*360/3)}) circle[radius=.03in];
      \node[anchor=east] at (axis cs:{cos(2*360/3)},{sin(2*360/3)}) {\scriptsize$e^{4\pi i/3} = \cos(4\pi/3) + i \sin(4\pi/3)$};
    \end{axis}
  \end{tikzpicture}  
  \]
In this case,
  \[
  x^3-1= (x-1)(x^2+x+1).
  \]
  We claim $x^2+x+1$ is irreducible in $\Q[x]$.  Since the degree of
  our polynomial is $2$ we can simply check for roots by
  Lemma~\ref{L:d23}. Since $x^2+x+1$ has no roots in $\Z_2[x]$ it is
  irreducible in $\Z_2[x]$, and hence irreducible in $\Z[x]$ by
  Lemma~\ref{L:RCP}; we could have also used the rational roots test
  here, Lemma~\ref{L:rr}. Now by Gauss' lemma, Lemma~\ref{L:G},
  $x^2+x+1$ is irreducible in $\Q[x]$. Let $\zeta$ be a root of $x^2 +
  x +1$. In this case
  \[
  [\Q(\zeta):\Q] = 2
  \]
  as $\Q(\zeta)$ has $\{1,\zeta\}$ as a basis as a $\Q$-vector space,
  and also because the minimum polynomial for $\zeta$ has degree
  $2$. Moreover, $\Q(\zeta) \iso \Q[x]/(x^2 + x+ 1)$.
\end{example}


\begin{example}[Fifth root of unity]
  Consider $x^5-1 \in \Q[x]$. By the fundamental theorem of algebra,
  Theorem~\ref{T:fta}, this polynomial has five roots in $\C$, all of
  whose cube root is one. Here is a plot of them in the complex plane:
\[
  \begin{tikzpicture}  
    \begin{axis}[  
        xmin=-1.2,  
        xmax=1.2,  
        ymin=-1.2,  
        ymax=1.2,  
        axis lines=center,
        ticks=none,
        xlabel=$\Re$,  
        ylabel=$\Im$,  
        every axis y label/.style={at=(current axis.above origin),anchor=south},  
        every axis x label/.style={at=(current axis.right of origin),anchor=west},
        axis equal image,clip mode=individual
      ]
      \draw[dashed] (axis cs: 0,0) circle[radius=.935in];
      
      \draw[fill=black] (axis cs: {1},{0}) circle[radius=.03in];
      \node[anchor=south west] at (axis cs:{1},{0}) {\scriptsize$1$};
      
      \draw[fill=black] (axis cs: {cos(360/5)},{sin(360/5)}) circle[radius=.03in];
      \node[anchor=south west] at (axis cs:{cos(360/5)},{sin(360/5)}) {\scriptsize$e^{2\pi i/5} = \cos(2\pi/5) + i \sin(2\pi/5)$};

      \draw[fill=black] (axis cs: {cos(2*360/5)},{sin(2*360/5)}) circle[radius=.03in];
      \node[anchor=east] at (axis cs:{cos(2*360/5)},{sin(2*360/5)}) {\scriptsize$e^{4\pi i/5} = \cos(4\pi/5) + i \sin(4\pi/5)$};

      \draw[fill=black] (axis cs: {cos(3*360/5)},{sin(3*360/5)}) circle[radius=.03in];
      \node[anchor=east] at (axis cs:{cos(3*360/5)},{sin(3*360/5)}) {\scriptsize$e^{6\pi i/5} = \cos(6\pi/5) + i \sin(6\pi/5)$};

      \draw[fill=black] (axis cs: {cos(4*360/5)},{sin(4*360/5)}) circle[radius=.03in];
      \node[anchor=north west] at (axis cs:{cos(4*360/5)},{sin(4*360/5)}) {\scriptsize$e^{8\pi i/5} = \cos(8\pi/5) + i \sin(8\pi/5)$};
    \end{axis}
  \end{tikzpicture}  
  \]
  In this case,
  \[
  x^5-1= (x-1)(x^4+x^3+x^2+x+1).
  \]
  We claim $f(x)=x^4+x^3+x^2+x+1$ is irreducible in $\Q[x]$.  By
  Gauss' lemma, Lemma~\ref{L:G}, it is sufficient to show that $f$ is
  irreducible in $\Z[x]$. By Exercise~\ref{E:et}, it is sufficient to
  show that
  \[
  f(y+1)  = x^4 + 5x^3 + 10x^2 + 10x +  5
  \]
  is irreducible. However, $f(y+1)$ is Eisenstein, and hence
  irreducible. Thus $f$ is irreducible in $\Q[x]$.  Let $\zeta =
  e^{2\pi i/5}$, a root of $f$. In this case
  \[
  [\Q(\zeta):\Q] = 4
  \]
  as $\Q(\zeta)$ has $\{1,\zeta,\zeta^2,\zeta^3\}$ as a basis as a $\Q$-vector space,
  and also because the minimum polynomial for $\zeta$ has degree
  $4$. Moreover, $\Q(\zeta) \iso \Q[x]/(x^4+x^3+x^2+1)$.
\end{example}


\begin{exercise}
  Let $K(\alpha)$ be a degree $n$ algebraic extension of
  $K$. Prove that
  \[
  \{\alpha,\alpha^2, \cdots, \alpha^{n-1},\alpha^n\}
  \]
  is a basis for $K(\alpha)$ as a $K$-vector space.
\end{exercise}




\begin{theorem}[Finite extensions are algebraic]\label{T:fea}
  If $L$ is a field extension of $K$ and $[L:K]<\infty$, then $L$ is
  algebraic over $K$.
  \begin{proof}
    Suppose that $[L:K]=n$. Then as a $K$-vector space, $\dim_K(L) =
    n$. This means that there is a basis for $L$ with $n$
    elements. Take any element $\alpha\in L$, and consider powers of
    this element, starting with $0$ and ending at $n$
    \[
    \{ \underbrace{1, \alpha, \alpha^2 ,\dots, \alpha^n}_{\text{$n+1$ elements}}\}.
    \]
    Since $n+1$ is larger than the vector space dimension, this is a
    linearly dependent set of elements. Thus, there exist $c_i\in K$
    (not all zero) such that
    \[
    c_n\alpha^n + c_{n-1}\alpha^{n-1} + \cdots + c_1\alpha + c_0 = 0,
    \]
    and so $\alpha$ is a root of the polynomial
    \[
    c_n x^n + c_{n-1}x^{n-1} + \cdots + c_1 x + c_0,
    \]
    and hence $\alpha$ is algebraic over $K$.
  \end{proof}
\end{theorem}



\begin{warning}[Algebraic extensions are not necessarily finite]
  Consider
  \[
  \Q(\sqrt{2},\sqrt{3},\sqrt{5},\dots)
  \]
  where we adjoin every root of a nonperfect square to $\Q$. This
  extension is an algebraic extension of of $\Q$, but it is
  \textbf{not} a finite extension of $\Q$.
\end{warning}





\begin{corollary}[Algebraic extensions of algebraic extensions]\label{C:feac}
  If $k\subset K\subset L$ is a tower of field extensions
  \[
  \begin{tikzcd}[column sep=0pt]
    L\ar[d,-,"\text{algebraic}",swap] & & L\ar[dd,-,"\text{algebraic}"]\\
    K \ar[d,-,"\text{algebraic}",swap] & \Rightarrow& \\
    k & \vphantom{K} & k
  \end{tikzcd}
  \]
  If $K$ is algebraic over $k$, and $L$ is algebraic over $K$, then $L$
  is algebraic over $k$.
  \begin{proof}
    Suppose that $\beta\in L$. Since $L$ is algebraic over $K$, there
    exists a polynomial $f(x)\in K[x]$ with $\beta$ as a root. Suppose that
    \[
    f(x) = \alpha_0 + \alpha_1 x + \cdots + \alpha_n x^n \in K[x].
    \]
    Since $K$ is algebraic over $k$, each $\alpha_i$ is algebraic over
    $k$. Let $g_i(x)\in k[x]$ be the minimal polynomial for each
    $\alpha_i$, and suppose that $\deg(g_i) = d_i$. Now, we can write
    a tower of field extensions:
    \[
    \begin{tikzcd}
      k(\alpha_0,\alpha_1,\dots,\alpha_n,\beta) \ar[d,-, "{[k(\alpha_0,\alpha_1,\dots, \alpha_n,\beta):k] = n}"]\\
      k(\alpha_0,\alpha_1,\dots,\alpha_n) \ar[d,-, "{[k(\alpha_0,\alpha_1,\dots, \alpha_n):k] \le d_n}"]\\
      \vdots  \ar[d,-]\\
      k(\alpha_0,\alpha_1) \ar[d,-, "{[k(\alpha_0,\alpha_1):k] \le d_1}"]\\
      k(\alpha_0) \ar[d,-, "{[k(\alpha_0):k] =d_0}"]\\
      k
    \end{tikzcd}
    \]
    Since the degrees in a tower of extension are multiplicative,
    Lemma~\ref{L:dm},
    \[
    [k(\beta):k]\le d_0\cdot d_1 \cdot \cdots \cdot d_n \cdot n.
    \]
    So we see that $k(\beta)\subseteq
    k(\alpha_0,\alpha_1,\dots,\alpha_n,\beta)$ is finite extension of
    $k$, hence algebraic over $k$ by Theorem~\ref{T:fea}. Since $\beta$
    is an arbitrary element of $L$, we see that every element of $L$
    is algebraic over $k$, and so $L$ is algebraic over $k$.
  \end{proof}
\end{corollary}





%% \begin{corollary}[Algebraic extensions are algebraic]
%%   Let $K$ be a field with $\alpha$ algebraic over $K$. In this case,
%%   $K(\alpha)$ is algebraic over $K$.
%%   \begin{sketch}
%%     The issue here is that we must show that every element of
%%     $K(\alpha)$ is algebraic over $K$. Use the corollary above.
%%   \end{sketch}
%% \end{corollary}



\begin{corollary}[Operations and algebraic elements]
  If $\alpha$ and $\beta$ are algebraic over $K$, then so are
  $\alpha+\beta$ and $\alpha\cdot \beta$.
  \begin{sketch}
    Show that $\alpha+\beta$ and $\alpha\cdot \beta$ are in
    $K(\alpha,\beta)$, and use Corollary~\ref{C:feac}.
  \end{sketch}
\end{corollary}




%% \begin{definition}
%%   Let $K$ be a field. An extension of the form $K(\alpha)$ is called a
%%   \dfn{simple extension} of $K$.
%% \end{definition}

%% \begin{lemma}[Algebraic extensions are simple]
%%   Let $K\subset \C$ and suppose that $\alpha,\beta\in \C$ are
%%   algebraic over $K$. Then there exists $\gamma\in\C$ such that
%%   $K(\alpha,\beta) = \K(\gamma)$.
%%   \begin{proof}
    
%%   \end{proof}
%% \end{lemma}




\end{document}
