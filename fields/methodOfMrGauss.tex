\documentclass{ximera}

\usepackage[T1]{fontenc}
\usepackage{stix2}
\usepackage{gillius}
\usepackage{resizegather}
%\usepackage{rsfso} fancy cal
\DeclareMathAlphabet{\mathcal}{OMS}{cmsy}{m}{n} %less fancy cal


\usepackage{multicol}


\usepackage{tikz-cd}
\usepackage{tkz-euclide} %% compass
\usetkzobj{all}  %% tkzCompass
\tikzset{>=stealth}
\tikzcdset{arrow style=tikz}
\usetikzlibrary{math} %% for assigning variables
%\usetikzlibrary{fadings}

\usepackage{colortbl,boldline,makecell} %% group tables


\usepackage[sans]{dsfont}

\usepackage{stmaryrd,pifont}

\graphicspath{
  {./}
  {fields/}
  }     



\let\oldbibliography\thebibliography%% to compact bib
\renewcommand{\thebibliography}[1]{%
  \oldbibliography{#1}%
  \setlength{\itemsep}{0pt}%
}
\renewcommand\refname{} %% no name needed!


\DefineVerbatimEnvironment{macaulay2}{Verbatim}{numbers=left,frame=lines,label=Macaulay2,labelposition=topline}

\DefineVerbatimEnvironment{gap}{Verbatim}{numbers=left,frame=lines,label=GAP,labelposition=topline}

%%% This next bit of code defines all our theorem environments
\makeatletter
\let\c@theorem\relax
\let\c@corollary\relax
%\let\c@example\relax
\makeatother

\let\definition\relax
\let\enddefinition\relax

\let\theorem\relax
\let\endtheorem\relax

\let\proposition\relax
\let\endproposition\relax

\let\exercise\relax
\let\endexercise\relax

\let\question\relax
\let\endquestion\relax

\let\remark\relax
\let\endremark\relax

\let\corollary\relax
\let\endcorollary\relax


\let\example\relax
\let\endexample\relax

\let\warning\relax
\let\endwarning\relax

\let\lemma\relax
\let\endlemma\relax


\let\algorithm\relax
\let\endalgorithm\relax
\usepackage{algpseudocode}

\newtheoremstyle{SlantTheorem}{\topsep}{\topsep}%%% space between body and thm
		{\slshape}                      %%% Thm body font
		{}                              %%% Indent amount (empty = no indent)
		{\bfseries\sffamily}            %%% Thm head font
		{}                              %%% Punctuation after thm head
		{3ex}                           %%% Space after thm head
		{\thmname{#1}\thmnumber{ #2}\thmnote{ \bfseries(#3)}}%%% Thm head spec
\theoremstyle{SlantTheorem}
\newtheorem{theorem}{Theorem}
%\newtheorem{definition}[theorem]{Definition}
%\newtheorem{proposition}[theorem]{Proposition}
%% \newtheorem*{dfnn}{Definition}
%% \newtheorem{ques}{Question}[theorem]
%% \newtheorem*{war}{WARNING}
%% \newtheorem*{cor}{Corollary}
%% \newtheorem*{eg}{Example}
\newtheorem*{remark}{Remark}
\newtheorem*{touchstone}{Touchstone}
\newtheorem{corollary}{Corollary}[theorem]
\newtheorem*{warning}{WARNING}
\newtheorem{example}[corollary]{Example}
\newtheorem{lemma}[theorem]{Lemma}




\newtheoremstyle{Definition}{\topsep}{\topsep}%%% space between body and thm
		{}                              %%% Thm body font
		{}                              %%% Indent amount (empty = no indent)
		{\bfseries\sffamily}            %%% Thm head font
		{}                              %%% Punctuation after thm head
		{3ex}                           %%% Space after thm head
		{\thmname{#1}\thmnumber{ #2}\thmnote{ \bfseries(#3)}}%%% Thm head spec
\theoremstyle{Definition}
\newtheorem{definition}[theorem]{Definition}



\let\algorithm\relax
\let\endalgorithm\relax
\newtheoremstyle{Alg}{\topsep}{\topsep}%%% space between body and thm
		{}                              %%% Thm body font
		{}                              %%% Indent amount (empty = no indent)
		{\bfseries\sffamily}            %%% Thm head font
		{}                              %%% Punctuation after thm head
		{3ex}                           %%% Space after thm head
		{\thmname{#1}\thmnumber{ #2}\thmnote{ \bfseries(#3)}}%%% Thm head spec
\theoremstyle{Alg}
\newtheorem*{algorithm}{Algorithm}
\newtheorem*{construction}{Construction}




\newtheoremstyle{Exercise}{\topsep}{\topsep} %%% space between body and thm
		{}                           %%% Thm body font
		{}                           %%% Indent amount (empty = no indent)
		{\bfseries\sffamily}         %%% Thm head font
		{)}                          %%% Punctuation after thm head
		{ }                          %%% Space after thm head
		{\thmnumber{#2}\thmnote{ \bfseries(#3)}}%%% Thm head spec
\theoremstyle{Exercise}
\newtheorem{exercise}[corollary]{}%[theorem]

%% \newtheoremstyle{Question}{\topsep}{\topsep} %%% space between body and thm
%% 		{\bfseries}                  %%% Thm body font
%% 		{3ex}                        %%% Indent amount (empty = no indent)
%% 		{}                           %%% Thm head font
%% 		{}                           %%% Punctuation after thm head
%% 		{}                           %%% Space after thm head
%% 		{\thmnumber{#2}\thmnote{ \bfseries(#3)}}%%% Thm head spec
\newtheoremstyle{Question}{3em}{3em} %%% space between body and thm
		{\large\bfseries}                           %%% Thm body font
		{}                           %%% Indent amount (empty = no indent)
		{}                         %%% Thm head font
		{}                          %%% Punctuation after thm head
		{0em}                          %%% Space after thm head
		{}%%% Thm head spec
\theoremstyle{Question}
\newtheorem*{question}{}






\renewcommand{\tilde}{\widetilde}
\renewcommand{\bar}{\overline}
\renewcommand{\hat}{\widehat}
\newcommand{\N}{\mathbb N}
\newcommand{\Z}{\mathbb Z}
\newcommand{\R}{\mathbb R}
\newcommand{\Q}{\mathbb Q}
\newcommand{\C}{\mathbb C}
\newcommand{\V}{\mathbb V}
\newcommand{\I}{\mathbb I}
\newcommand{\A}{\mathbb A}
\renewcommand{\o}{\mathbf o}
\newcommand{\iso}{\simeq}
\newcommand{\ph}{\varphi}
\newcommand{\Cf}{\mathcal{C}}
\newcommand{\IZ}{\mathrm{Int}(\Z)}
\newcommand{\dsum}{\oplus}
\newcommand{\directsum}{\bigoplus}
\newcommand{\union}{\bigcup}
\newcommand{\subgp}{\leq}
\newcommand{\normal}{\trianglelefteq}
\renewcommand{\i}{\mathfrak}
\renewcommand{\a}{\mathfrak{a}}
\renewcommand{\b}{\mathfrak{b}}
\newcommand{\m}{\mathfrak{m}}
\newcommand{\p}{\mathfrak{p}}
\newcommand{\q}{\mathfrak{q}}
\newcommand{\dfn}[1]{\textbf{#1}\index{#1}}
\let\hom\relax
\DeclareMathOperator{\mat}{Mat}
\DeclareMathOperator{\ann}{Ann}
\DeclareMathOperator{\h}{ht}
\DeclareMathOperator{\tr}{tr}
\DeclareMathOperator{\hom}{Hom}
\DeclareMathOperator{\Span}{Span}
\DeclareMathOperator{\spec}{Spec}
\DeclareMathOperator{\maxspec}{MaxSpec}
\DeclareMathOperator{\aut}{Aut}
\DeclareMathOperator{\ass}{Ass}
\DeclareMathOperator{\lcm}{lcm}
\DeclareMathOperator{\ff}{Frac}
\DeclareMathOperator{\im}{Im}
\DeclareMathOperator{\syz}{Syz}
\DeclareMathOperator{\gr}{Gr}
\DeclareMathOperator{\multideg}{multideg}
\renewcommand{\ker}{\mathop{\mathrm{Ker}}\nolimits}
\newcommand{\coker}{\mathop{\mathrm{Coker}}\nolimits}
\newcommand{\lps}{[\hspace{-0.25ex}[}
\newcommand{\rps}{]\hspace{-0.25ex}]}
\newcommand{\into}{\hookrightarrow}
\newcommand{\onto}{\twoheadrightarrow}
\newcommand{\tensor}{\otimes}
\newcommand{\x}{\mathbf{x}}
\newcommand{\X}{\mathbf X}
\newcommand{\Y}{\mathbf Y}
\renewcommand{\k}{\boldsymbol{\kappa}}
\renewcommand{\emptyset}{\varnothing}
\renewcommand{\qedsymbol}{$\blacksquare$}
\renewcommand{\l}{\ell}
\newcommand{\1}{\mathds{1}}
\newcommand{\lto}{\mathop{\longrightarrow\,}\limits}
\newcommand{\rad}{\sqrt}
\newcommand{\hf}{H}
\newcommand{\hs}{H\!S}
\newcommand{\hp}{H\!P}
\renewcommand{\vec}{\mathbf}
\let\temp\phi
\let\phi\varphi
\let\eulerphi\temp


\renewcommand{\epsilon}{\varepsilon}
\renewcommand{\subset}{\subseteq}
\renewcommand{\supset}{\supseteq}
\newcommand{\macaulay}{\normalfont\textsl{Macaulay2}}
\newcommand{\GAP}{\normalfont\textsf{GAP}}
\newcommand{\invlim}{\varprojlim}
\renewcommand{\le}{\leqslant}
\renewcommand{\ge}{\geqslant}
\newcommand{\valpha}{{\boldsymbol\alpha}}
\newcommand{\vbeta}{{\boldsymbol\beta}}
\newcommand{\vgamma}{{\boldsymbol\gamma}}
\newcommand{\dotp}{\bullet}
\newcommand{\lc}{\normalfont\textsc{lc}}
\newcommand{\lt}{\normalfont\textsc{lt}}
\newcommand{\lm}{\normalfont\textsc{lm}}
\newcommand{\from}{\leftarrow}
\newcommand{\transpose}{\intercal}
\newcommand{\grad}{\boldsymbol\nabla}
\newcommand{\curl}{\boldsymbol{\nabla\times}}
\renewcommand{\d}{\, d}
\newcommand{\<}{\langle}
\renewcommand{\>}{\rangle}

%\renewcommand{\proofname}{Sketch of Proof}


\renewenvironment{proof}[1][Proof]
  {\begin{trivlist}\item[\hskip \labelsep \itshape \bfseries #1{}\hspace{2ex}]\upshape}
{\qed\end{trivlist}}

\newenvironment{sketch}[1][Sketch of Proof]
  {\begin{trivlist}\item[\hskip \labelsep \itshape \bfseries #1{}\hspace{2ex}]\upshape}
{\qed\end{trivlist}}



\makeatletter
\renewcommand\section{\@startsection{paragraph}{10}{\z@}%
                                     {-3.25ex\@plus -1ex \@minus -.2ex}%
                                     {1.5ex \@plus .2ex}%
                                     {\normalfont\large\sffamily\bfseries}}
\renewcommand\subsection{\@startsection{subparagraph}{10}{\z@}%
                                    {3.25ex \@plus1ex \@minus.2ex}%
                                    {-1em}%
                                    {\normalfont\normalsize\sffamily\bfseries}}
\makeatother

%% Fix weird index/bib issue.
\makeatletter
\gdef\ttl@savemark{\sectionmark{}}
\makeatother


\makeatletter
%% no number for refs
\newcommand\frontstyle{%
  \def\activitystyle{activity-chapter}
  \def\maketitle{%
    \addtocounter{titlenumber}{1}%
                    {\flushleft\small\sffamily\bfseries\@pretitle\par\vspace{-1.5em}}%
                    {\flushleft\LARGE\sffamily\bfseries\@title \par }%
                    {\vskip .6em\noindent\textit\theabstract\setcounter{problem}{0}\setcounter{sectiontitlenumber}{0}}%
                    \par\vspace{2em}
                    \phantomsection\addcontentsline{toc}{section}{\textbf{\@title}}%
                  }}
\makeatother



\NewEnviron{annotate}{\vspace{-.3cm}\small \itshape \BODY \vspace{.3cm}}


%%%% TIKZ STUFF

%% N-GON code
\tikzset{
    pics/tikzngon/.style={
        code={
        \tikzmath{\xx = #1;\rr=1.7;}
        \draw[ultra thick,rounded corners=.05mm] ({\rr*sin(0*360/\xx)},{\rr*cos(0*360/\xx)})
        \foreach \x in {-1,0,...,\xx}
        {
        -- ({\rr*sin(\x*360/\xx)},{\rr*cos(\x*360/\xx)})
        }
           -- cycle;
  }}}

%% N-GON code (even)
\tikzset{
    pics/tikzegon/.style={
        code={
        \tikzmath{\xx = #1;\rr=1.7;}
        \draw[ultra thick,rounded corners=.05mm] ({\rr*sin(0*360/\xx+180/\xx)},{\rr*cos(0*360/\xx+180/\xx)})
        \foreach \x in {-1,0,...,\xx}
           {
           -- ({\rr*sin(\x*360/\xx+180/\xx)},{\rr*cos(\x*360/\xx+180/\xx)}) 
           }
           -- cycle;
  }}}




%% N-CLOCK code
\tikzset{
    pics/tikznclock/.style={
        code={
        \tikzmath{\xx = #1;\rr=1.7;\dd=.4;}
        \foreach \x in {1,...,\xx}
        \pgfmathtruncatemacro{\xy}{\x-1}
           {
             \node[circle,fill=black,inner sep=0pt, minimum size=13pt,text=white]
             at ({(\rr-\dd)*sin((\x-1)*360/(\xx)},{(\rr-\dd)*cos((\x-1)*360/\xx}) {\normalfont\bfseries\sffamily\small {\xy}};
           }
  \draw[thick] (0,0) circle (\rr);
  }}}



%% barcode from
%% https://tex.stackexchange.com/questions/6895/is-there-a-good-latex-package-for-generating-barcodes
%% NOT CURRENTLY USED!


\def\barcode#1#2#3#4#5#6#7{\begingroup%
  \dimen0=0.1em
  \def\stack##1##2{\oalign{##1\cr\hidewidth##2\hidewidth}}%
  \def\0##1{\kern##1\dimen0}%
  \def\1##1{\vrule height10ex width##1\dimen0}%
  \def\L##1{\ifcase##1\bc3211##1\or\bc2221##1\or\bc2122##1\or\bc1411##1%
    \or\bc1132##1\or\bc1231##1\or\bc1114##1\or\bc1312##1\or\bc1213##1%
    \or\bc3112##1\fi}%
  \def\R##1{\bgroup\let\next\1\let\1\0\let\0\next\L##1\egroup}%
  \def\G##1{\bgroup\let\bc\bcg\L##1\egroup}% reverse
  \def\bc##1##2##3##4##5{\stack{\0##1\1##2\0##3\1##4}##5}%
  \def\bcg##1##2##3##4##5{\stack{\0##4\1##3\0##2\1##1}##5}%
  \def\bcR##1##2##3##4##5##6{\R##1\R##2\R##3\R##4\R##5\R##6\11\01\11\09%
    \endgroup}%
  \stack{\09}#1\11\01\11\L#2%
  \ifcase#1\L#3\L#4\L#5\L#6\L#7\or\L#3\G#4\L#5\G#6\G#7%
    \or\L#3\G#4\G#5\L#6\G#7\or\L#3\G#4\G#5\G#6\L#7%
    \or\G#3\L#4\L#5\G#6\G#7\or\G#3\G#4\L#5\L#6\G#7%
    \or\G#3\G#4\G#5\L#6\L#7\or\G#3\L#4\G#5\L#6\G#7%
    \or\G#3\L#4\G#5\G#6\L#7\or\G#3\G#4\L#5\G#6\L#7%
  \fi\01\11\01\11\01\bcR}


\title{The method of Mr.\ Gauss}

\author{Bart Snapp}

\begin{document}
\begin{abstract}
  We prove that certain regular polygons are constructible.
\end{abstract}
\maketitle

When Gauss was around 16 years old, he put his massive brain to the
question of which regular $n$-gons are construcible with compass and
straightedge.


We too shall put our massive brains to this task. Key to understanding
this will be the maximum that we first encountered so long ago:
\begin{quote}
  To understand an object in mathematics, you should understand the
  functions on the object.
\end{quote}
The functions on a mathematical object are the automorphisms.  Recall
that $\aut(X)$\index{automorphism} is the set of bijective maps from
$X$ to $X$. When $G$ is a group, we \textit{insist} that $\aut(G)$ is
the set of group isomorphisms from $G$ to $G$. When $K$ is a field, we
insist that $\aut(K)$ we \textit{insist} that $\aut(K)$ is the set of
field isomorphisms from $K$ to $K$. In each case, above $\aut(X)$,
$\aut(G)$, and $\aut(K)$ form groups.



\begin{definition}\index{automorphism}\index{field automorphism}
  Let $K\subset L$ be fields and define the \dfn{Galois group}\textbf{
    of $\boldsymbol L$ over $\boldsymbol K$}, denoted
  \[
  \gal_K(L)
  \]
  to be the set of field automorphisms that fix elements of $K$. Hence
  $\gal_K(L)$ is a set of meaning bijective field isomorphims where if
  $a\in K$ and $\sigma \in \gal_K(L)$, then $\sigma(a) = a$.
\end{definition}

Galois groups are named after the mathematican \link[\'Evariste
  Galois]{https://en.wikipedia.org/wiki/Evariste\_Galois} (pronounced
\textit{gal-wah}). He lived a short and tragic life. He was shot dead
in a meaningless duel at the age of 20.

In retrospect, this was an act of suicide on Galois part. His work had
been overlooked multiple times, mostly by people who were careless,
ignorant, or just due to bad luck. Galois was lost in a world that did
not seem to value him.  It's sad that when we feel hopeless our brains
try to kill us with our thoughts and actions; especially when there
are \textbf{always other options.} 



\begin{exercise}
  Prove that $\gal_K(L)$ forms a group under composition.
\end{exercise}

%% \begin{example}[Galois group of a quadratic extension]
%%   If $\Q(\alpha)$ is a degree two extension of $\Q$, then
%%   \[
%%   \gal_\Q(\Q(\alpha)) \iso \Z_2^\times \iso {e}
%%   \]
%%   \begin{proof}
%%     We already know that $\Z_2^\times\iso {e}$. We will show that
%%     \[
%%     \gal_{\Q}(\Q(\alpha))\iso\Z_2^\times.
%%     \]
%%     Let
%%     \[
%%     ax^2 +bx + c
%%     \]
%%     be an irreducible quadratic polynomial, meaning $a \ne 0$, in
%%     $\Q[x]$. It's roots are
%%     \begin{align*}
%%     \alpha &= \frac{-b+\sqrt{b^2-4ac}}{2a} \\
%%     \beta &= \frac{-b-\sqrt{b^2-4ac}}{2a}.
%%     \end{align*}
%%     As an aside, note that
%%     \[
%%     \Q(\alpha) = \Q(\beta) 
%%     \]
%%     as
%%     \[
%%     \beta = \frac{-b-(b+2a\alpha)}{2}\in\Q(\alpha).
%%     \]
%%     Now by Lemma~\ref{L:fevs} and Theorem~\ref{T:dae}, $\Q(\alpha)$ is a
%%     $\Q$-vector space with basis
%%     \[
%%     \{1,\alpha\}.
%%     \]
%%     However, 
%%     \[
%%     \{\alpha,\alpha^2\}
%%     \]
%%     is also a basis, as $\dim_\Q(\Q(\alpha))=2$ and there are two
%%     linearly independent elements, behold
%%     \begin{align*}
%%       c_0 \alpha + c_1 \alpha^2 &= 0\\
%%       \alpha(c_0 + c_1 \alpha) &= 0.
%%     \end{align*}
%%     Since $\alpha\ne 0$, we see
%%     \[
%%     c_0 + c_1 \alpha = 0,
%%     \]
%%     and so $c_0 = c_1 = 0$.  This means that every element of
%%     $\Q(\alpha)$ can be expressed uniquely as
%%     \[
%%     c_0\alpha + c_1 \alpha^2 
%%     \]
%%     where $c_i\in\Q$. If $\sigma\in\gal_\Q(\Q(\alpha))$, then
%%     \begin{align*}
%%       \sigma(c_0 + c_1 \alpha) &= \sigma(c_0) + \sigma(c_1 \alpha)\\
%%       &= c_0 + c_1 \sigma(\alpha)\\
%%       &= c_0 + c_1 \sigma(\alpha).
%%     \end{align*}
%%     This means that an automorphisms of fields $\sigma \in
%%     \gal_\Q(\Q(\alpha))$ is uniquely determined by where it maps
%%     $\alpha$. In this case, there  are two choices
%%     \begin{align*}
%%       e :\alpha &\mapsto \alpha &  \sigma :\alpha &\mapsto \alpha^2\\
%%          \alpha^2 &\mapsto \alpha^2, &          \alpha^2 &\mapsto \alpha.
%%     \end{align*}
%%     Since $\gal_\Q(\Q(\alpha)) = \{e,\sigma\}$ as defined above, we
%%     see $\gal_\Q(\Q(\alpha))\iso\Z_3^\times$.
%%   \end{proof}
%% \end{example}
  
Finding Galois groups in general is hard, because they depend not only
on the degree of the minimal polynomial, but also on the explicit
coefficients of the polynomial. However, looking for Galois groups of
$n$th roots of unity is a much easier problem.

\begin{example}[$\boldsymbol{\gal_{\pmb \Q}\pmb(\pmb\Q\pmb(\zeta_3\pmb)\pmb)}$]
  Let $\zeta_3$ be a primitive third-root of
  unity. Now we have an isomorphism of groups:
  \[
  \gal_{\Q}(\Q(\zeta_3))\iso\Z_3^\times\iso \Z_2
  \]
  \begin{proof}
    We already know that $\Z_3^\times\iso \Z_2$ by
    Theorem~\ref{T:mgc}. We will show that
    \[
    \gal_{\Q}(\Q(\zeta_3))\iso\Z_3^\times.
    \]
    First note $\Q(\zeta_3)$ is an algebraic extension of degree $2$
    since $\zeta_3$ is a root of
    \[
    \Phi_3(x) = x^2 + x + 1. 
    \]
    When we substitute $x = y+ 1$, we find
    \[
    \Phi_3(x) = y^2 + 3y + 3
    \]
    which is irreducible by Eisenstein's criterion,
    Theorem~\ref{T:ec}. Thus $\Phi_3(x)$ is the minimal polynomial for
    $\zeta_3$. By Lemma~\ref{L:fevs} and Theorem~\ref{T:dae},
    $\Q(\zeta_3)$ is a $\Q$-vector space with basis
    \[
    \{1,\zeta_3\}.
    \]
    However, 
    \[
    \{\zeta_3,\zeta_3^2\}
    \]
    is also a basis, as $\dim_\Q(\Q(\zeta_3))=2$ and this is a minimal
    spanning set since
    \[
    \zeta_3 + \zeta_3^2 = -1.
    \]
    This means that every element of $\Q(\zeta_3)$ can be expressed
    uniquely as
    \[
    c_0\zeta_3 + c_1 \zeta_3^2
    \]
    where $c_i\in\Q$. If $\sigma\in\gal_\Q(\Q(\zeta_3))$, then
    \begin{align*}
      \sigma(c_0\zeta_3 + c_1 \zeta_3^2) &= \sigma(c_0\zeta_3) + \sigma(c_1 \zeta_3^2)\\
      &= c_0 \sigma(\zeta_3) + c_1 \sigma(\zeta_3^2)\\
      &= c_0 \sigma(\zeta_3) + c_1 \sigma(\zeta_3)^2.
    \end{align*}
    This means that an automorphisms of fields $\sigma \in
    \gal_\Q(\Q(\zeta_3))$ is uniquely determined by where it maps
    $\zeta_3$. In this case, there are two choices
    \begin{align*}
      e :\zeta_3 &\mapsto \zeta_3 &  \sigma :\zeta_3 &\mapsto \zeta_3^2\\
         \zeta_3^2 &\mapsto \zeta_3^2, &          \zeta_3^2 &\mapsto \zeta_3.
    \end{align*}
    Since $\gal_\Q(\Q(\zeta_3)) = \{e,\sigma\}$ as defined above, we
    see $\gal_\Q(\Q(\zeta_3))\iso\Z_3^\times$.
  \end{proof}
\end{example}


\begin{example}[$\boldsymbol{\gal_{\pmb\Q}\pmb(\pmb\Q\pmb(\zeta_5\pmb)\pmb)}$]
  Let $\zeta_5$ be a primitive fifth-root of
  unity. Now we have an isomorphism of groups:
  \[
  \gal_{\Q}(\Q(\zeta_5))\iso\Z_5^\times\iso \Z_4
  \]
  \begin{proof}
    We already know that $\Z_5^\times\iso \Z_4$ by
    Theorem~\ref{T:mgc}. We will show that
    \[
    \gal_{\Q}(\Q(\zeta_5))\iso\Z_5^\times.
    \]
    First note $\Q(\zeta_5)$ is an algebraic extension of degree $4$
    since the minimal polynomial for $\zeta_5$ is
    \[
    \Phi_5(x) = x^4+ +x^3+x^2 + x + 1.
    \]
    When we substitute $x = y+ 1$, we find
    \[
    \Phi_5(x) = y^{4} +5y^3+10y^2+10y+5
    \]
    which is irreducible by Eisenstein's criterion,
    Theorem~\ref{T:ec}. Thus $\Phi_5(x)$ is the minimal polynomial for
    $\zeta_5$.


    By Lemma~\ref{L:fevs} and Theorem~\ref{T:dae},
    $\Q(\zeta_5)$ is a $\Q$-vector space with basis
    \[
    \{1,\zeta_5,\zeta_5^2,\zeta_5^3\}.
    \]
    However,
    \[
    \{\zeta_5,\zeta_5^2,\zeta_5^3,\zeta_5^4\}
    \]
    is also a basis, as $\dim_\Q(\Q(\zeta_5))=4$ and this is a minimal
    spanning set since
    \[
    \zeta_5 + \zeta_5^2  + \zeta_5^3 + \zeta_5^4 = -1.
    \]
    This means that every element of $\Q(\zeta_5)$ can be expressed
    uniquely as
    \[
    c_0\zeta_5 + c_1\zeta_5^2  + c_2\zeta_5^3 + c_3\zeta_5^4 
    \]
    where $c_i\in\Q$. If $\sigma\in\gal_\Q(\Q(\zeta_5))$, then
    \begin{align*}
      \sigma(c_0\zeta_5 + c_1\zeta_5^2  + c_2\zeta_5^3 + c_3\zeta_5^4) &= \sigma(c_0\zeta_5) + \sigma(c_1 \zeta_5^2)+ \sigma(c_2 \zeta_5^3)+ \sigma(c_3 \zeta_5^4)\\
      &= c_0 \sigma(\zeta_5) + c_1 \sigma(\zeta_5^2) + c_2 \sigma(\zeta_5^3) + c_3 \sigma(\zeta_5^4)\\
      &= c_0 \sigma(\zeta_5) + c_1 \sigma(\zeta_5)^2 + c_2 \sigma(\zeta_5)^3 + c_3 \sigma(\zeta_5)^4.
    \end{align*}
    This means that an automorphisms of fields $\sigma \in
    \gal_\Q(\Q(\zeta_5))$ is uniquely determined by where it maps
    $\zeta_5$. In this case, there are four choices
    \begin{align*}
      e :\zeta_5 &\mapsto \zeta_5 &  \sigma :\zeta_5 &\mapsto \zeta_5^2\\
      \sigma^2 :\zeta_5 &\mapsto \zeta_5^4 &  \sigma^3 :\zeta_5 &\mapsto \zeta_5^3.
    \end{align*}
    Since
    \[
    \gal_\Q(\Q(\zeta_5)) = \{e,\sigma,\sigma^2,\sigma^3,\sigma^4\}
    \]
    as defined above, we see $\gal_\Q(\Q(\zeta_5))\iso\Z_5^\times$.
  \end{proof}
\end{example}

\begin{exercise}
  Let $p$ be a prime number. Prove that $\gal_\Q(\Q(\zeta_p))\iso
  \Z_p^\times$.
\end{exercise}

\begin{exercise}\index{automorphism}
  If $G$ is a group, define $\aut(G)$ to be the set of group
  automorphisms (bijective group homomorphisms from $G$ to
  itself). Prove that
  \[
  \aut(\Z_p) \iso \Z_p^\times.
  \]
\end{exercise}



Now it is time for a dirty lemma that we will use over and over again.


\begin{lemma}[Fixed elements are in the ground field]\label{L:fgf}
  Suppose $K\subset L$ is an algebraic extension of degree $n$
  with a basis
  \[
  \{\beta_1,\beta_2,\beta_3,\cdots,\beta_{n-1},\beta_n\}
  \]
  for $L$ as a $K$-vector space and
  \[
  \beta_1 + \beta_2 + \dots + \beta_n \in K.
  \]
  Additionally, suppose that $\gal_{K}(L)$ is cyclic on
  $\{\beta_1,\dots,\beta_n\}$, and that $\sigma\in\gal_{K}(L)$ is a
  generator. If $\alpha\in L$ with $\sigma(\alpha) = \alpha$, then
  $\alpha\in K$.
  \begin{proof}
    Since
    \[
    \{\beta_1,\beta_2,\beta_3,\cdots,\beta_{n-1},\beta_n\}
    \]
    is a basis for $K(\beta)$ as a $K$-vector space, we may  write
    \[
    \alpha = c_1 \beta_1 + c_2 \beta_2 + \cdots + c_n\beta_{n},
    \]
    where each $c_i\in K$. Applying $\sigma$ we find
    \[
    \sigma(\alpha) = c_1 \sigma(\beta) + c_2 \sigma(\beta_2) + \cdots + c_n\sigma(\beta_{n-1}).
    \]
    However, since the $\beta_i$ form a basis and $\sigma$ is a
    generator of the cyclic group $\gal_{K}(L)$, $\sigma$
    permutes the coefficients full-cycle, and hence
    \[
    c_1 = c_2 = \cdots = c_{n}\in K.
    \]
    Setting $c= c_1$, write
    \[
    \alpha = c \cdot (\beta_1 + \beta_2 + \cdots + \beta_{n}) \in K.
    \]
    This completes the proof.
  \end{proof}
\end{lemma}





\begin{example}[Regular $\boldsymbol{5}$-gon is constructible]
  We start with the polynomial
  \begin{align*}
    x^{5} -1 &= (x-1) \Phi_{5}(x)\\
    &= (x-1)(x^{4} + x^{3} + x^2 + x+1).
  \end{align*}
  By substituting $x=y+1$ into $x^{4} + x^3 +x^2 + x+1$, we find
  \[
  y^{4} +5y^3+10y^2+10y+5
  \]
  Since this polynomial satisfies Eisenstein's criterion,
  Theorem~\ref{T:ec}, we see $\Phi_5(x)$ is irreducible in $\Z[x]$. By
  Gauss' lemma, Lemma~\ref{L:G}, we see $\Phi_5(x)$ is irreducible in
  $\Q[x]$. Thus by Theorem~\ref{T:dae}, 
  \[
  \begin{tikzcd}
    \Q(\zeta_{5})\\
    \Q\ar[u, -, "4"]
  \end{tikzcd}
  \]
  hence we see that the regular $5$-gon may be construtible, but we
  must show that we can find a tower of degree two extensions,
  starting with $\Q$ and finishing with $\Q(\zeta_{5})$.  Consider
  $\gal_{\Q}(\Q(\zeta_{5}))$. In this case
  \[
  \gal_{\Q}(\Q(\zeta_{5}))\iso \Z_{5}^\times
  \]
  and is cyclic. There is no easy method other than trial-and-error to
  find the generator of this group, nevertheless we, like the heros
  before us, persist and find a generator
  \[
  \sigma(\zeta_{5}) = \zeta_{5}^2.
  \]
  Writing this map as the cycle on exponents, we have
  \begin{align*}
    \sigma &= (1\ 2\ 4\ 3),\\
    \sigma^2 &= (1\ 4)(2\ 3).
  \end{align*}
  Starting with $\sigma^2$, set
  \begin{align*}
    \alpha &= \zeta_{5}   + \zeta_{5}^4,\\
    \beta &= \zeta_{5}^2 + \zeta_{5}^{3}.
  \end{align*}
  The elements $\{\alpha,\beta\}$ form a basis for $\Q[\alpha,\beta]$ over
  $\Q$, as $\alpha$ and $\beta$ are linearly independent and they span
  $\Q[\alpha,\beta]$. In this case, $\sigma(\alpha) = \beta$ and
  $\sigma(\beta) = \alpha$. This means
  \begin{align*}
    \sigma(\alpha+\beta) &= \alpha + \beta,\\
    \sigma(\alpha\beta) &= \alpha\beta,
  \end{align*}
  which by Lemma~\ref{L:fgf} shows that
  $\alpha+\beta,\alpha\beta\in\Q$. However, $\alpha$ and $\beta$ are
  both roots of the polynomial
  \[
  x^2 - (\alpha +\beta) x + \alpha\beta\in\Q[x]
  \]
  which is irreducible in $\Q[x]$, since neither $\alpha$ nor $\beta$
  are rational. Hence $\Q[\alpha,\beta]= \Q(\alpha)$ and
  $[\Q(\alpha):\Q] = 2$:
  \[
  \begin{tikzcd}
    \Q(\alpha) \\ \Q \ar[u,-,"2"]
  \end{tikzcd}
  \]
  Since we know that $[\Q(\zeta_{5}):\Q] = 4$, and
  \[
  \Q(\alpha)\subset \Q(\zeta_{5})
  \]
  with $[\Q(\alpha):\Q]=2$, by Lemma~\ref{L:dm},
  \[
    [\Q(\zeta_{5}):\Q(\alpha)]=2.
  \]
  Since we have a tower of field extensions, each of degree two over
  the previous field, we see that the $5$-gon is constructible by
  compass and straightedge.
\end{example}

\begin{example}[Regular $\boldsymbol{17}$-gon is constructible]
  We start with the polynomial
  \begin{align*}
    x^{17} -1 &= (x-1) \Phi_{17}(x)\\
    &= (x-1)(x^{16} + x^{15} + \cdots + x+1).
  \end{align*}
  By substituting $x=y+1$ into $x^{16} + x^{15} + \cdots + x+1$, we find
  \begin{align*}
  y^{16} &+17 y^{15}+136 y^{14}+680 y^{13}+2380 y^{12} \\
  &+6188 y^{11}+12376 y^{10}+19448 y^9+24310 y^8+24310 y^7\\
  &+19448 y^6+12376 y^5+6188 y^4+2380 y^3+680 y^2+136 y+17.
  \end{align*}
  Since this polynomial satisfies Eisenstein's criterion,
  Theorem~\ref{T:ec}, we see $\Phi_{17}(x)$ is irreducible in
  $\Z[x]$. By Gauss' lemma, Lemma~\ref{L:G}, we see that $\Phi_{17}(x)$ is irreducible in
  $\Q[x]$. Thus by Theorem~\ref{T:dae}, 
  \[
  \begin{tikzcd}
    \Q(\zeta_{17})\\
    \Q\ar[u, -, "16"]
  \end{tikzcd}
  \]
  hence we see that the regular $17$-gon may be construtible, but we
  must show that we can find a tower of degree two extensions,
  starting with $\Q$ and finishing with $\Q(\zeta_{17})$.  Consider
  $\gal_{\Q}(\Q(\zeta_{17}))$. In this case
  \[
  \gal_{\Q}(\Q(\zeta_{17}))\iso \Z_{17}^\times
  \]
  and is cyclic. There is no easy method other than trial-and-error to
  find the generator of this group, nevertheless we, like the heros
  before us, persist and find a generator
  \[
  \sigma(\zeta_{17}) = \zeta_{17}^3.
  \]
  With this generator, we can produce subgroups of
  $\gal_\Q(\Q(\zeta_{17})) = \< \sigma \>$. We'll present it as a
  lattice:
  \[
    \begin{tikzcd}
    \Z_{16}\iso\gal_\Q(\Q(\zeta_{17})) = \< \sigma \>\ar[d,-]\\
    \< \sigma^2\> \ar[d,-]\\
    \<\sigma^4\> \ar[d,-]\\
    \<\sigma^8\> \ar[d,-]\\
    \<\sigma^{16}=e\> 
  \end{tikzcd}
  \]
  Let's examine $\<\sigma^2\>\subgp \gal_\Q(\Q(\zeta_{17}))$,
  \[
  \<\sigma^2\> = \{ \sigma^2, \sigma^4,\sigma^6,\sigma^8,\sigma^{10},\sigma^{12},\sigma^{14},\sigma^{16} = e\}.
  \]
  Each of these maps permute powers of $\zeta_{17}$ and fix elements
  of $\Q$. Quite explicitly,
  \begin{align*}
    \sigma^2: \zeta_{17} &\mapsto \zeta_{17}^9    & \sigma^{10}: \zeta_{17} &\mapsto \zeta_{17}^{8}\\
    \sigma^4: \zeta_{17} &\mapsto \zeta_{17}^{13} & \sigma^{12}: \zeta_{17} &\mapsto \zeta_{17}^{4}\\
    \sigma^6: \zeta_{17} &\mapsto \zeta_{17}^{15} & \sigma^{14}: \zeta_{17} &\mapsto \zeta_{17}^{2}\\
    \sigma^8: \zeta_{17} &\mapsto \zeta_{17}^{16} & \sigma^{16}: \zeta_{17} &\mapsto \zeta_{17}.
  \end{align*}
  There are eight field isomorphisms above, and hence there are eight
  more field isomorphisms in $\gal_\Q(\Q(\zeta_{17})) - \<
  \sigma^2\>$, and these are all different since the cosets of a
  subgroup partition the group, Theorem~\ref{T:CPG}. Thus if we set 
  \begin{align*}
    \alpha &= \zeta_{17}   + \zeta_{17}^9 + \zeta_{17}^{13} + \zeta_{17}^{15} + \zeta_{17}^{16} + \zeta_{17}^{8} + \zeta_{17}^{4} + \zeta_{17}^{2},\\
    \beta &= \zeta_{17}^3 + \zeta_{17}^{10} + \zeta_{17}^{5} + \zeta_{17}^{11} + \zeta_{17}^{14} + \zeta_{17}^{7} + \zeta_{17}^{12} + \zeta_{17}^{6},
  \end{align*} 
  the elements $\{\alpha,\beta\}$ form a basis for $\Q[\alpha,\beta]$
  over $\Q$, as $\alpha$ and $\beta$ are linearly independent and they
  span $\Q[\alpha,\beta]$. Moreover, by construction, $\sigma(\alpha)
  = \beta$ and $\sigma(\beta) = \alpha$. This means
  \begin{align*}
    \sigma(\alpha+\beta) &= \alpha + \beta,\\
    \sigma(\alpha\beta) &= \alpha\beta,
  \end{align*}
  which by Lemma~\ref{L:fgf} shows that
  $\alpha+\beta,\alpha\beta\in\Q$. However, $\alpha$ and $\beta$ are
  both roots of the polynomial
  \[
  x^2 - (\alpha +\beta) x + \alpha\beta\in\Q[x]
  \]
  which is irreducible in $\Q[x]$, since neither $\alpha$ nor $\beta$
  are rational. Hence $\Q[\alpha,\beta]= \Q(\alpha)$ and
  $[\Q(\alpha):\Q] = 2$:
  \[
  \begin{tikzcd}
    \Q(\alpha) \\ \Q \ar[u,-,"2"]
  \end{tikzcd}
  \]
  Now look at $\<\sigma^4\>\subgp \< \sigma^2\>$,
  \[
  \<\sigma^4\> = \{ \sigma^4, \sigma^8,\sigma^12,\sigma^{14},\sigma^{16} = e\}.
  \]
  Let's look at these field isomorphisms:
  \begin{align*}
    \sigma^4: \zeta_{17} &\mapsto \zeta_{17}^{13}    & \sigma^{12}: \zeta_{17} &\mapsto \zeta_{17}^{4}\\
    \sigma^8: \zeta_{17} &\mapsto \zeta_{17}^{16} & \sigma^{16}: \zeta_{17} &\mapsto \zeta_{17}.
  \end{align*}
  There are four field isomorphisms above, and hence there are four
  other field isomorphisms in $\<\sigma^2\> - \< \sigma^4\>$. Since 
  these are all different since the cosets of a subgroup partition the
  group, Theorem~\ref{T:CPG}. Thus if we set
  \begin{align*}
    \gamma &= \zeta_{17}   + \zeta_{17}^{13} + \zeta_{17}^{16} + \zeta_{17}^{4},\\
    \delta &= \zeta_{17}^9 + \zeta_{17}^{15} + \zeta_{17}^{8} + \zeta_{17}^{2},
  \end{align*}
  the elements $\{\gamma,\delta\}$ form a basis for
  $\Q[\alpha,\gamma,\delta]$ over $\Q(\alpha)$ as $\gamma$ and
  $\delta$ are linearly independent, and they span
  $\Q[\alpha,\gamma,\delta]$.  In this case, $\sigma^2(\gamma)= \delta$ and
  $\sigma^2(\delta) = \gamma$. This means
  \begin{align*}
    \sigma^2(\gamma+\delta) &= \gamma + \delta,\\
    \sigma^2(\gamma\delta) &= \gamma\delta,
  \end{align*}
  which by Lemma~\ref{L:fgf} shows that
  $\gamma+\delta,\gamma\delta\in\Q(\alpha)$. Again, both $\gamma$ and
  $\delta$ are roots of the polynomial
  \[
  x^2 - (\gamma +\delta) x + \gamma\delta\in\Q(\alpha)[x]
  \]
  which is irreducible in $\Q(\alpha)[x]$, Hence
  $\Q[\alpha,\gamma,\delta] = \Q(\alpha,\gamma)$ and
  $[\Q(\alpha,\gamma):\Q(\alpha)] = 2$:
  \[
  \begin{tikzcd}
    \Q(\alpha,\gamma)\\
    \Q(\alpha) \ar[u,-,"2"]\\
    \Q \ar[u,-,"2"]
  \end{tikzcd}
  \]
  Finally look at $\<\sigma^8\>\subgp \<\sigma^4\>$,
  \[
  \<\sigma^8\> = \{ \sigma^8,\sigma^{16} = e\}.
  \]
  There are two field isomorphisms above, and hence there are two
  other field isomorphisms in $\<\sigma^4\> - \< \sigma^8\>$, and
  these are all different since the cosets of a subgroup partition the
  group, Theorem~\ref{T:CPG}. Thus if we set
  \begin{align*}
    \epsilon &= \zeta_{17} + \zeta_{17}^{16},\\
    \eta &= \zeta_{17}^{13} + \zeta_{17}^4.
  \end{align*}
  The elements $\{\epsilon,\eta\}$ form a basis for
  $\Q[\alpha,\gamma,\epsilon,\eta]$ over $\Q(\alpha,\gamma)$ as
  $\epsilon$ and $\eta$ are linearly independent, and they span
  $\Q[\alpha,\gamma,\epsilon,\eta]$.  In this case,
  $\sigma^4(\epsilon) = \eta$ and $\sigma^4(\eta) = \epsilon$. This
  means that
  \begin{align*}
    \sigma^4(\epsilon+\eta) &= \epsilon + \eta\\
    \sigma^4(\epsilon\eta) &= \epsilon\eta,
  \end{align*}
  which by Lemma~\ref{L:fgf} shows that
  $\epsilon+\eta,\epsilon\eta\in\Q(\alpha,\gamma)$. Again, both
  $\epsilon$ and $\eta$ are roots of the polynomial
  \[
  x^2 - (\epsilon +\eta) x + \epsilon\eta\in\Q(\alpha,\gamma)[x]
  \]
  which is irreducible in $\Q(\alpha,\gamma)[x]$, Hence
  $\Q[\alpha,\gamma,\epsilon,\eta]= \Q(\alpha,\gamma,\epsilon)$ and
  $[\Q(\alpha,\gamma,\epsilon):\Q(\alpha,\gamma)] = 2$:
  \[
  \begin{tikzcd}
    \Q(\alpha,\gamma,\epsilon)\\
    \Q(\alpha,\gamma)\ar[u,-,"2"]\\
    \Q(\alpha) \ar[u,-,"2"]\\
    \Q \ar[u,-,"2"]
  \end{tikzcd}
  \]
  Since we know that $[\Q(\zeta_{17}):\Q] = 16$, and
  \[
  \Q(\alpha,\gamma,\epsilon)\subset \Q(\zeta_{17})
  \]
  with $[\Q(\alpha,\gamma,\epsilon):\Q]=8$, by Lemma~\ref{L:dm},
  \[
    [\Q(\zeta_{17}):\Q(\alpha,\gamma,\epsilon)]=2.
  \]
  Since we have a tower of field extensions, each of degree two over
  the previous field, we see that the $17$-gon is constructible by
  compass and straightedge.
\end{example}



\begin{theorem}[Constructible regular $\boldsymbol n$-gons]
  \begin{proof}
    Suppose you wish to construct a regular $p$-gon where $p$ is a
    prime number. This is equivalent to constructing the roots of
    $x^p-1$. Write:
    \begin{align*}
      x^{p} -1 &= (x-1) \Phi_{p}(x)\\
      &= (x-1)(x^{p-1} + x^{p-2} + \cdots + x+1).
    \end{align*}
    The polynomail $\Phi_{p}(x)$ can be shown to be irrducible by
    setting $x = y+1$ and using Eisenstein's criterion,
    Theorem~\ref{T:ec}, we see $\Phi_{p}(x)$ is irreducible in
    $\Z[x]$. By Gauss' lemma, Lemma~\ref{L:G}, see $\Phi_{p}(x)$ is
    irreducible in $\Q[x]$. Thus by Theorem~\ref{T:dae}, 
    \[
    \begin{tikzcd}
      \Q(\zeta_{p})\\
      \Q\ar[u, -, "p-1"]
    \end{tikzcd}
    \]
    Hence the regular $p$-gon is constructible only if $p-1 = 2^n$
    where $n\in \N$, by Theorem~\ref{T:dc}. However, this is not
    sufficient for a regular $p$-gon to be constructible. We must show
    that there is a tower of field extensions 
    \[
    \begin{tikzcd}
      K_n=\Q(\zeta_{p})\\
      K_{n-1}\ar[u, -, "2"]\\
      \vdots\ar[u, -, "2"]\\
      K_1\ar[u, -, "2"]\\
      K_0=\Q\ar[u, -, "2"]
    \end{tikzcd}
    \]
    where at each step, $[K_i:K_{i+1}]=2$. To do this, we will look at
    $\gal_\Q(\Q(\zeta_p))$. In this case
    \[
    \gal_\Q(\Q(\zeta_p))\iso\Z_p^\times \iso \Z_{p-1} = \Z_{2^n}
    \]
    with the right-most isomorphism holding by
    Corollary~\ref{C:mzpc}. Let $\sigma$ be a generator for the cyclic
    group $\gal_\Q(\Q(\zeta_p))$.

    HEREHERE


    Since $\Z_{2^n}$ has subgroups for each
    power of $2$ less than $2^n$, so does $\gal_\Q(\Q(\zeta_p))$. 
  \end{proof}
\end{theorem}

\end{document}
