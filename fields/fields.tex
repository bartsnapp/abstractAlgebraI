\documentclass{ximera}

\usepackage[T1]{fontenc}
\usepackage{stix2}
\usepackage{gillius}
\usepackage{resizegather}
%\usepackage{rsfso} fancy cal
\DeclareMathAlphabet{\mathcal}{OMS}{cmsy}{m}{n} %less fancy cal


\usepackage{multicol}


\usepackage{tikz-cd}
\usepackage{tkz-euclide} %% compass
\usetkzobj{all}  %% tkzCompass
\tikzset{>=stealth}
\tikzcdset{arrow style=tikz}
\usetikzlibrary{math} %% for assigning variables
%\usetikzlibrary{fadings}

\usepackage{colortbl,boldline,makecell} %% group tables


\usepackage[sans]{dsfont}

\usepackage{stmaryrd,pifont}

\graphicspath{
  {./}
  {fields/}
  }     



\let\oldbibliography\thebibliography%% to compact bib
\renewcommand{\thebibliography}[1]{%
  \oldbibliography{#1}%
  \setlength{\itemsep}{0pt}%
}
\renewcommand\refname{} %% no name needed!


\DefineVerbatimEnvironment{macaulay2}{Verbatim}{numbers=left,frame=lines,label=Macaulay2,labelposition=topline}

\DefineVerbatimEnvironment{gap}{Verbatim}{numbers=left,frame=lines,label=GAP,labelposition=topline}

%%% This next bit of code defines all our theorem environments
\makeatletter
\let\c@theorem\relax
\let\c@corollary\relax
%\let\c@example\relax
\makeatother

\let\definition\relax
\let\enddefinition\relax

\let\theorem\relax
\let\endtheorem\relax

\let\proposition\relax
\let\endproposition\relax

\let\exercise\relax
\let\endexercise\relax

\let\question\relax
\let\endquestion\relax

\let\remark\relax
\let\endremark\relax

\let\corollary\relax
\let\endcorollary\relax


\let\example\relax
\let\endexample\relax

\let\warning\relax
\let\endwarning\relax

\let\lemma\relax
\let\endlemma\relax


\let\algorithm\relax
\let\endalgorithm\relax
\usepackage{algpseudocode}

\newtheoremstyle{SlantTheorem}{\topsep}{\topsep}%%% space between body and thm
		{\slshape}                      %%% Thm body font
		{}                              %%% Indent amount (empty = no indent)
		{\bfseries\sffamily}            %%% Thm head font
		{}                              %%% Punctuation after thm head
		{3ex}                           %%% Space after thm head
		{\thmname{#1}\thmnumber{ #2}\thmnote{ \bfseries(#3)}}%%% Thm head spec
\theoremstyle{SlantTheorem}
\newtheorem{theorem}{Theorem}
%\newtheorem{definition}[theorem]{Definition}
%\newtheorem{proposition}[theorem]{Proposition}
%% \newtheorem*{dfnn}{Definition}
%% \newtheorem{ques}{Question}[theorem]
%% \newtheorem*{war}{WARNING}
%% \newtheorem*{cor}{Corollary}
%% \newtheorem*{eg}{Example}
\newtheorem*{remark}{Remark}
\newtheorem*{touchstone}{Touchstone}
\newtheorem{corollary}{Corollary}[theorem]
\newtheorem*{warning}{WARNING}
\newtheorem{example}[corollary]{Example}
\newtheorem{lemma}[theorem]{Lemma}




\newtheoremstyle{Definition}{\topsep}{\topsep}%%% space between body and thm
		{}                              %%% Thm body font
		{}                              %%% Indent amount (empty = no indent)
		{\bfseries\sffamily}            %%% Thm head font
		{}                              %%% Punctuation after thm head
		{3ex}                           %%% Space after thm head
		{\thmname{#1}\thmnumber{ #2}\thmnote{ \bfseries(#3)}}%%% Thm head spec
\theoremstyle{Definition}
\newtheorem{definition}[theorem]{Definition}



\let\algorithm\relax
\let\endalgorithm\relax
\newtheoremstyle{Alg}{\topsep}{\topsep}%%% space between body and thm
		{}                              %%% Thm body font
		{}                              %%% Indent amount (empty = no indent)
		{\bfseries\sffamily}            %%% Thm head font
		{}                              %%% Punctuation after thm head
		{3ex}                           %%% Space after thm head
		{\thmname{#1}\thmnumber{ #2}\thmnote{ \bfseries(#3)}}%%% Thm head spec
\theoremstyle{Alg}
\newtheorem*{algorithm}{Algorithm}
\newtheorem*{construction}{Construction}




\newtheoremstyle{Exercise}{\topsep}{\topsep} %%% space between body and thm
		{}                           %%% Thm body font
		{}                           %%% Indent amount (empty = no indent)
		{\bfseries\sffamily}         %%% Thm head font
		{)}                          %%% Punctuation after thm head
		{ }                          %%% Space after thm head
		{\thmnumber{#2}\thmnote{ \bfseries(#3)}}%%% Thm head spec
\theoremstyle{Exercise}
\newtheorem{exercise}[corollary]{}%[theorem]

%% \newtheoremstyle{Question}{\topsep}{\topsep} %%% space between body and thm
%% 		{\bfseries}                  %%% Thm body font
%% 		{3ex}                        %%% Indent amount (empty = no indent)
%% 		{}                           %%% Thm head font
%% 		{}                           %%% Punctuation after thm head
%% 		{}                           %%% Space after thm head
%% 		{\thmnumber{#2}\thmnote{ \bfseries(#3)}}%%% Thm head spec
\newtheoremstyle{Question}{3em}{3em} %%% space between body and thm
		{\large\bfseries}                           %%% Thm body font
		{}                           %%% Indent amount (empty = no indent)
		{}                         %%% Thm head font
		{}                          %%% Punctuation after thm head
		{0em}                          %%% Space after thm head
		{}%%% Thm head spec
\theoremstyle{Question}
\newtheorem*{question}{}






\renewcommand{\tilde}{\widetilde}
\renewcommand{\bar}{\overline}
\renewcommand{\hat}{\widehat}
\newcommand{\N}{\mathbb N}
\newcommand{\Z}{\mathbb Z}
\newcommand{\R}{\mathbb R}
\newcommand{\Q}{\mathbb Q}
\newcommand{\C}{\mathbb C}
\newcommand{\V}{\mathbb V}
\newcommand{\I}{\mathbb I}
\newcommand{\A}{\mathbb A}
\renewcommand{\o}{\mathbf o}
\newcommand{\iso}{\simeq}
\newcommand{\ph}{\varphi}
\newcommand{\Cf}{\mathcal{C}}
\newcommand{\IZ}{\mathrm{Int}(\Z)}
\newcommand{\dsum}{\oplus}
\newcommand{\directsum}{\bigoplus}
\newcommand{\union}{\bigcup}
\newcommand{\subgp}{\leq}
\newcommand{\normal}{\trianglelefteq}
\renewcommand{\i}{\mathfrak}
\renewcommand{\a}{\mathfrak{a}}
\renewcommand{\b}{\mathfrak{b}}
\newcommand{\m}{\mathfrak{m}}
\newcommand{\p}{\mathfrak{p}}
\newcommand{\q}{\mathfrak{q}}
\newcommand{\dfn}[1]{\textbf{#1}\index{#1}}
\let\hom\relax
\DeclareMathOperator{\mat}{Mat}
\DeclareMathOperator{\ann}{Ann}
\DeclareMathOperator{\h}{ht}
\DeclareMathOperator{\tr}{tr}
\DeclareMathOperator{\hom}{Hom}
\DeclareMathOperator{\Span}{Span}
\DeclareMathOperator{\spec}{Spec}
\DeclareMathOperator{\maxspec}{MaxSpec}
\DeclareMathOperator{\aut}{Aut}
\DeclareMathOperator{\ass}{Ass}
\DeclareMathOperator{\lcm}{lcm}
\DeclareMathOperator{\ff}{Frac}
\DeclareMathOperator{\im}{Im}
\DeclareMathOperator{\syz}{Syz}
\DeclareMathOperator{\gr}{Gr}
\DeclareMathOperator{\multideg}{multideg}
\renewcommand{\ker}{\mathop{\mathrm{Ker}}\nolimits}
\newcommand{\coker}{\mathop{\mathrm{Coker}}\nolimits}
\newcommand{\lps}{[\hspace{-0.25ex}[}
\newcommand{\rps}{]\hspace{-0.25ex}]}
\newcommand{\into}{\hookrightarrow}
\newcommand{\onto}{\twoheadrightarrow}
\newcommand{\tensor}{\otimes}
\newcommand{\x}{\mathbf{x}}
\newcommand{\X}{\mathbf X}
\newcommand{\Y}{\mathbf Y}
\renewcommand{\k}{\boldsymbol{\kappa}}
\renewcommand{\emptyset}{\varnothing}
\renewcommand{\qedsymbol}{$\blacksquare$}
\renewcommand{\l}{\ell}
\newcommand{\1}{\mathds{1}}
\newcommand{\lto}{\mathop{\longrightarrow\,}\limits}
\newcommand{\rad}{\sqrt}
\newcommand{\hf}{H}
\newcommand{\hs}{H\!S}
\newcommand{\hp}{H\!P}
\renewcommand{\vec}{\mathbf}
\let\temp\phi
\let\phi\varphi
\let\eulerphi\temp


\renewcommand{\epsilon}{\varepsilon}
\renewcommand{\subset}{\subseteq}
\renewcommand{\supset}{\supseteq}
\newcommand{\macaulay}{\normalfont\textsl{Macaulay2}}
\newcommand{\GAP}{\normalfont\textsf{GAP}}
\newcommand{\invlim}{\varprojlim}
\renewcommand{\le}{\leqslant}
\renewcommand{\ge}{\geqslant}
\newcommand{\valpha}{{\boldsymbol\alpha}}
\newcommand{\vbeta}{{\boldsymbol\beta}}
\newcommand{\vgamma}{{\boldsymbol\gamma}}
\newcommand{\dotp}{\bullet}
\newcommand{\lc}{\normalfont\textsc{lc}}
\newcommand{\lt}{\normalfont\textsc{lt}}
\newcommand{\lm}{\normalfont\textsc{lm}}
\newcommand{\from}{\leftarrow}
\newcommand{\transpose}{\intercal}
\newcommand{\grad}{\boldsymbol\nabla}
\newcommand{\curl}{\boldsymbol{\nabla\times}}
\renewcommand{\d}{\, d}
\newcommand{\<}{\langle}
\renewcommand{\>}{\rangle}

%\renewcommand{\proofname}{Sketch of Proof}


\renewenvironment{proof}[1][Proof]
  {\begin{trivlist}\item[\hskip \labelsep \itshape \bfseries #1{}\hspace{2ex}]\upshape}
{\qed\end{trivlist}}

\newenvironment{sketch}[1][Sketch of Proof]
  {\begin{trivlist}\item[\hskip \labelsep \itshape \bfseries #1{}\hspace{2ex}]\upshape}
{\qed\end{trivlist}}



\makeatletter
\renewcommand\section{\@startsection{paragraph}{10}{\z@}%
                                     {-3.25ex\@plus -1ex \@minus -.2ex}%
                                     {1.5ex \@plus .2ex}%
                                     {\normalfont\large\sffamily\bfseries}}
\renewcommand\subsection{\@startsection{subparagraph}{10}{\z@}%
                                    {3.25ex \@plus1ex \@minus.2ex}%
                                    {-1em}%
                                    {\normalfont\normalsize\sffamily\bfseries}}
\makeatother

%% Fix weird index/bib issue.
\makeatletter
\gdef\ttl@savemark{\sectionmark{}}
\makeatother


\makeatletter
%% no number for refs
\newcommand\frontstyle{%
  \def\activitystyle{activity-chapter}
  \def\maketitle{%
    \addtocounter{titlenumber}{1}%
                    {\flushleft\small\sffamily\bfseries\@pretitle\par\vspace{-1.5em}}%
                    {\flushleft\LARGE\sffamily\bfseries\@title \par }%
                    {\vskip .6em\noindent\textit\theabstract\setcounter{problem}{0}\setcounter{sectiontitlenumber}{0}}%
                    \par\vspace{2em}
                    \phantomsection\addcontentsline{toc}{section}{\textbf{\@title}}%
                  }}
\makeatother



\NewEnviron{annotate}{\vspace{-.3cm}\small \itshape \BODY \vspace{.3cm}}


%%%% TIKZ STUFF

%% N-GON code
\tikzset{
    pics/tikzngon/.style={
        code={
        \tikzmath{\xx = #1;\rr=1.7;}
        \draw[ultra thick,rounded corners=.05mm] ({\rr*sin(0*360/\xx)},{\rr*cos(0*360/\xx)})
        \foreach \x in {-1,0,...,\xx}
        {
        -- ({\rr*sin(\x*360/\xx)},{\rr*cos(\x*360/\xx)})
        }
           -- cycle;
  }}}

%% N-GON code (even)
\tikzset{
    pics/tikzegon/.style={
        code={
        \tikzmath{\xx = #1;\rr=1.7;}
        \draw[ultra thick,rounded corners=.05mm] ({\rr*sin(0*360/\xx+180/\xx)},{\rr*cos(0*360/\xx+180/\xx)})
        \foreach \x in {-1,0,...,\xx}
           {
           -- ({\rr*sin(\x*360/\xx+180/\xx)},{\rr*cos(\x*360/\xx+180/\xx)}) 
           }
           -- cycle;
  }}}




%% N-CLOCK code
\tikzset{
    pics/tikznclock/.style={
        code={
        \tikzmath{\xx = #1;\rr=1.7;\dd=.4;}
        \foreach \x in {1,...,\xx}
        \pgfmathtruncatemacro{\xy}{\x-1}
           {
             \node[circle,fill=black,inner sep=0pt, minimum size=13pt,text=white]
             at ({(\rr-\dd)*sin((\x-1)*360/(\xx)},{(\rr-\dd)*cos((\x-1)*360/\xx}) {\normalfont\bfseries\sffamily\small {\xy}};
           }
  \draw[thick] (0,0) circle (\rr);
  }}}



%% barcode from
%% https://tex.stackexchange.com/questions/6895/is-there-a-good-latex-package-for-generating-barcodes
%% NOT CURRENTLY USED!


\def\barcode#1#2#3#4#5#6#7{\begingroup%
  \dimen0=0.1em
  \def\stack##1##2{\oalign{##1\cr\hidewidth##2\hidewidth}}%
  \def\0##1{\kern##1\dimen0}%
  \def\1##1{\vrule height10ex width##1\dimen0}%
  \def\L##1{\ifcase##1\bc3211##1\or\bc2221##1\or\bc2122##1\or\bc1411##1%
    \or\bc1132##1\or\bc1231##1\or\bc1114##1\or\bc1312##1\or\bc1213##1%
    \or\bc3112##1\fi}%
  \def\R##1{\bgroup\let\next\1\let\1\0\let\0\next\L##1\egroup}%
  \def\G##1{\bgroup\let\bc\bcg\L##1\egroup}% reverse
  \def\bc##1##2##3##4##5{\stack{\0##1\1##2\0##3\1##4}##5}%
  \def\bcg##1##2##3##4##5{\stack{\0##4\1##3\0##2\1##1}##5}%
  \def\bcR##1##2##3##4##5##6{\R##1\R##2\R##3\R##4\R##5\R##6\11\01\11\09%
    \endgroup}%
  \stack{\09}#1\11\01\11\L#2%
  \ifcase#1\L#3\L#4\L#5\L#6\L#7\or\L#3\G#4\L#5\G#6\G#7%
    \or\L#3\G#4\G#5\L#6\G#7\or\L#3\G#4\G#5\G#6\L#7%
    \or\G#3\L#4\L#5\G#6\G#7\or\G#3\G#4\L#5\L#6\G#7%
    \or\G#3\G#4\G#5\L#6\L#7\or\G#3\L#4\G#5\L#6\G#7%
    \or\G#3\L#4\G#5\G#6\L#7\or\G#3\G#4\L#5\G#6\L#7%
  \fi\01\11\01\11\01\bcR}


\author{Bart Snapp}

\title{Fields}

\begin{document}
\begin{abstract}
  We introduce fields.
\end{abstract}
\maketitle

In this section we will introduce fields, which are objects that are a
bit like two groups at once.


\begin{definition}
  A \dfn{field} is a set $K$ with two operations that we will denote
  by $+$ and $\cdot$, and call addition and multiplication, such that
  the following properties hold:
  \begin{description}
  \item[Abelian group under addition] $(K,+)$ is an Abelian
    group with identity element denoted by $0$.
  \item[Abelian group under multiplication] $(K-\{0\},\cdot)$ is
    an Abelian group with identity element denoted by $1$.
  \item[Distributivity] The operation $\cdot$ distributes over $+$,
    meaning that if $a,b,c\in K$,
    \[
    a\cdot (b+ c) = a\cdot b+ a\cdot c \quad\text{and}\quad (a+ b)\cdot c  = a\cdot c+ b\cdot c.
    \]
  \end{description}
\end{definition}

\begin{exercise} 
  Which of the following are fields under the canonical operations?
  Select all that apply.
  \begin{selectAll}
    \choice[correct]{$\Q$}
    \choice[correct]{$\R$}
    \choice[correct]{$\C$}
    \choice{$\Z$}
    \choice[correct]{$\Z_p$ where $p$ is a prime number}
    \choice{$\Z_m^\times$}
  \end{selectAll}
\end{exercise}




\begin{exercise}
  Note, in the definition of a field, the elements ``$0$'' and ``$1$''
  are not necessarily the integers who look and sound the same. In
  particular, nowhere in the definition of a field does it state what
  how $0$ acts on field elements by field multiplication. Prove or
  disprove: $a\cdot 0 = 0$.
\end{exercise}

\begin{exercise}
  Let $K$ be a field. Prove that $(-1)\cdot a = (-a)$.
\end{exercise}





\begin{exercise}
  Let $K$ be a field. Prove that ${\left(ab^{-1}\right)}\hspace{0pt}^{-1} = a^{-1}
  b$. Note, here you are using the definition of a field to show
  \[
  \frac{1}{a/b} = \frac{b}{a}.
  \]
\end{exercise}


\begin{exercise} 
  Let $a$ and $b$ be elements of a field $K$. Prove that if $a\cdot b
  = 0$, then $a$ is zero or $b$ is zero. 
\end{exercise}




\begin{exercise} 
  When is $\Z_n$ definitely not a field?
\end{exercise}

\begin{exercise}\label{E:FF4}
  Describe a field with four elements.
  \begin{hint}
    Trial and error with a Cayley table for addition and a Cayley
    table for multiplication is not a bad way to start.
  \end{hint}
\end{exercise}


\begin{exercise}\label{E:FF6} Can you find a field with six elements?
\end{exercise}


\begin{exercise}
  Let $K$ be a field. Prove that $K\times K$ is never a field under
  componenetwise addition and componenetwise multiplication.
\end{exercise}


\begin{exercise} 
Let $K$ be a field. Find some quality inherent in $K$ that is
necessary and sufficient to make the set of ordered pairs in $K\times
K$ a field when ``addition'' is defined by $\diamond$ and
``multiplication'' is defined by $\star$:
\[
(a,b)\diamond (c,d) = (a+c, b+d)\qquad\text{and}\qquad (a,b)\star (c,d) =
(ac-bd,ad+bc).
\]
Hint: What if $K = \R$? What if $K = \C$? Keep on going!
\end{exercise}



\section{Homomorphisms of fields}


\begin{definition}\index{homomorphism!of fields}\index{field homomorphism}\index{isomorphism!of fields}\index{field isomorphism}
  Let $K$ and $L$ be fields. A function
  \[
  \phi:K\to L
  \]
  is a \dfn{homomorphism} of fields if
  \begin{align*}
    \phi(a+b) &= \phi(a)+\phi(b)\\
    \phi(a\cdot b) &= \phi(a)\cdot \phi(b)\\
    \phi(1_K) &= 1_L,
  \end{align*}
  where the operations on the left-hand side of the equations might
  not be the operations on right-hand side of the equations. A
  bijective homomorphism of fields is called a \dfn{field isomorphism}.
\end{definition}

\begin{exercise}
  Let $\phi:K\to L$ be a homomorphism of fields. Prove that $\phi(0_K)
  = 0_L$. Can you prove that $\phi(1_K) = 1_L$ strictly from the first
  two equations that define a field homomorphism? Why or why not?
\end{exercise}


\begin{definition}\index{kernel!field homomorphism}
  Let $\phi: K\to L$ be a homomorphism of fields. The \textbf{kernel}
  of $\phi$ is
  \[
  \ker(\phi) := \{a\in K:\phi(a) = 0_L\}\subset K.
  \]
  The \textbf{image} of $\phi$ is\index{image!field homomorphism}
  \[
  \im(\phi) = \{\phi(a):a\in K\}\subset L.
  \]
\end{definition}


\begin{exercise}
  Let $\phi:K\to L$ be a homomorphism of fields. Prove that $\phi$ is
  injective if and only if $\ker(\phi) = \{0\}$.
\end{exercise}

\begin{exercise}
  Let $\phi:K\to L$ be a homomorphism of fields. Prove that $\phi$ is
  injective.
\end{exercise}

\begin{exercise}
  Prove that every surjective homomorphism of fields is a bijective
  homomorphism of fields.
\end{exercise}







\section{The characteristic of a field}


Fields can be dichotomized by their \textit{characteristic}.

\begin{definition}
The \dfn{characteristic} of a field $K$ is the least positive
integer $n$ such that
\[
\underbrace{1 + 1 + \cdots +1}_{\text{$n$ times}} =0.
\]
If there is no such positive integer, then we say that $K$ has
characteristic $0$.
\end{definition}


\begin{exercise}
  Prove that if a field has characteristic $0$, then it contains a
  subgroup isomorphic to $\Z$.
\end{exercise}

\begin{exercise}
  Prove that if a field has characteristic $p$, then it contains a
  subgroup isomorphic to $\Z_p$.
\end{exercise}



\begin{exercise} 
Prove that if a field has finite characteristic, then its
characteristic is a prime number.
\end{exercise}

\begin{exercise}\index{freshman binomial theorem}\label{E:FBT}
Prove that in a field of positive characteristic $p$, the so-called
\textit{freshman binomial theorem} holds:
\[
(a+b)^p = a^p + b^p
\]
\end{exercise}



When working in fields of characteristic $p$, we have a new
homomorphism.

\begin{example}[The Frobenius homomorphism]\index{Frobenius homomorphism}
  Let $K$ be a field of characteristic $p$. Prove that
  \begin{align*}
    \phi:K &\to K\\
    a &\mapsto a^p
  \end{align*}
  is a homomorphism of fields. Is the Frobenius homomorphism surjective?
\end{example}





This simple idea of characteristic tells us something about fields.

\begin{definition}
  A $K$ is a \dfn{subfield} of a field $K$ if
  \begin{enumerate}
  \item $k\subset K$,
  \item $k$ has the same identity elements and operations as $K$.
  \end{enumerate}
\end{definition}


\begin{theorem}[Base fields of positive characteristic]
  If a field $K$ has positive characteristic, then $K$ contains a
  subfield $Z$ such that $Z\iso \Z_p$.
  \begin{sketch}
    Set
    \[
    Z =\left\{\sum_{i=1}^n 1:n\in\{1,2,\dots,p\}\right\}.
    \]
    Prove that $Z$ is a group of order $p$, and hence is isomorphic to
    $\Z_p$. Since $\Z_p$ is also a field, we are done.
  \end{sketch}
\end{theorem}





\begin{theorem}[Base fields of characteristic zero]
  If a field $K$ has positive characteristic, then $K$ contains a
  subfield $Q$ such that $Q\iso \Q$.
  \begin{sketch}
    Set
    \[
    Z =\left\{\sum_{i=1}^n \pm 1:n\in \N\cup\{0\}\right\}.
    \]
    Prove that $(Z,+)$ is an infinite cyclic group and hence is
    isomorphic to $(\Z,+)$.

    Now set
    \[
    Q = \{a^{-1}b : a,b\in Z\}.
    \]
    We claim that $Q$ is a field. First note that $Q$ is a subgroup of
    $K$ under field addition. Suppose $\alpha,\beta\in Q$. We'll
    show that $\alpha-\beta\in Q$. Let $\alpha = a^{-1}b$ and $\beta=
    c^{-1}d$. Write
    \[
    a^{-1}b - c^{-1} d = (ac)^{-1}(bc- ad)\in Q.
    \]
    Now we must show that $Q-\{0\}$ is a subgroup of $K$ under
    multiplication. Again let $\alpha = a^{-1}b$ and $\beta=
    c^{-1}d$. We'll show that $\alpha^{-1}\beta \in Q$. Write
    \[
    a^{-1}b (c^{-1}d)^{-1} = (ad)^{-1} bc\in Q.
    \]
    Since $Q\subset K$, multiplication distributes over addition.

    Now we will prove that $\Q\iso Q$. Define a map
    \begin{align*}
      \Q &\to Q\\
      b/a &\mapsto a^{-1} b.
    \end{align*}
    We must show that this map is \index{well-defined}well-defined.
    Suppose that
    \[
    \frac{b}{a} = \frac{d}{c},
    \]
    this happens if and only if
    \[
    ad = bc\quad\Leftrightarrow a^{-1} b = c^{-1} d.
    \]
    Thus our map is well-defined. Since our map is clearly a
    surjective homomorphism fields, it is an isomorphism of fields.
  \end{sketch}
\end{theorem}

\begin{corollary}[Number fields]
  Every subfield of $\C$ contains $\Q$. Subfields of $\C$ are often
  called \dfn{number fields}.
\end{corollary}












\end{document}
