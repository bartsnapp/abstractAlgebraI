\documentclass{ximera}

\usepackage[T1]{fontenc}
\usepackage{stix2}
\usepackage{gillius}
\usepackage{resizegather}
%\usepackage{rsfso} fancy cal
\DeclareMathAlphabet{\mathcal}{OMS}{cmsy}{m}{n} %less fancy cal


\usepackage{multicol}


\usepackage{tikz-cd}
\usepackage{tkz-euclide} %% compass
\usetkzobj{all}  %% tkzCompass
\tikzset{>=stealth}
\tikzcdset{arrow style=tikz}
\usetikzlibrary{math} %% for assigning variables
%\usetikzlibrary{fadings}

\usepackage{colortbl,boldline,makecell} %% group tables


\usepackage[sans]{dsfont}

\usepackage{stmaryrd,pifont}

\graphicspath{
  {./}
  {fields/}
  }     



\let\oldbibliography\thebibliography%% to compact bib
\renewcommand{\thebibliography}[1]{%
  \oldbibliography{#1}%
  \setlength{\itemsep}{0pt}%
}
\renewcommand\refname{} %% no name needed!


\DefineVerbatimEnvironment{macaulay2}{Verbatim}{numbers=left,frame=lines,label=Macaulay2,labelposition=topline}

\DefineVerbatimEnvironment{gap}{Verbatim}{numbers=left,frame=lines,label=GAP,labelposition=topline}

%%% This next bit of code defines all our theorem environments
\makeatletter
\let\c@theorem\relax
\let\c@corollary\relax
%\let\c@example\relax
\makeatother

\let\definition\relax
\let\enddefinition\relax

\let\theorem\relax
\let\endtheorem\relax

\let\proposition\relax
\let\endproposition\relax

\let\exercise\relax
\let\endexercise\relax

\let\question\relax
\let\endquestion\relax

\let\remark\relax
\let\endremark\relax

\let\corollary\relax
\let\endcorollary\relax


\let\example\relax
\let\endexample\relax

\let\warning\relax
\let\endwarning\relax

\let\lemma\relax
\let\endlemma\relax


\let\algorithm\relax
\let\endalgorithm\relax
\usepackage{algpseudocode}

\newtheoremstyle{SlantTheorem}{\topsep}{\topsep}%%% space between body and thm
		{\slshape}                      %%% Thm body font
		{}                              %%% Indent amount (empty = no indent)
		{\bfseries\sffamily}            %%% Thm head font
		{}                              %%% Punctuation after thm head
		{3ex}                           %%% Space after thm head
		{\thmname{#1}\thmnumber{ #2}\thmnote{ \bfseries(#3)}}%%% Thm head spec
\theoremstyle{SlantTheorem}
\newtheorem{theorem}{Theorem}
%\newtheorem{definition}[theorem]{Definition}
%\newtheorem{proposition}[theorem]{Proposition}
%% \newtheorem*{dfnn}{Definition}
%% \newtheorem{ques}{Question}[theorem]
%% \newtheorem*{war}{WARNING}
%% \newtheorem*{cor}{Corollary}
%% \newtheorem*{eg}{Example}
\newtheorem*{remark}{Remark}
\newtheorem*{touchstone}{Touchstone}
\newtheorem{corollary}{Corollary}[theorem]
\newtheorem*{warning}{WARNING}
\newtheorem{example}[corollary]{Example}
\newtheorem{lemma}[theorem]{Lemma}




\newtheoremstyle{Definition}{\topsep}{\topsep}%%% space between body and thm
		{}                              %%% Thm body font
		{}                              %%% Indent amount (empty = no indent)
		{\bfseries\sffamily}            %%% Thm head font
		{}                              %%% Punctuation after thm head
		{3ex}                           %%% Space after thm head
		{\thmname{#1}\thmnumber{ #2}\thmnote{ \bfseries(#3)}}%%% Thm head spec
\theoremstyle{Definition}
\newtheorem{definition}[theorem]{Definition}



\let\algorithm\relax
\let\endalgorithm\relax
\newtheoremstyle{Alg}{\topsep}{\topsep}%%% space between body and thm
		{}                              %%% Thm body font
		{}                              %%% Indent amount (empty = no indent)
		{\bfseries\sffamily}            %%% Thm head font
		{}                              %%% Punctuation after thm head
		{3ex}                           %%% Space after thm head
		{\thmname{#1}\thmnumber{ #2}\thmnote{ \bfseries(#3)}}%%% Thm head spec
\theoremstyle{Alg}
\newtheorem*{algorithm}{Algorithm}
\newtheorem*{construction}{Construction}




\newtheoremstyle{Exercise}{\topsep}{\topsep} %%% space between body and thm
		{}                           %%% Thm body font
		{}                           %%% Indent amount (empty = no indent)
		{\bfseries\sffamily}         %%% Thm head font
		{)}                          %%% Punctuation after thm head
		{ }                          %%% Space after thm head
		{\thmnumber{#2}\thmnote{ \bfseries(#3)}}%%% Thm head spec
\theoremstyle{Exercise}
\newtheorem{exercise}[corollary]{}%[theorem]

%% \newtheoremstyle{Question}{\topsep}{\topsep} %%% space between body and thm
%% 		{\bfseries}                  %%% Thm body font
%% 		{3ex}                        %%% Indent amount (empty = no indent)
%% 		{}                           %%% Thm head font
%% 		{}                           %%% Punctuation after thm head
%% 		{}                           %%% Space after thm head
%% 		{\thmnumber{#2}\thmnote{ \bfseries(#3)}}%%% Thm head spec
\newtheoremstyle{Question}{3em}{3em} %%% space between body and thm
		{\large\bfseries}                           %%% Thm body font
		{}                           %%% Indent amount (empty = no indent)
		{}                         %%% Thm head font
		{}                          %%% Punctuation after thm head
		{0em}                          %%% Space after thm head
		{}%%% Thm head spec
\theoremstyle{Question}
\newtheorem*{question}{}






\renewcommand{\tilde}{\widetilde}
\renewcommand{\bar}{\overline}
\renewcommand{\hat}{\widehat}
\newcommand{\N}{\mathbb N}
\newcommand{\Z}{\mathbb Z}
\newcommand{\R}{\mathbb R}
\newcommand{\Q}{\mathbb Q}
\newcommand{\C}{\mathbb C}
\newcommand{\V}{\mathbb V}
\newcommand{\I}{\mathbb I}
\newcommand{\A}{\mathbb A}
\renewcommand{\o}{\mathbf o}
\newcommand{\iso}{\simeq}
\newcommand{\ph}{\varphi}
\newcommand{\Cf}{\mathcal{C}}
\newcommand{\IZ}{\mathrm{Int}(\Z)}
\newcommand{\dsum}{\oplus}
\newcommand{\directsum}{\bigoplus}
\newcommand{\union}{\bigcup}
\newcommand{\subgp}{\leq}
\newcommand{\normal}{\trianglelefteq}
\renewcommand{\i}{\mathfrak}
\renewcommand{\a}{\mathfrak{a}}
\renewcommand{\b}{\mathfrak{b}}
\newcommand{\m}{\mathfrak{m}}
\newcommand{\p}{\mathfrak{p}}
\newcommand{\q}{\mathfrak{q}}
\newcommand{\dfn}[1]{\textbf{#1}\index{#1}}
\let\hom\relax
\DeclareMathOperator{\mat}{Mat}
\DeclareMathOperator{\ann}{Ann}
\DeclareMathOperator{\h}{ht}
\DeclareMathOperator{\tr}{tr}
\DeclareMathOperator{\hom}{Hom}
\DeclareMathOperator{\Span}{Span}
\DeclareMathOperator{\spec}{Spec}
\DeclareMathOperator{\maxspec}{MaxSpec}
\DeclareMathOperator{\aut}{Aut}
\DeclareMathOperator{\ass}{Ass}
\DeclareMathOperator{\lcm}{lcm}
\DeclareMathOperator{\ff}{Frac}
\DeclareMathOperator{\im}{Im}
\DeclareMathOperator{\syz}{Syz}
\DeclareMathOperator{\gr}{Gr}
\DeclareMathOperator{\multideg}{multideg}
\renewcommand{\ker}{\mathop{\mathrm{Ker}}\nolimits}
\newcommand{\coker}{\mathop{\mathrm{Coker}}\nolimits}
\newcommand{\lps}{[\hspace{-0.25ex}[}
\newcommand{\rps}{]\hspace{-0.25ex}]}
\newcommand{\into}{\hookrightarrow}
\newcommand{\onto}{\twoheadrightarrow}
\newcommand{\tensor}{\otimes}
\newcommand{\x}{\mathbf{x}}
\newcommand{\X}{\mathbf X}
\newcommand{\Y}{\mathbf Y}
\renewcommand{\k}{\boldsymbol{\kappa}}
\renewcommand{\emptyset}{\varnothing}
\renewcommand{\qedsymbol}{$\blacksquare$}
\renewcommand{\l}{\ell}
\newcommand{\1}{\mathds{1}}
\newcommand{\lto}{\mathop{\longrightarrow\,}\limits}
\newcommand{\rad}{\sqrt}
\newcommand{\hf}{H}
\newcommand{\hs}{H\!S}
\newcommand{\hp}{H\!P}
\renewcommand{\vec}{\mathbf}
\let\temp\phi
\let\phi\varphi
\let\eulerphi\temp


\renewcommand{\epsilon}{\varepsilon}
\renewcommand{\subset}{\subseteq}
\renewcommand{\supset}{\supseteq}
\newcommand{\macaulay}{\normalfont\textsl{Macaulay2}}
\newcommand{\GAP}{\normalfont\textsf{GAP}}
\newcommand{\invlim}{\varprojlim}
\renewcommand{\le}{\leqslant}
\renewcommand{\ge}{\geqslant}
\newcommand{\valpha}{{\boldsymbol\alpha}}
\newcommand{\vbeta}{{\boldsymbol\beta}}
\newcommand{\vgamma}{{\boldsymbol\gamma}}
\newcommand{\dotp}{\bullet}
\newcommand{\lc}{\normalfont\textsc{lc}}
\newcommand{\lt}{\normalfont\textsc{lt}}
\newcommand{\lm}{\normalfont\textsc{lm}}
\newcommand{\from}{\leftarrow}
\newcommand{\transpose}{\intercal}
\newcommand{\grad}{\boldsymbol\nabla}
\newcommand{\curl}{\boldsymbol{\nabla\times}}
\renewcommand{\d}{\, d}
\newcommand{\<}{\langle}
\renewcommand{\>}{\rangle}

%\renewcommand{\proofname}{Sketch of Proof}


\renewenvironment{proof}[1][Proof]
  {\begin{trivlist}\item[\hskip \labelsep \itshape \bfseries #1{}\hspace{2ex}]\upshape}
{\qed\end{trivlist}}

\newenvironment{sketch}[1][Sketch of Proof]
  {\begin{trivlist}\item[\hskip \labelsep \itshape \bfseries #1{}\hspace{2ex}]\upshape}
{\qed\end{trivlist}}



\makeatletter
\renewcommand\section{\@startsection{paragraph}{10}{\z@}%
                                     {-3.25ex\@plus -1ex \@minus -.2ex}%
                                     {1.5ex \@plus .2ex}%
                                     {\normalfont\large\sffamily\bfseries}}
\renewcommand\subsection{\@startsection{subparagraph}{10}{\z@}%
                                    {3.25ex \@plus1ex \@minus.2ex}%
                                    {-1em}%
                                    {\normalfont\normalsize\sffamily\bfseries}}
\makeatother

%% Fix weird index/bib issue.
\makeatletter
\gdef\ttl@savemark{\sectionmark{}}
\makeatother


\makeatletter
%% no number for refs
\newcommand\frontstyle{%
  \def\activitystyle{activity-chapter}
  \def\maketitle{%
    \addtocounter{titlenumber}{1}%
                    {\flushleft\small\sffamily\bfseries\@pretitle\par\vspace{-1.5em}}%
                    {\flushleft\LARGE\sffamily\bfseries\@title \par }%
                    {\vskip .6em\noindent\textit\theabstract\setcounter{problem}{0}\setcounter{sectiontitlenumber}{0}}%
                    \par\vspace{2em}
                    \phantomsection\addcontentsline{toc}{section}{\textbf{\@title}}%
                  }}
\makeatother



\NewEnviron{annotate}{\vspace{-.3cm}\small \itshape \BODY \vspace{.3cm}}


%%%% TIKZ STUFF

%% N-GON code
\tikzset{
    pics/tikzngon/.style={
        code={
        \tikzmath{\xx = #1;\rr=1.7;}
        \draw[ultra thick,rounded corners=.05mm] ({\rr*sin(0*360/\xx)},{\rr*cos(0*360/\xx)})
        \foreach \x in {-1,0,...,\xx}
        {
        -- ({\rr*sin(\x*360/\xx)},{\rr*cos(\x*360/\xx)})
        }
           -- cycle;
  }}}

%% N-GON code (even)
\tikzset{
    pics/tikzegon/.style={
        code={
        \tikzmath{\xx = #1;\rr=1.7;}
        \draw[ultra thick,rounded corners=.05mm] ({\rr*sin(0*360/\xx+180/\xx)},{\rr*cos(0*360/\xx+180/\xx)})
        \foreach \x in {-1,0,...,\xx}
           {
           -- ({\rr*sin(\x*360/\xx+180/\xx)},{\rr*cos(\x*360/\xx+180/\xx)}) 
           }
           -- cycle;
  }}}




%% N-CLOCK code
\tikzset{
    pics/tikznclock/.style={
        code={
        \tikzmath{\xx = #1;\rr=1.7;\dd=.4;}
        \foreach \x in {1,...,\xx}
        \pgfmathtruncatemacro{\xy}{\x-1}
           {
             \node[circle,fill=black,inner sep=0pt, minimum size=13pt,text=white]
             at ({(\rr-\dd)*sin((\x-1)*360/(\xx)},{(\rr-\dd)*cos((\x-1)*360/\xx}) {\normalfont\bfseries\sffamily\small {\xy}};
           }
  \draw[thick] (0,0) circle (\rr);
  }}}



%% barcode from
%% https://tex.stackexchange.com/questions/6895/is-there-a-good-latex-package-for-generating-barcodes
%% NOT CURRENTLY USED!


\def\barcode#1#2#3#4#5#6#7{\begingroup%
  \dimen0=0.1em
  \def\stack##1##2{\oalign{##1\cr\hidewidth##2\hidewidth}}%
  \def\0##1{\kern##1\dimen0}%
  \def\1##1{\vrule height10ex width##1\dimen0}%
  \def\L##1{\ifcase##1\bc3211##1\or\bc2221##1\or\bc2122##1\or\bc1411##1%
    \or\bc1132##1\or\bc1231##1\or\bc1114##1\or\bc1312##1\or\bc1213##1%
    \or\bc3112##1\fi}%
  \def\R##1{\bgroup\let\next\1\let\1\0\let\0\next\L##1\egroup}%
  \def\G##1{\bgroup\let\bc\bcg\L##1\egroup}% reverse
  \def\bc##1##2##3##4##5{\stack{\0##1\1##2\0##3\1##4}##5}%
  \def\bcg##1##2##3##4##5{\stack{\0##4\1##3\0##2\1##1}##5}%
  \def\bcR##1##2##3##4##5##6{\R##1\R##2\R##3\R##4\R##5\R##6\11\01\11\09%
    \endgroup}%
  \stack{\09}#1\11\01\11\L#2%
  \ifcase#1\L#3\L#4\L#5\L#6\L#7\or\L#3\G#4\L#5\G#6\G#7%
    \or\L#3\G#4\G#5\L#6\G#7\or\L#3\G#4\G#5\G#6\L#7%
    \or\G#3\L#4\L#5\G#6\G#7\or\G#3\G#4\L#5\L#6\G#7%
    \or\G#3\G#4\G#5\L#6\L#7\or\G#3\L#4\G#5\L#6\G#7%
    \or\G#3\L#4\G#5\G#6\L#7\or\G#3\G#4\L#5\G#6\L#7%
  \fi\01\11\01\11\01\bcR}


\title{Analogs to number theory}

\begin{document}
\begin{abstract}
  We repeat our ideas from number theory, but now with polynomials.
\end{abstract}
\maketitle


\section{Back to number theory}


To start, let's continue with number theory as our guide. We'll
introduce the GCD of two polynomials.

\begin{definition}
  Let $K$ be a field. A polynomial $g(x)\in K[x]$ is called a
  \dfn{greatest common divisor} of two polynomials $m(x),n(x)\in K[x]$
  provided that:
  \begin{enumerate}
  \item $g(x) | m(x)$ and $g(x) | n(x)$.
  \item If $d(x)\in K[x]$ is an element where $d(x)| m(x)$ and $d(x) | n(x)$, then $\deg(d(x))\le \deg(g(x))$.
  \end{enumerate}
  The GCD of two polynomials is \textbf{not} unique.  We denote the set of GCDs
  of $m(x)$ and $n(x)$ as $\gcd(m(x),n(x))$. If
  \[
  c\in K\quad\text{ and }c\in \gcd(m(x),n(x))
  \]
  we say that $m(x)$ and $n(x)$ are \dfn{coprime} or \dfn{relatively prime}.
\end{definition}


\begin{example}[Some polynomials in a GCD]
  Consider $x^4$ and $x^5$. Then we have
  \[
  x^4, 3x^4, x^4/2 \in \gcd(x^4,x^5).
  \]
\end{example}

\begin{theorem}[Euclid's lemma for polynomials, version 1]\label{T:ELP1}\index{Euclid's lemma}
  Let $K$ be a field. Given nonzero $m(x),n(x)\in K[x]$, if $G(x)\in
  \gcd(m(x),n(x))$, then
  \[
  \deg(G(x)) = \min\{\deg(f(x)\cdot m(x) +g(x)\cdot n(x)): f(x),g(x)\in K[x]\}.
  \]
  \begin{sketch} We will prove this in several steps.
  \begin{enumerate}
  \item Let $\mathcal S = \{\deg(f(x)\cdot m(x) +g(x)\cdot n(x)):
    f(x),g(x)\in K[x]\}$. Prove that $\mathcal S$ has a least element,
    call it $d$.
  \item Let $h(x) = f_d(x)\cdot m(x) +g_d(x)\cdot n(x)$ be such that
    $\deg(h(x)) = d$. Prove that $h(x)| (f(x)\cdot m(x) +g(x)\cdot n(x))$.
  \item Prove that $h(x)| m(x)$ and $h(x)| n(x)$. Explain why $\deg(h(x)) \le a(x)\in\gcd(m(x),n(x))$.
  \item Recall that $h(x) = f_d(x)\cdot m(x) +g_d(x)\cdot n(x)$, and prove that
    for any $a(x)\in\gcd(m,n)$, $a(x) | h(x)$. Explain why we must
    conclude that $h(x)\in\gcd(m(x),n(x))$.
  \end{enumerate}
  \end{sketch}
\end{theorem}

As before, when we first studied Euclid's lemma, Theorem~\ref{T:EL1},
if
\[
h(x)\in\gcd(m(x),n(x)),
\]
$h(x)$ is simultaneously the polynomial of \textbf{largest degree}
that divides both $m(x)$ and $n(x)$ and the polynomial of
\textbf{smallest degree} of the form $f(x)\cdot m(x) + g(x)\cdot
n(x)$. However consider this
\[
\gcd(m(x),n(x)) = \{\text{common divisors of $m(x)$ and $n(x)$}\}\cap \mathcal{S}
\]
where $\mathcal{S}$ is the set from the proof of Euclid's lemma for
polynomials. Here is a picture that attempts to convey the situation:
\[
\begin{tikzpicture}
  \draw[white,thin,shading = axis, top color = white, bottom color = gray] (-2,2) -- (0,0)-- (2,2) -- (-2,2) -- cycle;
  \draw[white,thin,shading = axis, bottom color = white, top color = gray] (-2,-2) -- (0,0)-- (2,-2) -- (-2,-2) -- cycle;
  \filldraw (0,0) circle (3pt);
  \node[right] at (0,0) {$\scriptstyle\gcd(m(x),n(x))$};
  \node at (0,1.5) {$\scriptstyle f(x)\cdot m(x) + b(x)\cdot n(x)$};
  \node at (0,-1.5) {\text{\tiny common divisors of $m(x)$ and $n(x)$}};
  \draw[->,ultra thick] (-3,-2) -- (-3,2);
  \node[left] at (-3,-1.5) {\text{\tiny smaller degree}};
  \node[left] at (-3,1.5) {\text{\tiny larger degree}};
\end{tikzpicture}
\]
In the cone above $\gcd(m(x),n(x))$ is all polynomials of the form
$f(x)\cdot m(x) + b(x)\cdot n(x)$. The cone below $\gcd(m(x),n(x))$ is the
set of common divisors of $m(x)$ and $n(x)$.




\begin{corollary}[Euclid's lemma for polynomials, version 2]\label{C:ELP2}\index{Euclid's lemma}
  Let $K$ be a field and $\l(x)$, $m(x)$, and $n(x)$ be nonzero
  polynomials in $K[x]$. If
  \[
  \l(x)|m(x)n(x)\qquad\text{and}\qquad \gcd(\l(x),m(x))\subset K,
  \]
  then $\l(x)|n(x)$.
\end{corollary}


\begin{corollary}[Unique factorization of polynomials]\label{C:UFP}
  Every nonzero polynomial can be expressed as a unique (up to order,
  and up to constant multiple) product of irreducible polynomials.
  \begin{sketch}
    See the proof of unique factorization of natural numbers,
    Corollary~\ref{C:UFNN}.
  \end{sketch}
\end{corollary}







\section{Congruence modulo a polynomial}

Think about the group $\Z_n$. In this group,
\[
a\equiv b \pmod{n} \quad\Leftrightarrow \quad n| (a-b).
\]
In Lemma~\ref{L:mawd} and Lemma~\ref{L:mmwd} we show that both
addition and multiplication are well defined on this group. Recall
that this is called \textit{congruence modulo an integer.}  If $K[x]$
is the set of polynomials in terms of $x$ with coefficients in $K$, we
can think about \textit{congruence modulo a polynomial.}


\begin{definition}[Congruence modulo a polynomial]
  Given a field $K$, and polynomials in $K[x]$, we define
  $f(x),g(x)\in K[x]$ to be \dfn{congruent modulo a polynomial}
  $n(x)\in K[x]$:
  \begin{align*}
    f(x) \equiv g(x) \pmod{n(x)}\quad &\Leftrightarrow \quad n(x) | (f(x)-g(x))\\
    &\Leftrightarrow \quad f(x)-g(x) = n(x)\cdot q(x).
  \end{align*}
  for some $q(x)\in K[x]$. We'll denote the set of polynomials in
  $K[x]$ congruent modulo $n(x)$ as $K[x]/(n(x))$.
\end{definition}


We should now show that addition and multiplication modulo a
polynomial are well defined.

\begin{lemma}[Arithmetic modulo a polynomial]
  Let $K$ be a field and consider $K[x]/(n(x))$. Both addition and
  multiplication in $K[x]/(n(x))$ are
  well-defined.\index{well-defined} This means that if
  \begin{align*}
    f(x) &\equiv f'(x) \pmod{n(x)}\\
    g(x) &\equiv g'(x) \pmod{n(x)},
  \end{align*}
  then
  \begin{align*}
    f(x) + g(x) &\equiv f'(x) + g'(x) \pmod{n(x)} & \text{and}\\
    f(x) \cdot g(x) &\equiv f'(x) \cdot g'(x) \pmod{n(x)}. & 
  \end{align*}
  \begin{sketch}
    Use the fact that
    \[
    f(x) \equiv g(x) \pmod{n(x)}\quad \Leftrightarrow \quad f(x)-g(x) = n(x)\cdot q(x)
    \]
    for some $q(x) \in K[x]$.
  \end{sketch}
\end{lemma}



\begin{lemma}[$\boldsymbol{K\pmb[x\pmb]\pmb/\pmb(n\pmb(x\pmb)\pmb)}$ is a vector space]\label{L:mpvs}
  Let $K$ be a field and consider $K[x]/(n(x))$. The group
  $K[x]/(n(x))$ is a $K$-vector space.
  \begin{sketch}
    Check the definition of a vector space.
  \end{sketch}
\end{lemma}


\begin{example}[Congruence modulo a polynomial]
  Let's examine the elements of $K[x]/(n(x))$ in some specific cases:
  \begin{align*}
    \Z_2[x]/(x^2) &= \{a + b x: a,b\in \Z_2\}.\\
    \Z_3[x]/(x^2) &= \{a + b x: a,b\in \Z_3\}.\\
    \Z_5[x]/(x^3) &= \{a + b x + c x^2: a,b,c\in \Z_5\}.\\
    \Z_3[x]/(x^2+1) &= \{a + b x: a,b\in \Z_3\}.
  \end{align*}
\end{example}

\begin{exercise}
  Let $K$ be a finite field of order $m$. Let $n(x)$ be a polynomial
  of degree $d$. How many elements are in $K[x]/(n(x))$?
\end{exercise}


\begin{exercise}
  Is $\Z_2[x]/(x^3)$ a field?
\end{exercise}




From our work before, we know that $\Z_n$ is a field if and only if
$n$ is prime. Since $\Z_n$ is a group, and multiplication distributes
over addition, we had to show that every element of $\Z_p-\{0\}$ had
an inverse. We did this in Lemma~\ref{L:mi}.


\begin{theorem}[Fields and irreducible polynomials]\label{T:cmip}
  Let $K$ be a field and $n(x)\in K[x]$ be a polynomial of positive
  degree. The set $K[x]/(n(x))$ is a field if and only if $n(x)$ is
  irreducible in $K[x]$.
  \begin{proof}
    $(\Rightarrow)$ Working by contraposition, suppose $n(x)$ is not
    an irreducible polynomial. Then $n(x) = a(x)\cdot b(x)$ where
    $\deg(a(x))>0$ and $\deg(b(x))>0$. Since both $a(x)$ and $b(x)$
    have degree less than $n(x)$, both
    \[
    a(x),b(x)\in K[x]/(n(x)).
    \]
    However, $a(x)\cdot b(x) \equiv 0 \pmod{n(x)}$. Hence $K[x]/(n(x))$
    cannot be a field.
    


    $(\Leftarrow)$ First note that $K[x]/(n(x))$ is an additive group
    under modular addition, Lemma~\ref{L:mpvs}. Next note that
    multiplication distributes over addition (show this). The
    multiplicative identity is $1\in K[x]/(n(x))$. All that remains to
    be shown is that multiplicative inverses exist in
    $K[x]/(n(x))-\{0\}$. Consider $a(x)\in K[x]/(n(x))$. Since $n(x)$
    is irreducible,
    \[
    \gcd(a(x),n(x)) \subset K.
    \]
    Take $c\in \gcd(a(x),n(x))$ and write
    \begin{align*}
      c &= a(x) \cdot f(x) + n(x) \cdot g(x) \\
      1 &= c^{-1}a(x) \cdot f(x) + c^{-1}n(x) \cdot g(x) \\
      1 - c^{-1}a(x) \cdot f(x)  &=c^{-1}n(x) \cdot g(x) \\
      (c^{-1} f(x)) \cdot a(x) &\equiv 1 \pmod{n(x)}.
    \end{align*}
    Thus, $K[x]/(n(x))$ is a field.
  \end{proof}
\end{theorem}

Let us recall the first time we learned about the number $i$\dots
Roman numerals notwithstanding, $i$ is the ``number'' that when you
square $i$, you obtain $-1$. Of course, there is no real number with
this property.

All of this may seem quite artificial. Why are we allowed to just
``declare'' that something like $i$ exists, especially when it plainly
doesn't?

So here is something of a response to the sentiment above. Let's
\textit{construct} $i$, and hence all of $\C$, in terms of conrguences
modulo a polynomial.

\begin{example}[A construction of $\pmb\C$]
  Consider $\R[x]/(x^2+1)$. Since $x^2+1$ is irreducible in $\R[x]$,
  $\R[x]/(x^2+1)$ is a field. Consider the map
  \begin{align*}
    \pi: \R[x]/(x^2+1) &\to \C\\
    a  + bx + (x^2+1) &\mapsto a+ bi.
  \end{align*}
  This map is a surjection. You should show that this map is a
  homomorphism (and hence and isomorphism) of fields.
\end{example}




\end{document}
