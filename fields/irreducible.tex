\documentclass{ximera}

\usepackage[T1]{fontenc}
\usepackage{stix2}
\usepackage{gillius}
\usepackage{resizegather}
%\usepackage{rsfso} fancy cal
\DeclareMathAlphabet{\mathcal}{OMS}{cmsy}{m}{n} %less fancy cal


\usepackage{multicol}


\usepackage{tikz-cd}
\usepackage{tkz-euclide} %% compass
\usetkzobj{all}  %% tkzCompass
\tikzset{>=stealth}
\tikzcdset{arrow style=tikz}
\usetikzlibrary{math} %% for assigning variables
%\usetikzlibrary{fadings}

\usepackage{colortbl,boldline,makecell} %% group tables


\usepackage[sans]{dsfont}

\usepackage{stmaryrd,pifont}

\graphicspath{
  {./}
  {fields/}
  }     



\let\oldbibliography\thebibliography%% to compact bib
\renewcommand{\thebibliography}[1]{%
  \oldbibliography{#1}%
  \setlength{\itemsep}{0pt}%
}
\renewcommand\refname{} %% no name needed!


\DefineVerbatimEnvironment{macaulay2}{Verbatim}{numbers=left,frame=lines,label=Macaulay2,labelposition=topline}

\DefineVerbatimEnvironment{gap}{Verbatim}{numbers=left,frame=lines,label=GAP,labelposition=topline}

%%% This next bit of code defines all our theorem environments
\makeatletter
\let\c@theorem\relax
\let\c@corollary\relax
%\let\c@example\relax
\makeatother

\let\definition\relax
\let\enddefinition\relax

\let\theorem\relax
\let\endtheorem\relax

\let\proposition\relax
\let\endproposition\relax

\let\exercise\relax
\let\endexercise\relax

\let\question\relax
\let\endquestion\relax

\let\remark\relax
\let\endremark\relax

\let\corollary\relax
\let\endcorollary\relax


\let\example\relax
\let\endexample\relax

\let\warning\relax
\let\endwarning\relax

\let\lemma\relax
\let\endlemma\relax


\let\algorithm\relax
\let\endalgorithm\relax
\usepackage{algpseudocode}

\newtheoremstyle{SlantTheorem}{\topsep}{\topsep}%%% space between body and thm
		{\slshape}                      %%% Thm body font
		{}                              %%% Indent amount (empty = no indent)
		{\bfseries\sffamily}            %%% Thm head font
		{}                              %%% Punctuation after thm head
		{3ex}                           %%% Space after thm head
		{\thmname{#1}\thmnumber{ #2}\thmnote{ \bfseries(#3)}}%%% Thm head spec
\theoremstyle{SlantTheorem}
\newtheorem{theorem}{Theorem}
%\newtheorem{definition}[theorem]{Definition}
%\newtheorem{proposition}[theorem]{Proposition}
%% \newtheorem*{dfnn}{Definition}
%% \newtheorem{ques}{Question}[theorem]
%% \newtheorem*{war}{WARNING}
%% \newtheorem*{cor}{Corollary}
%% \newtheorem*{eg}{Example}
\newtheorem*{remark}{Remark}
\newtheorem*{touchstone}{Touchstone}
\newtheorem{corollary}{Corollary}[theorem]
\newtheorem*{warning}{WARNING}
\newtheorem{example}[corollary]{Example}
\newtheorem{lemma}[theorem]{Lemma}




\newtheoremstyle{Definition}{\topsep}{\topsep}%%% space between body and thm
		{}                              %%% Thm body font
		{}                              %%% Indent amount (empty = no indent)
		{\bfseries\sffamily}            %%% Thm head font
		{}                              %%% Punctuation after thm head
		{3ex}                           %%% Space after thm head
		{\thmname{#1}\thmnumber{ #2}\thmnote{ \bfseries(#3)}}%%% Thm head spec
\theoremstyle{Definition}
\newtheorem{definition}[theorem]{Definition}



\let\algorithm\relax
\let\endalgorithm\relax
\newtheoremstyle{Alg}{\topsep}{\topsep}%%% space between body and thm
		{}                              %%% Thm body font
		{}                              %%% Indent amount (empty = no indent)
		{\bfseries\sffamily}            %%% Thm head font
		{}                              %%% Punctuation after thm head
		{3ex}                           %%% Space after thm head
		{\thmname{#1}\thmnumber{ #2}\thmnote{ \bfseries(#3)}}%%% Thm head spec
\theoremstyle{Alg}
\newtheorem*{algorithm}{Algorithm}
\newtheorem*{construction}{Construction}




\newtheoremstyle{Exercise}{\topsep}{\topsep} %%% space between body and thm
		{}                           %%% Thm body font
		{}                           %%% Indent amount (empty = no indent)
		{\bfseries\sffamily}         %%% Thm head font
		{)}                          %%% Punctuation after thm head
		{ }                          %%% Space after thm head
		{\thmnumber{#2}\thmnote{ \bfseries(#3)}}%%% Thm head spec
\theoremstyle{Exercise}
\newtheorem{exercise}[corollary]{}%[theorem]

%% \newtheoremstyle{Question}{\topsep}{\topsep} %%% space between body and thm
%% 		{\bfseries}                  %%% Thm body font
%% 		{3ex}                        %%% Indent amount (empty = no indent)
%% 		{}                           %%% Thm head font
%% 		{}                           %%% Punctuation after thm head
%% 		{}                           %%% Space after thm head
%% 		{\thmnumber{#2}\thmnote{ \bfseries(#3)}}%%% Thm head spec
\newtheoremstyle{Question}{3em}{3em} %%% space between body and thm
		{\large\bfseries}                           %%% Thm body font
		{}                           %%% Indent amount (empty = no indent)
		{}                         %%% Thm head font
		{}                          %%% Punctuation after thm head
		{0em}                          %%% Space after thm head
		{}%%% Thm head spec
\theoremstyle{Question}
\newtheorem*{question}{}






\renewcommand{\tilde}{\widetilde}
\renewcommand{\bar}{\overline}
\renewcommand{\hat}{\widehat}
\newcommand{\N}{\mathbb N}
\newcommand{\Z}{\mathbb Z}
\newcommand{\R}{\mathbb R}
\newcommand{\Q}{\mathbb Q}
\newcommand{\C}{\mathbb C}
\newcommand{\V}{\mathbb V}
\newcommand{\I}{\mathbb I}
\newcommand{\A}{\mathbb A}
\renewcommand{\o}{\mathbf o}
\newcommand{\iso}{\simeq}
\newcommand{\ph}{\varphi}
\newcommand{\Cf}{\mathcal{C}}
\newcommand{\IZ}{\mathrm{Int}(\Z)}
\newcommand{\dsum}{\oplus}
\newcommand{\directsum}{\bigoplus}
\newcommand{\union}{\bigcup}
\newcommand{\subgp}{\leq}
\newcommand{\normal}{\trianglelefteq}
\renewcommand{\i}{\mathfrak}
\renewcommand{\a}{\mathfrak{a}}
\renewcommand{\b}{\mathfrak{b}}
\newcommand{\m}{\mathfrak{m}}
\newcommand{\p}{\mathfrak{p}}
\newcommand{\q}{\mathfrak{q}}
\newcommand{\dfn}[1]{\textbf{#1}\index{#1}}
\let\hom\relax
\DeclareMathOperator{\mat}{Mat}
\DeclareMathOperator{\ann}{Ann}
\DeclareMathOperator{\h}{ht}
\DeclareMathOperator{\tr}{tr}
\DeclareMathOperator{\hom}{Hom}
\DeclareMathOperator{\Span}{Span}
\DeclareMathOperator{\spec}{Spec}
\DeclareMathOperator{\maxspec}{MaxSpec}
\DeclareMathOperator{\aut}{Aut}
\DeclareMathOperator{\ass}{Ass}
\DeclareMathOperator{\lcm}{lcm}
\DeclareMathOperator{\ff}{Frac}
\DeclareMathOperator{\im}{Im}
\DeclareMathOperator{\syz}{Syz}
\DeclareMathOperator{\gr}{Gr}
\DeclareMathOperator{\multideg}{multideg}
\renewcommand{\ker}{\mathop{\mathrm{Ker}}\nolimits}
\newcommand{\coker}{\mathop{\mathrm{Coker}}\nolimits}
\newcommand{\lps}{[\hspace{-0.25ex}[}
\newcommand{\rps}{]\hspace{-0.25ex}]}
\newcommand{\into}{\hookrightarrow}
\newcommand{\onto}{\twoheadrightarrow}
\newcommand{\tensor}{\otimes}
\newcommand{\x}{\mathbf{x}}
\newcommand{\X}{\mathbf X}
\newcommand{\Y}{\mathbf Y}
\renewcommand{\k}{\boldsymbol{\kappa}}
\renewcommand{\emptyset}{\varnothing}
\renewcommand{\qedsymbol}{$\blacksquare$}
\renewcommand{\l}{\ell}
\newcommand{\1}{\mathds{1}}
\newcommand{\lto}{\mathop{\longrightarrow\,}\limits}
\newcommand{\rad}{\sqrt}
\newcommand{\hf}{H}
\newcommand{\hs}{H\!S}
\newcommand{\hp}{H\!P}
\renewcommand{\vec}{\mathbf}
\let\temp\phi
\let\phi\varphi
\let\eulerphi\temp


\renewcommand{\epsilon}{\varepsilon}
\renewcommand{\subset}{\subseteq}
\renewcommand{\supset}{\supseteq}
\newcommand{\macaulay}{\normalfont\textsl{Macaulay2}}
\newcommand{\GAP}{\normalfont\textsf{GAP}}
\newcommand{\invlim}{\varprojlim}
\renewcommand{\le}{\leqslant}
\renewcommand{\ge}{\geqslant}
\newcommand{\valpha}{{\boldsymbol\alpha}}
\newcommand{\vbeta}{{\boldsymbol\beta}}
\newcommand{\vgamma}{{\boldsymbol\gamma}}
\newcommand{\dotp}{\bullet}
\newcommand{\lc}{\normalfont\textsc{lc}}
\newcommand{\lt}{\normalfont\textsc{lt}}
\newcommand{\lm}{\normalfont\textsc{lm}}
\newcommand{\from}{\leftarrow}
\newcommand{\transpose}{\intercal}
\newcommand{\grad}{\boldsymbol\nabla}
\newcommand{\curl}{\boldsymbol{\nabla\times}}
\renewcommand{\d}{\, d}
\newcommand{\<}{\langle}
\renewcommand{\>}{\rangle}

%\renewcommand{\proofname}{Sketch of Proof}


\renewenvironment{proof}[1][Proof]
  {\begin{trivlist}\item[\hskip \labelsep \itshape \bfseries #1{}\hspace{2ex}]\upshape}
{\qed\end{trivlist}}

\newenvironment{sketch}[1][Sketch of Proof]
  {\begin{trivlist}\item[\hskip \labelsep \itshape \bfseries #1{}\hspace{2ex}]\upshape}
{\qed\end{trivlist}}



\makeatletter
\renewcommand\section{\@startsection{paragraph}{10}{\z@}%
                                     {-3.25ex\@plus -1ex \@minus -.2ex}%
                                     {1.5ex \@plus .2ex}%
                                     {\normalfont\large\sffamily\bfseries}}
\renewcommand\subsection{\@startsection{subparagraph}{10}{\z@}%
                                    {3.25ex \@plus1ex \@minus.2ex}%
                                    {-1em}%
                                    {\normalfont\normalsize\sffamily\bfseries}}
\makeatother

%% Fix weird index/bib issue.
\makeatletter
\gdef\ttl@savemark{\sectionmark{}}
\makeatother


\makeatletter
%% no number for refs
\newcommand\frontstyle{%
  \def\activitystyle{activity-chapter}
  \def\maketitle{%
    \addtocounter{titlenumber}{1}%
                    {\flushleft\small\sffamily\bfseries\@pretitle\par\vspace{-1.5em}}%
                    {\flushleft\LARGE\sffamily\bfseries\@title \par }%
                    {\vskip .6em\noindent\textit\theabstract\setcounter{problem}{0}\setcounter{sectiontitlenumber}{0}}%
                    \par\vspace{2em}
                    \phantomsection\addcontentsline{toc}{section}{\textbf{\@title}}%
                  }}
\makeatother



\NewEnviron{annotate}{\vspace{-.3cm}\small \itshape \BODY \vspace{.3cm}}


%%%% TIKZ STUFF

%% N-GON code
\tikzset{
    pics/tikzngon/.style={
        code={
        \tikzmath{\xx = #1;\rr=1.7;}
        \draw[ultra thick,rounded corners=.05mm] ({\rr*sin(0*360/\xx)},{\rr*cos(0*360/\xx)})
        \foreach \x in {-1,0,...,\xx}
        {
        -- ({\rr*sin(\x*360/\xx)},{\rr*cos(\x*360/\xx)})
        }
           -- cycle;
  }}}

%% N-GON code (even)
\tikzset{
    pics/tikzegon/.style={
        code={
        \tikzmath{\xx = #1;\rr=1.7;}
        \draw[ultra thick,rounded corners=.05mm] ({\rr*sin(0*360/\xx+180/\xx)},{\rr*cos(0*360/\xx+180/\xx)})
        \foreach \x in {-1,0,...,\xx}
           {
           -- ({\rr*sin(\x*360/\xx+180/\xx)},{\rr*cos(\x*360/\xx+180/\xx)}) 
           }
           -- cycle;
  }}}




%% N-CLOCK code
\tikzset{
    pics/tikznclock/.style={
        code={
        \tikzmath{\xx = #1;\rr=1.7;\dd=.4;}
        \foreach \x in {1,...,\xx}
        \pgfmathtruncatemacro{\xy}{\x-1}
           {
             \node[circle,fill=black,inner sep=0pt, minimum size=13pt,text=white]
             at ({(\rr-\dd)*sin((\x-1)*360/(\xx)},{(\rr-\dd)*cos((\x-1)*360/\xx}) {\normalfont\bfseries\sffamily\small {\xy}};
           }
  \draw[thick] (0,0) circle (\rr);
  }}}



%% barcode from
%% https://tex.stackexchange.com/questions/6895/is-there-a-good-latex-package-for-generating-barcodes
%% NOT CURRENTLY USED!


\def\barcode#1#2#3#4#5#6#7{\begingroup%
  \dimen0=0.1em
  \def\stack##1##2{\oalign{##1\cr\hidewidth##2\hidewidth}}%
  \def\0##1{\kern##1\dimen0}%
  \def\1##1{\vrule height10ex width##1\dimen0}%
  \def\L##1{\ifcase##1\bc3211##1\or\bc2221##1\or\bc2122##1\or\bc1411##1%
    \or\bc1132##1\or\bc1231##1\or\bc1114##1\or\bc1312##1\or\bc1213##1%
    \or\bc3112##1\fi}%
  \def\R##1{\bgroup\let\next\1\let\1\0\let\0\next\L##1\egroup}%
  \def\G##1{\bgroup\let\bc\bcg\L##1\egroup}% reverse
  \def\bc##1##2##3##4##5{\stack{\0##1\1##2\0##3\1##4}##5}%
  \def\bcg##1##2##3##4##5{\stack{\0##4\1##3\0##2\1##1}##5}%
  \def\bcR##1##2##3##4##5##6{\R##1\R##2\R##3\R##4\R##5\R##6\11\01\11\09%
    \endgroup}%
  \stack{\09}#1\11\01\11\L#2%
  \ifcase#1\L#3\L#4\L#5\L#6\L#7\or\L#3\G#4\L#5\G#6\G#7%
    \or\L#3\G#4\G#5\L#6\G#7\or\L#3\G#4\G#5\G#6\L#7%
    \or\G#3\L#4\L#5\G#6\G#7\or\G#3\G#4\L#5\L#6\G#7%
    \or\G#3\G#4\G#5\L#6\L#7\or\G#3\L#4\G#5\L#6\G#7%
    \or\G#3\L#4\G#5\G#6\L#7\or\G#3\G#4\L#5\G#6\L#7%
  \fi\01\11\01\11\01\bcR}


\author{Bart Snapp}

\title{Irreducible polynomials}

\begin{document}
\begin{abstract}
  We discuss how to identify irreducible polynomials.
\end{abstract}
\maketitle

From our previous work, we know that $K[x]/(n(x))$ is a field if and
only if $n(x)$ is an irreducible polynomial. Let's figure out some
strategies to determine if a polynomial is irreducible. In particular,
we are interested in showing that polynomials in $\Q[x]$ are
irreducible.

\section{Degree two or three}

\begin{lemma}[Irreducibility for degree two and three]\label{L:d23}
  Let $K$ be a field and $f(x)\in K[x]$ where $2 \le deg(f(x))\le 3$. In
  this case, $f(x)$ is irreducible if and only if $f(x)$ has no roots
  in $K$.
  \begin{sketch}
    If $2 \le deg(f(x))\le 3$ and $f(x)$ factors, then it must have a
    factor of degree $1$.
  \end{sketch}
\end{lemma}






\section{Changing coefficients}

So far, we've been thinking about polynomials whose coefficients are
field elements. This doesn't have to be the case. Define
\[
\Z[x] = \{\text{polynomials in terms of $x$ with coefficients in
  $\Z$}\}.
\]

When the polynomial has integer coefficients, there is an easy way to
find rational roots.

\begin{lemma}[Rational roots test]\label{L:rr}
  Consider $f(x) \in \Z[x]$,
  \[
  f(x) = c_nx^n + c_{n-1}x^{n-1} + \cdots + c_1 x + c_0.
  \]
  If $\frac{r}{s}\in\Q$ is in lowest terms and $\frac{r}{s}$ is a root
  of $f(x)$, then
  \[
  r | c_0 \quad\text{and}\quad s | c_n.
  \]
  \begin{sketch}
    Evaluate $f\left(\frac{r}{s}\right)$ and clear denominators.
  \end{sketch}
\end{lemma}

Even if the degree of the polynomial you are working with is higher
than two or three, the rational roots test allows you to easily find
linear factors.


\begin{lemma}[Reduction to characteristic $\boldsymbol p$]\label{L:RCP}
  Given a polynomial in $\Z_p[x]$, if $f(x)$ is irreducible in
  $\Z_p[x]$ for some prime $p$, then $f(x)$ is irreducible in $\Z[x]$.
  \begin{sketch}
    Prove the contrapositive of the statement.
  \end{sketch}
\end{lemma}


\begin{lemma}[Gauss' lemma]\index{Gauss' lemma}\label{L:G}
  Given a polynomial $f(x)\in\Z[x]$, if $f(x)$ is irreducible in
  $\Z[x]$, then $f(x)$ is irreducible in $\Q[x]$.
  \begin{proof}
    We'll prove the contrapositive of this statement. Suppose $f(x) =
    g(x)\cdot h(x)$ where $g(x),h(x)\in\Q[x]$. Let $d$ be a common
    denominator for the coefficients of $g(x)$ and $h(x)$. Hence
    \[
    d\cdot f(x) = g'(x)\cdot h'(x)
    \]
    where $g'(x)\in \Z[x]$ and $h'(x) \in\Z[x]$. We may factor
    \[
    d = p_1,\dots, p_n.
    \]
    Since $p_1$ divides $d$, $p_1$ divides the coefficients of
    $g'(x)\cdot h'(x)$.


    We claim that $p_1$ divides all the coefficients of $g'(x)$ or
    $p_1$ divides all the coefficients of $h'(x)$.  Let $a_i$
    represent the coefficients of $g'(x)$ and $b_j$ represent the
    coefficients of $h'(x)$. Seeking a contradiction, suppose 
    \begin{itemize}
      \item $a_n$ be the first coefficient of $g'(x)$ that is not
        divisible by $p_1$.
      \item $b_m$ be the first coefficient of $h'(x)$ that is not
        divisible by $p_1$.
    \end{itemize}
    Writing out the coefficient of $x_{m+n}$ of $d\cdot f(x)$, we have
    \[
    \underbrace{a_0b_{m+n}  + \cdots + a_{n-1}b_{m+1}}_{\text{divisible by $p_1$}} + a_nb_m + \underbrace{a_{n+1}b_{m-1} + \cdots + a_{m+n} b_0}_{\text{divisible by $p_1$}}
    \]
    Since this entire expression is divisible by $p_1$, we conclude
    that $a_nb_m$ is divisible by $p_1$. Now by Euclid's lemma,
    Corollary~\ref{C:EL2}, this means $p_1|a_n$ or $p_1|b_m$. Hence $p_1$
    divides all the coefficients of $g'(x)$ or $p_1$ divides all the
    coefficients of $h'(x)$. This means that
    \[
    g'(x)/p \in \Z[x] \quad\text{or}\quad h'(x)/p \in \Z[x]
    \]
    Repeat for every prime factor of $d$ to see that $f(x)$ factors in
    $\Z[x]$.
  \end{proof}
\end{lemma}


Now we have the following:
\[
\text{Irreducible in $\Z_p[x]$} \Rightarrow \text{Irreducible in $\Z[x]$}  \Rightarrow \text{Irreducible in $\Q[x]$}
\]


\section{Eisenstein's criterion}



\begin{theorem}[Eisenstein's criterion]\index{Eisenstein's criterion}\label{T:ec}
  Given a polynomial $f(x) \in\Z[x]$
  \[
  f(x) = c_nx^n + c_{n-1}x^{n-1} + \cdots + c_1 x + c_0,
  \]
  if there exists a prime $p$ such that
  \begin{enumerate}
  \item $p|c_i$ for all $i \ne n$,
  \item $p\nmid c_n$, and
  \item $p^2\nmid c_0$,
  \end{enumerate}
  then $f(x)$ is irreducible in $\Z[x]$.
  \begin{proof}
    Seeking a contradiction, suppose that $f(x)$ factors in $\Z[x]$ and there exists a prime $p$ such that
  \begin{enumerate}
  \item $p|c_i$ for $i \ne n$,
  \item $p\nmid c_n$,
  \item $p^2\nmid c_0$.
  \end{enumerate}
  Write
  \[
  f(x) = g(x) \cdot h(x)
  \]
  where
  \begin{align*}
    g(x) &= a_rx^r + \cdots + a_1 x+ a_0\\
    h(x) &= b_sx^s + \cdots + b_1 x+ b_0,
  \end{align*}
  and $r + s = n = \deg(f(x))$. Multiplying $g(x)$ and $h(x)$ and
  examining coefficients we find
  \begin{align*}
    c_0 &= a_0b_0\\
    c_1 &= a_0b_1 + a_1 b_0\\
    &\vdots \\
    c_n &= a_r b_s.
  \end{align*}
  If $p| c_0$, then by Euclid's lemma, Corollary~\ref{C:EL2}, we have
    that $p|a_0$ or $p|b_0$, but not both as $p^2\nmid c_0$. WLOG
    assume $p|a_0$ and $p\nmid b_0$.

    Since $p|c_1$, then by Euclid's lemma, Corollary~\ref{C:EL2}, we
    have that $p|a_1$ or $p|b_0$. Since $p\nmid b_0$, we see
    $p|a_1$. Continuing on in this fashion, we see $p|c_i$ for all $i$
    implies $p| a_i$, $i=1,\dots, r$. But this means $p|a_rb_s$ which
    means $p|c_n$, a contradiction.
  \end{proof}
\end{theorem}

Some folks say that a polynomial is \textbf{Eisenstein} if it satisfies
the conditions of Eisenstein's criterion. 

\begin{exercise}\label{E:et}
  Eisenstein's criterion is often combined with the following
  exercise: Prove that if $f(ax+b)$ is irreducible, then $f(x)$ is
  irreducible.


  So now, if you want to show a polynomial $f$ is irreducible, one
  strategy is to see if $f(ax+b)$ is Eisenstein for some values of
  $a$ and $b$.
\end{exercise}


\begin{exercise}
  Prove that $x^4 + 10x + 5$ is irreducible in $\Q[x]$.
\end{exercise}



\begin{exercise}
  Prove that $x^4 + 1$ is irreducible in $\Q[x]$.
  \begin{hint}
    Set $x = 1+y$.
  \end{hint}
\end{exercise}


\begin{exercise}
  Determine whether the following polynomials are irreducible. For those
  are are reducible, give their factorizations into irreducible
  polynomials. For those that are irreducible, give a proof explaining
  your conclusion.
  \begin{enumerate}
  \item $x^2 + x +1$ in $\Z_2[x]$.
  \item $x^3 + x + 1$ in $\Z_3[x]$.
  \item $x^4+1$ in $\Z_5[x]$.
  \item $x^2 + x + 4$ in $\Z_{11}[x]$.
  \item $x^5 + 9x^4 + 12x^2 + 6$ in  $\Q[x]$.
  \item $17 x^3 - 7x^2 + 34x + 1$ in $\Q[x]$.
  \item $x^4 + 10x + 1$ in $\Q[x]$.
  \item $x^4 + 10x^2 + 1$ in $\Q[x]$. This is known as a
    \dfn{Swinnerton-Dyer polynomial}.
  \item $4x^3 -3x -1/2$ in $\Q[x]$.
  \end{enumerate}
  \begin{hint}
    Note, each of these may require different techniques. Some will
    probably require `brute-force.'
  \end{hint}
\end{exercise}


\end{document}

