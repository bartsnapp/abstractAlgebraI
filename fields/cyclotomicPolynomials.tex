\documentclass{ximera}

\author{Bart Snapp}

\usepackage[T1]{fontenc}
\usepackage{stix2}
\usepackage{gillius}
\usepackage{resizegather}
%\usepackage{rsfso} fancy cal
\DeclareMathAlphabet{\mathcal}{OMS}{cmsy}{m}{n} %less fancy cal


\usepackage{multicol}


\usepackage{tikz-cd}
\usepackage{tkz-euclide} %% compass
\usetkzobj{all}  %% tkzCompass
\tikzset{>=stealth}
\tikzcdset{arrow style=tikz}
\usetikzlibrary{math} %% for assigning variables
%\usetikzlibrary{fadings}

\usepackage{colortbl,boldline,makecell} %% group tables


\usepackage[sans]{dsfont}

\usepackage{stmaryrd,pifont}

\graphicspath{
  {./}
  {fields/}
  }     



\let\oldbibliography\thebibliography%% to compact bib
\renewcommand{\thebibliography}[1]{%
  \oldbibliography{#1}%
  \setlength{\itemsep}{0pt}%
}
\renewcommand\refname{} %% no name needed!


\DefineVerbatimEnvironment{macaulay2}{Verbatim}{numbers=left,frame=lines,label=Macaulay2,labelposition=topline}

\DefineVerbatimEnvironment{gap}{Verbatim}{numbers=left,frame=lines,label=GAP,labelposition=topline}

%%% This next bit of code defines all our theorem environments
\makeatletter
\let\c@theorem\relax
\let\c@corollary\relax
%\let\c@example\relax
\makeatother

\let\definition\relax
\let\enddefinition\relax

\let\theorem\relax
\let\endtheorem\relax

\let\proposition\relax
\let\endproposition\relax

\let\exercise\relax
\let\endexercise\relax

\let\question\relax
\let\endquestion\relax

\let\remark\relax
\let\endremark\relax

\let\corollary\relax
\let\endcorollary\relax


\let\example\relax
\let\endexample\relax

\let\warning\relax
\let\endwarning\relax

\let\lemma\relax
\let\endlemma\relax


\let\algorithm\relax
\let\endalgorithm\relax
\usepackage{algpseudocode}

\newtheoremstyle{SlantTheorem}{\topsep}{\topsep}%%% space between body and thm
		{\slshape}                      %%% Thm body font
		{}                              %%% Indent amount (empty = no indent)
		{\bfseries\sffamily}            %%% Thm head font
		{}                              %%% Punctuation after thm head
		{3ex}                           %%% Space after thm head
		{\thmname{#1}\thmnumber{ #2}\thmnote{ \bfseries(#3)}}%%% Thm head spec
\theoremstyle{SlantTheorem}
\newtheorem{theorem}{Theorem}
%\newtheorem{definition}[theorem]{Definition}
%\newtheorem{proposition}[theorem]{Proposition}
%% \newtheorem*{dfnn}{Definition}
%% \newtheorem{ques}{Question}[theorem]
%% \newtheorem*{war}{WARNING}
%% \newtheorem*{cor}{Corollary}
%% \newtheorem*{eg}{Example}
\newtheorem*{remark}{Remark}
\newtheorem*{touchstone}{Touchstone}
\newtheorem{corollary}{Corollary}[theorem]
\newtheorem*{warning}{WARNING}
\newtheorem{example}[corollary]{Example}
\newtheorem{lemma}[theorem]{Lemma}




\newtheoremstyle{Definition}{\topsep}{\topsep}%%% space between body and thm
		{}                              %%% Thm body font
		{}                              %%% Indent amount (empty = no indent)
		{\bfseries\sffamily}            %%% Thm head font
		{}                              %%% Punctuation after thm head
		{3ex}                           %%% Space after thm head
		{\thmname{#1}\thmnumber{ #2}\thmnote{ \bfseries(#3)}}%%% Thm head spec
\theoremstyle{Definition}
\newtheorem{definition}[theorem]{Definition}



\let\algorithm\relax
\let\endalgorithm\relax
\newtheoremstyle{Alg}{\topsep}{\topsep}%%% space between body and thm
		{}                              %%% Thm body font
		{}                              %%% Indent amount (empty = no indent)
		{\bfseries\sffamily}            %%% Thm head font
		{}                              %%% Punctuation after thm head
		{3ex}                           %%% Space after thm head
		{\thmname{#1}\thmnumber{ #2}\thmnote{ \bfseries(#3)}}%%% Thm head spec
\theoremstyle{Alg}
\newtheorem*{algorithm}{Algorithm}
\newtheorem*{construction}{Construction}




\newtheoremstyle{Exercise}{\topsep}{\topsep} %%% space between body and thm
		{}                           %%% Thm body font
		{}                           %%% Indent amount (empty = no indent)
		{\bfseries\sffamily}         %%% Thm head font
		{)}                          %%% Punctuation after thm head
		{ }                          %%% Space after thm head
		{\thmnumber{#2}\thmnote{ \bfseries(#3)}}%%% Thm head spec
\theoremstyle{Exercise}
\newtheorem{exercise}[corollary]{}%[theorem]

%% \newtheoremstyle{Question}{\topsep}{\topsep} %%% space between body and thm
%% 		{\bfseries}                  %%% Thm body font
%% 		{3ex}                        %%% Indent amount (empty = no indent)
%% 		{}                           %%% Thm head font
%% 		{}                           %%% Punctuation after thm head
%% 		{}                           %%% Space after thm head
%% 		{\thmnumber{#2}\thmnote{ \bfseries(#3)}}%%% Thm head spec
\newtheoremstyle{Question}{3em}{3em} %%% space between body and thm
		{\large\bfseries}                           %%% Thm body font
		{}                           %%% Indent amount (empty = no indent)
		{}                         %%% Thm head font
		{}                          %%% Punctuation after thm head
		{0em}                          %%% Space after thm head
		{}%%% Thm head spec
\theoremstyle{Question}
\newtheorem*{question}{}






\renewcommand{\tilde}{\widetilde}
\renewcommand{\bar}{\overline}
\renewcommand{\hat}{\widehat}
\newcommand{\N}{\mathbb N}
\newcommand{\Z}{\mathbb Z}
\newcommand{\R}{\mathbb R}
\newcommand{\Q}{\mathbb Q}
\newcommand{\C}{\mathbb C}
\newcommand{\V}{\mathbb V}
\newcommand{\I}{\mathbb I}
\newcommand{\A}{\mathbb A}
\renewcommand{\o}{\mathbf o}
\newcommand{\iso}{\simeq}
\newcommand{\ph}{\varphi}
\newcommand{\Cf}{\mathcal{C}}
\newcommand{\IZ}{\mathrm{Int}(\Z)}
\newcommand{\dsum}{\oplus}
\newcommand{\directsum}{\bigoplus}
\newcommand{\union}{\bigcup}
\newcommand{\subgp}{\leq}
\newcommand{\normal}{\trianglelefteq}
\renewcommand{\i}{\mathfrak}
\renewcommand{\a}{\mathfrak{a}}
\renewcommand{\b}{\mathfrak{b}}
\newcommand{\m}{\mathfrak{m}}
\newcommand{\p}{\mathfrak{p}}
\newcommand{\q}{\mathfrak{q}}
\newcommand{\dfn}[1]{\textbf{#1}\index{#1}}
\let\hom\relax
\DeclareMathOperator{\mat}{Mat}
\DeclareMathOperator{\ann}{Ann}
\DeclareMathOperator{\h}{ht}
\DeclareMathOperator{\tr}{tr}
\DeclareMathOperator{\hom}{Hom}
\DeclareMathOperator{\Span}{Span}
\DeclareMathOperator{\spec}{Spec}
\DeclareMathOperator{\maxspec}{MaxSpec}
\DeclareMathOperator{\aut}{Aut}
\DeclareMathOperator{\ass}{Ass}
\DeclareMathOperator{\lcm}{lcm}
\DeclareMathOperator{\ff}{Frac}
\DeclareMathOperator{\im}{Im}
\DeclareMathOperator{\syz}{Syz}
\DeclareMathOperator{\gr}{Gr}
\DeclareMathOperator{\multideg}{multideg}
\renewcommand{\ker}{\mathop{\mathrm{Ker}}\nolimits}
\newcommand{\coker}{\mathop{\mathrm{Coker}}\nolimits}
\newcommand{\lps}{[\hspace{-0.25ex}[}
\newcommand{\rps}{]\hspace{-0.25ex}]}
\newcommand{\into}{\hookrightarrow}
\newcommand{\onto}{\twoheadrightarrow}
\newcommand{\tensor}{\otimes}
\newcommand{\x}{\mathbf{x}}
\newcommand{\X}{\mathbf X}
\newcommand{\Y}{\mathbf Y}
\renewcommand{\k}{\boldsymbol{\kappa}}
\renewcommand{\emptyset}{\varnothing}
\renewcommand{\qedsymbol}{$\blacksquare$}
\renewcommand{\l}{\ell}
\newcommand{\1}{\mathds{1}}
\newcommand{\lto}{\mathop{\longrightarrow\,}\limits}
\newcommand{\rad}{\sqrt}
\newcommand{\hf}{H}
\newcommand{\hs}{H\!S}
\newcommand{\hp}{H\!P}
\renewcommand{\vec}{\mathbf}
\let\temp\phi
\let\phi\varphi
\let\eulerphi\temp


\renewcommand{\epsilon}{\varepsilon}
\renewcommand{\subset}{\subseteq}
\renewcommand{\supset}{\supseteq}
\newcommand{\macaulay}{\normalfont\textsl{Macaulay2}}
\newcommand{\GAP}{\normalfont\textsf{GAP}}
\newcommand{\invlim}{\varprojlim}
\renewcommand{\le}{\leqslant}
\renewcommand{\ge}{\geqslant}
\newcommand{\valpha}{{\boldsymbol\alpha}}
\newcommand{\vbeta}{{\boldsymbol\beta}}
\newcommand{\vgamma}{{\boldsymbol\gamma}}
\newcommand{\dotp}{\bullet}
\newcommand{\lc}{\normalfont\textsc{lc}}
\newcommand{\lt}{\normalfont\textsc{lt}}
\newcommand{\lm}{\normalfont\textsc{lm}}
\newcommand{\from}{\leftarrow}
\newcommand{\transpose}{\intercal}
\newcommand{\grad}{\boldsymbol\nabla}
\newcommand{\curl}{\boldsymbol{\nabla\times}}
\renewcommand{\d}{\, d}
\newcommand{\<}{\langle}
\renewcommand{\>}{\rangle}

%\renewcommand{\proofname}{Sketch of Proof}


\renewenvironment{proof}[1][Proof]
  {\begin{trivlist}\item[\hskip \labelsep \itshape \bfseries #1{}\hspace{2ex}]\upshape}
{\qed\end{trivlist}}

\newenvironment{sketch}[1][Sketch of Proof]
  {\begin{trivlist}\item[\hskip \labelsep \itshape \bfseries #1{}\hspace{2ex}]\upshape}
{\qed\end{trivlist}}



\makeatletter
\renewcommand\section{\@startsection{paragraph}{10}{\z@}%
                                     {-3.25ex\@plus -1ex \@minus -.2ex}%
                                     {1.5ex \@plus .2ex}%
                                     {\normalfont\large\sffamily\bfseries}}
\renewcommand\subsection{\@startsection{subparagraph}{10}{\z@}%
                                    {3.25ex \@plus1ex \@minus.2ex}%
                                    {-1em}%
                                    {\normalfont\normalsize\sffamily\bfseries}}
\makeatother

%% Fix weird index/bib issue.
\makeatletter
\gdef\ttl@savemark{\sectionmark{}}
\makeatother


\makeatletter
%% no number for refs
\newcommand\frontstyle{%
  \def\activitystyle{activity-chapter}
  \def\maketitle{%
    \addtocounter{titlenumber}{1}%
                    {\flushleft\small\sffamily\bfseries\@pretitle\par\vspace{-1.5em}}%
                    {\flushleft\LARGE\sffamily\bfseries\@title \par }%
                    {\vskip .6em\noindent\textit\theabstract\setcounter{problem}{0}\setcounter{sectiontitlenumber}{0}}%
                    \par\vspace{2em}
                    \phantomsection\addcontentsline{toc}{section}{\textbf{\@title}}%
                  }}
\makeatother



\NewEnviron{annotate}{\vspace{-.3cm}\small \itshape \BODY \vspace{.3cm}}


%%%% TIKZ STUFF

%% N-GON code
\tikzset{
    pics/tikzngon/.style={
        code={
        \tikzmath{\xx = #1;\rr=1.7;}
        \draw[ultra thick,rounded corners=.05mm] ({\rr*sin(0*360/\xx)},{\rr*cos(0*360/\xx)})
        \foreach \x in {-1,0,...,\xx}
        {
        -- ({\rr*sin(\x*360/\xx)},{\rr*cos(\x*360/\xx)})
        }
           -- cycle;
  }}}

%% N-GON code (even)
\tikzset{
    pics/tikzegon/.style={
        code={
        \tikzmath{\xx = #1;\rr=1.7;}
        \draw[ultra thick,rounded corners=.05mm] ({\rr*sin(0*360/\xx+180/\xx)},{\rr*cos(0*360/\xx+180/\xx)})
        \foreach \x in {-1,0,...,\xx}
           {
           -- ({\rr*sin(\x*360/\xx+180/\xx)},{\rr*cos(\x*360/\xx+180/\xx)}) 
           }
           -- cycle;
  }}}




%% N-CLOCK code
\tikzset{
    pics/tikznclock/.style={
        code={
        \tikzmath{\xx = #1;\rr=1.7;\dd=.4;}
        \foreach \x in {1,...,\xx}
        \pgfmathtruncatemacro{\xy}{\x-1}
           {
             \node[circle,fill=black,inner sep=0pt, minimum size=13pt,text=white]
             at ({(\rr-\dd)*sin((\x-1)*360/(\xx)},{(\rr-\dd)*cos((\x-1)*360/\xx}) {\normalfont\bfseries\sffamily\small {\xy}};
           }
  \draw[thick] (0,0) circle (\rr);
  }}}



%% barcode from
%% https://tex.stackexchange.com/questions/6895/is-there-a-good-latex-package-for-generating-barcodes
%% NOT CURRENTLY USED!


\def\barcode#1#2#3#4#5#6#7{\begingroup%
  \dimen0=0.1em
  \def\stack##1##2{\oalign{##1\cr\hidewidth##2\hidewidth}}%
  \def\0##1{\kern##1\dimen0}%
  \def\1##1{\vrule height10ex width##1\dimen0}%
  \def\L##1{\ifcase##1\bc3211##1\or\bc2221##1\or\bc2122##1\or\bc1411##1%
    \or\bc1132##1\or\bc1231##1\or\bc1114##1\or\bc1312##1\or\bc1213##1%
    \or\bc3112##1\fi}%
  \def\R##1{\bgroup\let\next\1\let\1\0\let\0\next\L##1\egroup}%
  \def\G##1{\bgroup\let\bc\bcg\L##1\egroup}% reverse
  \def\bc##1##2##3##4##5{\stack{\0##1\1##2\0##3\1##4}##5}%
  \def\bcg##1##2##3##4##5{\stack{\0##4\1##3\0##2\1##1}##5}%
  \def\bcR##1##2##3##4##5##6{\R##1\R##2\R##3\R##4\R##5\R##6\11\01\11\09%
    \endgroup}%
  \stack{\09}#1\11\01\11\L#2%
  \ifcase#1\L#3\L#4\L#5\L#6\L#7\or\L#3\G#4\L#5\G#6\G#7%
    \or\L#3\G#4\G#5\L#6\G#7\or\L#3\G#4\G#5\G#6\L#7%
    \or\G#3\L#4\L#5\G#6\G#7\or\G#3\G#4\L#5\L#6\G#7%
    \or\G#3\G#4\G#5\L#6\L#7\or\G#3\L#4\G#5\L#6\G#7%
    \or\G#3\L#4\G#5\G#6\L#7\or\G#3\G#4\L#5\G#6\L#7%
  \fi\01\11\01\11\01\bcR}


\title{Cyclotomic polynomials}

\begin{document}
\begin{abstract}
  We introduce cyclotomic, or ``circle-dividing'' polynomials.
\end{abstract}
\maketitle

\begin{theorem}[DeMoivre's Theorem]\index{DeMoivre's Theorem} 
Let $\alpha$ be any nonzero complex number. Then the equation
\[
x^n - \alpha = 0
\]
has exactly $n$ distinct roots, all of which are complex numbers.
\begin{proof} 
We will give a direct proof of DeMoivre's Theorem \textit{without}
appealing to the Fundamental Theorem of Algebra.
\begin{enumerate}
\item Prove that $e^{i \theta} = \cos(\theta) + i \sin(\theta)$.
\item Prove that any complex number $a + bi$ can be expressed as $R
  e^{i\theta}$ where $R\in \R$ and $\theta \in [0,2\pi)$.
\item Explain why if $\alpha = R(\cos(\theta) +i \sin(\theta))$, then 
\[
\sqrt[n]{R} \cdot  e^{\frac{i(\theta + 2k\pi)}{n}}
\]
is a root of $x^n - \alpha$.
\item Plot the roots on the unit circle, how many are there?
\end{enumerate}
Explain how we have proved the theorem.
\end{proof}
\end{theorem}


\begin{definition}\index{root of unity}\index{zetan@$\zeta_n$} 
For $n\in \N$, let 
\[
\zeta_n = e^{2\pi i/n}
\]
In this case we call $\zeta_n^k$ an \textbf{$\boldsymbol{n}$th-root of unity}, where $k\in
\N$.
\end{definition}

\begin{exercise} 
Prove that $\zeta_n^k\bar{\zeta_n^k} = 1$, where $\bar{\zeta_n^k}$ is
the complex conjugate of $\zeta_n^k$. Explain what this means
geometrically.
\end{exercise}

\begin{exercise}\label{E:KN} Prove that the set 
\[
K_n = \{\zeta_n^k = e^{2k\pi i/n}: k\in \N\}
\]
is a cyclic group under multiplication. Further prove that $K_n \iso
\Z_n$ as groups.
\end{exercise}



\begin{exercise}
If $n = 2m$, prove that 
\[
x^n -1 = (x-1)(x+1)q_1(x) \cdots q_{m-1}(x)
\]
where $q_i(x)$ are distinct irreducible polynomials in $\R[x]$. Hint, see (\ref{E:RQ}).
\end{exercise}

\begin{exercise}
If $n = 2m+1$, prove that 
\[
x^n -1 = (x-1)q_1(x) \cdots q_{m}(x)
\]
where $q_i(x)$ are distinct irreducible polynomials in $\R[x]$. Hint, see (\ref{E:RQ}).
\end{exercise}


\begin{definition}\index{primitive root of unity} 
A number $\zeta$ is called a \textbf{primitive} $n$th root of unity if
\[
\zeta^n = 1 \qquad\text{and}\qquad \zeta^{m} \ne 1
\]
for all $m< n$ where $m,n\in\N$.
\end{definition}


\begin{theorem} 
A number $\zeta_n^k$, where $1\le k\le n$, is a primitive $n$th root
of unity if and only if $(k,n) = 1$.

\begin{proof} 
$(\Rightarrow)$ Seek a contradiction supposing that $\zeta_n^k$ is a
  primitive $n$th root of unity and that $(k,n) \ne 1$.

$(\Leftarrow)$ Seek a contradiction supposing $(k,n) = 1$ and
  $\zeta_n^k$ is not a primitive $n$th root of unity. Use the fact
  that the order of $\zeta_n$ in $K_n$ is $n$ and Euclid's Lemma
  (\ref{E:EL2}).\index{Euclid's Lemma}
\end{proof}
\end{theorem}



\begin{exercise}\label{E:KP}
Prove that given a prime $p\nmid n$, the primitive $n$th roots of
unity are given by
\[
\{\zeta_n^{kp}: 1\le k \le n\}.
\]
\end{exercise}


\begin{exercise}\index{Euler phi-function@Euler $\phi$-function}\index{phi@$\phi$}
Prove that there are exactly $\phi(n)$ primitive roots of unity in
$\C$.
\end{exercise}



\begin{definition}\index{cyclotomic polynomial}
The $n$th \dfn{cyclotomic polynomial} is defined as:
\[
\Phi_n(x) = \prod_{\substack{(k,n)=1 \\ k<n}}(x-\zeta_n^k)
\]
\end{definition}

\begin{exercise}
  What is the degree of $\Phi_n(x)$?
\end{exercise}

\begin{exercise}
  Carefully explain each step shown below:
  \begin{align*}
    x^n - 1 &= \prod_{k=1}^n (x-\zeta_n^k) \\
    &=\prod_{d|n}\prod_{\substack{(k,d) = 1\\ \zeta_d^k \in K_n}}(x-\zeta_d^k) \\
    &=\prod_{d|n} \Phi_d(x).
  \end{align*}
\end{exercise}

\begin{corollary}
  The upshot of the above exercise is that 
  \[
  \Phi_n(x) = \frac{x^n-1}{\prod_{d|n} \Phi_d(x)}
  \]
\end{corollary}

\begin{exercise}
  Given a prime $p$, compute $\Phi_p(x)$. 
\end{exercise}

\begin{exercise}
  Compute $\Phi_n$ for $n = 1, \dots, 10$.
\end{exercise}

\begin{exercise}
  Prove that 
  \[
  n = \sum_{d|n} \phi(d).
  \]
\end{exercise}


\begin{theorem} 
  The $n$th cyclotomic polynomial $\Phi_n(x)$ is a monic polynomial in
  $\Z[x]$ for all $n\in \N$.
  
  \begin{proof} Proceed by induction on $n$.
    \begin{enumerate}
    \item Check the case when $n = 1$. 
    \item Now suppose our statement holds for all values less than $n$.
    \item Noting that 
      \[
      x^n - 1 = \prod_{d|n}\Phi_d(x) = \Phi_n(x)\cdot \prod_{\substack{d|n\\ d \ne n}}\Phi_d(x).
      \]
      Hence $\prod_{\substack{d|n\\ d \ne n}}\Phi_d(x)$ divides $x^n-1$ in
      $\Q(\zeta_n)[x]$. Explain why this is also true in $\Q[x]$. Big hint,
      see the Division Theorem for polynomials, see
      (\ref{T:divThmPoly}). Use this theorem in $\Q(\zeta_n)[x]$ and $\Q[x]$
      utilizing uniqueness!
    \end{enumerate}
    Explain how to finish the proof. Hint, use (\ref{D:cont}).
  \end{proof}
\end{theorem}

\begin{theorem} 
  The $n$th cyclotomic polynomial $\Phi_n(x)$ is irreducible in $\Q[x]$
  for all $n\in \N$.
  \begin{proof} 
    Seeking a contradiction, suppose that $\Phi_n(x)$ is not irreducible.
    \begin{enumerate}
    \item Explain why we may assume that $\Phi_n(x)$ is reducible in
      $\Z[x]$.
    \item Explain why we may assume that $\Phi_n(x) = f(x)\cdot g(x)$
      where $f(x)$ and $g(x)$ are monic polynomials in $\Z[x]$ with
      \[
      \deg(f) < \phi(n) \qquad \text{and}\qquad \deg(g) < \phi(n).
      \]
      Hint, use (\ref{D:cont}).
\item Explain why we may assume that $f$ is a minimal polynomial for
  $\zeta_n$.
    \end{enumerate}
    Now we claim that given a prime $p \nmid n$, $\zeta_n^p$ is also a
    root of $f(x)$. Further seek another contradiction, and suppose that
    $\zeta_n^p$ is not a root of $f(x)$.
    \begin{enumerate}
    \item Explain why $\zeta_n^p$ is a root of $g(x)$.
    \item Explain why $f(x) | g(x^p)$ and write $g(x^p) = f(x) \cdot
      h(x)$, where $h(x)\in \Q[x]$.
    \item Explain why we may conclude that $h(x)$ is a monic polynomial
      $\Z[x]$, use (\ref{D:cont}).
    \item Reduce the equation $g(x^p) = f(x) \cdot h(x)$ modulo $p$,
      writing the image of the polynomials $f$, $g$, and $h$ as $\bar{f}$,
      $\bar{g}$, and $\bar{h}$:
      \[
      \bar{g}(x^p) = \bar{f}(x) \cdot \bar{h}(x)
      \]\index{Freshman Binomial Theorem}\index{Fermat's Little Theorem}
    \item Explain how to use the Freshman Binomial Theorem (\ref{E:FBT}),
      or Fermat's Little Theorem (\ref{T:FLT}), to conclude that
      \[
      \bar{g}(x)^p = \bar{f}(x) \cdot \bar{h}(x)
      \]
    \item Explain how the fact that $\Z_p[x]$ is a UFD allows us to
      conclude that $\bar{f}$ and $\bar{g}$ have a common factor.
    \item Explain why we must conclude that $x^n-1$ has a multiple root
      modulo $p$. Further explain why $\bar{f}$ and $\bar{g}$ both have
  $x$ as a factor, see (\ref{E:mr}). Explain why this is a
      contradiction.
    \end{enumerate}
    Thus we conclude that $\zeta^p_n$ is a root of $f(x)$. Since this
    argument will work for any prime $p$ with $p\nmid n$, we can show that
    every primitive $n$th root of unity is a root of $f$, see (\ref{E:KP}). This will
    show that $\Phi_n(x) = f$, and is hence irreducible.
  \end{proof}
\end{theorem}




\end{document}
