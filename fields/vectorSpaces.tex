\documentclass{ximera}

\usepackage[T1]{fontenc}
\usepackage{stix2}
\usepackage{gillius}
\usepackage{resizegather}
%\usepackage{rsfso} fancy cal
\DeclareMathAlphabet{\mathcal}{OMS}{cmsy}{m}{n} %less fancy cal


\usepackage{multicol}


\usepackage{tikz-cd}
\usepackage{tkz-euclide} %% compass
\usetkzobj{all}  %% tkzCompass
\tikzset{>=stealth}
\tikzcdset{arrow style=tikz}
\usetikzlibrary{math} %% for assigning variables
%\usetikzlibrary{fadings}

\usepackage{colortbl,boldline,makecell} %% group tables


\usepackage[sans]{dsfont}

\usepackage{stmaryrd,pifont}

\graphicspath{
  {./}
  {fields/}
  }     



\let\oldbibliography\thebibliography%% to compact bib
\renewcommand{\thebibliography}[1]{%
  \oldbibliography{#1}%
  \setlength{\itemsep}{0pt}%
}
\renewcommand\refname{} %% no name needed!


\DefineVerbatimEnvironment{macaulay2}{Verbatim}{numbers=left,frame=lines,label=Macaulay2,labelposition=topline}

\DefineVerbatimEnvironment{gap}{Verbatim}{numbers=left,frame=lines,label=GAP,labelposition=topline}

%%% This next bit of code defines all our theorem environments
\makeatletter
\let\c@theorem\relax
\let\c@corollary\relax
%\let\c@example\relax
\makeatother

\let\definition\relax
\let\enddefinition\relax

\let\theorem\relax
\let\endtheorem\relax

\let\proposition\relax
\let\endproposition\relax

\let\exercise\relax
\let\endexercise\relax

\let\question\relax
\let\endquestion\relax

\let\remark\relax
\let\endremark\relax

\let\corollary\relax
\let\endcorollary\relax


\let\example\relax
\let\endexample\relax

\let\warning\relax
\let\endwarning\relax

\let\lemma\relax
\let\endlemma\relax


\let\algorithm\relax
\let\endalgorithm\relax
\usepackage{algpseudocode}

\newtheoremstyle{SlantTheorem}{\topsep}{\topsep}%%% space between body and thm
		{\slshape}                      %%% Thm body font
		{}                              %%% Indent amount (empty = no indent)
		{\bfseries\sffamily}            %%% Thm head font
		{}                              %%% Punctuation after thm head
		{3ex}                           %%% Space after thm head
		{\thmname{#1}\thmnumber{ #2}\thmnote{ \bfseries(#3)}}%%% Thm head spec
\theoremstyle{SlantTheorem}
\newtheorem{theorem}{Theorem}
%\newtheorem{definition}[theorem]{Definition}
%\newtheorem{proposition}[theorem]{Proposition}
%% \newtheorem*{dfnn}{Definition}
%% \newtheorem{ques}{Question}[theorem]
%% \newtheorem*{war}{WARNING}
%% \newtheorem*{cor}{Corollary}
%% \newtheorem*{eg}{Example}
\newtheorem*{remark}{Remark}
\newtheorem*{touchstone}{Touchstone}
\newtheorem{corollary}{Corollary}[theorem]
\newtheorem*{warning}{WARNING}
\newtheorem{example}[corollary]{Example}
\newtheorem{lemma}[theorem]{Lemma}




\newtheoremstyle{Definition}{\topsep}{\topsep}%%% space between body and thm
		{}                              %%% Thm body font
		{}                              %%% Indent amount (empty = no indent)
		{\bfseries\sffamily}            %%% Thm head font
		{}                              %%% Punctuation after thm head
		{3ex}                           %%% Space after thm head
		{\thmname{#1}\thmnumber{ #2}\thmnote{ \bfseries(#3)}}%%% Thm head spec
\theoremstyle{Definition}
\newtheorem{definition}[theorem]{Definition}



\let\algorithm\relax
\let\endalgorithm\relax
\newtheoremstyle{Alg}{\topsep}{\topsep}%%% space between body and thm
		{}                              %%% Thm body font
		{}                              %%% Indent amount (empty = no indent)
		{\bfseries\sffamily}            %%% Thm head font
		{}                              %%% Punctuation after thm head
		{3ex}                           %%% Space after thm head
		{\thmname{#1}\thmnumber{ #2}\thmnote{ \bfseries(#3)}}%%% Thm head spec
\theoremstyle{Alg}
\newtheorem*{algorithm}{Algorithm}
\newtheorem*{construction}{Construction}




\newtheoremstyle{Exercise}{\topsep}{\topsep} %%% space between body and thm
		{}                           %%% Thm body font
		{}                           %%% Indent amount (empty = no indent)
		{\bfseries\sffamily}         %%% Thm head font
		{)}                          %%% Punctuation after thm head
		{ }                          %%% Space after thm head
		{\thmnumber{#2}\thmnote{ \bfseries(#3)}}%%% Thm head spec
\theoremstyle{Exercise}
\newtheorem{exercise}[corollary]{}%[theorem]

%% \newtheoremstyle{Question}{\topsep}{\topsep} %%% space between body and thm
%% 		{\bfseries}                  %%% Thm body font
%% 		{3ex}                        %%% Indent amount (empty = no indent)
%% 		{}                           %%% Thm head font
%% 		{}                           %%% Punctuation after thm head
%% 		{}                           %%% Space after thm head
%% 		{\thmnumber{#2}\thmnote{ \bfseries(#3)}}%%% Thm head spec
\newtheoremstyle{Question}{3em}{3em} %%% space between body and thm
		{\large\bfseries}                           %%% Thm body font
		{}                           %%% Indent amount (empty = no indent)
		{}                         %%% Thm head font
		{}                          %%% Punctuation after thm head
		{0em}                          %%% Space after thm head
		{}%%% Thm head spec
\theoremstyle{Question}
\newtheorem*{question}{}






\renewcommand{\tilde}{\widetilde}
\renewcommand{\bar}{\overline}
\renewcommand{\hat}{\widehat}
\newcommand{\N}{\mathbb N}
\newcommand{\Z}{\mathbb Z}
\newcommand{\R}{\mathbb R}
\newcommand{\Q}{\mathbb Q}
\newcommand{\C}{\mathbb C}
\newcommand{\V}{\mathbb V}
\newcommand{\I}{\mathbb I}
\newcommand{\A}{\mathbb A}
\renewcommand{\o}{\mathbf o}
\newcommand{\iso}{\simeq}
\newcommand{\ph}{\varphi}
\newcommand{\Cf}{\mathcal{C}}
\newcommand{\IZ}{\mathrm{Int}(\Z)}
\newcommand{\dsum}{\oplus}
\newcommand{\directsum}{\bigoplus}
\newcommand{\union}{\bigcup}
\newcommand{\subgp}{\leq}
\newcommand{\normal}{\trianglelefteq}
\renewcommand{\i}{\mathfrak}
\renewcommand{\a}{\mathfrak{a}}
\renewcommand{\b}{\mathfrak{b}}
\newcommand{\m}{\mathfrak{m}}
\newcommand{\p}{\mathfrak{p}}
\newcommand{\q}{\mathfrak{q}}
\newcommand{\dfn}[1]{\textbf{#1}\index{#1}}
\let\hom\relax
\DeclareMathOperator{\mat}{Mat}
\DeclareMathOperator{\ann}{Ann}
\DeclareMathOperator{\h}{ht}
\DeclareMathOperator{\tr}{tr}
\DeclareMathOperator{\hom}{Hom}
\DeclareMathOperator{\Span}{Span}
\DeclareMathOperator{\spec}{Spec}
\DeclareMathOperator{\maxspec}{MaxSpec}
\DeclareMathOperator{\aut}{Aut}
\DeclareMathOperator{\ass}{Ass}
\DeclareMathOperator{\lcm}{lcm}
\DeclareMathOperator{\ff}{Frac}
\DeclareMathOperator{\im}{Im}
\DeclareMathOperator{\syz}{Syz}
\DeclareMathOperator{\gr}{Gr}
\DeclareMathOperator{\multideg}{multideg}
\renewcommand{\ker}{\mathop{\mathrm{Ker}}\nolimits}
\newcommand{\coker}{\mathop{\mathrm{Coker}}\nolimits}
\newcommand{\lps}{[\hspace{-0.25ex}[}
\newcommand{\rps}{]\hspace{-0.25ex}]}
\newcommand{\into}{\hookrightarrow}
\newcommand{\onto}{\twoheadrightarrow}
\newcommand{\tensor}{\otimes}
\newcommand{\x}{\mathbf{x}}
\newcommand{\X}{\mathbf X}
\newcommand{\Y}{\mathbf Y}
\renewcommand{\k}{\boldsymbol{\kappa}}
\renewcommand{\emptyset}{\varnothing}
\renewcommand{\qedsymbol}{$\blacksquare$}
\renewcommand{\l}{\ell}
\newcommand{\1}{\mathds{1}}
\newcommand{\lto}{\mathop{\longrightarrow\,}\limits}
\newcommand{\rad}{\sqrt}
\newcommand{\hf}{H}
\newcommand{\hs}{H\!S}
\newcommand{\hp}{H\!P}
\renewcommand{\vec}{\mathbf}
\let\temp\phi
\let\phi\varphi
\let\eulerphi\temp


\renewcommand{\epsilon}{\varepsilon}
\renewcommand{\subset}{\subseteq}
\renewcommand{\supset}{\supseteq}
\newcommand{\macaulay}{\normalfont\textsl{Macaulay2}}
\newcommand{\GAP}{\normalfont\textsf{GAP}}
\newcommand{\invlim}{\varprojlim}
\renewcommand{\le}{\leqslant}
\renewcommand{\ge}{\geqslant}
\newcommand{\valpha}{{\boldsymbol\alpha}}
\newcommand{\vbeta}{{\boldsymbol\beta}}
\newcommand{\vgamma}{{\boldsymbol\gamma}}
\newcommand{\dotp}{\bullet}
\newcommand{\lc}{\normalfont\textsc{lc}}
\newcommand{\lt}{\normalfont\textsc{lt}}
\newcommand{\lm}{\normalfont\textsc{lm}}
\newcommand{\from}{\leftarrow}
\newcommand{\transpose}{\intercal}
\newcommand{\grad}{\boldsymbol\nabla}
\newcommand{\curl}{\boldsymbol{\nabla\times}}
\renewcommand{\d}{\, d}
\newcommand{\<}{\langle}
\renewcommand{\>}{\rangle}

%\renewcommand{\proofname}{Sketch of Proof}


\renewenvironment{proof}[1][Proof]
  {\begin{trivlist}\item[\hskip \labelsep \itshape \bfseries #1{}\hspace{2ex}]\upshape}
{\qed\end{trivlist}}

\newenvironment{sketch}[1][Sketch of Proof]
  {\begin{trivlist}\item[\hskip \labelsep \itshape \bfseries #1{}\hspace{2ex}]\upshape}
{\qed\end{trivlist}}



\makeatletter
\renewcommand\section{\@startsection{paragraph}{10}{\z@}%
                                     {-3.25ex\@plus -1ex \@minus -.2ex}%
                                     {1.5ex \@plus .2ex}%
                                     {\normalfont\large\sffamily\bfseries}}
\renewcommand\subsection{\@startsection{subparagraph}{10}{\z@}%
                                    {3.25ex \@plus1ex \@minus.2ex}%
                                    {-1em}%
                                    {\normalfont\normalsize\sffamily\bfseries}}
\makeatother

%% Fix weird index/bib issue.
\makeatletter
\gdef\ttl@savemark{\sectionmark{}}
\makeatother


\makeatletter
%% no number for refs
\newcommand\frontstyle{%
  \def\activitystyle{activity-chapter}
  \def\maketitle{%
    \addtocounter{titlenumber}{1}%
                    {\flushleft\small\sffamily\bfseries\@pretitle\par\vspace{-1.5em}}%
                    {\flushleft\LARGE\sffamily\bfseries\@title \par }%
                    {\vskip .6em\noindent\textit\theabstract\setcounter{problem}{0}\setcounter{sectiontitlenumber}{0}}%
                    \par\vspace{2em}
                    \phantomsection\addcontentsline{toc}{section}{\textbf{\@title}}%
                  }}
\makeatother



\NewEnviron{annotate}{\vspace{-.3cm}\small \itshape \BODY \vspace{.3cm}}


%%%% TIKZ STUFF

%% N-GON code
\tikzset{
    pics/tikzngon/.style={
        code={
        \tikzmath{\xx = #1;\rr=1.7;}
        \draw[ultra thick,rounded corners=.05mm] ({\rr*sin(0*360/\xx)},{\rr*cos(0*360/\xx)})
        \foreach \x in {-1,0,...,\xx}
        {
        -- ({\rr*sin(\x*360/\xx)},{\rr*cos(\x*360/\xx)})
        }
           -- cycle;
  }}}

%% N-GON code (even)
\tikzset{
    pics/tikzegon/.style={
        code={
        \tikzmath{\xx = #1;\rr=1.7;}
        \draw[ultra thick,rounded corners=.05mm] ({\rr*sin(0*360/\xx+180/\xx)},{\rr*cos(0*360/\xx+180/\xx)})
        \foreach \x in {-1,0,...,\xx}
           {
           -- ({\rr*sin(\x*360/\xx+180/\xx)},{\rr*cos(\x*360/\xx+180/\xx)}) 
           }
           -- cycle;
  }}}




%% N-CLOCK code
\tikzset{
    pics/tikznclock/.style={
        code={
        \tikzmath{\xx = #1;\rr=1.7;\dd=.4;}
        \foreach \x in {1,...,\xx}
        \pgfmathtruncatemacro{\xy}{\x-1}
           {
             \node[circle,fill=black,inner sep=0pt, minimum size=13pt,text=white]
             at ({(\rr-\dd)*sin((\x-1)*360/(\xx)},{(\rr-\dd)*cos((\x-1)*360/\xx}) {\normalfont\bfseries\sffamily\small {\xy}};
           }
  \draw[thick] (0,0) circle (\rr);
  }}}



%% barcode from
%% https://tex.stackexchange.com/questions/6895/is-there-a-good-latex-package-for-generating-barcodes
%% NOT CURRENTLY USED!


\def\barcode#1#2#3#4#5#6#7{\begingroup%
  \dimen0=0.1em
  \def\stack##1##2{\oalign{##1\cr\hidewidth##2\hidewidth}}%
  \def\0##1{\kern##1\dimen0}%
  \def\1##1{\vrule height10ex width##1\dimen0}%
  \def\L##1{\ifcase##1\bc3211##1\or\bc2221##1\or\bc2122##1\or\bc1411##1%
    \or\bc1132##1\or\bc1231##1\or\bc1114##1\or\bc1312##1\or\bc1213##1%
    \or\bc3112##1\fi}%
  \def\R##1{\bgroup\let\next\1\let\1\0\let\0\next\L##1\egroup}%
  \def\G##1{\bgroup\let\bc\bcg\L##1\egroup}% reverse
  \def\bc##1##2##3##4##5{\stack{\0##1\1##2\0##3\1##4}##5}%
  \def\bcg##1##2##3##4##5{\stack{\0##4\1##3\0##2\1##1}##5}%
  \def\bcR##1##2##3##4##5##6{\R##1\R##2\R##3\R##4\R##5\R##6\11\01\11\09%
    \endgroup}%
  \stack{\09}#1\11\01\11\L#2%
  \ifcase#1\L#3\L#4\L#5\L#6\L#7\or\L#3\G#4\L#5\G#6\G#7%
    \or\L#3\G#4\G#5\L#6\G#7\or\L#3\G#4\G#5\G#6\L#7%
    \or\G#3\L#4\L#5\G#6\G#7\or\G#3\G#4\L#5\L#6\G#7%
    \or\G#3\G#4\G#5\L#6\L#7\or\G#3\L#4\G#5\L#6\G#7%
    \or\G#3\L#4\G#5\G#6\L#7\or\G#3\G#4\L#5\G#6\L#7%
  \fi\01\11\01\11\01\bcR}


\author{Bart Snapp}

\title{Vector spaces}

\begin{document}
\begin{abstract}
  We review vector spaces.
\end{abstract}
\maketitle



\begin{definition}\index{K-vector space@$K$-vector space}\index{vector space@$K$-vector space}
  A \textbf{$\boldsymbol{K}$-vector space} is an Abelian group $(V,+)$
  along with a field $K$ such that we may multiply group elements by
  field elements, meaning that there is a binary operation $-\cdot-:
  K\times V \to V$ such that if $\nu,\mu\in V$ and $a,b,\in K$ we
  have:
\begin{description}
\item[Compatibility with scalars] $(ab)\cdot \nu = a\cdot (b\cdot \nu)$.
\item[Vectors distribute over scalars] $(a+b)\cdot \nu =
  a\cdot\nu + b\cdot \nu$.
\item[Scalars distribute over vectors] $a\cdot (\nu+\mu) =
  a\cdot \nu + a\cdot \mu$.
\item[Identity is respected] $1_K\cdot \nu = \nu$.
\end{description}
In this case, elements of the group $V$ are called \dfn{vectors} and
elements of the field $K$ are called \dfn{scalars}.
\end{definition}

\begin{exercise}
  Let $V$ be a $K$-vector space. If $\nu\in V$, prove that
  $0_K\cdot \nu = \vec{0}$.
\end{exercise}


\begin{example}[Euclidean space]
  Perhaps the most obvious vector space would be $\R^3$. This is an
  $\R$-vector space. It is also a $\Q$-vector space.
\end{example}



\begin{example}[Complex numbers]
  Recall $\C = \{a+bi:a,b\in\R\}$. The set $\C$ is an $\C$-vector
  space, an $\R$-vector space, and a $\Q$-vector space.
\end{example}


\begin{example}[Polynomials]
  Let $K[x]$ be the set of all formal sums
  \[
  c_nx^n + c_{n-1}x^{n-1} + \dots + c_1 x + c_0
  \]
  where $n$ is a nonnegative integer and each $c_i \in K$. In this
  case, $K[x]$ is a $K$-vector space.
\end{example}




\begin{definition}
  Let $V$ be a $K$-vector space. A subset $W\subset V$ is a
  \textbf{$\boldsymbol K$-vector} \dfn{subspace} of $V$ if $W\subgp V$ and $W$
  is also a $K$-vector space.
\end{definition}

\begin{lemma}[Subspace criterion]\index{subspace criterion}
  Let $V$ be a $K$-vector space. $W\subset K$ is a subspace of $V$ if
  and only if
  \begin{enumerate}
  \item $W\ne \emptyset$.
  \item $W$ is closed under multiplication by scalars.
  \item $W$ is closed under vector addition.
  \end{enumerate}
  \begin{sketch}
    Check the definition of a vector space.
  \end{sketch}
\end{lemma}


\begin{exercise}
  Let $V$ be a $K$-vector space with subspaces $U$ and $W$. Prove that
  $U\cap W$ is a $K$-vector subspace of $V$.
\end{exercise}

\begin{definition}
  Let $V$ and $W$ be $K$-vector spaces. A \dfn{linear transformation}
  for $\nu,\mu\in V$ and $s\in K$ is a function $T:V\to W$ such that
    \begin{enumerate}
    \item $T(\nu+\mu) = T(\nu)+T(\mu)$.
    \item $T(s \nu) = sT(\nu)$.
    \end{enumerate}
\end{definition}

\begin{remark}
  We proved that every linear transformation is a matrix in
  Lemma~\ref{L:MT}.
\end{remark}

\begin{definition}\index{isomorphic!vector spaces}
  Two $K$-vector spaces $V$ and $W$ are said to be \textbf{isomorphic
    vector spaces} if there exists a bijective linear transformation
  $T:V\to W$. In this case we write, $V\iso W$.
\end{definition}

\begin{exercise}\index{kernel!linear transformation}
  Let $V$ and $W$ be $K$-vector spaces and let $T:V\to W$ be a linear
  transformation. Define the \textbf{kernel} of $T$ as follows:
  \[
  \ker(T) = \{\nu\in V: T(\nu) = 0\}\subset V.
  \]
  Prove that $\ker(T)$ is a $K$-vector subspace of $V$.
\end{exercise}

\begin{exercise}\index{image!linear transformation}
  Let $V$ and $W$ be $K$-vector spaces and let $T:V\to W$ be a linear
  transformation. Define the \textbf{image} of $T$ as follows:
  \[
  \im(T) = \{T(\nu): \nu\in V\}\subset W.
  \]
  Prove that $\im(T)$ is a $K$-vector subspace of $W$.
\end{exercise}







\begin{definition}
  Given a set of vectors $S$, in a $K$-vector space, $V$, the
  \dfn{span} of the vectors in $S$ is
  \[
  \Span(S) = \left\{\sum_{i=1}^n a_i\sigma_i:\text{$n\in \N$,
    $\sigma_i\in S$, and $a_i\in K$}\right\}.
  \]
  If $\Span(S) = V$, then we say $S$ is a \dfn{spanning set}.
\end{definition}





\begin{definition}
  Given a $K$-vector space $V$, a finite set of vectors
  \[
  \{\lambda_1,\dots,\lambda_n\}
  \]
  is said to be \dfn{linearly independent} if
  \[
  a_1\lambda_1 + a_2\lambda_2 +\cdots + a_n\lambda_n = 0\quad \Rightarrow \quad a_1= \cdots =a_n = 0.
  \]
  A finite set of vectors is set to be \dfn{linearly dependent} if
  they are not linearly independent.
\end{definition}



\begin{exercise}
  Let $V$ be a $K$-vector space. Let $S= \{\sigma_1,\sigma_2,
  \dots,\sigma_n\}$. Prove that if $\tau\notin\Span(S)$, that $\tau$
  and $\sigma_i$ are linearly independent.
\end{exercise}


%% \begin{exercise}
%%   Let $V$ be a $K$-vector space. Let $L= \{\lambda_1,\lambda_2,
%%   \dots,\lambda_n\}$. Prove that if , that $\tau$
%%   and $\sigma_i$ are linearly independent.
%% \end{exercise}



\begin{theorem}[Bases equivalences]
  Define a partial ordering on sets where $S \preceq T$ if $S\subset
  T$. Let $B= \{\beta_1,\dots,\beta_n\}$ be a finite set of vectors in
  a $K$-vector space $V$. The following are equivalent:
  \begin{enumerate}
  \item $B = \min_{\preceq}\{S\subset V:\Span(S) = V\}$.
  \item $B$ is a linearly independent spanning set of vectors.
  \item $B = \max_{\preceq}\{S\subset V:\text{$S$ is a linearly independent set of vectors}\}$.
  \end{enumerate}
  Any finite set of vectors $B \subset V$ satisfying any of the
  equivalent conditions above is called a \dfn{basis} for $V$.
  \begin{proof}
    We will prove this in a ``circle.''


    $(\mathrm a)\Rightarrow(\mathrm b)$ We will assume that
    \[
    B = \min_{\preceq}\{S\subset V:\Span(S) = V\}.
    \]
    We must show that $B$ is a linearly independent spanning set of
    vectors.  If the set $B$ is minimal with respect to spanning $V$,
    no subset of $B$ will span $V$. In particular, if
    \[
    B = \{\beta_1,\beta_2,\dots,\beta_n\},
    \]
    then
    \[
    \beta_n \notin \Span(\beta_1,\dots,\beta_{n-1}).
    \]
    This means $\beta_n$ and $\beta_i$ are linearly independent for $i
    = 1,\dots, n-1$. Repeating this process with each $\beta_i$, we
    see that $B$ is a spanning set of linearly independent vectors.


    $(\mathrm b)\Rightarrow(\mathrm c)$ We will assume that $B$ is a
    linearly independent spanning set of vectors. We must show that
    \[
    B = \max_{\preceq}\{S\subset V:\text{$S$ is a linearly independent set of vectors}\}.
    \]
    If $B$ is a set of linearly independent vectors that span $V$,
    then no vector can be added to this list while maintaining linear
    independence. Hence $B$ is a maximal linearly independent set.
    

    $(\mathrm c)\Rightarrow(\mathrm a)$ We will assume that
    \[
    B = \max_{\preceq}\{S\subset V:\text{$S$ is a linearly independent set of vectors}\}.
    \]
    We must show that
    \[
    B = \min_{\preceq}\{S\subset V:\Span(S) = V\}.
    \]
    If $B$ is a maximal linearly independent set, no vectors can be
    added while maintaining linear independence, thus $\Span(B) =
    V$. If any element were removed from $B$, it would no longer span
    $V$, hence $B$ is a minimal spanning set.
  \end{proof}
\end{theorem}



\begin{theorem}[Bases have the same order]
  Let $V$ be a $K$-vector space with bases
  \begin{align*}
    B &= \{\beta_1,\beta_2,\dots, \beta_n\}, \\
    C &= \{\gamma_1,\gamma_2, \dots, \gamma_m\}.
  \end{align*}
  In this case, $n = m$.
  \begin{proof}
    Let $B= \{\beta_1,\beta_2,\dots, \beta_n\}$ be a basis for $V$.
    Suppose you want to construct another basis, $C$.  Let $\gamma_1$
    be the first vector you choose to be in $C$. Since $\Span(B) = V$,
    we may write
    \begin{align}
      \gamma_1 &= c_1\beta_1 + c_2\beta_2 + \dots + c_n\beta_n \tag{$\bigstar$}\\
      \beta_1 &= c_1^{-1}(\gamma_1 -  c_2\beta_2 - \dots - c_n\beta_n.\notag
    \end{align}
    Note, WLOG $c_1 \ne 0$, as if it was zero, we could renumber our
    basis vectors.  Letting $B_1 = \{\gamma_1, \beta_2,\dots,
    \beta_n\}$, this means that $\beta_1\in \Span(B_1)$. We claim that
    $B_1$ is a set of linearly independent vectors. Suppose that for
    $a_i\in K$
    \[
    a_1 \gamma_1 + a_2 \beta_2 + \dots + a_n \beta_n = 0.
    \]
    Using $(\bigstar)$ to substitute, write
    \begin{align*}
      a_1(c_1\beta_1 + c_2\beta_2 + \dots + c_n\beta_n) + a_2 \beta_2 + \dots + a_n \beta_n &= 0\\
      (a_1c_1) \beta_1 + (a_2+a_1c_2)\beta_2 + \dots + (a_n+a_1c_n)\beta_n &= 0.
    \end{align*}
    Since $B$ is a set of linearly independent vectors, we conclude
    \begin{align*}
      (a_1c_1) &=0\\
      (a_2+a_1c_2) &=0\\
      &\vdots \\      
      (a_n+a_1c_n) &=0.
    \end{align*}
    Since $c_1\ne 0$, we see that $a_i = 0$, and hence $B_1$ is a set
    of linearly independent vectors. Inductively repeating this
    process for $\gamma_2\notin \Span(\gamma_1)$, and then for
    $\gamma_3 \notin \Span(\gamma_1,\gamma_2)$, we construct sets of
    linearly independent spanning vectors
    \begin{align*}
      B &= \{\beta_1,\beta_2,\dots,\beta_n\},\\
      B_1 &= \{\gamma_1,\beta_2,\dots,\beta_n\},\\
      B_2 &= \{\gamma_1,\gamma_2,\dots,\beta_n\},\\
      &\vdots \\
      C = B_n &= \{\gamma_1,\gamma_2,\dots,\gamma_n\}.
    \end{align*}
    Since $C$ is a set spanning linearly independent vectors, we
    cannot add another linearly independent vector, thus $|C| = |B|$.
  \end{proof}
\end{theorem}


Here is a picture that attempts to convey the situation:
\[
\begin{tikzpicture}
  \draw[white,thin,shading = axis, top color = white, bottom color = gray] (-2,2) -- (0,0)-- (2,2) -- (-2,2) -- cycle;
  \draw[white,thin,shading = axis, bottom color = white, top color = gray] (-2,-2) -- (0,0)-- (2,-2) -- (-2,-2) -- cycle;
  \filldraw (0,0) circle (3pt);
  \node[right] at (0,0) {$B$ is linearly independent and spans};
  \node at (0,1.5) {$\scriptstyle \Span(B) = V$};
  \node at (0,-1.5) {\tiny $B$  is linearly independent};
\end{tikzpicture}
\]
The cone above $B$ are all sets of vectors that span $V$. The cone
below $B$ are all sets of vectors that are linearly independent.



\begin{definition}
  Let $B$ be a basis for a $K$-vector space $V$. The \dfn{vector space
    dimension} of $V$ is the order of $B$,
  \[
  \dim_K(V) = |B|.
  \]
  If a vector space does not have a finite basis, then we say it is
  \dfn{infinite dimensional}.
\end{definition}


\begin{example}[Euclidean space]
  The $\R$-vector space $\R^3$ has dimension $3$ since it has basis vectors
  \[
  \begin{bmatrix}
    1\\
    0\\
    0
  \end{bmatrix},\quad
  \begin{bmatrix}
    0\\
    1\\
    0
  \end{bmatrix},\quad
   \begin{bmatrix}
    0\\
    0\\
    1
  \end{bmatrix}.
  \]
\end{example}


\begin{example}[Complex numbers]
   Recall $\C = \{a+bi:a,b\in\R\}$. The complex numbers are a vector
   space over several different fields. Here we have
  \[
  \dim_{\C}(\C) = 1, \quad \dim_{\R}(\C) = 2, \quad \dim_{\Q}(\C) =\infty.
  \]
\end{example}

\begin{exercise}
  Find a basis for $\C$ as a $\C$-vector space. Find a basis for $\C$
  as an $\R$-vector space. In each case, prove your answer is correct.
\end{exercise}



\begin{example}[Polynomials of fixed degree]
  Let $\mathcal{P}_3\subset \Q[x]$ be the polynomials up to and including degree
  $3$. In this case $\dim_{\Q}(\Q[x]) = 4$.
\end{example}

\begin{exercise}
  Find a basis for $\mathcal{P}_3\subset \Q[x]$, the polynomials up to and
  including degree $3$, as a $\Q$-vector space. Prove your answer is
  correct.
\end{exercise}


\begin{theorem}[Bases and unique representation]
  Let $V$ be a $K$-vector space and $B = \{\beta_1,\dots,\beta_n$ be a
  basis for $V$. If
  \begin{align*}
  \alpha = a_1\beta_1 + a_2\beta_2 + \dots + a_n\beta_n\\
  \alpha = c_1\beta_1 + c_2\beta_2 + \dots + c_n\beta_n,
  \end{align*}
  where $a_i, c_i\in K$, then $a_i = c_i$ for $i =1,\dots, n$.
  \begin{sketch}
    Subtract the two equations above, combine like terms, and use the
    fact that a basis is a linearly independent set of vectors.
  \end{sketch}
\end{theorem}


\begin{corollary}
  If $V$ is an $n$ dimensional $K$-vector space, then $V\iso K^n$.
\end{corollary}

\begin{exercise}
  Let $V$ be a $K$-vector space of dimension $4$. Let $U$ and $W$ be
  subspaces of $V$. If $\dim_K(U) = 3$ and $\dim_K(V) =3$, what could
  be the value of $\dim_K(W\cap V)$.
\end{exercise}

\begin{exercise}
  Let $V$ be a $\Z_5$-vector space of dimension $3$. How many elements
  are in $\V$?
\end{exercise}



\begin{theorem}[Quotient vector spaces]
  Given a $K$-vector space $V$ and a subspace $W\subset V$, the set of
  left cosets of $W$ form a $K$-vector space under coset
  addition. This $K$-vector space is denoted by $V/W$ and is
  pronounced ``$V$ modulo $W$.''
  \begin{proof}
    First note that by Theorem\ref{T:quotient}, $V/W$ is an Abelian
    group under coset addition. Now we must show that multiplication
    by field elements is \index{well-defined}well-defined and that
    $V/W$ is a $K$-vector space. Suppose that
    \[
    \nu + W = \mu + W.
    \]
    We must show that if $a\in K$,
    \[
    a(\nu + W) = a(\mu + W).
    \]
    Since $aW = W$, this is true. All other conditions for $V/W$ to be
    a vector space are inherited from the fact that $V$ is a $K$-vector
    space.
  \end{proof}
\end{theorem}

\begin{corollary}[Dimensions and quotients]
  Let $V$ be a $K$-vector space and $W\subset V$ be a subspace. In
  this case,
  \[
  \dim_K(V) = \dim_K(W) + \dim_K(V/W)
  \]
\end{corollary}

%% \begin{example}[$\R^n$]
%% \end{example}


\begin{corollary}[Dimension and linear transformations]
  Let $V$ and $W$ be a $K$-vector spaces and let $T:V\to W$ be a
  linear transformation. In this case
  \[
  \dim_K(V) = \dim_K(\ker(T)) + \dim_K(\im(T)).
  \]
  \begin{sketch}
    Use Noether's isomorphism theorem, Theorem~\ref{T:NI}.
  \end{sketch}
\end{corollary}


\end{document}
