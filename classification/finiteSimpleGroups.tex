\documentclass{ximera}

\usepackage[T1]{fontenc}
\usepackage{stix2}
\usepackage{gillius}
\usepackage{resizegather}
%\usepackage{rsfso} fancy cal
\DeclareMathAlphabet{\mathcal}{OMS}{cmsy}{m}{n} %less fancy cal


\usepackage{multicol}


\usepackage{tikz-cd}
\usepackage{tkz-euclide} %% compass
\usetkzobj{all}  %% tkzCompass
\tikzset{>=stealth}
\tikzcdset{arrow style=tikz}
\usetikzlibrary{math} %% for assigning variables
%\usetikzlibrary{fadings}

\usepackage{colortbl,boldline,makecell} %% group tables


\usepackage[sans]{dsfont}

\usepackage{stmaryrd,pifont}

\graphicspath{
  {./}
  {fields/}
  }     



\let\oldbibliography\thebibliography%% to compact bib
\renewcommand{\thebibliography}[1]{%
  \oldbibliography{#1}%
  \setlength{\itemsep}{0pt}%
}
\renewcommand\refname{} %% no name needed!


\DefineVerbatimEnvironment{macaulay2}{Verbatim}{numbers=left,frame=lines,label=Macaulay2,labelposition=topline}

\DefineVerbatimEnvironment{gap}{Verbatim}{numbers=left,frame=lines,label=GAP,labelposition=topline}

%%% This next bit of code defines all our theorem environments
\makeatletter
\let\c@theorem\relax
\let\c@corollary\relax
%\let\c@example\relax
\makeatother

\let\definition\relax
\let\enddefinition\relax

\let\theorem\relax
\let\endtheorem\relax

\let\proposition\relax
\let\endproposition\relax

\let\exercise\relax
\let\endexercise\relax

\let\question\relax
\let\endquestion\relax

\let\remark\relax
\let\endremark\relax

\let\corollary\relax
\let\endcorollary\relax


\let\example\relax
\let\endexample\relax

\let\warning\relax
\let\endwarning\relax

\let\lemma\relax
\let\endlemma\relax


\let\algorithm\relax
\let\endalgorithm\relax
\usepackage{algpseudocode}

\newtheoremstyle{SlantTheorem}{\topsep}{\topsep}%%% space between body and thm
		{\slshape}                      %%% Thm body font
		{}                              %%% Indent amount (empty = no indent)
		{\bfseries\sffamily}            %%% Thm head font
		{}                              %%% Punctuation after thm head
		{3ex}                           %%% Space after thm head
		{\thmname{#1}\thmnumber{ #2}\thmnote{ \bfseries(#3)}}%%% Thm head spec
\theoremstyle{SlantTheorem}
\newtheorem{theorem}{Theorem}
%\newtheorem{definition}[theorem]{Definition}
%\newtheorem{proposition}[theorem]{Proposition}
%% \newtheorem*{dfnn}{Definition}
%% \newtheorem{ques}{Question}[theorem]
%% \newtheorem*{war}{WARNING}
%% \newtheorem*{cor}{Corollary}
%% \newtheorem*{eg}{Example}
\newtheorem*{remark}{Remark}
\newtheorem*{touchstone}{Touchstone}
\newtheorem{corollary}{Corollary}[theorem]
\newtheorem*{warning}{WARNING}
\newtheorem{example}[corollary]{Example}
\newtheorem{lemma}[theorem]{Lemma}




\newtheoremstyle{Definition}{\topsep}{\topsep}%%% space between body and thm
		{}                              %%% Thm body font
		{}                              %%% Indent amount (empty = no indent)
		{\bfseries\sffamily}            %%% Thm head font
		{}                              %%% Punctuation after thm head
		{3ex}                           %%% Space after thm head
		{\thmname{#1}\thmnumber{ #2}\thmnote{ \bfseries(#3)}}%%% Thm head spec
\theoremstyle{Definition}
\newtheorem{definition}[theorem]{Definition}



\let\algorithm\relax
\let\endalgorithm\relax
\newtheoremstyle{Alg}{\topsep}{\topsep}%%% space between body and thm
		{}                              %%% Thm body font
		{}                              %%% Indent amount (empty = no indent)
		{\bfseries\sffamily}            %%% Thm head font
		{}                              %%% Punctuation after thm head
		{3ex}                           %%% Space after thm head
		{\thmname{#1}\thmnumber{ #2}\thmnote{ \bfseries(#3)}}%%% Thm head spec
\theoremstyle{Alg}
\newtheorem*{algorithm}{Algorithm}
\newtheorem*{construction}{Construction}




\newtheoremstyle{Exercise}{\topsep}{\topsep} %%% space between body and thm
		{}                           %%% Thm body font
		{}                           %%% Indent amount (empty = no indent)
		{\bfseries\sffamily}         %%% Thm head font
		{)}                          %%% Punctuation after thm head
		{ }                          %%% Space after thm head
		{\thmnumber{#2}\thmnote{ \bfseries(#3)}}%%% Thm head spec
\theoremstyle{Exercise}
\newtheorem{exercise}[corollary]{}%[theorem]

%% \newtheoremstyle{Question}{\topsep}{\topsep} %%% space between body and thm
%% 		{\bfseries}                  %%% Thm body font
%% 		{3ex}                        %%% Indent amount (empty = no indent)
%% 		{}                           %%% Thm head font
%% 		{}                           %%% Punctuation after thm head
%% 		{}                           %%% Space after thm head
%% 		{\thmnumber{#2}\thmnote{ \bfseries(#3)}}%%% Thm head spec
\newtheoremstyle{Question}{3em}{3em} %%% space between body and thm
		{\large\bfseries}                           %%% Thm body font
		{}                           %%% Indent amount (empty = no indent)
		{}                         %%% Thm head font
		{}                          %%% Punctuation after thm head
		{0em}                          %%% Space after thm head
		{}%%% Thm head spec
\theoremstyle{Question}
\newtheorem*{question}{}






\renewcommand{\tilde}{\widetilde}
\renewcommand{\bar}{\overline}
\renewcommand{\hat}{\widehat}
\newcommand{\N}{\mathbb N}
\newcommand{\Z}{\mathbb Z}
\newcommand{\R}{\mathbb R}
\newcommand{\Q}{\mathbb Q}
\newcommand{\C}{\mathbb C}
\newcommand{\V}{\mathbb V}
\newcommand{\I}{\mathbb I}
\newcommand{\A}{\mathbb A}
\renewcommand{\o}{\mathbf o}
\newcommand{\iso}{\simeq}
\newcommand{\ph}{\varphi}
\newcommand{\Cf}{\mathcal{C}}
\newcommand{\IZ}{\mathrm{Int}(\Z)}
\newcommand{\dsum}{\oplus}
\newcommand{\directsum}{\bigoplus}
\newcommand{\union}{\bigcup}
\newcommand{\subgp}{\leq}
\newcommand{\normal}{\trianglelefteq}
\renewcommand{\i}{\mathfrak}
\renewcommand{\a}{\mathfrak{a}}
\renewcommand{\b}{\mathfrak{b}}
\newcommand{\m}{\mathfrak{m}}
\newcommand{\p}{\mathfrak{p}}
\newcommand{\q}{\mathfrak{q}}
\newcommand{\dfn}[1]{\textbf{#1}\index{#1}}
\let\hom\relax
\DeclareMathOperator{\mat}{Mat}
\DeclareMathOperator{\ann}{Ann}
\DeclareMathOperator{\h}{ht}
\DeclareMathOperator{\tr}{tr}
\DeclareMathOperator{\hom}{Hom}
\DeclareMathOperator{\Span}{Span}
\DeclareMathOperator{\spec}{Spec}
\DeclareMathOperator{\maxspec}{MaxSpec}
\DeclareMathOperator{\aut}{Aut}
\DeclareMathOperator{\ass}{Ass}
\DeclareMathOperator{\lcm}{lcm}
\DeclareMathOperator{\ff}{Frac}
\DeclareMathOperator{\im}{Im}
\DeclareMathOperator{\syz}{Syz}
\DeclareMathOperator{\gr}{Gr}
\DeclareMathOperator{\multideg}{multideg}
\renewcommand{\ker}{\mathop{\mathrm{Ker}}\nolimits}
\newcommand{\coker}{\mathop{\mathrm{Coker}}\nolimits}
\newcommand{\lps}{[\hspace{-0.25ex}[}
\newcommand{\rps}{]\hspace{-0.25ex}]}
\newcommand{\into}{\hookrightarrow}
\newcommand{\onto}{\twoheadrightarrow}
\newcommand{\tensor}{\otimes}
\newcommand{\x}{\mathbf{x}}
\newcommand{\X}{\mathbf X}
\newcommand{\Y}{\mathbf Y}
\renewcommand{\k}{\boldsymbol{\kappa}}
\renewcommand{\emptyset}{\varnothing}
\renewcommand{\qedsymbol}{$\blacksquare$}
\renewcommand{\l}{\ell}
\newcommand{\1}{\mathds{1}}
\newcommand{\lto}{\mathop{\longrightarrow\,}\limits}
\newcommand{\rad}{\sqrt}
\newcommand{\hf}{H}
\newcommand{\hs}{H\!S}
\newcommand{\hp}{H\!P}
\renewcommand{\vec}{\mathbf}
\let\temp\phi
\let\phi\varphi
\let\eulerphi\temp


\renewcommand{\epsilon}{\varepsilon}
\renewcommand{\subset}{\subseteq}
\renewcommand{\supset}{\supseteq}
\newcommand{\macaulay}{\normalfont\textsl{Macaulay2}}
\newcommand{\GAP}{\normalfont\textsf{GAP}}
\newcommand{\invlim}{\varprojlim}
\renewcommand{\le}{\leqslant}
\renewcommand{\ge}{\geqslant}
\newcommand{\valpha}{{\boldsymbol\alpha}}
\newcommand{\vbeta}{{\boldsymbol\beta}}
\newcommand{\vgamma}{{\boldsymbol\gamma}}
\newcommand{\dotp}{\bullet}
\newcommand{\lc}{\normalfont\textsc{lc}}
\newcommand{\lt}{\normalfont\textsc{lt}}
\newcommand{\lm}{\normalfont\textsc{lm}}
\newcommand{\from}{\leftarrow}
\newcommand{\transpose}{\intercal}
\newcommand{\grad}{\boldsymbol\nabla}
\newcommand{\curl}{\boldsymbol{\nabla\times}}
\renewcommand{\d}{\, d}
\newcommand{\<}{\langle}
\renewcommand{\>}{\rangle}

%\renewcommand{\proofname}{Sketch of Proof}


\renewenvironment{proof}[1][Proof]
  {\begin{trivlist}\item[\hskip \labelsep \itshape \bfseries #1{}\hspace{2ex}]\upshape}
{\qed\end{trivlist}}

\newenvironment{sketch}[1][Sketch of Proof]
  {\begin{trivlist}\item[\hskip \labelsep \itshape \bfseries #1{}\hspace{2ex}]\upshape}
{\qed\end{trivlist}}



\makeatletter
\renewcommand\section{\@startsection{paragraph}{10}{\z@}%
                                     {-3.25ex\@plus -1ex \@minus -.2ex}%
                                     {1.5ex \@plus .2ex}%
                                     {\normalfont\large\sffamily\bfseries}}
\renewcommand\subsection{\@startsection{subparagraph}{10}{\z@}%
                                    {3.25ex \@plus1ex \@minus.2ex}%
                                    {-1em}%
                                    {\normalfont\normalsize\sffamily\bfseries}}
\makeatother

%% Fix weird index/bib issue.
\makeatletter
\gdef\ttl@savemark{\sectionmark{}}
\makeatother


\makeatletter
%% no number for refs
\newcommand\frontstyle{%
  \def\activitystyle{activity-chapter}
  \def\maketitle{%
    \addtocounter{titlenumber}{1}%
                    {\flushleft\small\sffamily\bfseries\@pretitle\par\vspace{-1.5em}}%
                    {\flushleft\LARGE\sffamily\bfseries\@title \par }%
                    {\vskip .6em\noindent\textit\theabstract\setcounter{problem}{0}\setcounter{sectiontitlenumber}{0}}%
                    \par\vspace{2em}
                    \phantomsection\addcontentsline{toc}{section}{\textbf{\@title}}%
                  }}
\makeatother



\NewEnviron{annotate}{\vspace{-.3cm}\small \itshape \BODY \vspace{.3cm}}


%%%% TIKZ STUFF

%% N-GON code
\tikzset{
    pics/tikzngon/.style={
        code={
        \tikzmath{\xx = #1;\rr=1.7;}
        \draw[ultra thick,rounded corners=.05mm] ({\rr*sin(0*360/\xx)},{\rr*cos(0*360/\xx)})
        \foreach \x in {-1,0,...,\xx}
        {
        -- ({\rr*sin(\x*360/\xx)},{\rr*cos(\x*360/\xx)})
        }
           -- cycle;
  }}}

%% N-GON code (even)
\tikzset{
    pics/tikzegon/.style={
        code={
        \tikzmath{\xx = #1;\rr=1.7;}
        \draw[ultra thick,rounded corners=.05mm] ({\rr*sin(0*360/\xx+180/\xx)},{\rr*cos(0*360/\xx+180/\xx)})
        \foreach \x in {-1,0,...,\xx}
           {
           -- ({\rr*sin(\x*360/\xx+180/\xx)},{\rr*cos(\x*360/\xx+180/\xx)}) 
           }
           -- cycle;
  }}}




%% N-CLOCK code
\tikzset{
    pics/tikznclock/.style={
        code={
        \tikzmath{\xx = #1;\rr=1.7;\dd=.4;}
        \foreach \x in {1,...,\xx}
        \pgfmathtruncatemacro{\xy}{\x-1}
           {
             \node[circle,fill=black,inner sep=0pt, minimum size=13pt,text=white]
             at ({(\rr-\dd)*sin((\x-1)*360/(\xx)},{(\rr-\dd)*cos((\x-1)*360/\xx}) {\normalfont\bfseries\sffamily\small {\xy}};
           }
  \draw[thick] (0,0) circle (\rr);
  }}}



%% barcode from
%% https://tex.stackexchange.com/questions/6895/is-there-a-good-latex-package-for-generating-barcodes
%% NOT CURRENTLY USED!


\def\barcode#1#2#3#4#5#6#7{\begingroup%
  \dimen0=0.1em
  \def\stack##1##2{\oalign{##1\cr\hidewidth##2\hidewidth}}%
  \def\0##1{\kern##1\dimen0}%
  \def\1##1{\vrule height10ex width##1\dimen0}%
  \def\L##1{\ifcase##1\bc3211##1\or\bc2221##1\or\bc2122##1\or\bc1411##1%
    \or\bc1132##1\or\bc1231##1\or\bc1114##1\or\bc1312##1\or\bc1213##1%
    \or\bc3112##1\fi}%
  \def\R##1{\bgroup\let\next\1\let\1\0\let\0\next\L##1\egroup}%
  \def\G##1{\bgroup\let\bc\bcg\L##1\egroup}% reverse
  \def\bc##1##2##3##4##5{\stack{\0##1\1##2\0##3\1##4}##5}%
  \def\bcg##1##2##3##4##5{\stack{\0##4\1##3\0##2\1##1}##5}%
  \def\bcR##1##2##3##4##5##6{\R##1\R##2\R##3\R##4\R##5\R##6\11\01\11\09%
    \endgroup}%
  \stack{\09}#1\11\01\11\L#2%
  \ifcase#1\L#3\L#4\L#5\L#6\L#7\or\L#3\G#4\L#5\G#6\G#7%
    \or\L#3\G#4\G#5\L#6\G#7\or\L#3\G#4\G#5\G#6\L#7%
    \or\G#3\L#4\L#5\G#6\G#7\or\G#3\G#4\L#5\L#6\G#7%
    \or\G#3\G#4\G#5\L#6\L#7\or\G#3\L#4\G#5\L#6\G#7%
    \or\G#3\L#4\G#5\G#6\L#7\or\G#3\G#4\L#5\G#6\L#7%
  \fi\01\11\01\11\01\bcR}


\author{Bart Snapp}

\title{Finite simple groups}

\begin{document}
\begin{abstract}
  We give a historical overview of the classification of the finite
  simple groups. 
\end{abstract}
\maketitle


\begin{definition}
  A group $G$ is called \dfn{simple}, if the only normal subgroups of
  $G$ are $\{e\}$ and $G$.
\end{definition}


\begin{exercise}
  Prove that every group of prime order is simple.
\end{exercise}


\section{Timeline of the proof}


Now, to give you an idea of how much work went into this
classification, we will copy (almost directly) from this
\link[Wikipedia]{https://en.wikipedia.org/wiki/Classification\_of\_finite\_simple\_groups\#Timeline\_of\_the\_proof}
article. Many of the items in the list below are taken from Solomon
(2001). The date given is usually the publication date of the complete
proof of a result, which is sometimes several years later than the
proof or first announcement of the result, so some of the items appear
in the ``wrong'' order.

\begin{description}
\item[1832] Galois introduces normal subgroups and finds the simple
  groups $A_n$ for $n\ge 5$ and $PSL_2(\Z_p)$ for $p\ge 5$.
\item[1854]	Cayley defines abstract groups.
\item[1861] Mathieu describes the first two Mathieu groups $M_{11}$,
  $M_{12}$, the first sporadic simple groups, and announces the
  existence of $M_{24}$.
\item[1870] Jordan lists some simple groups: the alternating and
  projective special linear ones, and emphasizes the importance of the
  simple groups.
\item[1872] Sylow proves the Sylow theorems.
\item[1873] Mathieu introduces three more Mathieu groups $M_{22}$,
  $M_{23}$, $M_{24}$.
\item[1892] Otto H\"older proves that the order of any non-Abelian
  finite simple group must be a product of at least four (not
  necessarily distinct) primes, and asks for a classification of
  finite simple groups.
\item[1893] Cole classifies simple groups of order up to $660$.
\item[1896]	Frobenius and Burnside begin the study of character theory of finite groups.
\item[1899] Burnside classifies the simple groups such that the
  centralizer of every involution is a non-trivial elementary Abelian
  $2$-group.
\item[1901] Frobenius proves that a Frobenius group has a Frobenius
  kernel, so in particular is not simple.
\item[1901] Dickson defines classical groups over arbitrary finite
  fields, and exceptional groups of type $G_2$ over fields of odd
  characteristic.
\item[1901] Dickson introduces the exceptional finite simple groups of
  type $E_6$.
\item[1904] Burnside uses character theory to prove Burnside's theorem
  that the order of any non-Abelian finite simple group must be
  divisible by at least $3$ distinct primes.
\item[1905] Dickson introduces simple groups of type $G_2$ over fields
  of even characteristic.
\item[1911] Burnside conjectures that every non-Abelian finite simple
  group has even order.
\item[1928] Philip Hall proves the existence of Hall subgroups of
  solvable groups.
\item[1933] Hall begins his study of $p$-groups.
\item[1935] Brauer begins the study of modular characters.
\item[1936] Zassenhaus classifies finite sharply $3$-transitive
  permutation groups.
\item[1938] Fitting introduces the Fitting subgroup and proves
  Fitting's theorem that for solvable groups the Fitting subgroup
  contains its centralizer.
\item[1942] Brauer describes the modular characters of a group
  divisible by a prime to the first power.
\item[1954]	Brauer classifies simple groups with $GL_2(\Z_p)$ as the centralizer of an involution. %% was
                                                                                                       %% \F_q,
                                                                                                       %% does
                                                                                                       %% q
                                                                                                       %% =
                                                                                                       %% p^n?
\item[1955] The Brauer-Fowler theorem implies that the number of
  finite simple groups with given centralizer of involution is finite,
  suggesting an attack on the classification using centralizers of
  involutions.
\item[1955] Chevalley introduces the Chevalley groups, in particular
  introducing exceptional simple groups of types $F_4$, $E_7$, and
  $E_8$.
\item[1956] Hall-Higman theorem.
\item[1957] Suzuki shows that all finite simple (CA)-groups of odd
  order are cyclic.
\item[1958] The Brauer-Suzuki-Wall theorem characterizes the
  projective special linear groups of rank $1$, and classifies the
  simple (CA)-groups.
\item[1959] Steinberg introduces the Steinberg groups, giving some new
  finite simple groups, of types $^3D_4$ and $^2E_6$ (the latter were
  independently found at about the same time by Jacques Tits).
\item[1959] The Brauer-Suzuki theorem about groups with generalized
  quaternion Sylow $2$-subgroups shows in particular that none of them
  are simple.
\item[1960] Thompson proves that a group with a fixed-point-free
  automorphism of prime order is nilpotent.
\item[1960] Feit, Marshall Hall, and Thompson show that all finite
  simple (CN)-groups of odd order are cyclic.
\item[1960] Suzuki introduces the Suzuki groups, with types $^2B_2$.
\item[1961] Ree introduces the Ree groups, with types $^2F_4$ and
  $^2G_2$.
\item[1963]	Feit and Thompson prove the odd order theorem.
\item[1964] Tits introduces (BN) pairs for groups of Lie type and
  finds the Tits group.
\item[1965] The Gorenstein-Walter theorem classifies groups with a
  dihedral Sylow $2$-subgroup.
\item[1966] Glauberman proves the $Z^*$ theorem.
\item[1966] Janko introduces the Janko group $J_1$, the first new
  sporadic group for about a century.
\item[1968] Glauberman proves the $ZJ$ theorem.
\item[1968] Higman and Sims introduce the Higman-Sims group.
\item[1968] Conway introduces the Conway groups.
\item[1969] Walter's theorem classifies groups with Abelian Sylow
  $2$-subgroups.
\item[1969] Introduction of the Suzuki sporadic group, the Janko group
  $J_2$, the Janko group $J_3$, the McLaughlin group, and the Held
  group.
\item[1969] Gorenstein introduces signalizer functors based on
  Thompson's ideas.
\item[1970] MacWilliams shows that the $2$-groups with no normal
  Abelian subgroup of rank $3$ have sectional $2$-rank at most
  $4$. The simple groups with Sylow subgroups satisfying the latter
  condition were later classified by Gorenstein and Harada.
\item[1970] Bender introduced the generalized Fitting subgroup.
\item[1970] The Alperin-Brauer-Gorenstein theorem classifies groups
  with quasi-dihedral or wreathed Sylow $2$-subgroups, completing the
  classification of the simple groups of $2$-rank at most $2$.
\item[1971] Fischer introduces the three Fischer groups.
\item[1971] Thompson classifies quadratic pairs.
\item[1971] Bender classifies group with a strongly embedded subgroup.
\item[1972] Gorenstein proposes a $16$-step program for classifying
  finite simple groups; the final classification follows his outline
  quite closely.
\item[1972] Lyons introduces the Lyons group.
\item[1973] Rudvalis introduces the Rudvalis group.
\item[1973] Fischer discovers the baby monster group (unpublished),
  which Fischer and Griess use to discover the monster group, which in
  turn leads Thompson to the Thompson sporadic group and Norton to the
  Harada-Norton group (also found in a different way by Harada).
\item[1974] Thompson classifies $N$-groups, groups all of whose local
  subgroups are solvable.
\item[1974] The Gorenstein-Harada theorem classifies the simple groups
  of sectional $2$-rank at most $4$, dividing the remaining finite
  simple groups into those of component type and those of
  characteristic $2$ type.
\item[1974] Tits shows that groups with (BN) pairs of rank at least $3$
  are groups of Lie type.
\item[1974] Aschbacher classifies the groups with a proper $2$-generated
  core.
\item[1975] Gorenstein and Walter prove the $L$-balance theorem.
\item[1976] Glauberman proves the solvable signalizer functor theorem.
\item[1976] Aschbacher proves the component theorem, showing roughly
  that groups of odd type satisfying some conditions have a component
  in standard form. The groups with a component of standard form were
  classified in a large collection of papers by many authors.
\item[1976] O'Nan introduces the O'Nan group.
\item[1976]	Janko introduces the Janko group $J_4$, the last sporadic group to be discovered.
\item[1977] Aschbacher characterizes the groups of Lie type of odd
  characteristic in his classical involution theorem. After this
  theorem, which in some sense deals with ``most'' of the simple groups,
  it was generally felt that the end of the classification was in
  sight.
\item[1978] Timmesfeld proves the $O_2$ extraspecial theorem, breaking
  the classification of groups of $GF(2)$-type into several smaller
  problems.
\item[1978] Aschbacher classifies the thin finite groups, which are
  mostly rank $1$ groups of Lie type over fields of even
  characteristic.
\item[1981] Bombieri uses elimination theory to complete Thompson's
  work on the characterization of Ree groups, one of the hardest steps
  of the classification.
\item[1982] McBride proves the signalizer functor theorem for all
  finite groups.
\item[1982] Griess constructs the monster group by hand.
\item[1983] The Gilman-Griess theorem classifies groups of
  characteristic $2$ type and rank at least $4$ with standard
  components, one of the three cases of the trichotomy theorem.
\item[1983] Aschbacher proves that no finite group satisfies the
  hypothesis of the uniqueness case, one of the three cases given by
  the trichotomy theorem for groups of characteristic $2$ type.
\item[1983] Gorenstein and Lyons prove the trichotomy theorem for
  groups of characteristic $2$ type and rank at least $4$, while
  Aschbacher does the case of rank $3$. This divides these groups into
  3 subcases: the uniqueness case, groups of GF(2) type, and groups
  with a standard component.
\item[1983] Gorenstein announces the proof of the classification is
  complete, somewhat prematurely as the proof of the quasithin case
  was incomplete.
\item[1994] Gorenstein, Lyons, and Solomon begin publication of the
  revised classification.
\item[2004] Aschbacher and Smith publish their work on quasithin
  groups (which are mostly groups of Lie type of rank at most $2$ over
  fields of even characteristic), filling the last gap in the
  classification known at that time.
\item[2008] Harada and Solomon fill a minor gap in the classification
  by describing groups with a standard component that is a cover of
  the Mathieu group $M_{22}$, a case that was accidentally omitted
  from the proof of the classification due to an error in the
  calculation of the Schur multiplier of $M_{22}$.
\item[2012] Georges Gonthier and collaborators announce a
  computer-checked version of the Feit-Thompson theorem using the \texttt{Coq}
  proof assistant.
\end{description}



In some sense, the classification of the finite simple groups
``solves'' finite group theory---what's left are ``details'' left to
the reader. In constrast to this view, let me point out that some of
these ``details'' are quite complex and interesting. For example, we
know that if $p$ is a prime
\[
|\Z_{p-1}|= | \Z_p^\times|,
\]
and later, we will show that
\[
\Z_{p-1}\iso \Z_p^\times.
\]
However, the exact nature of this isomorphism will remain
mysterious. We know that this isomorphism must be of the form
\begin{align*}
  \Z_{p-1} &\to \Z_p^\times\\
  x &\mapsto a^x
\end{align*}
since homomorphisms of cyclic groups are determined by where the
generator maps to. But what should $a$ be?  Understanding this problem
amounts to solving what is known as the \index{discrete log problem}\link[discrete log problem]{https://en.wikipedia.org/wiki/Discrete_logarithm}.








For some interesting extra reading check out:
\begin{itemize}
\item \link[\textit{A brief history of the classification of the finite simple groups}, R.\ Solomon, Bulletin of the American Mathematical Society, Volume 38, Number 1, March 2001, Pages 315--352]{https://www.ams.org/journals/bull/2001-38-03/S0273-0979-01-00909-0/S0273-0979-01-00909-0.pdf}.


\end{itemize}


\end{document}
