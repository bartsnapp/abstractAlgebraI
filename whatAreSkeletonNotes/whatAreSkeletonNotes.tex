\begin{document}

\author{Bart Snapp}

\title{What are ``skeleton notes?''}

\begin{abstract}
  What is this document and how do I use it?
\end{abstract}
\maketitle


\section{Why?}

I call this document ``skeleton notes'' because it represents the
``bones'' of the course. This is in contrast to the ``flesh, blood,
heart,'' and dare I say, (yes I do) ``soul.''  These notes obviously q
have none of that.


It's \textit{your job} to supply \textbf{that.} These notes will give
you structure and support along the way.






\section{How to \sout{read} [do] mathematics}

Reading mathematics is \textbf{not} the same as reading a novel---it's
more fun, and more interactive!  To read mathematics you need
\begin{enumerate}
\item a pencil or pen that you enjoy writing with,
\item plenty of blank paper, and
\item the courage to write down everything---even ``obvious'' things.
\end{enumerate}
As you read a math book, you work along with me, the author repeating
many of the same calculations I did to write this book, trying to
anticipate my next thoughts.


\textbf{You} must \textbf{write} down each proposition, \textbf{discover}
proofs that convince \textbf{you}, and constantly \textbf{think} about what \textbf{you}
are doing.

\textbf{You} should work \textbf{examples}.  \textbf{You} should
\textbf{fill-in the details} I left out.  This is not an easy task; it
is \textbf{hard} work. However, it is work that is \textbf{rewarding in the
  end.}



\textbf{Mathematics is not a passive endeavor.}  Some may call me an ``author''
and you a ``reader.'' I call us ``new friends'' who are exploring
mathematics together. \textbf{Let's get going!}



\end{document}
